\documentclass[fleqn]{article}
\usepackage{geometry}
\usepackage{enumerate}
\usepackage{enumitem}
\usepackage{amssymb}
\usepackage{fancyhdr}
\usepackage{multicol}
\usepackage{blindtext}
\usepackage{color}
\usepackage[fontsize=16pt]{fontsize}
\usepackage{lipsum}
\usepackage{pgfplots}
\usepackage{physics}
\usepackage{mathtools}
\usepackage[makeroom]{cancel}
\usepackage{ulem}
\usepackage{gensymb}
\usepackage{graphicx}

\graphicspath{ {../Images/} }
\setlength{\columnsep}{1cm}
\geometry{margin = 2cm}
\addtolength{\jot}{0.1cm}
\def\columnseprulecolor{\color{blue}}
\date{Fall 2024}

\newcommand{\textoverline}[1]{$\overline{\mbox{#1}}$}

\newcommand{\hp}{\hspace{1cm}}

\newcommand{\const}{\textrm{const}}

\newcommand{\del}{\partial}

\newcommand{\pdif}[2]{ \frac{\partial #1}{ \partial #2} }

\newcommand{\pderiv}[1]{ \frac{\partial}{ \partial #1} }

\newcommand{\comment}[1]{}

\newcommand{\equations} [1] {
\begin{gather*}
#1
\end{gather*}
}

\newcommand{\twovec}[2]{ 
\begin{pmatrix}
#1 \\ 
#2
\end{pmatrix}
}

\begin{document}
\pagestyle{empty}

\section{midterm 2 is on thursday}
Luckily you're starting early-ish and you have enough homework drops to ignore all of it. Tech week is inconveniently timed but like you'd live.

\section{Oscillators}
write out all the shenanigans with $G$ and $\phi$ and $\omega_N$ and $\zeta$ and whatever 

When doing the big ass derivation, make sure to state explicitly why 
there is no phase function 

write out the rules for damping (underdamped, overdamped, etc.)

be real comfortable with finding the position function for any oscillator 
(damped or undamped)
explicitly state why the phase is 0 if undamped 

\section{Green's Function}
Green's Function itself is derived from the homogeneous solution of the equation of 
motion. 

The Green's function convolution can be derived from the fact that 
\[
F(t) = \sum^N_{n = 1} \delta(t - t_n)
\]
Just take that to infinity and boom you have any function you want.

\[
G(t) = e^{- \frac{ct}{2m} t} \frac{\sin(\omega_d t)}{m \omega_d}
\]

when in doubt use the equation 

\[
x(t) = \int^t_{-\infty} G(t - t') F(t') \, dt'
\]
lower bound is determined by the force 

for just a basic delta function 
\[
x(t) = \int^{\infty}_{-\infty} G(t - t') \delta(t - t_0) \, dt' = G(t - t_0)
\hp 
t > t_0
\]
It's just the same $G(t)$ except the impulse isnt at $t = 0$

\section{Reference Frames}
Linear reference frames arent bad and you can definitely intuit them out 

Rotational reference frames have some math derivations behind them. 

write out $e$, $n$, $u$ coordinates because theyre kind of stupid. 

$n$ and $e$ are parallel to the surface of the sphere and $u$ is perpendicular, or 
"upwards". 

write out position, velocity, acceleration, (and probably forces)
for non-inertial frames 

KNOW VECTOR IDENTITIES




\subsection{Maybe important??}
I'm just gonna write stuff from office hours
\[
\int^{\epsilon}_{- \epsilon} \ddot + \omega^2 \dot x + \gamma x = 
\int^{\epsilon}_{- \epsilon} \delta(t - 0)
\]
looking at it I think this is how you determine the initial conditions for 
Green's function?

if you have to write out the steady state solution, DEFINE IT.


\end{document}