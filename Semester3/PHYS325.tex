\documentclass[fleqn]{report}
\usepackage{geometry}
\usepackage{amssymb}
\usepackage{fancyhdr}
\usepackage{multicol}
\usepackage{blindtext}
\usepackage{color}
\usepackage[fontsize=16pt]{fontsize}
\usepackage{lipsum}
\usepackage{pgfplots}
\usepackage{physics}
\usepackage{mathtools}
\usepackage[makeroom]{cancel}
\usepackage{ulem}

\setlength{\columnsep}{1cm}
\addtolength{\jot}{0.1cm}
\def\columnseprulecolor{\color{blue}}
\date{Fall 2024}

\newcommand{\textoverline}[1]{$\overline{\mbox{#1}}$}

\newcommand{\hp}{\hspace{1cm}}

\newcommand{\del}{\partial}

\newcommand{\pdif}[2]{ \frac{\partial #1}{ \partial #2} }

\newcommand{\pderiv}[1]{ \frac{\partial}{ \partial #1} }

\newcommand{\comment}[1]{}

\newcommand{\equations} [1] {
\begin{gather*}
#1
\end{gather*}
}

\newcommand{\twovec}[2]{ 
\begin{pmatrix}
#1 \\ 
#2
\end{pmatrix}
}

\title{PHYS 325}
\author{Aiden Sirotkine}

\begin{document}

\pagestyle{fancy}
\maketitle
\tableofcontents
\clearpage

\chapter{PHYS325}
I missed everything from the first 2 weeks because my laptop exploded whoopsies

\chapter{Equations of motion}
derive the equations of motion to solve for a trajectory $\vec r(t)$ which is a position vector 
with respect to time

\section{Newton's Second Law}
\[
\vec F = m \vec a = m \frac{d^2 \vec r(t)}{dt^2} 
\hp
\vec F = \frac{d \vec p(t)}{dt}
\]

\section{Strategy}
\begin{enumerate}
\item
choose a reference frame and coordinates
\item identify all the relevant forces (external forces)

make a force diagram lol
\item
integrate N2 for a given force $\vec F(\vec r, \dot \vec r, t)$ to
find $\vec r(t)$
\item
fix integration constants from inital or boundary conditions. 
(e.g. $\vec v_0 = \vec v(t = 0) = 0)$ 

\item 
\equations{
\vec F = 0
\\
from N2L 0 = \vec F = m \vec a = m \frac{d \vec v}{dt}
\\
\rightarrow \vec v =  const = \vec v_0
\\
\vec r(t) = \int v_0 dt = v_0 t + r_0
}
\item
if F is constant
\equations{
\dot \vec v = \ddot \vec r = \vec a = \frac{\vec F_0}{m}
\\
\int dv = \int \frac{F_0}{m} dt \rightarrow \vec v(t) = 
\frac{F_0}{m} t + \vec v_0
\\
\vec r(t) = \int dr = \int \vec v dt =
\\
\frac{1}{2} \frac{\vec F_0}{m} t^2 + \vec v_0 t + \vec r_0
}

only valid for constant force
\end{enumerate}

\subsection{Time Dependent Force}
\equations{
\frac{d \vec v}{dt} = \vec a = \frac{\vec F}{m}
\\
\textrm{separation of variables}
\\
d \vec v = \frac{\vec F(t)}{m} dt 
\\
\int d \vec v = \int \frac{\vec F(t)}{m} dt
\\
\vec v(t) = \frac{\vec F}{m} + \vec C \hp \vec F = \int F(t) dt
\\
\textrm{C is the integration constant}
\\
\vec v = \frac{d \vec r}{dt}
\\
\vec r = \int \vec v dt
}

\subsection{Example: Forced Harmonic Oscillator}
A particle $m$ moves along $-\infty < x <\infty$. It is subjected to a force 
$F = F_0 \cos(\alpha t)$.
It starts at time $t = 0, x_0 = x(t = 0) = 0, v_0 = v(t = 0) = 0$

\begin{enumerate}
\item
coordinate system is just 1D
\item
force is $F = F_0 \cos(\alpha t)$
\item
equation of motion from N2L is $\frac{dv}{dt} = a = \frac{F}{m}$
\item
separation of variables
\equations{
    dv = \frac{F}{m} dt = \frac{F_0}{m} dt = \frac{F_0}{m} \cos{\alpha t} dt
    \\
    v(t) = \int dv = \frac{F_0}{m} \int \cos(\alpha t) dt = 
    \frac{F_0}{m} \frac{1}{2} \sin{\alpha t} + C_1
    \\
    x(t) = int dx = \int v dt =
    \\
    \int \frac{F_0}{\alpha m} \sin(\alpha t) + C_1 dt =
    \\
    - \frac{F_0}{\alpha^2 m} \cos(\alpha t) + C_1(t) + C_2
}
\item
find initial conditions

\equations{
0 = v_0 = \frac{F_0}{\alpha m} \sin{\alpha 0} + C_1 \rightarrow C_1 = 0
\\
0 = x_0 = - \frac{F_0}{\alpha^2 m} \cos(\alpha 0) + 0 + C_2 \rightarrow 
\\
C_2 = \frac{F_0}{\alpha^2 m}
\\
x(t) = \frac{F_0}{\alpha^2 m} \left(1 - \cos(\alpha t) \right)
\\
v(t) = \frac{F_0}{\alpha m} \sin(\alpha t)
}
\end{enumerate}

\section{Position Dependent Force}
focusing on 1 dimension for simplicity

get the equation of motion from Newton's 2nd Law

\equations{
F(x) = m a = m \frac{dv}{dt}
\\
\textrm{chain rule}
\\
\frac{dv}{dt} = \frac{dv}{dx} \frac{dx}{dt} = v \frac{dv}{dx}
\\
m v \frac{dv}{dx} = F(x)
\\
\textrm{separation of variables}
\\
mv \, dv = F(x) dx
\\
\textrm{Use definite integrals. relabel v = v' and x = x' (not derivatives)}
\\
m \int^v_{v_0} v' \, dv' = \int^x_{x_0} F(x') \, dx' 
\\
\frac{1}{2} m(v^2 - v_0^2) = \int^x_{x_0} F(x') \, dx'
\\
\textrm{solve for v}
\\
\textrm{It looks like change in kinetic energy and work}
\\
\Delta T = \frac{1}{2} m (v^2 - v_0^2) \hp W = \int F(x) \, dx
\\
\textrm{if force is conservative, it is path independent, }
\\
\textrm{and it can be written as a
gradient of a potential $\vec F = -\nabla U$}
\\
F = -\frac{dU}{dx}
\\
T - T_0 = \int^x_{x_0} - \frac{dU}{dx} \, dx' = - \left( U(x) - U(x_0) \right)
}
\equations{
E = T + U(x) = T_0 + U(x_0)
\\
\textrm{Conservation of Mechanical Energy}
\\
E = T + U(x) = \frac{1}{2} mv^2 + U(x)
\\
v = \pm \sqrt{\frac{2}{m} (E - U(x))}
\\
\textrm{use $v = \frac{dx}{dt}$ to find $x(t)$}
}

\section{Analyzing the Potential}
infer velocity and position

given energy, what is motion

$E = E_0 = E(x_0) = T(x_0) + U(x_0)$

$U(x_0) = E_0 =$ const $\rightarrow T(x_0) = 0 \rightarrow v(x_0) = C$

\subsection{case 2: energy of particle is \_}
\equations{
    E = E_1 = T(x) + U(x)
    \\
    \textrm{particle can move between $x_{1a}$ and $x_{1b}$}
    \\
    x_0: E_1 = E = T + U(x_0) 
    \\
    \textrm{ potential energy is at minimum, kinetic at max}
    \\
    \textrm{$v(x_0)$ is maximal}
}

\subsection{Case 3: $E = E_2 = U(x_2)$}
$T(x_2) = 0$ so $v = 0$





























\end{document}