\documentclass[fleqn]{report}
\usepackage{geometry}
\usepackage{amssymb}
\usepackage{fancyhdr}
\usepackage{multicol}
\usepackage{blindtext}
\usepackage{color}
\usepackage[fontsize=16pt]{fontsize}
\usepackage{lipsum}
\usepackage{pgfplots}
\usepackage{physics}
\usepackage{mathtools}
\usepackage[makeroom]{cancel}
\usepackage{ulem}

\setlength{\columnsep}{1cm}
\addtolength{\jot}{0.1cm}
\def\columnseprulecolor{\color{blue}}
\date{Fall 2024}

\newcommand{\textoverline}[1]{$\overline{\mbox{#1}}$}

\newcommand{\hp}{\hspace{1cm}}

\newcommand{\del}{\partial}

\newcommand{\pdif}[2]{ \frac{\partial #1}{ \partial #2} }

\newcommand{\pderiv}[1]{ \frac{\partial}{ \partial #1} }

\newcommand{\comment}[1]{}

\newcommand{\equations} [1] {
\begin{gather*}
#1
\end{gather*}
}

\newcommand{\twovec}[2]{ 
\begin{pmatrix}
#1 \\ 
#2
\end{pmatrix}
}

\title{PHYS 325}
\author{Aiden Sirotkine}

\begin{document}

\pagestyle{fancy}
\maketitle
\tableofcontents
\clearpage

\chapter{PHYS325}
I missed everything from the first 2 weeks because my laptop exploded whoopsies

\chapter{Equations of motion}
derive the equations of motion to solve for a trajectory $\vec r(t)$ which is a position vector 
with respect to time

\section{Newton's Second Law}
\[
\vec F = m \vec a = m \frac{d^2 \vec r(t)}{dt^2} 
\hp
\vec F = \frac{d \vec p(t)}{dt}
\]

\section{Strategy}
\begin{enumerate}
\item
choose a reference frame and coordinates
\item identify all the relevant forces (external forces)

make a force diagram lol
\item
integrate N2 for a given force $\vec F(\vec r, \dot \vec r, t)$ to
find $\vec r(t)$
\item
fix integration constants from inital or boundary conditions. 
(e.g. $\vec v_0 = \vec v(t = 0) = 0)$ 

\item 
\equations{
\vec F = 0
\\
from N2L 0 = \vec F = m \vec a = m \frac{d \vec v}{dt}
\\
\rightarrow \vec v =  const = \vec v_0
\\
\vec r(t) = \int v_0 dt = v_0 t + r_0
}
\item
if F is constant
\equations{
\dot \vec v = \ddot \vec r = \vec a = \frac{\vec F_0}{m}
\\
\int dv = \int \frac{F_0}{m} dt \rightarrow \vec v(t) = 
\frac{F_0}{m} t + \vec v_0
\\
\vec r(t) = \int dr = \int \vec v dt =
\\
\frac{1}{2} \frac{\vec F_0}{m} t^2 + \vec v_0 t + \vec r_0
}

only valid for constant force
\end{enumerate}

\subsection{Time Dependent Force}
\equations{
\frac{d \vec v}{dt} = \vec a = \frac{\vec F}{m}
\\
\textrm{separation of variables}
\\
d \vec v = \frac{\vec F(t)}{m} dt 
\\
\int d \vec v = \int \frac{\vec F(t)}{m} dt
\\
\vec v(t) = \frac{\vec F}{m} + \vec C \hp \vec F = \int F(t) dt
\\
\textrm{C is the integration constant}
\\
\vec v = \frac{d \vec r}{dt}
\\
\vec r = \int \vec v dt
}

\subsection{Forced Harmonic Oscillator}
A particle $m$ moves along $-\infty < x <\infty$. It is subjected to a force 
$F = F_0 \cos(\alpha t)$.
It starts at time $t = 0, x_0 = x(t = 0) = 0, v_0 = v(t = 0) = 0$

\begin{enumerate}
\item
coordinate system is just 1D
\item
force is $F = F_0 \cos(\alpha t)$
\item
equation of motion from N2L is $\frac{dv}{dt} = a = \frac{F}{m}$
\item
separation of variables
\equations{
    dv = \frac{F}{m} dt = \frac{F_0}{m} dt = \frac{F_0}{m} \cos{\alpha t} dt
    \\
    v(t) = \int dv = \frac{F_0}{m} \int \cos(\alpha t) dt = 
    \frac{F_0}{m} \frac{1}{2} \sin{\alpha t} + C_1
    \\
    x(t) = int dx = \int v dt =
    \\
    \int \frac{F_0}{\alpha m} \sin(\alpha t) + C_1 dt =
    \\
    - \frac{F_0}{\alpha^2 m} \cos(\alpha t) + C_1(t) + C_2
}
\item
find initial conditions

\equations{
0 = v_0 = \frac{F_0}{\alpha m} \sin{\alpha 0} + C_1 \rightarrow C_1 = 0
\\
0 = x_0 = - \frac{F_0}{\alpha^2 m} \cos(\alpha 0) + 0 + C_2 \rightarrow 
\\
C_2 = \frac{F_0}{\alpha^2 m}
\\
x(t) = \frac{F_0}{\alpha^2 m} \left(1 - \cos(\alpha t) \right)
\\
v(t) = \frac{F_0}{\alpha m} \sin(\alpha t)
}
\end{enumerate}

\section{Position Dependent Force}
focusing on 1 dimension for simplicity

get the equation of motion from Newton's 2nd Law

\equations{
F(x) = m a = m \frac{dv}{dt}
\\
\textrm{chain rule}
\\
\frac{dv}{dt} = \frac{dv}{dx} \frac{dx}{dt} = v \frac{dv}{dx}
\\
m v \frac{dv}{dx} = F(x)
\\
\textrm{separation of variables}
\\
mv \, dv = F(x) dx
\\
\textrm{Use definite integrals. relabel v = v' and x = x' (not derivatives)}
\\
m \int^v_{v_0} v' \, dv' = \int^x_{x_0} F(x') \, dx' 
\\
\frac{1}{2} m(v^2 - v_0^2) = \int^x_{x_0} F(x') \, dx'
\\
\textrm{solve for v}
\\
\textrm{It looks like change in kinetic energy and work}
\\
\Delta T = \frac{1}{2} m (v^2 - v_0^2) \hp W = \int F(x) \, dx
\\
\textrm{if force is conservative, it is path independent, }
\\
\textrm{and it can be written as a
gradient of a potential $\vec F = -\nabla U$}
\\
F = -\frac{dU}{dx}
\\
T - T_0 = \int^x_{x_0} - \frac{dU}{dx} \, dx' = - \left( U(x) - U(x_0) \right)
}
\equations{
E = T + U(x) = T_0 + U(x_0)
\\
\textrm{Conservation of Mechanical Energy}
\\
E = T + U(x) = \frac{1}{2} mv^2 + U(x)
\\
v = \pm \sqrt{\frac{2}{m} (E - U(x))}
\\
\textrm{use $v = \frac{dx}{dt}$ to find $x(t)$}
}

\section{Analyzing the Potential}
infer velocity and position

given energy, what is motion

$E = E_0 = E(x_0) = T(x_0) + U(x_0)$

$U(x_0) = E_0 =$ const $\rightarrow T(x_0) = 0 \rightarrow v(x_0) = C$

\subsection{case 2: energy of particle is \_}
\equations{
    E = E_1 = T(x) + U(x)
    \\
    \textrm{particle can move between $x_{1a}$ and $x_{1b}$}
    \\
    x_0: E_1 = E = T + U(x_0) 
    \\
    \textrm{ potential energy is at minimum, kinetic at max}
    \\
    \textrm{$v(x_0)$ is maximal}
}

\subsection{Case 3: $E = E_2 = U(x_2)$}
$T(x_2) = 0$ so $v = 0$

\section{Analyzing Extrema of Potential}
To find the extrema points, find the derivative of the potential and set it 
equal to 0.

To find the type of extrema point, take the second derivative and then check the
sign

\subsection{Push Particle Towards Extrema}
\begin{enumerate}
\item
$x = x_1$ is a minimum

If the particle is slightly moved from the minimum, it will return to the minimum
 because that is where the lowest energy point is. 

This point is called stable for the particle.
\item
$x = x_2$ is a maximum

The particle will move farther away from the equilibrium point.

This point is unstable
\item
$x = x_3$ is a saddle point

The particle is stable in one direction and unstable in the other

marginally stable point 
\end{enumerate}

\section{Simple Harmonic Oscillator}
Simple Harmonic Oscillator

Figure out motion near the equilibrium point

To approximate a function only near a specific point, use a Taylor Series

Taylor expand potential around $x_0$
\equations{
    U(x) \approx U(x_0) + U'|_{x = x_0}(x - x_0) + \frac{1}{2} U''|_{x = x_0} (x - x_0)^2 + \ldots 
    \\
    \textrm{choose $U(x_0) = 0$ extremum}
    \\
    U(x) \approx \frac{1}{2} kx^2
}

Can also taylor expand force near $x_0$
\equations{
    F(x) \approx F(x_0) + F'|_{x = x_0} (x - x_0) + 
    \frac{1}{2} F''|_{x_0} (x - x_0)^2
    \\
    -U'|_{x_0} = 0 \hp -U''|_{x_0} = -k \hp \textrm{rest is small}  
}

\newpage
insert equation of motion 
\equations{
    m \ddot x = F(x) \approx F'|_{x_0} (x - x_0) = -kx
    \\
    m \ddot x + kx = 0
    \\
    \textrm{ansatz: guess the form of the solution (trig or exponent)}
    \\
    x(t) = A \sin(\omega t + \varphi)
    \\
    \dot x(t) = \frac{dx}{dt} = A \omega \cos (\omega t + \varphi)
    \\
    \ddot x(t) = -A \omega^2 \sin(\omega t + \varphi)
    \\
    \textrm{insert ansatz into equation of motion}
    \\
    m (-A) \omega^2 \sin(\omega t + \varphi) + 
    k A \sin(\omega t + \varphi) = 0
    \rightarrow 
    \\
    \omega = \sqrt{\frac{k}{m}}
    \hp
    \textrm{period = } T = \frac{2 \pi }{\omega} = 2 \pi \sqrt{\frac{m}{k}}
}

\section{velocity Dependent Force}
$\vec F = \vec F(\vec v)$

drag forces, friction, air resistance

\subsection{Types of drag forces}
\begin{enumerate}
    \item
    Stoke's drag (linear)
    \[
    \vec F = -c \vec v
    \]
    laminar flow, valid for small velocities and viscous fluids
    \item
    Newtonian Drag (nonlinear)
    \[
    \vec F(\vec v) = -k \vec v^2
    \]
    valid for larger velocities and less viscous fluids.
\end{enumerate}

The type of drag is applicable based of Reynold's Number
\[
R = \frac{\rho v L}{\mu} \hp
\textrm{ density, fluid flow, size, viscosity}
\]

\section{Linear Drag}
particle of mass $m$ and initial velocity $v_0$ in laminar flow

goal, derive velocity after a long time (should be 0)

choose coordinate system: 1D along x direction $v = \dot x$

force $F(v) = -cv$

introduce new constant $\kappa  = \frac{c}{m}$
\[
[k] = [\frac{F}{m * v}] = [\frac{1}{t}]
\]

\equations{
    F = -cv = -m \kappa v
    \\
    F = m \ddot x = m \frac{dv}{dt} = -m \kappa v
    \\
    \textrm{separation of variables}
    \\
    \frac{1}{v} dv = - \kappa dt 
    \\
    \textrm{use definite integrals, relabel variables}
    \\
    \int^v_{v_0} \frac{1}{v'} dv' = - \kappa \int^t_{t_0} dt'
    \\
    - \kappa t = \ln(v') \Big |^v_{v_0} = \ln(\frac{v}{v_0})
    \\
    v(t) = v_0 \exp(- \kappa t)
}
exponential decay with a rate of $\kappa$

For some sanity checks, you can try $t = \infty$ and $t = 0$ to 
make sure they make sense

\section{non-linear drag with gravity}
motion is 1D along the z axis 

drag force $F = -c v^2$ sgn(v)

(sgn(v) is $\pm 1$ depending on if v is greater or less than 0)

new constant 
\[
\sigma = \frac{c}{m}
\]

\equations{
    F = -mg + cv^2
    \rightarrow
    -mg + \frac{C}{m}mv^2 \rightarrow
    m(g - \sigma v^2)
    \rightarrow
    \\
    \frac{dv}{dt} = -g + \sigma v^2
    \rightarrow
    \int^v_0 \frac{1}{-g + \sigma v^2} dv = \int^t_0 dt
    \\
    t = \frac{-1}{\sqrt{g \sigma}}
    \tanh^{-1}(v \sqrt{\frac{\sigma}{g}})
    \\
    v = - \sqrt{\frac{g}{\sigma}} \tanh(\sqrt{gh} t)
    \\
    t \to \infty \hp v \to - \sqrt{\frac{g}{\sigma}}
}
You can see that $v$ asymptotically approaches a constant
velocity with large time which you can kind of see 
from the initial equation.

For most cases $m \frac{dv}{dt} = \vec F(\vec r, \vec v, t)$ is 
unsolvable. However, for some cases
$F = f(v) g(t)$, you get
$\int \frac{m}{f(v)} dv = \int g(t) dt$

For another special case $F = f(v) h(x)$, you get
\equations{
    m \frac{dv}{dt} \frac{1}{f(v)} = h(x) 
    \\
    \frac{dv}{dt} = \frac{dv}{dx} \frac{dx}{dt} = v \frac{dv}{dx}
    \\
    m v \frac{dv}{dx} \frac{1}{f(v)} = h(x)
    \rightarrow
    \int \frac{mv}{f(v)} dv = \int h(x) dx 
}

\section{Time Varying Mass $M(t)$}
$u$ is the speed of the mass being projected out.
\equations{
    P_i = M_i v_i
    \\
    P_f = dm(v - u)
    \rightarrow
    t(m_f v_f)
    =
    fm(v - u)
    \\
    P_i = P_f \Rightarrow Mv_i = -u dm + m_i v_i - dM dv 
    \\
    dM dv \approx 0 \textrm{ and cancel some stuff}
    \\
    Mdv = t u dm
    \Longrightarrow
    dv = \frac{-u}{M} dm
    \rightarrow
    \int dv = \int \frac{-u}{M} dm
    \rightarrow 
    \\
    v - v_0 = -u \ln(\frac{M}{M_0})
    \rightarrow
    v = v_0 - u \ln(\frac{M}{M_0})
}
Because its logarithmic, its really hard to boost something by 
chucking mass out the back.

Conservation of momentum is not true in gravity 
$P_i \neq P_f$

\equations{
    dp = P_f - P_i \neq 0
    \\
    dp = -Mg dt 
    -Mg \, dt = M dv + u dM 
    \\
    \frac{dv}{dt} = -g - \frac{u}{M} \frac{dM}{dt}
    \hp 
    \textrm{ THIS IS THE IMPORTANT EQUATION}
}
$u, M$, and $dM/dt$ are all controllable by the rocket

\subsection{velocity from acceleration}
\[
v(t) = v_0 -gt - u \ln(\frac{M_0 - dt}{M_0})
\]

\section{2D Rocket}
\begin{gather*}
    \vec F(\vec r, \dot \vec r, t) = m \vec a = m \ddot \vec r
\end{gather*}
Cartesian Coordinates 
\equations{
    F_x(\vec r, \dot \vec r, t) = m \ddot x
    \hp
    F_y(\vec r, \dot \vec r, t) = m \ddot y
    \hp
    F_z(\vec r, \dot \vec r, t) = m \ddot z
}
Consider a magnetic field $B e_z$
\equations{
    \vec F = q \vec v \times \vec B = q(\vec v \times \vec B)
    \rightarrow
    \\
    qB(v_y e_x - v_x e_y)
    \hp 
    F_x = qBv_y = m \ddot x
    \hp 
    F_y = -qB v_x = m \ddot y 
    \\
    m \ddot z = 0 \Longrightarrow \dot z = \textrm{const}
    \\
    qB \dot y = m \ddot x
    \hp 
    qB \dot x = m \ddot y
    \\
    qB \ddot y = m \dddot x
    \rightarrow
    qB (\frac{-qB}{m}) \dot x = m \dddot x
    \rightarrow
    \ddot v_x = - \omega^2 v_x
    \\
    \textrm{this is the spring equation}
    \\
    \ddot x = \frac{k}{m} x
    \longrightarrow
    v_x = A \sin(\omega t + \phi)
    \hp 
    v_y = A \cos(\omega t + \phi)
    \\
    \textrm{it goes in a circle}
}

\subsection{something}
\equations{
    \ddot v_x = \omega \dot v_y
    \\
    \ddot v_x = -\omega^2 v_x
    \hp 
    \ddot v_y = - \omega^2 v_y
    \\
    \dot \nu = \dot v_x + i \dot v_y
    \\
    =
    \omega v_y - i \omega v_x
    \\
    \dot \nu = -i \omega \nu 
    \\
    \textrm{and some stuff I missed}
}

\subsection{Particle in Field}
Motion of a charged particle in a homogeneous magnetic field
\equations{
\dot v_x = \omega v_y
\hp 
\dot v_y = - \omega v_x
\hp
w / \omega = \frac{qB}{m}
\mu = v_x + i v_y
\\
\dot mu = \dot v_x + i \dot v_y
\longrightarrow
\omega v_y - i \omega v_x
\rightarrow
-i \omega (v_x + i v_y)
\\
\dot \mu = -i \omega \mu
\\
\textrm{solve with separation of variables}
\\
\frac{1}{\mu} \, d \mu = - i \omega \, dt
\rightarrow
\ln(\mu) + \tilde{C} = \ln(\frac{\mu}{C})
\\
\mu = C \exp(-i \omega t)
\hp
C = A e^{i \delta}
\\
\mu = A \exp(i(\delta - \omega t))
}
Use the euler identity to keep doing math
\equations{
	\mu = A(\cos(\omega t - \delta) - i \sin(\omega t - \delta))
	\\
	\textrm{the signs reversed and im not entirely sure why or how}
	\\
	v_x = \Re(\mu) = A \cos(\omega t - \delta)
	\hp
	v_y = \Im(\mu) = -A \sin(\omega t - \delta)
    \\
    |v^2| = A^2 \cos(\omega t - \delta) + A_2 \sin(\omega t - \delta)
    = A^2
    \\
    \textrm{double check this is true by 
    showing derivative of $v$ is 0}
    \\
    \frac{1}{2} \frac{d}{dt}|v|^2 = 
    \frac{1}{2}m 2v \cdot \frac{d}{dt}v
    \\
    = v \cdot q(v \times B) = 0
    \\
    = 0 \hp A = \textrm{constant}
}

Now you figure out the trajectory
\equations{
    \vec v = \frac{dr}{dt} \rightarrow v \, dt = dr 
    \\
    x(t) = \int v_x \, dt = \int A \cos(\omega t - \delta) \, dt 
    =
    \frac{A}{\omega} \sin(\omega t - \delta) + C_y
    \\
    z(t) = \int v_z \, dt = \int v_{z0} \, dt = 
    v_{z0} t + z_0
    \\
    A = \sqrt{v_x^2 + v_y^2}
}

\section{Curvilinear Coordinates}
non-Cartesian coordinates 

\subsection{Examples}
2d polar coordinates $(r, \theta)$
\[
x = r \cos(\phi(t)) \hp
y = r \sin(\phi(t))
\hp 
r = \sqrt{x^2 + y^2}
\]

3d cylindrical coordinates $(r, \phi, z)$
\[
x \ r \cos(\phi) \hp y = r \sin(\phi) \hp z = z
\]

3d spherical coordinates $(r, \theta, \phi)$
\[
x = r \cos(\phi) \sin(\theta)
\hp 
y = r \sin(\phi) \sin(\theta)
\hp 
z = r \cos(\theta) 
\]

\section{Whirling Stick}
A rigid stick whriling with fixed $\omega$

use polar coordinates to have the easiest math 
\equations{
    \omega = \varphi t = \textrm{const}
    \\
    e_r = e_r (t) \hp e_{\varphi} = e_{\varphi} (t)
    \\
    \textrm{relate $r$ and $\varphi$ to $x$ and $y$ basis vectors}
    \\
    e_r(t) = \cos(\varphi(t)) e_x + \sin(\varphi(t))e_y
    \\
    e_{\varphi}(t) = -\sin(\varphi(t)) e_x + \cos(\varphi(t)) e_y
    \\
    \textrm{position vector } r(t) = r(t) e_r(t)
}

Determine the first derivatives
\equations{
    \dot e_r = \frac{d}{dt}\cos(\varphi(t)) e_x + \sin(\varphi(t))e_y
    \rightarrow
    \\
    \frac{d}{dt}\cos(\varphi(t)) e_x + 
    \cos(\varphi) \frac{d}{dt}e_x + 
    \frac{d}{dt} \sin(\varphi(t))e_y + 
    \sin(\varphi(t)) \frac{d}{dt} e_y
    \\
    \frac{d}{dt}\cos(\varphi(t)) e_x + 
    \frac{d}{dt} \sin(\varphi(t))e_y 
    \\
    \dot e_r = - \dot \varphi \sin(\varphi) e_x + 
    \dot \varphi \cos(\varphi) e_y = \dot \varphi e_\varphi
    \\
    \cdot e_{\varphi} = - \dot \varphi e_{\varphi}
    \\
    v(t) = \dot r(t) e_r + r \dot \varphi e_\varphi
    =
    v_r e_r + v_\varphi e_\varphi 
}

now acceleration lol
\equations{
    a = \frac{dv}{dt} = 
    (\ddot r - r \dot \varphi^2) e_r + 
    (r \ddot \varphi + 2 \dot r \dot \varphi) e_{\varphi}
    \\
    \varphi = \omega t \hp r \dot \varphi^2  = r \omega^2
    \\
    \textrm{centripetal force}
}

\section{Bead on a Whirling Rod}
Use polar coordinates for sake of convenience. 
Use basis vectors $e_r$ and $e_\phi$

Rod whirls at a rate of $\omega$, so $\phi(t) = \omega t$

Our general strategy is draw a sketch and then figure out 
your coordinates and then write out position, velocity, and 
acceleration vectors.

Used Newton's 2nd law to to get a differential equation.

Solve the differential equation.
\equations{
    a(t) = [\ddot r - r \dot \phi^2] \vec e_r + 
    [r \ddot \phi + 2 \dot r \dot \phi] \vec e_\phi 
    \\
    F = ma 
    \\
    \textrm{no force in the radial direction because 
    it's all normal force}
    \\
    F_{net} = F_n = F_n(t) e_\phi(t) + 0 e_r(t)
    \\
    F_n e_\phi (t) = m [\ddot r - r \dot \phi^2] \vec e_r + 
    m[r \ddot \phi + 2 \dot r \dot \phi] \vec e_\phi 
    \\
    e_r = 0 = m (\ddot r - r \dot \phi^2)
    \hp 
    e_\phi = F_n = m (r \ddot \phi + 2 \dot r \dot \phi )
    \\
    0 = \ddot r - r \dot \phi^2 \rightarrow
    \ddot r = r \dot \phi^2 \hp 
    \phi(t) = \omega t, \dot \phi(t) = \omega
    \\
    \ddot r = \omega^2 r
    \hp 
    \textrm{use an ansatz/guess}
    \\
    r = e^{\lambda t} \hp 
    \dot r = \lambda e^{\lambda t} \hp 
    \ddot r = \lambda^2 e^{\lambda t}
    \\
    \lambda^2 e^{\lambda t} = \omega^2 e^{\lambda t}
    \hp \lambda^2 = \omega^2
    \hp \lambda = \pm \omega 
    \\
    r(t) = Ae^{\omega t} + B e^{- \omega t}
    \hp \textrm{solve with initial condition}
    \\
    \dot r(t) = \omega A e^{\omega t} - \omega B e^{\omega t}
    \\
    r(t = 0) = r_0 
    \hp v(t = 0) = v_0
    \\
    A + B = r_0
    \hp A - B = 0
    \rightarrow 
    A = B
    \hp A = B = \frac{r_0}{2}
    \\
    r(t) = \frac{r_0}{2} 
    (e^{\omega t} + e^{-\omega t})
    =
    \frac{r_0}{2} \cosh(\omega t)
    \\
    F_n = m (r \ddot \phi + 2 \dot r \dot \phi)
    \hp \phi = \omega t
    \\
    F_n = m2 \dot r \omega 
    = m \omega^2 r_0 \sinh(\omega t) 
}

\subsection{Bead on Spinning Loop}
Loop of fixed radius R spinning about vertical axis at fixed rate 
$\Omega$. Bead of mass $m$, free to move along the loop. Everything 
is in a gravitational field. 

Use spherical coordinates for math convenience. 

\equations{
    x = r\sin(\theta) \cos(\phi)
    \hp
    y = r\sin(\theta) \sin(\phi)
    \hp 
    z = r \cos(\theta)
    \\
    r^2 = x^2 + y^2 + z^2
    \hp 
    \tan(\theta) = \frac{\sqrt{x^2 + y^2}}{x^2}
    \\
    \textrm{the coordinate vectors are real difficult to find}
    \\
    \textrm{coords, position, velocity, acceleration}
    \\
    e_r = \sin(\theta) \cos(\phi) e_x + 
    \sin(\theta) \sin(\phi) e_y + 
    \cos(\theta) e_z
    \\
    e_\theta = \cos(\theta) \cos(\phi) e_x + 
    \cos(\theta) \sin(\phi) e_y -
    \sin(\theta) e_z
    \\
    e_\phi = - \sin \phi e_x + \cos(\phi) e_y
    \\
    r(t) = R = \textrm{const}
    \hp \phi(t) = \Omega t \hp \Omega = \textrm{const}
    \hp
    \theta(t)
    \\
    \dot e_r = \textrm{blah}
    \dot e_\theta = \textrm{blah}
    \dot e_\phi = \textrm{blah}
    \\
    r(t) = r(t) e_r(t)
    \hp v(t) = \frac{d}{dt} \left( r(t) e_r(t) \right)
    =
    \dot r e_r + r \dot e_r 
    =
    \\
    \dot r e_r + r \left( 
    \dot \theta e_\theta + \dot \phi \sin(\theta) e_\phi
    \right)
    =
    \dot r e_r + r \dot \theta e_\theta + 
    r \sin(\theta) \dot \phi e_\phi
    \\
    a = \frac{dv}{dt} = \frac{d}{dt}
    \dot r e_r + r \dot \theta e_\theta + 
    r \sin(\theta) \dot \phi e_\phi
    =
    \\
    \ddot r e_r + \dot r \dot e_r + 
    \dot r (\dot \theta e_\theta) + r(\ddot \theta e_\theta + 
    \dot \theta \dot e_\theta) + \\
    \dot r (\sin(\theta) \dot \phi e_\phi) +
    r (
        \cos(\theta) \dot \phi e_\phi + 
        \sin(\theta)(\ddot \phi e_\phi + \dot \phi \dot e_\phi)
    )
    \\
    \textrm{what insanity}
    \textrm{plug in all the coordinate vectors to the thingy}
    \\
    \vec a(t) = 
    [\ddot r - r \dot \theta^2 - 
    r \sin^2 \theta \dot \phi^2] e_r 
    +
    r[2 \dot r \dot \theta + r \ddot \theta  - 
    r \sin(\theta) \cos(\theta) \dot \phi^2] e_\theta 
    + \\
    [2\dot r \sin(\theta) \dot \phi + 
    2 r \cos(\theta) \dot \theta \dot \phi +
    r \sin \theta \ddot \phi] e_\phi
}

\equations{
    \textrm{now use $F = ma$ to do some bullshit}
    \\
    F = ma \hp F_g = -mg e_z = 
    -mg (\cos \theta e_r - \sin(\theta) e_\theta )
    \\
    F_n = N_r e_r + N_\phi e_\phi
    \\
    F_{net} = F_G + F_n = 
    F_n = N_r e_r + N_\phi e_\phi
    -mg (\cos \theta e_r - \sin(\theta) e_\theta )
    = \\
    [N_r - mg \cos(\theta)] e_r +
    mg \sin (\theta) e_\theta + 
    N_\phi e_\phi
    \\
    F_\theta = m a_\theta
    \\
    mg \sin(\theta) = \\
    [\ddot r - r \dot \theta^2 - 
    r \sin^2 \theta \dot \phi^2] e_r 
    +
    r[2 \dot r \dot \theta + r \ddot \theta  - 
    r \sin(\theta) \cos(\theta) \dot \phi^2] e_\theta 
    + \\
    [2\dot r \sin(\theta) \dot \phi + 
    2 r \cos(\theta) \dot \theta \dot \phi +
    r \sin \theta \ddot \phi] e_\phi
    \\
    g \sin(\theta) = r \ddot \theta -
    R \sin(\theta) \cos(\theta) \Omega^2
    = \\
    \ddot \theta = - \frac{g}{R} \sin(\theta) + 
    \Omega^2 \sin(\theta) \cos(\theta)
    \hp \textrm{$r \rightarrow R$ because it's constant}
}























\end{document}
