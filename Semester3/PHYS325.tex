\documentclass[fleqn]{report}
\usepackage{geometry}
\usepackage{amssymb}
\usepackage{fancyhdr}
\usepackage{multicol}
\usepackage{blindtext}
\usepackage{color}
\usepackage[fontsize=16pt]{fontsize}
\usepackage{lipsum}
\usepackage{pgfplots}
\usepackage{physics}
\usepackage{mathtools}
\usepackage[makeroom]{cancel}
\usepackage{ulem}

\setlength{\columnsep}{1cm}
\addtolength{\jot}{0.1cm}
\def\columnseprulecolor{\color{blue}}
\date{Fall 2024}

\newcommand{\textoverline}[1]{$\overline{\mbox{#1}}$}

\newcommand{\hp}{\hspace{1cm}}

\newcommand{\const}{\textrm{const}}

\newcommand{\del}{\partial}

\newcommand{\pdif}[2]{ \frac{\partial #1}{ \partial #2} }

\newcommand{\pderiv}[1]{ \frac{\partial}{ \partial #1} }

\newcommand{\comment}[1]{}

\newcommand{\equations} [1] {
\begin{gather*}
#1
\end{gather*}
}

\newcommand{\numequations} [1] {
\begin{gather}
#1
\end{gather}
}

\newcommand{\twovec}[2]{ 
\begin{pmatrix}
#1 \\ 
#2
\end{pmatrix}
}

\title{PHYS 325}
\author{Aiden Sirotkine}

\begin{document}

\pagestyle{fancy}
\maketitle
\tableofcontents
\clearpage

\chapter{PHYS325}
I missed everything from the first 2 weeks because my laptop exploded whoopsies

\chapter{Equations of motion}
derive the equations of motion to solve for a trajectory $\vec r(t)$ which is a position vector 
with respect to time

\section{Newton's Second Law}
\[
\vec F = m \vec a = m \frac{d^2 \vec r(t)}{dt^2} 
\hp
\vec F = \frac{d \vec p(t)}{dt}
\]

\section{Strategy}
\begin{enumerate}
\item
choose a reference frame and coordinates
\item identify all the relevant forces (external forces)

make a force diagram lol
\item
integrate N2 for a given force $\vec F(\vec r, \dot \vec r, t)$ to
find $\vec r(t)$
\item
fix integration constants from inital or boundary conditions. 
(e.g. $\vec v_0 = \vec v(t = 0) = 0)$ 

\item 
\equations{
\vec F = 0
\\
from N2L 0 = \vec F = m \vec a = m \frac{d \vec v}{dt}
\\
\rightarrow \vec v =  const = \vec v_0
\\
\vec r(t) = \int v_0 dt = v_0 t + r_0
}
\item
if F is constant
\equations{
\dot \vec v = \ddot \vec r = \vec a = \frac{\vec F_0}{m}
\\
\int dv = \int \frac{F_0}{m} dt \rightarrow \vec v(t) = 
\frac{F_0}{m} t + \vec v_0
\\
\vec r(t) = \int dr = \int \vec v dt =
\\
\frac{1}{2} \frac{\vec F_0}{m} t^2 + \vec v_0 t + \vec r_0
}

only valid for constant force
\end{enumerate}

\subsection{Time Dependent Force}
\equations{
\frac{d \vec v}{dt} = \vec a = \frac{\vec F}{m}
\\
\textrm{separation of variables}
\\
d \vec v = \frac{\vec F(t)}{m} dt 
\\
\int d \vec v = \int \frac{\vec F(t)}{m} dt
\\
\vec v(t) = \frac{\vec F}{m} + \vec C \hp \vec F = \int F(t) dt
\\
\textrm{C is the integration constant}
\\
\vec v = \frac{d \vec r}{dt}
\\
\vec r = \int \vec v dt
}

\subsection{Forced Harmonic Oscillator}
A particle $m$ moves along $-\infty < x <\infty$. It is subjected to a force 
$F = F_0 \cos(\alpha t)$.
It starts at time $t = 0, x_0 = x(t = 0) = 0, v_0 = v(t = 0) = 0$

\begin{enumerate}
\item
coordinate system is just 1D
\item
force is $F = F_0 \cos(\alpha t)$
\item
equation of motion from N2L is $\frac{dv}{dt} = a = \frac{F}{m}$
\item
separation of variables
\equations{
    dv = \frac{F}{m} dt = \frac{F_0}{m} dt = \frac{F_0}{m} \cos{\alpha t} dt
    \\
    v(t) = \int dv = \frac{F_0}{m} \int \cos(\alpha t) dt = 
    \frac{F_0}{m} \frac{1}{2} \sin{\alpha t} + C_1
    \\
    x(t) = int dx = \int v dt =
    \\
    \int \frac{F_0}{\alpha m} \sin(\alpha t) + C_1 dt =
    \\
    - \frac{F_0}{\alpha^2 m} \cos(\alpha t) + C_1(t) + C_2
}
\item
find initial conditions

\equations{
0 = v_0 = \frac{F_0}{\alpha m} \sin{\alpha 0} + C_1 \rightarrow C_1 = 0
\\
0 = x_0 = - \frac{F_0}{\alpha^2 m} \cos(\alpha 0) + 0 + C_2 \rightarrow 
\\
C_2 = \frac{F_0}{\alpha^2 m}
\\
x(t) = \frac{F_0}{\alpha^2 m} \left(1 - \cos(\alpha t) \right)
\\
v(t) = \frac{F_0}{\alpha m} \sin(\alpha t)
}
\end{enumerate}

\section{Position Dependent Force}
focusing on 1 dimension for simplicity

get the equation of motion from Newton's 2nd Law

\equations{
F(x) = m a = m \frac{dv}{dt}
\\
\textrm{chain rule}
\\
\frac{dv}{dt} = \frac{dv}{dx} \frac{dx}{dt} = v \frac{dv}{dx}
\\
m v \frac{dv}{dx} = F(x)
\\
\textrm{separation of variables}
\\
mv \, dv = F(x) dx
\\
\textrm{Use definite integrals. relabel v = v' and x = x' (not derivatives)}
\\
m \int^v_{v_0} v' \, dv' = \int^x_{x_0} F(x') \, dx' 
\\
\frac{1}{2} m(v^2 - v_0^2) = \int^x_{x_0} F(x') \, dx'
\\
\textrm{solve for v}
\\
\textrm{It looks like change in kinetic energy and work}
\\
\Delta T = \frac{1}{2} m (v^2 - v_0^2) \hp W = \int F(x) \, dx
\\
\textrm{if force is conservative, it is path independent, }
\\
\textrm{and it can be written as a
gradient of a potential $\vec F = -\nabla U$}
\\
F = -\frac{dU}{dx}
\\
T - T_0 = \int^x_{x_0} - \frac{dU}{dx} \, dx' = - \left( U(x) - U(x_0) \right)
}
\equations{
E = T + U(x) = T_0 + U(x_0)
\\
\textrm{Conservation of Mechanical Energy}
\\
E = T + U(x) = \frac{1}{2} mv^2 + U(x)
\\
v = \pm \sqrt{\frac{2}{m} (E - U(x))}
\\
\textrm{use $v = \frac{dx}{dt}$ to find $x(t)$}
}

\section{Analyzing the Potential}
infer velocity and position

given energy, what is motion

$E = E_0 = E(x_0) = T(x_0) + U(x_0)$

$U(x_0) = E_0 =$ const $\rightarrow T(x_0) = 0 \rightarrow v(x_0) = C$

\subsection{case 2: energy of particle is \_}
\equations{
    E = E_1 = T(x) + U(x)
    \\
    \textrm{particle can move between $x_{1a}$ and $x_{1b}$}
    \\
    x_0: E_1 = E = T + U(x_0) 
    \\
    \textrm{ potential energy is at minimum, kinetic at max}
    \\
    \textrm{$v(x_0)$ is maximal}
}

\subsection{Case 3: $E = E_2 = U(x_2)$}
$T(x_2) = 0$ so $v = 0$

\section{Analyzing Extrema of Potential}
To find the extrema points, find the derivative of the potential and set it 
equal to 0.

To find the type of extrema point, take the second derivative and then check the
sign

\subsection{Push Particle Towards Extrema}
\begin{enumerate}
\item
$x = x_1$ is a minimum

If the particle is slightly moved from the minimum, it will return to the minimum
 because that is where the lowest energy point is. 

This point is called stable for the particle.
\item
$x = x_2$ is a maximum

The particle will move farther away from the equilibrium point.

This point is unstable
\item
$x = x_3$ is a saddle point

The particle is stable in one direction and unstable in the other

marginally stable point 
\end{enumerate}

\section{Simple Harmonic Oscillator}
Simple Harmonic Oscillator

Figure out motion near the equilibrium point

To approximate a function only near a specific point, use a Taylor Series

Taylor expand potential around $x_0$
\equations{
    U(x) \approx U(x_0) + U'|_{x = x_0}(x - x_0) + \frac{1}{2} U''|_{x = x_0} (x - x_0)^2 + \ldots 
    \\
    \textrm{choose $U(x_0) = 0$ extremum}
    \\
    U(x) \approx \frac{1}{2} kx^2
}

Can also taylor expand force near $x_0$
\equations{
    F(x) \approx F(x_0) + F'|_{x = x_0} (x - x_0) + 
    \frac{1}{2} F''|_{x_0} (x - x_0)^2
    \\
    -U'|_{x_0} = 0 \hp -U''|_{x_0} = -k \hp \textrm{rest is small}  
}

\newpage
insert equation of motion 
\equations{
    m \ddot x = F(x) \approx F'|_{x_0} (x - x_0) = -kx
    \\
    m \ddot x + kx = 0
    \\
    \textrm{ansatz: guess the form of the solution (trig or exponent)}
    \\
    x(t) = A \sin(\omega t + \varphi)
    \\
    \dot x(t) = \frac{dx}{dt} = A \omega \cos (\omega t + \varphi)
    \\
    \ddot x(t) = -A \omega^2 \sin(\omega t + \varphi)
    \\
    \textrm{insert ansatz into equation of motion}
    \\
    m (-A) \omega^2 \sin(\omega t + \varphi) + 
    k A \sin(\omega t + \varphi) = 0
    \rightarrow 
    \\
    \omega = \sqrt{\frac{k}{m}}
    \hp
    \textrm{period = } T = \frac{2 \pi }{\omega} = 2 \pi \sqrt{\frac{m}{k}}
}

\section{velocity Dependent Force}
$\vec F = \vec F(\vec v)$

drag forces, friction, air resistance

\subsection{Types of drag forces}
\begin{enumerate}
    \item
    Stoke's drag (linear)
    \[
    \vec F = -c \vec v
    \]
    laminar flow, valid for small velocities and viscous fluids
    \item
    Newtonian Drag (nonlinear)
    \[
    \vec F(\vec v) = -k \vec v^2
    \]
    valid for larger velocities and less viscous fluids.
\end{enumerate}

The type of drag is applicable based of Reynold's Number
\[
R = \frac{\rho v L}{\mu} \hp
\textrm{ density, fluid flow, size, viscosity}
\]

\section{Linear Drag}
particle of mass $m$ and initial velocity $v_0$ in laminar flow

goal, derive velocity after a long time (should be 0)

choose coordinate system: 1D along x direction $v = \dot x$

force $F(v) = -cv$

introduce new constant $\kappa  = \frac{c}{m}$
\[
[k] = [\frac{F}{m * v}] = [\frac{1}{t}]
\]

\equations{
    F = -cv = -m \kappa v
    \\
    F = m \ddot x = m \frac{dv}{dt} = -m \kappa v
    \\
    \textrm{separation of variables}
    \\
    \frac{1}{v} dv = - \kappa dt 
    \\
    \textrm{use definite integrals, relabel variables}
    \\
    \int^v_{v_0} \frac{1}{v'} dv' = - \kappa \int^t_{t_0} dt'
    \\
    - \kappa t = \ln(v') \Big |^v_{v_0} = \ln(\frac{v}{v_0})
    \\
    v(t) = v_0 \exp(- \kappa t)
}
exponential decay with a rate of $\kappa$

For some sanity checks, you can try $t = \infty$ and $t = 0$ to 
make sure they make sense

\section{non-linear drag with gravity}
motion is 1D along the z axis 

drag force $F = -c v^2$ sgn(v)

(sgn(v) is $\pm 1$ depending on if v is greater or less than 0)

new constant 
\[
\sigma = \frac{c}{m}
\]

\equations{
    F = -mg + cv^2
    \rightarrow
    -mg + \frac{C}{m}mv^2 \rightarrow
    m(g - \sigma v^2)
    \rightarrow
    \\
    \frac{dv}{dt} = -g + \sigma v^2
    \rightarrow
    \int^v_0 \frac{1}{-g + \sigma v^2} dv = \int^t_0 dt
    \\
    t = \frac{-1}{\sqrt{g \sigma}}
    \tanh^{-1}(v \sqrt{\frac{\sigma}{g}})
    \\
    v = - \sqrt{\frac{g}{\sigma}} \tanh(\sqrt{gh} t)
    \\
    t \to \infty \hp v \to - \sqrt{\frac{g}{\sigma}}
}
You can see that $v$ asymptotically approaches a constant
velocity with large time which you can kind of see 
from the initial equation.

For most cases $m \frac{dv}{dt} = \vec F(\vec r, \vec v, t)$ is 
unsolvable. However, for some cases
$F = f(v) g(t)$, you get
$\int \frac{m}{f(v)} dv = \int g(t) dt$

For another special case $F = f(v) h(x)$, you get
\equations{
    m \frac{dv}{dt} \frac{1}{f(v)} = h(x) 
    \\
    \frac{dv}{dt} = \frac{dv}{dx} \frac{dx}{dt} = v \frac{dv}{dx}
    \\
    m v \frac{dv}{dx} \frac{1}{f(v)} = h(x)
    \rightarrow
    \int \frac{mv}{f(v)} dv = \int h(x) dx 
}

\section{Time Varying Mass $M(t)$}
$u$ is the speed of the mass being projected out.
\equations{
    P_i = M_i v_i
    \\
    P_f = dm(v - u)
    \rightarrow
    t(m_f v_f)
    =
    fm(v - u)
    \\
    P_i = P_f \Rightarrow Mv_i = -u dm + m_i v_i - dM dv 
    \\
    dM dv \approx 0 \textrm{ and cancel some stuff}
    \\
    Mdv = t u dm
    \Longrightarrow
    dv = \frac{-u}{M} dm
    \rightarrow
    \int dv = \int \frac{-u}{M} dm
    \rightarrow 
    \\
    v - v_0 = -u \ln(\frac{M}{M_0})
    \rightarrow
    v = v_0 - u \ln(\frac{M}{M_0})
}
Because its logarithmic, its really hard to boost something by 
chucking mass out the back.

Conservation of momentum is not true in gravity 
$P_i \neq P_f$

\equations{
    dp = P_f - P_i \neq 0
    \\
    dp = -Mg dt 
    -Mg \, dt = M dv + u dM 
    \\
    \frac{dv}{dt} = -g - \frac{u}{M} \frac{dM}{dt}
    \hp 
    \textrm{ THIS IS THE IMPORTANT EQUATION}
}
$u, M$, and $dM/dt$ are all controllable by the rocket

\subsection{velocity from acceleration}
\[
v(t) = v_0 -gt - u \ln(\frac{M_0 - dt}{M_0})
\]

\section{2D Rocket}
\begin{gather*}
    \vec F(\vec r, \dot \vec r, t) = m \vec a = m \ddot \vec r
\end{gather*}
Cartesian Coordinates 
\equations{
    F_x(\vec r, \dot \vec r, t) = m \ddot x
    \hp
    F_y(\vec r, \dot \vec r, t) = m \ddot y
    \hp
    F_z(\vec r, \dot \vec r, t) = m \ddot z
}
Consider a magnetic field $B e_z$
\equations{
    \vec F = q \vec v \times \vec B = q(\vec v \times \vec B)
    \rightarrow
    \\
    qB(v_y e_x - v_x e_y)
    \hp 
    F_x = qBv_y = m \ddot x
    \hp 
    F_y = -qB v_x = m \ddot y 
    \\
    m \ddot z = 0 \Longrightarrow \dot z = \textrm{const}
    \\
    qB \dot y = m \ddot x
    \hp 
    qB \dot x = m \ddot y
    \\
    qB \ddot y = m \dddot x
    \rightarrow
    qB (\frac{-qB}{m}) \dot x = m \dddot x
    \rightarrow
    \ddot v_x = - \omega^2 v_x
    \\
    \textrm{this is the spring equation}
    \\
    \ddot x = \frac{k}{m} x
    \longrightarrow
    v_x = A \sin(\omega t + \phi)
    \hp 
    v_y = A \cos(\omega t + \phi)
    \\
    \textrm{it goes in a circle}
}

\subsection{something}
\equations{
    \ddot v_x = \omega \dot v_y
    \\
    \ddot v_x = -\omega^2 v_x
    \hp 
    \ddot v_y = - \omega^2 v_y
    \\
    \dot \nu = \dot v_x + i \dot v_y
    \\
    =
    \omega v_y - i \omega v_x
    \\
    \dot \nu = -i \omega \nu 
    \\
    \textrm{and some stuff I missed}
}

\subsection{Particle in Field}
Motion of a charged particle in a homogeneous magnetic field
\equations{
\dot v_x = \omega v_y
\hp 
\dot v_y = - \omega v_x
\hp
w / \omega = \frac{qB}{m}
\mu = v_x + i v_y
\\
\dot mu = \dot v_x + i \dot v_y
\longrightarrow
\omega v_y - i \omega v_x
\rightarrow
-i \omega (v_x + i v_y)
\\
\dot \mu = -i \omega \mu
\\
\textrm{solve with separation of variables}
\\
\frac{1}{\mu} \, d \mu = - i \omega \, dt
\rightarrow
\ln(\mu) + \tilde{C} = \ln(\frac{\mu}{C})
\\
\mu = C \exp(-i \omega t)
\hp
C = A e^{i \delta}
\\
\mu = A \exp(i(\delta - \omega t))
}
Use the euler identity to keep doing math
\equations{
	\mu = A(\cos(\omega t - \delta) - i \sin(\omega t - \delta))
	\\
	\textrm{the signs reversed and im not entirely sure why or how}
	\\
	v_x = \Re(\mu) = A \cos(\omega t - \delta)
	\hp
	v_y = \Im(\mu) = -A \sin(\omega t - \delta)
    \\
    |v^2| = A^2 \cos(\omega t - \delta) + A_2 \sin(\omega t - \delta)
    = A^2
    \\
    \textrm{double check this is true by 
    showing derivative of $v$ is 0}
    \\
    \frac{1}{2} \frac{d}{dt}|v|^2 = 
    \frac{1}{2}m 2v \cdot \frac{d}{dt}v
    \\
    = v \cdot q(v \times B) = 0
    \\
    = 0 \hp A = \textrm{constant}
}

Now you figure out the trajectory
\equations{
    \vec v = \frac{dr}{dt} \rightarrow v \, dt = dr 
    \\
    x(t) = \int v_x \, dt = \int A \cos(\omega t - \delta) \, dt 
    =
    \frac{A}{\omega} \sin(\omega t - \delta) + C_y
    \\
    z(t) = \int v_z \, dt = \int v_{z0} \, dt = 
    v_{z0} t + z_0
    \\
    A = \sqrt{v_x^2 + v_y^2}
}

\chapter{Curvilinear Coordinates}
non-Cartesian coordinates 

\subsection{Examples}
2d polar coordinates $(r, \theta)$
\[
x = r \cos(\theta(t)) \hp
y = r \sin(\theta(t))
\hp 
r = \sqrt{x^2 + y^2}
\]

3d cylindrical coordinates $(r, \phi, z)$
\[
x = r \cos(\phi) \hp y = r \sin(\phi) \hp z = z
\]

3d spherical coordinates $(r, \theta, \phi)$
\[
x = r \cos(\phi) \sin(\theta)
\hp 
y = r \sin(\phi) \sin(\theta)
\hp 
z = r \cos(\theta) 
\]

\section{Whirling Stick}
A rigid stick whriling with fixed $\omega$

use polar coordinates to have the easiest math 
\equations{
    \omega = \varphi t = \textrm{const}
    \\
    e_r = e_r (t) \hp e_{\varphi} = e_{\varphi} (t)
    \\
    \textrm{relate $r$ and $\varphi$ to $x$ and $y$ basis vectors}
    \\
    e_r(t) = \cos(\varphi(t)) e_x + \sin(\varphi(t))e_y
    \\
    e_{\varphi}(t) = -\sin(\varphi(t)) e_x + \cos(\varphi(t)) e_y
    \\
    \textrm{position vector } r(t) = r(t) e_r(t)
}

Determine the first derivatives
\equations{
    \dot e_r = \frac{d}{dt}\cos(\varphi(t)) e_x + \sin(\varphi(t))e_y
    \rightarrow
    \\
    \frac{d}{dt}\cos(\varphi(t)) e_x + 
    \cos(\varphi) \frac{d}{dt}e_x + 
    \frac{d}{dt} \sin(\varphi(t))e_y + 
    \sin(\varphi(t)) \frac{d}{dt} e_y
    \\
    \frac{d}{dt}\cos(\varphi(t)) e_x + 
    \frac{d}{dt} \sin(\varphi(t))e_y 
    \\
    \dot e_r = - \dot \varphi \sin(\varphi) e_x + 
    \dot \varphi \cos(\varphi) e_y = \dot \varphi e_\varphi
    \\
    \cdot e_{\varphi} = - \dot \varphi e_{\varphi}
    \\
    v(t) = \dot r(t) e_r + r \dot \varphi e_\varphi
    =
    v_r e_r + v_\varphi e_\varphi 
}

now acceleration lol
\equations{
    a = \frac{dv}{dt} = 
    (\ddot r - r \dot \varphi^2) e_r + 
    (r \ddot \varphi + 2 \dot r \dot \varphi) e_{\varphi}
    \\
    \varphi = \omega t \hp r \dot \varphi^2  = r \omega^2
    \\
    \textrm{centripetal force}
}

\section{Bead on a Whirling Rod}
Use polar coordinates for sake of convenience. 
Use basis vectors $e_r$ and $e_\phi$

Rod whirls at a rate of $\omega$, so $\phi(t) = \omega t$

Our general strategy is draw a sketch and then figure out 
your coordinates and then write out position, velocity, and 
acceleration vectors.

Used Newton's 2nd law to to get a differential equation.

Solve the differential equation.
\equations{
    a(t) = [\ddot r - r \dot \phi^2] \vec e_r + 
    [r \ddot \phi + 2 \dot r \dot \phi] \vec e_\phi 
    \\
    F = ma 
    \\
    \textrm{no force in the radial direction because 
    it's all normal force}
    \\
    F_{net} = F_n = F_n(t) e_\phi(t) + 0 e_r(t)
    \\
    F_n e_\phi (t) = m [\ddot r - r \dot \phi^2] \vec e_r + 
    m[r \ddot \phi + 2 \dot r \dot \phi] \vec e_\phi 
    \\
    e_r = 0 = m (\ddot r - r \dot \phi^2)
    \hp 
    e_\phi = F_n = m (r \ddot \phi + 2 \dot r \dot \phi )
    \\
    0 = \ddot r - r \dot \phi^2 \rightarrow
    \ddot r = r \dot \phi^2 \hp 
    \phi(t) = \omega t, \dot \phi(t) = \omega
    \\
    \ddot r = \omega^2 r
    \hp 
    \textrm{use an ansatz/guess}
    \\
    r = e^{\lambda t} \hp 
    \dot r = \lambda e^{\lambda t} \hp 
    \ddot r = \lambda^2 e^{\lambda t}
    \\
    \lambda^2 e^{\lambda t} = \omega^2 e^{\lambda t}
    \hp \lambda^2 = \omega^2
    \hp \lambda = \pm \omega 
    \\
    r(t) = Ae^{\omega t} + B e^{- \omega t}
    \hp \textrm{solve with initial condition}
    \\
    \dot r(t) = \omega A e^{\omega t} - \omega B e^{\omega t}
    \\
    r(t = 0) = r_0 
    \hp v(t = 0) = v_0
    \\
    A + B = r_0
    \hp A - B = 0
    \rightarrow 
    A = B
    \hp A = B = \frac{r_0}{2}
    \\
    r(t) = \frac{r_0}{2} 
    (e^{\omega t} + e^{-\omega t})
    =
    \frac{r_0}{2} \cosh(\omega t)
    \\
    F_n = m (r \ddot \phi + 2 \dot r \dot \phi)
    \hp \phi = \omega t
    \\
    F_n = m2 \dot r \omega 
    = m \omega^2 r_0 \sinh(\omega t) 
}

\subsection{Bead on Spinning Loop}
Loop of fixed radius R spinning about vertical axis at fixed rate 
$\Omega$. Bead of mass $m$, free to move along the loop. Everything 
is in a gravitational field. 

Use spherical coordinates for math convenience. 

\equations{
    x = r\sin(\theta) \cos(\phi)
    \hp
    y = r\sin(\theta) \sin(\phi)
    \hp 
    z = r \cos(\theta)
    \\
    r^2 = x^2 + y^2 + z^2
    \hp 
    \tan(\theta) = \frac{\sqrt{x^2 + y^2}}{x^2}
    \\
    \textrm{the coordinate vectors are real difficult to find}
    \\
    \textrm{coords, position, velocity, acceleration}
    \\
    e_r = \sin(\theta) \cos(\phi) e_x + 
    \sin(\theta) \sin(\phi) e_y + 
    \cos(\theta) e_z
    \\
    e_\theta = \cos(\theta) \cos(\phi) e_x + 
    \cos(\theta) \sin(\phi) e_y -
    \sin(\theta) e_z
    \\
    e_\phi = - \sin \phi e_x + \cos(\phi) e_y
    \\
    r(t) = R = \textrm{const}
    \hp \phi(t) = \Omega t \hp \Omega = \textrm{const}
    \hp
    \theta(t)
    \\
    \dot e_r = \textrm{blah}
    \dot e_\theta = \textrm{blah}
    \dot e_\phi = \textrm{blah}
    \\
    r(t) = r(t) e_r(t)
    \hp v(t) = \frac{d}{dt} \left( r(t) e_r(t) \right)
    =
    \dot r e_r + r \dot e_r 
    =
    \\
    \dot r e_r + r \left( 
    \dot \theta e_\theta + \dot \phi \sin(\theta) e_\phi
    \right)
    =
    \dot r e_r + r \dot \theta e_\theta + 
    r \sin(\theta) \dot \phi e_\phi
    \\
    a = \frac{dv}{dt} = \frac{d}{dt}
    \dot r e_r + r \dot \theta e_\theta + 
    r \sin(\theta) \dot \phi e_\phi
    =
    \\
    \ddot r e_r + \dot r \dot e_r + 
    \dot r (\dot \theta e_\theta) + r(\ddot \theta e_\theta + 
    \dot \theta \dot e_\theta) + \\
    \dot r (\sin(\theta) \dot \phi e_\phi) +
    r (
        \cos(\theta) \dot \phi e_\phi + 
        \sin(\theta)(\ddot \phi e_\phi + \dot \phi \dot e_\phi)
    )
    \\
    \textrm{what insanity}
    \textrm{plug in all the coordinate vectors to the thingy}
    \\
    \vec a(t) = 
    [\ddot r - r \dot \theta^2 - 
    r \sin^2 \theta \dot \phi^2] e_r 
    +
    r[2 \dot r \dot \theta + r \ddot \theta  - 
    r \sin(\theta) \cos(\theta) \dot \phi^2] e_\theta 
    + \\
    [2\dot r \sin(\theta) \dot \phi + 
    2 r \cos(\theta) \dot \theta \dot \phi +
    r \sin \theta \ddot \phi] e_\phi
}

\equations{
    \textrm{now use $F = ma$ to do some bullshit}
    \\
    F = ma \hp F_g = -mg e_z = 
    -mg (\cos \theta e_r - \sin(\theta) e_\theta )
    \\
    F_n = N_r e_r + N_\phi e_\phi
    \\
    F_{net} = F_G + F_n = 
    F_n = N_r e_r + N_\phi e_\phi
    -mg (\cos \theta e_r - \sin(\theta) e_\theta )
    = \\
    [N_r - mg \cos(\theta)] e_r +
    mg \sin (\theta) e_\theta + 
    N_\phi e_\phi
    \\
    F_\theta = m a_\theta
    \\
    mg \sin(\theta) = \\
    [\ddot r - r \dot \theta^2 - 
    r \sin^2 \theta \dot \phi^2] e_r 
    +
    r[2 \dot r \dot \theta + r \ddot \theta  - 
    r \sin(\theta) \cos(\theta) \dot \phi^2] e_\theta 
    + \\
    [2\dot r \sin(\theta) \dot \phi + 
    2 r \cos(\theta) \dot \theta \dot \phi +
    r \sin \theta \ddot \phi] e_\phi
    \\
    g \sin(\theta) = r \ddot \theta -
    R \sin(\theta) \cos (\theta) \Omega^2
}

\chapter{Types of Forces}

\subsection{Conservative Forces in 2d and 3d}

\subsection{Conservative Force}
a force $\vec F( \vec r)$ is conservative if it can be written 
as the gradient of a potential $\vec U(\vec r)$.
\[
\vec F(\vec r) = - \vec \nabla U(\vec r)
\]

\subsection{$\nabla$ operator}
\equations{
    \vec \nabla =
    \vec e_x \frac{d}{dx} +
    \vec e_y \frac{d}{dy} +
    \vec e_z \frac{d}{dz}
    \\
    \textrm{if you have another coordinate system}
    \\
    \vec \nabla =
    \vec e_r \frac{d}{dr} +
    \vec e_\phi \frac{d}{d\phi} +
    \vec e_z \frac{d}{dz}
}

\subsection{Curl of Conservative Force}
\[
\nabla \times \vec F(\vec r) = 
\vec \nabla \times (- \nabla U) = 0
\]

NOT every force is conservative. 

\equations{
    F = F_{cons} + F_{diss} 
    \\
    \nabla \times F = \nabla \times F_{cons} +
    \nabla \times F_{diss} \neq 0
}

If $\nabla \times F = 0$, then the forces can be written 
as $F = - \nabla U$ (has a potential) and $\vec F$ is converative.
(useful to check)

\subsection{Work along a path}
\equations{
    w = \int_{p_2}^{p_1} \vec F \cdot d \vec r
    = - \int_{p_2}^{p_1} \vec \nabla U \, dr 
    \hp \nabla \approx \frac{d}{dr}
    \\
    - \int_{p_2}^{p_1} dU = -(U(p_2) - U(p_1))
}

If path is closed $-(U(p_1) - U(p_1)) = 0$


A conservative force does NO work along a closed path. 

\subsection{Energy Conservation} 
If a force is conservative, $F = - \nabla U$, then the total 
energy $\vec E = T + U =$ const. (constant of motion)

How to show that this is true?
\equations{
    \frac{d}{dt} E = \frac{d}{dt} (T + U) = 0 = E = \textrm{const}
    \\
    \frac{d}{dt} T = \frac{d}{dt} \frac{1}{2}mv^2 = mv \cdot \frac{dv}{dt} = \vec v \cdot \vec F
    \\
    \textrm{2nd term}
    \\
    \frac{d}{dt} U(t, \vec r(t)) =
    \frac{d}{dt} U + \frac{dx}{dt} \frac{dU}{dx} + 
    \frac{dy}{dt} \frac{dU}{dy} + \frac{dz}{dt} \frac{dU}{dz}
    \\
    \frac{d}{dt} U(t, r(t)) = \frac{dU}{dt} + \frac{dr}{dt} \cdot \nabla U
    = - \vec v \cdot \vec F + \frac{dU}{dt}
    \\
    \frac{dE}{dt} = 
    \frac{dT}{dt} + \frac{dU}{dt} = \vec v \cdot \vec F - 
    \vec v \cdot \vec F + \frac{dU}{dt} = 0
}

$E = T + U =$ const is conserved, if $\vec F$ is conservative, and 
$U = U(\vec r)$ (ie, no explicit time dependence)

\subsection{How to Find Potential}
given some force $F$

option a: show this with indefinite integrals
\equations{
	\vec F = - \vec \nabla U
}


option b: use indefinite integrals
\equations{
	F = - \nabla U 
	\rightarrow U = \int^{p_2}_{p_1} = -\int^{p_2}_{p_1} \vec F \cdot \, dr
}

\subsection{example}
\[
	\vec F = (2xy + 1) \vec e_x + 
	(x^2 + 2) \vec e_y
\]
find $U$

\equations{
	-U = \int \vec F \cdot d \vec r = 
	\int (F_x dx + F_y dy + F_z dz)
	\\
	\int^x_0 F_x(x', 0, 0) dx + \int^y_0 F_y(x, y', 0) dy + \int^z_0 F_z(x,y, z') dz
	\\
	\int^x_0 1 dx' + \int^y_0 (x^2 + 2) dy' + \int^z_0 0 dz'
	\rightarrow
	\\
	-U = x + (x^2 + 2)y + 0 =
	-x - x^2y - 2y
}

\section{Central Forces}
force that points towards or away from a point radially and is dependent on distance $r$
\[
	F = f(r, \theta, \phi) \vec e_r
\]

\subsection{Conservation of Angular Momentum}
Define angular momentum $\vec L$ with respect to a reference point $\mathrm{o}$

\subsection{YOUR LAPTOP DIED FIGURE OUT LATER}
I still don't get anything about orbits or angular momentum of anything like that.

You missed the last half of the friday lecture on like september 20th ish

\section{Conservative Central Forces}
\[
\vec L = \vec r \times m \vec v
\]
If you have a conservative central force, you have conservation of angular momentum.

\equations{
    L = \textrm{const}
    \\
    \textrm{to prove, take the derivative with respect to time}
    \\
    \frac{dL}{dt} = \frac{d}{dt} \left( \vec r \times m \vec v \right)
    =
    \dot \vec r \times m \vec v + \vec r \times m \frac{dv}{dt}
    \\
    \left( m \frac{dv}{dt} = \vec F \right)
    \hp \dot \vec r \times m \vec v = \vec v \times m \vec v = 0
    \\
    F = f(r, \theta, \phi) \vec e_r \hp \vec r = r \vec e_r
    \\
    =
    r \vec e_r \times m f(r, \theta, \phi) \vec e_r = 0
    \textrm{ because the unit vectors are the same}
}

\subsection{useful math}
\equations{
    f(r, \theta, \phi) \hp 
    e_r \times e_\phi = - \vec e_\theta 
    \hp
    e_\theta = - e_z
}


\subsection{Consequences}
\begin{itemize}
    \item
Motion remains in $\vec r - \vec v$ plane 
\item
can rotate coords s.t. $\theta = \pi/2 \hp \dot \theta = 0$
\item 
consider 
\equations{
    \vec r = r(t) \vec e_r \hp v = 
    \dot r \vec e_r + 
    r \dot \theta e_\theta + 
    r \dot \phi \sin(\theta) e_\phi
    =
    \dot r \vec e_r + 
    r \dot \phi e_\phi
    \\
    \vec L = \vec r \times m \vec v = m r \vec e_r \times 
    ( \dot r \vec e_r + r \dot \phi e_\phi) =
    \\
    mr \vec e_r \times \dot r \vec e_r \approx \vec e_r \times e_r = 0
    \\
    \vec L = - mr^2 \dot \phi \vec e_\theta
    \hp
    L \vec e_z = + mr^2 \dot \phi \vec e_z
    \\
    \textrm{magnitude } L = m r^2 \dot \phi 
    \\
    \textrm{ direction in $e_z$} 
    \\
    \textrm{transformed our 3d equation to 2d because of }
    \\
    \textrm{conservation of angular momentum}
}
\end{itemize}

\subsection{Conservative, Central Force}
\equations{
    \vec F = f \vec e_r = - \nabla U
    \\
    \textrm{no dependence in $\vec e_\phi, \vec e_\theta$}
    \\
    \frac{dU}{d\theta} = 0 \hp \frac{dU}{d\phi} = 0
    \\
    \vec F = - \frac{dU}{dr} \vec e_r \rightarrow 
    U = - \int f \, dr 
}

Examples of Conservative, Central Forces:
gravity, coulomb potential, Yukawa potential (nuclear force)

\subsection{IMPORTANT}
\[
L =  m r^2 \dot \phi 
\]


\subsection{back to example}
\equations{
    \vec F = - \frac{K}{r^2} \vec e_r \hp K > 0
    \\
    U = - \int -\frac{K}{r^2} \, dr = \frac{-K}{r} (\lim_{r \to \infty} = 0)
    \\
    L = \textrm{const}
    \\
    \textrm{choose $L = L e_z, \theta = \pi/2, \dot \theta = 0$}
    \\
    \dot \phi = \frac{L}{mr^2}
    \hp 
    E = T + U = \const
    \\
    T = \frac{1}{2} m |\vec v|^2 \hp 
    \vec v = \dot r \vec e_r + r \dot \phi \vec e_\phi
    \\
    T = \frac{1}{2} m (\dot r \vec e_r + r \dot \phi \vec e_\phi)
    =
    \frac{1}{2}m \dot r^2 + \frac{1}{2} m \left(\frac{r^2 L^2}{m^2 r^4} \right)
    =
    \frac{1}{2} m \dot r^2 + \frac{L^2}{2 m r^2} 
    \\
    E = T + U
    =
    \frac{1}{2} m \dot r^2 + \frac{L^2}{2 m r^2} - \frac{K}{r}
    \\
    T = \frac{1}{2} m \dot r^2
    \hp 
    U_{eff} = \frac{L^2}{2 m r^2} - \frac{K}{r}
    \\
    \textrm{EFFECTIVE POTENTIAL IS IMPORTANT} 
    \\
    U_{eff} = \frac{L^2}{2 m r^2} + U
}

\section{Analyze Effective Potential}
\equations{
    U_{eff} = \frac{L^2}{2 m r^2} - \frac{K}{r}
    \hp K > 0
}

You can circular, elliptical, parabolic, and hyperbolic orbits. 

\subsection{Case: $E > 0$}


$\frac{L^2}{2 m r^2}$

particle moves towards $r = 0$ and then has a turning point where $\frac{dr}{dt} = 0$. The particle then returns back to $r \to \infty$

This is a hyperbolic trajectory. 


\subsection{Case: $E < 0$}
$\frac{K}{r} > (\frac{L^2}{2 m r^2} + T_r)$

$E$ is stuck in a potential well and will oscillate between turning points in the potential. 

elliptical orbit. 

\subsection{Case: $E < 0$ and $U_{eff}' \Big|_{r = r_c} = 0$}
$U_{eff}' = \frac{dU_{eff}}{dr}$

orbit is circular with radius $r_c$

\equations{
    r = r_c = \const \hp 
    T_r = \frac{1}{2} m \dot r^2 = 0
    \\
    \textrm{because the radius never changes.}
}

$E = 0$ is a parabolic orbit but eh.


\chapter{Gravity}
It's a pretty important thing in our day to day life. 

Need to think about both gravitational force and potential. 

2 masses a distance $r$ apart experience a force in the $\vec e_r$ direction
\[
\vec F_G = - \frac{G m_1 m_2}{r^2} \vec e_r
\hp G = 6.67 * 10^{-11} \frac{N m^2}{kg^2}
\]

consider extended mass $m_1$
\equations{
    m_1 = \int dm_1 = \int^{V_1} \rho_1 (\vec r_1) d \vec r_1
    \\
    F_G = -Gm_2 \int \frac{\rho_1 (r_1)}{|r_2 - r_1|^3}(r_2 - r_1) dr_1
    \\
    \vec g = \frac{\vec F_G}{m_2} = 
    -G \int \rho (\vec r_1) 
    \frac{(\vec r_2 - \vec r_1)}{|\vec r_2 - \vec r_1|^3} d^3 r
    \\
    \frac{1}{|r_2 - r_1|^2} \cdot \frac{\vec r_2 - \vec r_1}{|\vec r_2 - \vec r_1|}
    \\
    \frac{1}{|r_2 - r_1|^2} \equiv "\frac{1}{r^2}"
    \hp  \frac{\vec r_2 - \vec r_1}{|\vec r_2 - \vec r_1|} = \vec e_r
}

\section{Gravitational Potential}
\equations{
    \vec g = - \nabla \Phi 
    \\
    \textrm{analogous to $U = m_2 \Phi$, $\vec F_G = - \nabla U$}
    \\
    U = \frac{G m_1 m_2}{r}
    \hp 
    \Phi = \frac{U}{m_2} = \frac{-G m_1}{r}
    \\
    \Phi = -G \int \limits_{V} \frac{\rho (\vec r_1)}{|\vec r_2 - \vec r_1|} d^3 r
}

\subsection{$\Phi$ of a Spherical Shell}
Setup: given a uniform spherical shell with mass $M$, radius $R$, thickness $h$, mass density $\phi = H/V = M / (4 \pi R^2 h)$.

Compute $\Phi$

What we do is make an arbitrary point $p$ in the mass such that the distance from that point to $m_2$ is $\vec r$ and for some point $R$ away from $p$, we can see the distance from that point to $m_2$ is $\vec r - \vec r_1$.
We also make a function $s$ such that $s(\theta) = |\vec r - \vec r_1|$

the thickness is very small so $h = dr$
Strategy:
\[
\Phi = \int d \Phi \hp d \Phi = \frac{-G}{|\vec r - \vec r_1|} dm
\]

\equations{
    dm(\vec r_1) = \rho (\vec r_1) d^3 \vec r_1 = \rho R^2 dr \sin (\theta) d \theta d \phi 
    \\
    R^2 dr = \textrm{radius} \hp \sin (\theta) d \theta d \phi = \textrm{Surface of sphere}
    \\
    d \Phi = \frac{G dm}{|\vec r - \vec r_1|} = \frac{-G}{s(\theta)} dm 
}



\equations{
    \Phi = \int d \Phi = - G \int \frac{dm}{s(\theta)}
    =
    -G \rho h R^2 \int^\pi_{\theta = 0} \int^{2 \pi}_{\phi = 0} \frac{\sin (\theta)}{s(\theta)} d \theta d \phi  
    =
    \\
    -2 \pi G h \rho R^2 \int^{\pi}_{0} \frac{\sin(\theta)}{s(\theta)} d \theta
    \\
    \textrm{solve integral (substitute $\int d \theta \to \int ds)$}
    \\
    s(\theta) = |\vec r - \vec r_1| 
    \hp 
    s^2(\theta) = |\vec r - \vec r_1|^2 = {r}^2 - 2 \vec r \cdot \vec r_1 + \vec r_1^2
    =
}
okay so how $s$ works is $\vec r$ is a vector at the radius of the sphere that moves constantly in order for you to integrate over the whole sphere and $\vec r_1$ is a vector that point solely to $x$ which is the point that you are trying to find the potential at. 
\equations{
    x^2 + R^2 - 2R x \cos(\theta)
    \\
    \textrm{why is that $\theta$ the same as spherical coords $\theta$ (angle from $\vec z$ line)}
    \\
    \frac{d (s^2)}{d \theta} = 2 s \frac{ds}{d\theta} = 2 R x \sin (\theta)
    \\
    \frac{\sin (\theta)}{s} \, d \theta = \frac{1}{Rx} ds 
    \\
    \int^\pi_0 \frac{\sin (\theta)}{s} \, d \theta = 
    \int^{s_{max}}_{s_{min}}\frac{1}{Rx} ds 
    \\
    s_{min} = s(\theta = 0) = \pm (R - x) > 0
    \\
    s_{max} = s(\theta = \pi) = (R + x) > 0
    \\
    \Phi = - 2 \pi G \frac{\rho h R}{x} (s_{max} - s_{min})
}

\subsection{case: $m_2$ outside of mass}
\equations{
    s_{min} = -(R - x) = x - R 
    \hp
    s_{max} = (R + x) 
    \\
    \Phi = - 4 \pi G \frac{\rho h R^2}{x} = 
    \frac{-GM}{x}
}
Outside of an extended mass, the gravitational potential is the same as $\Phi$ of a point mass with the same $M$

\subsection{case: $m_2$ inside of mass}
\equations{
    s_{min} = (R - x) 
    \hp
    s_{max} = (R + x)
    \\
    \Phi = - 4 \pi G \rho h R = 
    \frac{-GM}{R} = \const
}

\section{$\Phi$ of a uniform sphere}
Radius $R$, mass $M$, constant mass density $\rho = \begin{cases} \rho^* , r < R \\ 0 , r > R \end{cases}$

Find $\Phi = \Phi (x)$

Strategy: Use infinitely many thin shells 

\equations{
    d \phi = 
    \begin{cases}
    -4 \pi G \frac{\rho r^2}{x} dr \hp x > r
    \\
    -4 \pi G \rho r dr \hp x < r
    \end{cases}
    \\
    \Phi = \int d \phi = -4 \pi G \int^R_{0} \frac{\rho r^2}{x} dr 
    =
    -4 \pi G \int^R_{0} \frac{\rho^* r^2}{x} dr
    =
    \frac{- GM}{x}
}

\subsection{inside sphere}
\equations{
    \Phi = \int^{x}_{0} d \Phi_{inner} + \int^{R}_{x} d \Phi_{out}
    =
    \\
    -4 \pi G \frac{\rho^*}{x} \int^x_0 r^2 dr -
    -4 \pi G \rho^* \int^R_x r dr 
    =
    \\
    \frac{1}{2} GM \frac{3 R^2 - x^2}{R^3}
}

\section{Gravitational Orbits}
\subsection{Kepler's Laws}
\begin{itemize}
    \item 
    Everything orbits in planar ellipses
    \item 
    the line between the sun and a planet sweeps equal areas in equal times 
    \item 
    The orbital period around the sun is proportional to the 3/2 power of the semimajor axis
\end{itemize}

\equations{
    L = m r^2 \dot \varphi \textrm{ conservation of angular momentum}
    \\
    \textrm{per unit mass } l = \frac{L}{m} = r^2 \dot \varphi = \const 
    \\
    \textrm{energy (per unit mass) conservation}
    \\
    \varepsilon = \frac{E}{m}
    =
    \frac{1}{m} (T + U_{grav})
    \hp 
    U_{grav} = - \frac{GMm}{r} 
    \\
    \varepsilon = \frac{1}{2} \vec v^2 - \frac{GM}{r}
    \\
    \vec v = \dot r \vec e_r + r \dot \varphi e_{\varphi}
    \hp 
    v^2 = \vec v \cdot \vec v = \dot r^2 + r^2 \dot \varphi^2
    \\
    \varepsilon = \frac{1}{2} \dot r^2 + \frac{l^2}{2r^2} - \frac{GM}{r}
    =
    T + U_{eff}
}

Now I should be able to write out the orbit as a 1d equation. 

perigee $r_p$ is the minimal distance from $M$

apogee $r_a$ is the maximal distance from $M$

velocity vectors $v_p$ and $v_a$ perpendicular to the eerigee and apogee 

$v_p \cdot v_a = 0$ and the angle between $r_p$ and $r_a$ is $\pi / 2$

angular momentum 
\equations{
    \vec l = \vec r \times \vec v \hp l = |\vec r| |\vec v| \sin(\alpha)
    \\
    \textrm{in perigee and apogee, $\sin(\alpha) = 1$}
    \\
    l = r_p v_p = r_a v_a = \const
}

\subsection{Energy (per unit mass)}
\equations{
    \varepsilon = \frac{l^2}{2 r_{p, a}^2} - \frac{GM}{r_{p, a}}
    \\
    = \frac{1}{2} v_{p, a} - \frac{GM}{r}
    \\
    \varepsilon = 
    \frac{1}{2} v_{p, a}^2 - \frac{GM}{l} v_{p, a}
    \\
    v_{p, a} = \frac{GM}{l} \pm \sqrt{\left(\frac{GM}{l}\right)^2 + 2 \varepsilon}
    \\
    \textrm{$+$ is perigee and $-$ is apogee}
    \\
    r_{p, a} = \frac{l}{v_{p, a}} = 
    l \left( \frac{GM}{l} \pm \sqrt{\left( \frac{GM}{l}\right)^2 + 2 \varepsilon}\right)^{-1}
}

\section{Effective Potential and Orbits}
type of orbit depends on the sign of the total energy of the particle 
\[
\varepsilon = \frac{1}{2} \dot r^2 + U_{eff} 
\hp 
U_{eff} = \frac{l^2}{2r^2} - \frac{GM}{r}
\]

\subsection{case: $\varepsilon < 0$ and $U_{eff}' = 0$}
Circular orbit. Absolute minimum in effective potential 

if $U_{eff}'' > 0$
Find speed 

\equations{
    -m \frac{v^2}{r} \vec e_r = -ma_c \vec e_r = m \vec a = - \frac{GMm}{r^2} \vec e_r
    \\
    v = \sqrt{\frac{GM}{r}}
}

Find orbital period 
\equations{
    p = \frac{2 \pi}{\omega} = \frac{2 \pi r}{v} = 
    2 \pi \sqrt{\frac{r^3}{GM}}
}

Angular momentum per unit mass 
\equations{
    l = |\vec r||\vec v| \sin(\alpha) = rv 
    \\
    v = \frac{GM}{l} \hp r = \frac{l^2}{GM}
}

energy 
\equations{
    \varepsilon = \frac{1}{2} v^2 - \frac{GM}{r} \hp v^2 = \frac{GM}{r}
    \\
    \varepsilon = - \frac{1}{2} v^2 = - \frac{1}{2} \left( \frac{GM}{l^2} \right)^2
    < 0
}

escape velocity (velocity such that $E = 0$)
\[
v_{esc} = \sqrt{\frac{2GM}{r}}
\]

\subsection{case: Elliptic Orbit}
\[
- \frac{1}{2} \left(\frac{GM}{l}\right)^2 = \varepsilon_c < \varepsilon < 0
\]

particle moves between perigee and apogee determined by $(\varepsilon, l)$

\subsection{case: $\varepsilon = 0$}
parabolic orbit 

perigee $v_p = \frac{2GM}{l}$

$r_p = \frac{l}{v_p} = \frac{l^2}{2GM}$

at the "apogee", $v_a \to 0$, $r_a = \frac{l}{v_a} \to \infty$

marginally unbound orbit 

\subsection{case: $\varepsilon > 0$}
hypoerbolic orbit 

\equations{
    v_{p, a} = \frac{GM}{l} \pm \sqrt{\left( \frac{GM}{l} \right)^2 + 2 \varepsilon}
}
$v > 0, r > 0$ hyperbola $v < 0$ unphysical speed 


\[
    v_{p, a} = \frac{GM}{l} \pm \sqrt{\left( \frac{GM}{l} \right)^2 + 2 \varepsilon}
\]

if $\varepsilon > 0$ then it has to be plus 

if $\varepsilon = 0$ then $v_p = \frac{2GM}{l}, v_a = 0$

if $\varepsilon < 0$ then you can have $+$ or $-$



\section{2 body problem}
\equations{
    F = m_1 \ddot r_1 = - \frac{Gm_1 m_2}{|\vec r_2 - \vec r_1|^2} 
    \frac{(\vec r_1 - \vec r_2)}{|\vec r_1 - \vec r_2|}
    \\
    F = m_2 \ddot r_1 = - \frac{Gm_1 m_2}{|\vec r_2 - \vec r_1|^2} 
    \frac{(\vec r_2 - \vec r_1)}{|\vec r_1 - \vec r_2|}
}

where $(\vec r_1 - \vec r_2)$ is the distance between the two masses. 

Instead of dealing with 6 differential equations. We can use the center of mass frame to treat the 2 body problem as a 1 body problem. 

1 mass of total mass $M = m_1 + m_2$

1 orbiting object of reduced mass $\mu = \frac{m_1 * m_2}{M}$

Center of mass 
\[
\vec C = \frac{m_1 \vec r_1 + m_2 \vec r_2}{M}
\]

Check that there is not external force so this reference frame is inertial 
\equations{
    F_{net} = ma + ma = m_1 \ddot r_1 + m_2 \ddot r_2 = 
    \\
    \frac{Gm_1 m_2}{|r_1 - r_2|^2} \frac{\vec r_2 - \vec r_1}{|r_2 - r_1|} + 
    \frac{Gm_1 m_2}{|r_1 - r_2|^2} \frac{\vec r_1 - \vec r_2}{|r_2 - r_1|}
    =
    0
}

\subsection{Equations of Motion}
\equations{
    \ddot \vec r = \frac{GM}{|r|^2} \frac{\vec r}{|r|}
}

The relative position vector $\vec r$ is governed by the same equations of motion 
as a test mass $\mu$ in the gravitational field of $M$.

\subsection{Energy}
\equations{
    E = T + U =
    \frac{1}{2} m_1 \dot r_1^2 + \frac{1}{2} m_2 \dot r_2^2 - 
    \frac{G m_1 m_2}{|r_2 - r_1|}
    \\
    \rightarrow 
    \\
    E = 
    \frac{1}{2} \mu \dot r^2 - \frac{GM \mu}{|\vec r|} + 
    \frac{1}{2} M \dot {\vec C}^2
    \\
    \frac{1}{2} M \dot {\vec C}^2 = \const
    \hp
    \frac{1}{2} \mu \dot r^2 - \frac{GM \mu}{|\vec r|} = \const
}

\subsection{Trajectory}
Choose Spherical coordinates $(r, \theta, \varphi)$

For a central force, angular momentum is conserved. $\dot L = 0$

Choose $\vec L = L \vec e_z$ so that $\theta = \frac{\pi}{2}, \dot \theta = 0$

$\vec v = \dot r \vec e_r + r \dot \varphi \vec e_\varphi$

\equations{
    l = \frac{L}{\mu} = |\frac{1}{\mu} r \times \mu \vec v| = \const
    \ldots = r^2 \dot \varphi 
    \\
    \dot \varphi = \frac{l}{r^2}
    \\
    T = \frac{1}{2} m {\vec v}^2 = \frac{1}{2} \mu (\dot r^2 + r^2 \dot \varphi)
    =
    \frac{1}{2} \mu (\dot r^2 + \frac{l^2}{r^2})
    \\
    \varepsilon = \frac{E}{\mu} = \frac{1}{2} \dot r^2 + U_{eff} = 
    \frac{1}{2} \dot r^2 + \frac{l^2}{2r^2} - \frac{GM}{r}
    \\
    \ldots 
    \\
    - \dot r = \sqrt{2 (\varepsilon - U_{eff})} = \frac{dr}{dt}
    \rightarrow 
    \int dt = \int \frac{dr}{\sqrt{2 (\varepsilon - U_{eff})}}
    \\
    \textrm{calculate and invert to get $r(t)$}
}

\subsection{Kepler's Orbits $r(\varphi)$}
Goal: find orbits parameterized by $r(\varphi)$. 

Introduce a new variable that's the inverse of $r$ and then write down Newton's Second Law (N2L) to get a new differential equation and then solve it. 

\equations{
    u = \frac{1}{r}
    \\
    e_r: \ddot r - \frac{l^2}{r^3} = - \frac{GM}{r^2}
    \\
    e_\varphi : \frac{1}{r} \frac{d}{dt} (r^2 \dot \varphi) = 0
}
Derivation is left as an exercise to the reader (fucking killing myself)

Now we have to relate the time derivatives of $r$ and $u$

\equations{
    \dot r = \frac{d}{dt} \frac{1}{u} = 
    -u^{-2} \dot u = -u^{-2} \frac{du}{d \varphi } \frac{d \varphi}{dt}
    = \ldots =
    -l u'
    \\
    \ddot r = - l^2 u^2 u''
    \\
    \textrm{insert $e_r$}
    \\
    \frac{d^2 u}{d \varphi^2} + u = \frac{GM}{l^2}
    \\
    \textrm{solution}
    \\
    u(\varphi) = D \cos(\varphi - \varphi_0) + \frac{GM}{l^2}
}

choose coords such that $\varphi_0 = 0$

Get $D$ from $\varepsilon = \const$ 
\[
D = \sqrt{\frac{2 \varepsilon}{l^2} + \left(\frac{GM}{l^2}\right)^2}
\]

put $D$ into $u$ and then invert to get $1/u = r$

$\epsilon$ and $\varepsilon$ are different things 
\equations{
    r =
    \frac{\alpha}{1 + \epsilon \cos (\varphi)}
    \\
    \alpha = \frac{l^2}{GM} \textrm{ latus rectum}
    \hp
    \epsilon = \sqrt{1 + \frac{2 \varepsilon l^2}{G^2M^2}}
    \textrm{ eccentricity}
    \\
    r_{perigee} = 
    \frac{\alpha}{1 + \epsilon }
    \hp
    r_{apogee} = 
    \frac{\alpha}{1 - \epsilon }
}

\subsection{Cases: $\epsilon > 1$}
the eccentricity is greater than 1 so we get a hyperbola 

the critical angle is that such $1 + \epsilon \cos \varphi = 0$ because thats where the particle asymptotically reaches

\subsection{Case: $\epsilon = 1$}
parabolic tragectory 

perigee of $\alpha / 2$ at the turning point 

the critical angle is $\pi$ because the object reaches that largest angle asymptotically. 

If you write $r$ in cartesian coordinates you get 
\equations{
    x = \pm \left( -\frac{y^3}{2 \alpha } + \frac{\alpha}{2} \right)
}

\subsection{Case: $\epsilon < 1$}
Elliptic Orbit 
\equations{
    \frac{-1}{2} \left( \frac{GM}{l} \right)^2 = \epsilon_{min} < \epsilon < 0
    \\
    0 < \varepsilon < 1
    \\
    r = \frac{\alpha}{1 + \varepsilon \cos(\varphi)}
    \\
    a = \textrm{ semimajor axis } = \frac{r_{min} + r_{max}}{2}
    \\
    b = \textrm{ semiminor axis } = a \sqrt{1 - \varepsilon^2} = 
    \frac{\alpha}{\sqrt{1 - \varepsilon^2}} = \frac{l}{\sqrt{|2 \varepsilon |}}
    \\
    \frac{\alpha}{1 - \varepsilon^2} = \frac{GM}{2 |\epsilon|}
}


\subsection{Case: Circulat Orbit}
$\varepsilon < 0$ and $U_{eff}' \Big|_{r = r_c} = 0$

\equations{
    r_c = \frac{l^2}{GM}
    \hp
    \epsilon = \frac{-1}{2} \left( \frac{GM}{l} \right)^2
    \hp
    \varepsilon = 0
}

\section{More with Kepler's Laws}

\subsection{2nd Law}
\equations{
    dA = \frac{1}{2} r d \varphi
    \\
    \frac{dA}{dt} = \frac{1}{2} r \frac{d \varphi}{dt}
    =
    \frac{1}{2} r^2 \frac{l}{r^2} = \frac{l}{2} = \const
}


\subsection{3rd Law}
\equations{
    a^3 \equiv p^2
    \textrm{ (a is semimajor axis and p is period)}
    \\
    \textrm{ area of ellipse } A = \pi a b 
    \\
    \int dA = \frac{l}{2} \int dt 
    \rightarrow 
    \pi a b = \frac{l}{2} p
    \\
    p^2 = \frac{4 \pi^2 a^2 b^2}{l^2} = \frac{4 \pi^2}{GM} a^3
    \\
    \frac{p^2}{a^3} = \frac{4 \pi^2}{GM}
}

\chapter{Harmonic and Damped Motion}

\section{Equations of Motion}
carm of mass $m$ attached to a spring with spring constant $k$ and length $l$.

$x(t)$ is the time dependent position 

$x = 0$ wall, $x_0 = L$ equilibrium. 

\equations{
    \textrm{ spring force } \vec F_s = -k (x(t) - L)
    \\
    m \vec a = \vec F_s
    \hp 
    m \ddot x = -k (x(t) - L)
    \rightarrow
    m \ddot x + k(x - L) = 0
}

You can also derive it from energy conservation 

\equations{
    E = T + U = 
    \frac{1}{2} m \dot x^2 + \frac{1}{2} k (x - L)^2
    \\
    \textrm{derive with respect to time}
    \\
    \frac{dE}{dt} = 0 = 
    \frac{1}{2}m * 2 \dot x * \ddot x + \frac{1}{2} k * 2(x - L) \dot x 
    \rightarrow
    \\
    \dot x (m \ddot x  + k(x - L)) = 0
    \hp 
    \dot x \neq 0 
    \\
    m \ddot x  + k(x - L) = 0
}

\subsection{Solving EoM}
Choose reference frame such that the equilibrium point is at $x = 0$ 

\equations{
    x = x_0 = L \rightarrow y = 0 = x - L, \hp \ddot y = \ddot x
    \\
    \textrm{ make an ansatz for the solution }
    \\
    y(t) \approx e^{\lambda t}
    \hp 
    \dot y(t) = \lambda e^{\lambda t}
    \hp
    \ddot y(t) = \lambda^2 e^{\lambda t}
    \\
    \textrm{insert}
    \\
    m \lambda^2 e^{\lambda t} + k e^{\lambda t} = 0
    \\
    \lambda^2 = - \frac{k}{m} \rightarrow y = 
    \pm i\sqrt{\frac{k}{m}} = \pm i \omega
    \\
    y(t) = D_1 e^{i \omega t} + D_2 e^{- i \omega t}
    \hp D_1, D_1 \in \mathbb{C}
    \\
    \textrm{use Euler's Identity}
    \\
    y(t) = A \cos(\omega t) + B \sin(\omega t)
    \hp 
    A = D_1 + D_2, B = i(D_1 - D_2)
    \\
    y(t) = C \cos(\omega t + \phi)
}
period $= \frac{2 \pi}{\omega}$ and $\omega = \sqrt{\frac{k}{m}}$ 


\subsection{Energy}
\equations{
    U = \frac{1}{2} k y^2 = \frac{1}{2} kC^2 \cos^2(\omega t - \phi)
    \\
    U \geq 0 
    \hp 
    U_{max} = \frac{1}{2} kC^2
    \\
    T = \frac{1}{2} m \dot y^2 
    =
    \frac{m}{2} C^2 \omega^2 \sin^2(\omega t - \phi)
    \\
    T \geq 0 \hp T_{max} = \frac{m}{2} C^2 \omega^2
    \\
    E = T + U = 
    \frac{1}{2} mC^2 \omega^2 \sin^2(\omega t - \phi)
    +
    \frac{1}{2} k C^2 \cos^2(\omega t - \phi)
    =
    \const
    \\
    E = T_{max} = U_{max} = \const
}

\section{Harmonic Oscillator}

simple harmonic oscillator
\equations{
    m_{eff} \ddot x + k_{eff}x = 0
    \hp 
    \ddot x + \omega^2 x = 0
    \hp 
    \omega = \sqrt{\frac{k_{eff}}{m_{eff}}}
}

The spring constant and mass are now effective because we're in a dampened oscillator instead of an ideal one. 

\subsection{EOM}
Consider a vertical spring with height equation $h = \alpha (s - x)^2$

Use conservation of energy 

\equations{
    U = U_{spring} + U_{g}
    \hp 
    U_{spring} = \frac{1}{2} k (x - L)^2
    \\
    U_g = mgh = mg \alpha (s - x)^2
    \\
    \frac{dU}{dx} = 0
    \rightarrow 
    x_{eq} = \frac{kL + 2 \alpha m g s}{k + 2 \alpha mg}
    \hp 
    \textrm{do yourself :(}
}

Now find the effective spring constant

Taylor expand $U$ near $x_{eq}$

\equations{
    U = U(x_{eq}) + U' \Big|_{xeq} (x - x_{eq}) + \frac{1}{2} U'' \Big|_{xeq} (x - x_{eq})^2
    \hp 
    U'(x_{eq}) = 0
    \\
    =
    U(x_{eq}) + \frac{1}{2} U''(x_{eq}) (x - x_{eq})^2
    \hp 
    k_{eff} = U''(x_{eq})
    \\
    k_{eff} = U''(x_{eq}) = k + 2 \alpha mg 
}

Now to find kinetic energy 

\equations{
    T = \frac{1}{2} m \vec v^2 = 
    \frac{1}{2} m (\dot x^2 + \dot y^2)
}

if $h$ is small you can Taylor expand around $\frac{\dot y^2}{\dot x^2} = 4 \alpha^2 (x - s)^2 << 1$

\equations{
    T \approx \frac{1}{2} m \dot x^2
    \\
    E = T + U = 
    \frac{1}{2} m \dot x^2 + U_{x_{eq}} + \frac{1}{2}k_{eff}(x - x_{eq})^2
}

Shift the coordinates for convenience

\equations{
    \bar x = x - x_{eq} 
    \hp 
    \dot \bar x = \dot x 
}

back to the equation of motion

\equations{
    \frac{1}{2} m \dot \bar x^2 + \frac{1}{2} k_{eff} \bar x^2 
    =
    E - U(x_{eq}) = \const
    \\
    \textrm{take the derivative}
    \\
    m \ddot {\bar x} + k_{eff} \bar x = 0
    \rightarrow 
    \ddot \bar x + \omega^2 \bar x = 0
    \\
    \omega = \sqrt{\frac{k_{eff}}{m}}
    =
    \sqrt{\frac{k + 2 \alpha m g}{m}}
}

some stuff with $\dot y$ and $\dot x$

\equations{
    y = h = \alpha (s -x)^2
    \hp 
    \dot y = 2 \alpha (s - x)(-x) \dot x
    \\
    \textrm{take derivative}
    \\
    \frac{d}{dt} (\frac{1}{2} m \dot {\bar x}^2 + \frac{1}{2} k_{eff} \bar x^2)
    \rightarrow 
    \frac{1}{2} m 2 \dot {\bar x} \ddot {\bar x} + \frac{1}{2} k_{eff} 2 \bar x \dot {\bar x} = 0
}

\section{Simple Pendulum}


choose polar coordinates 
\[
r = L = \const \hp \vec v = \dot r \vec e_r + r \dot \theta e_\theta = L \dot \theta \vec e_\theta
\]

Use energy consevation to find equation of motion 
\equations{
    U = mgh = mgL(1 - \cos(\theta))
    \\
    T = \frac{1}{2} m \vec v^2 = \frac{1}{2} m L^2 \dot \theta^2
    \\
    \frac{dE}{dt} = \frac{d}{dt} (\frac{1}{2} m L^2 \dot \theta^2 +  mgL(1 - \cos(\theta)) )
    \rightarrow 
    \\
    L \ddot \theta + g \sin(\theta) = 0 
    \\
    \textrm{small angle approximation}
    \\
    \ddot \theta + \omega^2 \theta = 0
    \hp 
    \omega = \sqrt{\frac{g}{L}}
    \\
    \theta(t) = A \cos(\omega t - \phi)
}
$A$ is amplitude and $\phi$ is the phase which are both derived from initial conditions 

\section{Physical Pendulum}
Let the pendulum have like an actual mass distribution with a center of mass and a pivot that it rotates around. 

choose coordinate system (polar coordinates)
\[
r = L \hp \theta = \theta(t) \hp \vec v = L \dot \theta \vec e_\theta 
\]

use kinetic energy (but you have to use moment of inertia and whatnot instead of just regular kinetic energy)

\equations{
    T = \frac{1}{2} I_p \dot \theta^2
    \hp 
    U = mgh = mgL(1 - \cos(\theta))
    \\
    \frac{dE}{dt} = \frac{d}{dt} (T + U) = I_p \ddot \theta + mgL\sin(\theta) = 0
    \\
    \textrm{small angle}
    \\
    I_p \ddot \theta + mgL\theta = 0
    \rightarrow 
    \ddot \theta + \omega^2 \theta = 0 
    \hp 
    \omega = \sqrt{\frac{mgL}{I_p}}
}

\section{Dampened}
oscillator that's damped due to friction. $E$ is no longer conserved 

Imagine a spring connected to a cart with friction force $\vec F = -C \vec v$

\subsection{EOM}
\equations{
    \frac{dE}{dt} = P_{lost} = \frac{dE_{lost}}{dt} 
    =
    \vec F \frac{dx}{dt} = \vec F \cdot \vec v = -c \vec v^2
    \\
    \textrm{mechanical energy}
    \\
    E = T + U_s = \frac{1}{2} m \dot x^2 + \frac{1}{2} k (x - L)^2
    \\
    \textrm{something?}
    \\
    -c \dot x^2 = P_{lost} = \frac{dE}{dt} =
     \frac{m}{2} 2 \dot x \ddot x + k (x - L) \dot x \rightarrow 
     \\
     m \ddot x  + c \dot x + k (x - L) = 0
}
Translate the coordinates such that $y = x - L$ 

\equations{
    m \ddot y + c \dot y + ky = 0
    \hp 
    \ddot y + 2 \zeta \omega_N \dot y + \omega^2_N y = 0
}
That's the equation for a damped harmonic oscillator $\omega_N^2 = \frac{k_{eff}}{m_{eff}}$ and a damping parameter $\zeta = \frac{c}{2m \omega_N}$

Take an ansatz $y = e^{\lambda t}, \dot y = \lambda e^{\lambda t}, \ddot y = \lambda^2 e^{\lambda t}$

\equations{
    \lambda^2 + 2 \zeta \omega_N \lambda + \omega_N^2 = 0
    \rightarrow 
    \lambda_\pm = -\omega_N(\zeta \pm \sqrt{\zeta^2 - \lambda})
    \\
    y = A e^{\lambda_+ t} + B e^{\lambda_- t}
}

\subsection{$\zeta = 0$}
undamped
\equations{
    \ddot y + \omega^2_N y = 0 \hp \lambda = \pm i \omega_N
    \\
    y(t) = 
    A e^{i \omega_N t} + B e^{-i \omega_N t}
}


\subsection{$\zeta > 1$}
overdamped 

\equations{
    \lambda_\pm = \omega_N(\zeta \pm \sqrt{\zeta^2 - 1}) < 0
    \\
    y(t) = 
    A e^{-|\lambda_+|t} + Be^{-|\lambda_-| t}
}

\subsection{$\zeta = 1$}
critically damped 

\equations{
    \lambda_\pm = - \omega_N
    \\
    y(t) = 
    A e^{- \omega_N t} + Bt e^{- \omega_N t}
}


\subsection{$0 < \zeta < 1$}
Underdamped

\equations{
    \lambda_\pm = -\omega_N (\zeta \pm \sqrt{\zeta^2 - 1})
    =
    \omega_N \zeta \pm i \omega_N \sqrt{1 - \zeta^2}
    =
    - \zeta \omega_N \pm i \omega_d
    \\
    y(t) = e^{-\zeta \omega_N t} (D_1 e^{i \omega_d t} + D_2 e^{-i \omega_d t})
    \hp 
    \textrm{ or }
    \\
    y(t) = e^{-\zeta \omega_N t} (A \cos(\omega_d t) + B \sin(\omega_d t))
    \hp 
    \textrm{ or }
    \\
    y(t) = Ce^{- \zeta \omega_N t} \cos(\omega_d t - \phi)
}
$\omega_d$ is the damping frequency 

\subsection{$\zeta < 0$}
accelerated oscillator?

\equations{
    \lambda_\pm = +\omega_N |\zeta| \pm \sqrt{\ldots}
    \\
    y(t) \approx e^{\omega_N |\zeta| t} \ldots 
}
It's gonna blow up exponentially 


\section{Forced Oscillator}
oscillator with external driving force 

Energy is NOT conserved 

\equations{
    \frac{dE}{dt} + P_{diss} = (\frac{dW}{t})_{ext} = 
    \ldots = \vec F_{ext} \cdot \vec v
    \\
    m_{eff} \ddot {\vec x} + c_{eff} \dot {\vec x} + k_{eff} \vec x = F_{ext}
}
That's the equation for a damped harmonic oscillator in 3d 

\subsection{EOM}
no damping 

use coordinates $(r, \theta)$ 

use modified energy method 

\equations{
    \dot E + P_{diss} = \vec F_{ext} \cdot \vec v
    \\
    P_{diss} = 0 \hp T = \frac{1}{2} m \vec v^2 
    \hp 
    v = L \dot \theta \vec e_{\theta} 
    \\
    U = mgh = mgL(1 - \cos(\theta))
    \\
    \vec F_{ext} \cdot \vec v = 
    L \dot \theta \vec F \cdot e_\theta 
    =
    L \dot \theta F \cos(\theta)
    \\
    \dot E = \frac{1}{2} m L^2 2 \dot \theta \ddot \theta + 
    mgL \dot \theta \sin(\theta)
    =
    FL \dot \theta \cos(\theta)
    \rightarrow 
    \\
    mL \ddot \theta + mg\sin(\theta) = F(t) \cos(\theta)
}

Figure out the eom for small $\theta$ so $\sin(\theta) = \theta$ and $\cos(\theta) \approx 1$

\equations{
    mL \ddot \theta + mg\theta = F(t)
    \rightarrow 
    m_{eff} \ddot \theta + k_{eff} \theta = F(t)
}

\subsection{example}
cart on a cart 

spring of constant $k$ and length $L$ is connected to cart of mass $m$ with position $x(t)$ and that whole thing is on a bigger cart with position $y(t)$ 

net displacemen of cart = $x(t) + y(t)$. Spring force $\vec F_s = -k \vec x$

use N2L 
\equations{
    \vec F = m \vec a \rightarrow -kx = m (\ddot x + \ddot y)
    \rightarrow 
    m \ddot x + kx = -m \ddot y
}

\section{Forced Harmonic Oscillator}
\[
m_{eff} \frac{d^2 \vec x}{dt^2} + c_{eff} \frac{d \vec x}{dt} + k_{eff} \vec x = \vec F_{ext}(t)
\]

How to solve: introduce linear operator $\mathcal{L}[x] = F$

The linear operator is a map of a function into a function 
\[
\mathcal{L} f(t, x) \rightarrow g(t, x) = \mathcal{L}[f(t, x)]
\]
Follows linear property
\[
\mathcal{L}[a f(t, x) + b f_2(t, x)] = a \mathcal{L}[f(t, x)] + b \mathcal{L}[f_2(t, x)]
\]

for example 
\[
\mathcal{L} = A \frac{d}{dx}
\rightarrow 
L[g] = A \frac{dg}{dx}
\]

\equations{
    \mathcal{L} = 
    m_{eff} \frac{d^2}{dt^2} + c_{eff} \frac{d}{dt} + k_{eff} 
    \\
    \mathcal{L}[\vec x(t)] = 
    m_{eff} \frac{d^2 \vec x}{dt^2} + c_{eff} \frac{d \vec x}{dt} + k_{eff} \vec x 
}

so for a forced oscillator 
\[
\mathcal{L}[\vec x(t)] = \vec F_{ext}(t)
\]

\subsection{General Solution}
\[
x(t) = x_h(t) + x_p(t)
\]
where $x_n(t)$ is the homogeneous solution without the external force and $x_p(t)$ is the particular solution of just the external force 

homogeneous solution solves $\mathcal{L}[x(t)] = 0$

particular solution needs just a single solution to $\mathcal{L}[x(t)] = F_{ext}$

\equations{
    \omega_n = \sqrt{\frac{k_{eff}}{m_{eff}}}
    \hp 
    \zeta = \frac{c_{eff}}{2 m_{eff} \omega_n}
    \hp 
    \omega_N \sqrt{1 - \zeta^2}
}
$\omega_n$ is natural frequency. $\zeta$ is dimensionless damping parameter. $\omega_d$ is dampened frequency. 

\subsection{Force as a power series}
\[
F_{ext} = a + bt + ct^2 + \ldots
\]
\equations{
    m_{eff} \ddot x + c_{eff} \dot x + k_{eff} x 
    =
    a + bt + ct^2 + \ldots
}
Strategy 
\begin{itemize}
    \item
    solve for $x_h(t)$
    \item 
    determine 1 possible $x_p(t)$
    \item 
    find initial conditions 
\end{itemize}

ansatz for $x_p(t)$
\equations{
    x_p(t) = \alpha + \beta t + \gamma t^2
}
and then just solve 

\subsection{Example}
\equations{
    F_{ext} = t^2 \hp \textrm{ no damping}
    \\
    c_{eff} = 0 \hp k_{eff} = 1 \hp m_{eff} = 1
    \hp 
    \ddot x + \omega_n x = 0
    \\
    x(0) = 0 \hp \dot x(0) = 0
    \\
    \textrm{solve particular solution}
    \\
    x(t) = \alpha + \beta t + \gamma t^2 
    \hp 
    \dot x = \beta + 2 \gamma t 
    \hp 
    \ddot x = 2 \gamma 
    \\
    2 \gamma + \alpha + \beta t + \gamma t^2 = t^2
    \hp 
    \beta = 0 \hp \gamma = 1 \hp \alpha = -2 
    \\
    x_p = -2 + t^2
    \\
    \textrm{find general solution (simple harmonic oscillator)}
    \\
    x_h(t) = A \cos(\omega_n t) + B\sin(\omega_n t)
    \\
    x = x_h(t) + x_p(t) = A \cos(\omega_n t) + B \sin(\omega_n t) + t^2 - 2
    \\
    \textrm{initial conitions}
    \\
    x(0) = 0 \rightarrow A = 2
    \\
    \dot x(0) = 0 \rightarrow B = B = 0
    \\
    x(t) = 2 \cos(\omega_n t) + t^2 - 2
}

\subsection{Exponential Force}
\equations{
    m_{eff} \ddot x + c_{eff} \dot x + k_{eff} x = F_0 e^{\alpha t}
    \\
    \textrm{$x_p$ ansatz $x_p = A e^{\alpha t}$}
    \\
    m_{eff} \alpha^2 A e^{\alpha t} + c_{eff} A \alpha e^{\alpha t} + k_{eff} A e^{\alpha t} = F_0 e^{\alpha t}
    \rightarrow 
    \\
    m_{eff} \alpha^2 A + c_{eff} A \alpha + k_{eff} A  
    = F_0 
    \rightarrow
    A = 
    \frac{F_0}{m_{eff} \alpha^2 + c_{eff} \alpha + k_{eff}}
    \\
    x = x_h + x_0 = 
    \\
    e^{-\zeta \omega_n t} (A \cos(\omega_d t) + B \sin(\omega_d t)) + 
    \frac{F_0}{m_{eff} \alpha^2 + c_{eff} \alpha + k_{eff}} e^{\alpha t}
}

\subsection{Harmonic Force}
\[
    F = F_0 \cos(\omega t - \theta)
\]
$F_0$ is amplitude, $\omega$ is angular speed, $\theta$ is phase 

\equations{
    m \ddot x + x \dot x + kx = F_0 \cos(\omega t - \theta)
    \\
    \textrm{ansatz $x_p = D \cos(\omega t - \psi)$}
}

That does work but theres a trick 
\equations{
    e^{i \Omega t} = \cos(\Omega t) + i \sin(\Omega t)
    \\
    F(x) = \Re (F_0 e^{i(\omega t - \theta)})
    \\
    \textrm{ansatz $x_p = \Re (X e^{i \omega t}) X \in \mathbb{C}$}
    \\
    \Re (-m \omega^2 X e^{i \omega t} + i \omega c X e^{i \omega t} + k X e^{i \omega t}) = \Re (F_0 (e^{i \omega t} e^{-i \theta }))
    \rightarrow 
    \\
    X = \frac{F_0 e^{-i \theta}}{-m \omega^2 + i \omega c + k}
    =
    \frac{F_0}{k} e^{-i \theta} \left[
        1 + 2 i \zeta \frac{\omega}{\omega_N} - 
        \left( \frac{\omega}{\omega_N}\right)^2
    \right]^{-1}
    \\
    =
    \overset{N}{G}(\omega) F_0 e^{-i \theta}
    \\
    \overset{N}{G}(\omega) = \frac{1}{k} \left(
        1 + 2 i \zeta \frac{\omega}{\omega_N} - \left( \frac{\omega}{\omega_N}\right)^2
    \right)^{-1}
    =
    G(\omega) e^{i \phi(\omega)}
    \\
    G(\omega) = |\overset{N}{G}(\omega)| = \frac{1}{k} 
    \left[
        \left(
            1 - \frac{\omega^2}{\omega_N^2}
        \right)^2
        +
        \left(
            2 \zeta \frac{\omega}{\omega_N}
        \right)^2
    \right]^{-1/2}
    \\
    \phi(\omega) = Arg(\overset{N}{G}(\omega))
    =
    \arctan(\frac{2 \zeta \omega \omega_N}{\omega_N^2 - \omega^2})
    \hp 
    0 \leq \phi < \pi 
    \\
    x_p(t) = \Re
}

trying this again because it's a new lecture 

\equations{
    m \ddot x + c \dot x + k x = F_0 \cos(\omega t - \theta)
    \\
    x_p(t) = F_0 G(\omega) \cos(\omega t - \theta - \phi(t))
    \\
    G(\omega) = 
    \frac{1}{k}
    \left[
        \left(
            1 - (\frac{\omega}{\omega_N})^2
        \right)^2
        +
        \left(
            2 \zeta \frac{\omega}{\omega_N}
        \right)^2
    \right]^{-1/2}
    \\
    \phi(t) = \arctan(\frac{2 \zeta \omega \omega_N}{\omega_N^2 - \omega^2})
}
I get to derive this at home oh boy such fun 

$x_p(t)$ is oscillating with a frequency $\omega \neq \omega_N$ 
that is the same as the driving force.

The amplitude $|x_p(t)| = F_0 G(\omega)$ is frequency dependent. 

The phase $\phi(\omega)$ also depends on the frequency. 

At late times, the oscillation will be just the driven force because the 
regular oscillator will have been dampened to nothing. $x_p(t)$ is the 
"steady state" solution. 
\equations{
    e^{- \zeta \omega_N t} = e^{- t / \tau} 
    \hp
    \tau = \frac{1}{\zeta \omega_N}
}

\section{Analyzing $x_p(t)$}

\subsection{peak (resonance)}
at the peak, the system is in resonance (critical point)
\equations{
    \frac{d G(\omega)}{d \omega} \Big|_{\omega = \omega_R} = 0
    \hp 
    \omega_R = \pm \omega_N \sqrt{1 - 2 \zeta^2}
}

magnitude of the peak 
\equations{
    G(\omega = \omega_R) = \frac{1}{2k \zeta \sqrt{1- \zeta^2}}
}
for a small damping parameter, the peak magnitude is very large. $\omega_R \approx \omega_N$

\subsection{small $\omega$}
$\omega / \omega_N << 1$ and $\omega < \omega_R$
\equations{
    G(\omega) \approx \frac{1}{k} + 
    \frac{1 - 2 \zeta^2}{k} \left( \frac{\omega}{\omega_N} \right)^2
    \approx 
    \frac{1}{k} + O (\frac{\omega}{\omega_N})^2
    \\
    \phi(t) \simeq 2 \zeta \frac{\omega}{\omega_N} \approx 0
}

\subsection{very large $\omega$}
\equations{
    G(\omega) \simeq 
    \frac{1}{k} \frac{\omega_N}{\omega} + O(\frac{\omega_N}{\omega})^3
    \\
    \phi(\omega) \approx - 2 \zeta \frac{\omega_N}{\omega} + n \pi \simeq \pi
    \\
    x_p(t) = 
    \frac{F_0}{k} (\frac{\omega_N}{\omega})^2 \cos(\omega t - \theta)
    \hp 
    \omega_N^2 = \frac{k}{m}
    \\
    x_p \approx \frac{F_0}{m \omega^2} \cos(\omega t - \theta)
}

for the different phases 
\begin{itemize}
    \item 
    at resonance 

    $\phi(\omega_R) \simeq \pi/2$ so $x$ is $\pi/2$ behind force 
    \item 
    $\omega << \omega_N$

    $\phi \simeq 0$ so
    x and F are in phase 
    \item
    $\omega >> \omega_N$
    
    $\phi \simeq \pi$ so $x$ is $\pi$ behind the force 
\end{itemize}


\chapter{Periodic Forces and Fourier Series}
The force can be all sorts of wacky as long as it's periodic along a period $T$

$F(t + T) = F(t)$ and $T = \frac{2 \pi}{\Omega}$ where $\Omega$ is the frequency of 
the periodic force. 

Describe the solution for the periodic force as a superposition of 
harmonic functions (sin and cos). 
We can do this because the differential operator 
$\mathcal{L} = m \frac{d^2}{dt^2} + c \frac{d}{dt} + k$ is linear 

\section{Fourier Series}
\equations{
    F(t) = \frac{a_0}{2} + \sum^{\infty}_{n = 1} a_N \cos(n \Omega t) + 
    \sum^{\infty}_{p = 1} b_p \sin(p \Omega t)
    \\
    n, p \in \mathbb{N}
    \hp 
    a_n, b_p = \textrm{ Fourier Coefficients}
}
An even function is a function such that $f(t) = f(-t)$ (like cos)

$f_{even}(t) = \frac{1}{2} (F(t) + F(-t))$

odd function $f(-t) = - f(t)$ (like sin)

$f_{odd} = \frac{1}{2} (F(t) - F(-t))$

if your function is even, the fourier series will have 0 for all sin coefficients.

If it's odd, then all the cos coefficients will be 0

\subsection{Coefficients}
\equations{
    a_0 = \frac{2}{T} \int^T_0 F(t) dt 
    \\
    a_N = \frac{2}{T} \int^0_T F(t) \cos(n \Omega t) dt 
    \\
    b_p = \frac{2}{T} \int^0_T F(t) \sin(p \Omega t) dt 
}
The way the integral bounds work is you just need to integrate for 1 period of $T$.

Verify the coeffients 

\equations{
    \int \cos(p \Omega t) \cos(n \Omega t) dt 
    =
    \\
    \frac{1}{2} \int \cos((n + p) \Omega t) dt  + 
    \int \cos((n - p) \Omega t) dt 
}
it vanishes so I think something's good 

\section{Fouier Series}
This is just a new lecture so I'm writing all the stuff 

\equations{
    F(t) = 
    \frac{a_0}{2} + 
    \sum^\infty_{p = 1} a_p \cos(p \Omega t)
    + \sum^\infty_{p = 1} b_p \sin(p \Omega t)
    \\
    a_0 = \frac{2}{T} F(t) dt 
    \\
    a_p = \frac{2}{t} \int_T F(t) \cos(p \Omega t) \, dt 
    \hp 
    b_p = \frac{2}{t} \int_T F(t) \sin(p \Omega t) \, dt 
}
In order to verify $a_0, a_p, b_p$, we need to show that the thing is orthogonal 

\equations{
    \int \limits_T \cos(m \Omega t) \cos(p \Omega t) \, dt = 
    \begin{cases}
    0 \hp m \neq p \\
    \frac{T}{2} \hp m = p
    \end{cases}
    \\
    \int \limits_T \sin(m \Omega t) \sin(p \Omega t) \, dt = 
    \begin{cases}
    0 \hp m \neq p \\
    \frac{T}{2} \hp m = p
    \end{cases}
    \\
    \int \limits_T \cos(m \Omega t) \sin(p \Omega t) \, dt = 0
}

insert $F(t)$ to get something 

\equations{
    a_m = \frac{2}{T} \int \limits_T F(t) \cos(m \Omega t) \, dt 
    = 
    \\
    \frac{2}{T} \int \limits_T dt \cos(m \Omega t) 
    \left[
        \frac{a_0}{2} + 
        \sum^\infty_{p = 1} a_p \cos(p \Omega t)
        + \sum^\infty_{p = 1} b_p \sin(p \Omega t)
    \right]
    \\
    a_m = \frac{2}{T} \int \limits_T dt \left[
        \frac{a_0}{2} \cos(m \Omega t)
        + 
        \cos(m \Omega t) \sum^\infty_{p = 1} a_p \cos(p \Omega t)
    \right]
    \\
    \textrm{if } m = 0: a_0 = \frac{2}{T} \int \limits_T dt 
    \rightarrow \frac{a_0}{2} = a_0
    \\
    \textrm{if } m \neq 0 : a_m = \frac{2}{T} \int \limits_T
    dt \cos(m \Omega t) \sum_p  a_p \cos(p \Omega t) = a_p = m
}

\subsection{Example: Square Wave}
it's 1 from $\frac{-T}{2}$ to 0 and -1 from 0 to $\frac{T}{2}$
 
because the function is odd, $a_n = 0$

Also because the function is odd, \\
$a_0 = \langle F(t) \rangle = 
\frac{1}{T} \int \limits_T F(t) dt = 0$

$b_p$ is the only one that involves actual calculation 

\equations{
    b_p = \frac{2}{T} \int \limits_T F(t) \sin(p \Omega t) \, dt 
}
integrate from $-\frac{T}{2}$ to $\frac{T}{2}$ 

use integration by parts as well 

\equations{
    u = F(t) \rightarrow u' = 0 
    \hp 
    v' = \sin(p \Omega t) \rightarrow v = 
    \frac{-1}{p \Omega} \cos(p \Omega t) 
    \\
    b_p = 
    \frac{2}{T}
    \left[
        F(t) (\frac{-1}{p \Omega}) \cos(p \Omega t)
    \right]
    \Big|^{T/2}_{-T/2} - 
    \int \limits_T 0 * \frac{-1}{p \Omega} \cos(p \Omega t) dt 
    \rightarrow \\
    \frac{2}{T} 
    \left[
        \frac{-1}{p \Omega} F(t) \cos(p \Omega t) \Big|^0_{-T/2} + 
        \frac{-1}{p \Omega} F(t) \cos(p \Omega t) \Big|^{T/2}_{0}
    \right]
}

You're going to get 
\equations{
    b_p = 
    \begin{cases}
        0 \hp \textrm{p is even} \\
        \frac{-4}{p \pi} \hp \textrm{p is odd}
    \end{cases}
    \\
    F(t) = \sum^\infty_{p = 1} b_p \sin(p \Omega t)
}
That is the truncated series 

If you take only the first $n$ terms of a fourier series, youre going to get a noticeably wave-y function but it will look closer to your original function that just a sine or cosine wave. 

\subsection{Gibb's Phenomenon}
When doing the fourier series of a square wave, the edges of the square are going to have little horns becauseidk math. 

This is characteristic of a truncated series. 

\subsection{Summary}
equation of motion 
\[
m \ddot x + c \dot x + kx = F(t) 
\]
with periodic force $F(t)$ 

Fourier series 
\[
    F(t) = 
    \frac{a_0}{2} + 
    \sum^\infty_{p = 1} a_p \cos(p \Omega t)
    + \sum^\infty_{p = 1} b_p \sin(p \Omega t)
\]

with coefficients 
\equations{
    a_p = \frac{2}{T} \int \limits_T F(t) \cos(p \Omega t) \, dt
    \hp 
    b_p = \frac{2}{T} \int \limits_T F(t) \sin(p \Omega t) \, dt
    \\
    a_0 = \frac{2}{T} \int \limits_T F(t) \, dt \hp 
    \textrm{ time average } 
}

Some shenanigans 

equation of motion with linear operator $\mathcal L = F(t)$

with homogeneous solution $x_h(t)$ for $\mathcal L(x) = 0$

and a general solution $x(t) = x_h(t) + x_p(t)$

How to find the particular solution? 

\equations{
    \mathcal L(x) = 
    \left(
        m \frac{d^2}{dt^2} + c \frac{d}{dt} + k
    \right)
    x
    =
    \sum_q F_q(t)
    \\
    x(t) = x_1(t) + x_2(t) + \ldots = \sum^\infty_{f= 1} x_f(t)
    \hp 
    m \ddot x_1 + c \dot x_1 + k x_1 = F_1(t) 
    \\
    m \ddot x_q + c \dot x_q + k x_q = F_q(t)
}
each $F_q(t)$ is a harmonic function 

\equations{ 
    x_q(t) = f(q) \cos(q \Omega t - \theta - \phi(q \Omega))
    \hp 
    q \Omega = \omega
    \\
    x_{p}(t) = 
    \\
    \frac{a_0}{2k} + 
    \sum^\infty_{q = 1} a_q G(q \Omega)
    \cos(q \Omega t - \phi(q \Omega)) + 
    \sum^\infty_{q = 1} b_q G(q \Omega) 
    \sin(q \Omega t - \phi(q \Omega))
    \\
    G(q \Omega) = 
    \frac{1}{k} 
    \left[
        \left( 
            1 - \left( \frac{q \Omega}{\omega_N}\right)^2
        \right)^2
        +
        \left(
            2 \zeta \frac{2 \Omega}{\omega_n}
        \right)^2
    \right]^{-1/2}
    \\
    \phi(q \Omega) = \arctan(\frac{2 \zeta q \Omega \omega_N}{\omega_n^2 - (q \Omega)^2})
}

\section{Steady State Solution}
$x(t)$ at late times $(t \to \infty)$

for an undamped oscillator 
\[
x(t) = A \cos(\omega_n t) + B \sin(\omega_n t) + x_p(t) 
\]

for a damped oscillator 
\[
x(t) = x_p(t) 
\]
Can there be resonance? Yes. 
if 
$q \Omega = \omega_R = \omega_n\sqrt{1 - 2 \zeta^2} \approx
 \omega_n (\textrm{ if $\zeta << 1$})$

$q \simeq \omega_R / \Omega$

$G = \sum G(q \Omega)$

You can have resonance for each $q$

\chapter{Impulse Response and Green's Function}
Consider a force that exerts an impulse 

The function is $0$ everywhere except for the time of impulse in which it is instantaneously non-zero and then it goes back to $0$

\section{Math Break: Delta Distribution}

Describe this with a \textbf{Delta Distribution} or "delta function". 

\equations{
    F(t) = \delta (t - a) = 
    \begin{cases}
    \infty \hp t = a 
    \\
    0 \hp \textrm{otherwise}
    \end{cases}
}


consider a rectangle with width $\epsilon$ and height $1 / \epsilon$ at time $t = a$. and take the limit as $\epsilon \to 0$

This gets you the delta distribution $\delta(t - a)$
\[
f(t = a) = \int^{\infty}_{- \infty} f(t) \delta(t - a) \, dt
\]
for functions continuous at $t = a$ and with limits 

$\delta(\infty), \delta(- \infty) = 0$.

\subsection{example}
\[
\int^{\infty}_{- \infty} \sin(t) \delta(t - \frac{3}{2} \pi ) =
sin(3/2)
\]

You can do a $u$ sub to get to the right form of the thing 
you want. 

There's also a thing you can do with integration by parts 
\equations{
\int^{\infty}_{-\infty} f(t) \frac{d}{dt} \delta(t - a)dt 
\rightarrow 
f(t) \delta(t - a)  
\Big|^{\infty}_{-\infty}
-
\int^{\infty}_{-\infty} (\frac{df}{dt}) \delta(t - a) dt 
\\
=
- \frac{df}{dt} 
\Big|_{t = a}
}

\subsection{Impulse Forces}
Consider a force acting for $\Delta t << \omega_N$

It can be described using a delta distribution. 

\equations{
    F(t) = I \delta(t)
    \hp 
    I = \Delta p = \Delta(F * t)
}
Response to impulse force 

\subsection{Green's Function}
Let force be at $t = 0$ with impulse of $1$ and 
$x(t = 0) = 0$ and $\dot x(t = 0) = \frac{1}{m}$. 

\equations{
    G(t) = \begin{cases}
    0 \hp t < 0 \\ 
    \exp(- \frac{C}{2m} t) \frac{\sin(\omega_d t)}{m \omega_d}
    \end{cases}
}

That is the response for 
\[
m \ddot x + c \dot x + kx = \delta(t - a)
\hp
\omega_N = \sqrt{\frac{k}{m}}, \omega_d = \omega_N \sqrt{1 - \zeta^2}, \zeta = \frac{c}{2m \omega_N}
\]

\subsection{Math Break: Heaviside}
It's just a step function 
\[
H(y - y_0) = \begin{cases}
1 \hp y > y_0 \\ 
0 \hp y < y_0 
\end{cases}
\]

\subsection{Back to Green's Function}

\equations{
    G(t) = \begin{cases}
    0 \hp t < 0 \\ 
    \exp(- \frac{C}{2m} t) \frac{\sin(\omega_d t)}{m \omega_d}
    \end{cases}
    =
    H(t) 
    \exp(- \frac{C}{2m} t) \frac{\sin(\omega_d t)}{m \omega_d}
    \\
    \textrm{at late times }
    \\
    x(t) = \begin{cases}
    0 \hp t < a \\ 
    \exp(- \frac{C}{2m} (t - a)) \frac{\sin(\omega_d (t - a))}{m \omega_d}
    \end{cases}
    =
    \\
    =
    H(t - a) 
    \exp(- \frac{C}{2m} (t - a)) \frac{\sin(\omega_d (t - a))}{m \omega_d}
    \\
    = G(t - a)
}

recognize response to impulse as Green's Function $G(t - a)$

satisfies equation of motion 
\equations{ 
    \mathcal L(G(t - a)) = 
    (m \frac{d^2}{d^2 t} + c \frac{d}{dt} + k) 
    G(t - a) = \delta(t - a)
}

\section{Arbitrary Forcing and Convolution}
Imagine a force with 2 consecutive impulses 
$I_1(t = t_1)$ and $I_2(t = t_2)$

Goal: figure out the forces and the particular solution 

\equations{
    F(t) = I_1 \delta(t - t_1) + I_2 \delta(t - t_2)
    \\
    x_p(t) = I_1 G(t - t_1) + I_2 G(t - t_2)
}

to generalize 
\equations{
    F(t) = \sum^N_{q = 1} I_q \delta(t - t_q)
    \rightarrow
    \\
    x_p(t) = 
    \sum^N_{q = 1} I_q G(t - t_q)
    \\
    =
    \sum^N_{q = 1} I_q H(t - t_q) \exp(\frac{c}{2m}(t - t_q)) 
    \frac{\sin(\omega_d (t - t_q))}{m \omega_d}  
}

consider a continuous force 
\[
F(t) = \sum^{N}_{q = 1} I_q \delta(t - t_q)
=
\sum^{N}_{q = 1} F(t_q) \delta(t - t_q) \Delta t
\]

if we consider the limits $\Delta t \to 0$

\[
F(t) = \int^{\infty}_{- \infty} F(t') \delta(t - t') dt'
\]

the response $x_p(t)$ is 
\[
x_p(t) = \int^{\infty}_{- \infty} F(t') G(t - t') dt'
\]
solution for a damped oscillator with an arbitrary driving force. 

verify 
\equations{
    \mathcal L[x] = 
    \mathcal L_t \left[
        \int^{\infty}_{- \infty} F(t') G(t - t') dt'
    \right]
    =
    \int^{\infty}_{- \infty} F(t') \mathcal L_t 
    \left[ 
        G(t - t') dt' 
    \right]
    =
    \\
    \int^{\infty}_{- \infty} F(t') \delta(t - t') = F(t)
}

G(t) is called Green's Function 

x(t) is called convolution 

\equations{
    x = F \otimes G = 
    \int^{\infty}_{- \infty} F(t') G(t - t') dt'
    =
    \int^{\infty}_{- \infty} F(t'') G(t'') dt''
    =
    G \otimes F
}
Its commutative

\subsection{Integration Bounds}
if F is off for $t - t' < 0$
\[
G(t - t') = \begin{cases}
    = 0 \hp t - t' < 0 \\
    \neq 0 \hp t - t' > 0
\end{cases}
\]

this means that 
\equations{
    x(T) = \int^{\infty}_{- \infty} F(t') G(t - t') dt' 
}
integrate to where $t' = t$

If the force starts at $t = t_0$ then $F(t) = 0$ for $t < t_0$
so our lower integration boundary can be $t_0$

so our new integral is 
\equations{
    x(T) = \int^{t' = t}_{t = t_0} F(t') G(t - t') dt' 
}

motion depends on the entire past. 

new lecture

Green's function is the solution to 
\[
\mathcal L[G(t - t')] =
\left(
    m \frac{d^2}{dt^2} + c \frac{d}{dt} + k 
\right) 
[G(t - t')]
=
\delta(t - t')
\]

\equations{
    x(t) = F \otimes G =
     \int^{\infty}_{-\infty}
    F(t') G(t - t') dt'
    =
    \\
    \int^{\infty}_{-\infty}
    F(t') H(t - t') \exp({- \frac{c (t - t')}{2m}})
    *
    \frac{\sin(\omega_d (t - t'))}{m \omega_d}
    dt'
    \\
    H = \begin{cases}
        1 \hp t > t' \\
        0 \hp t < t'
    \end{cases}
    \\
    \omega_d = \omega_n \sqrt{1 - \zeta^2}
    \hp 
    \omega_n = \sqrt{\frac{k}{m}}
    \hp 
    \zeta = \frac{c}{2m \omega_n}
}


\section{Undamped Oscillator}
constant force $F = F_0$ switched on at $t = 0$

$x(0) = 0$ and $\dot x(0) = 0$

find $x(t)$

Green's Function Approach 
\equations{
    x(t) = \int^{\infty}_{- \infty} F_0 * H(t - t') * 
    \exp(\frac{-c (t - t')}{2m}) * 
    \frac{\sin(\omega_d (t - t'))}{m \omega_d}
    \\
    \textrm{undamped oscillator means $c = 0$ plus fix bounds}
    \\
    \int^{t}_{0} F_0(t) 
    \frac{\sin(\omega_d (t - t'))}{m \omega_d}
    \, dt 
    =
    \frac{F_0}{m \omega_n^2} (1 - \cos(\omega_n t))
}

\section{Fourier Transform}
Tranform a function of time into a function of frequency 

rewrite differential equation of $t$ into algebraic equation of $\omega$

fourier transform = trafo lmao 

\equations{
    \tilde f(\omega) = 
    \int^{\infty}_{- \infty} f(t) e^{-i \omega t}
}

Inverse trafo 
\[
f(t) = \frac{1}{2 \pi} \int^{\infty}_{- \infty} 
\tilde f(t) e^{+i \omega t} \, d \omega
\]

$*$ and $**$ are the trafo pair 

convention: 

$*$ = prefactor = $1$

$** = \frac{1}{2 \pi}$

or 

$* = \frac{1}{\sqrt{2 \pi}}$

$** = \frac{1}{\sqrt{2 \pi}}$

\subsection{Example}
$f(t) = \delta(t)$

\equations{
    \tilde f(\omega) = 
    \int^{\infty}_{- \infty} \delta(t - 0) e^{-i\omega t}
    = 1
}

inverse 
\equations{
    \delta(t) = \frac{1}{2\pi} \int^{\infty}_{- \infty} 
    1 e^{i \omega t} d\omega 
}

\subsection{Square pulse}
$f(t) = 1$ for $-T < t < T$ and 0 everywhere else 

\equations{
    f(t) = H(T + t) * H(T - t)
    \\
    \delta f(\omega) = 
    \int^{\infty}_{- \infty} f(t) e^{-i \omega t} \, dt 
    =
    \int^{\infty}_{- \infty} 
    H(T + t) * H(T - t) e^{-i \omega t} \, dt 
    =
    \\
    \int^{T}_{-T} 
    e^{-i \omega t} \, dt  
    =
    \frac{2}{\omega} \sin(\omega t)
}

That function has zeroes at $\omega = n \frac{\pi}{T}$

\subsection{Differential vs Algebraic}
\equations{
    \mathcal L = m \ddot x + c \dot x + kx = F(t)
    \\
    \tilde f(\omega) = 
    \\
    m \int^{\infty}_{- \infty} \ddot x e^{-i \omega t} +
    c \int^{\infty}_{- \infty} \dot x e^{-i \omega t} +
    k \int^{\infty}_{-\infty}  x e^{-i \omega t} =
    \int^{\infty}_{-\infty} F(T) e^{-i \omega t} dt
    \\
    FT[x] = \int^{\infty}_{-\infty} e^{- i\omega t} dt 
    \\
    \textrm{integration by parts}
    \\
    x e^{- \omega t} 
    \Big|^{\infty}_{-\infty} - 
    \int^u x (- i \omega ) e^{- i \omega t} dt 
    = 
    \int^u x (i \omega ) e^{- i \omega t} dt = \tilde x(\omega)
    \\
    -m \omega^2 \tilde x(\omega) + 
    i c \omega \tilde x(\omega)  + k \tilde x(\omega) = 
    \tilde F(\omega)
}

if $F(t) = \delta(t)$ then $x(t) = G(t)$
\[
F(\omega) = \int \delta(t) e^{- i \omega t} = 1
\]

\equations{
    \mathcal L_\omega [\tilde G(\omega)] = (-m \omega^2 + i \omega c + k) \tilde G(\omega) = 1
    \\
    \tilde G(\omega) = 
    \frac{1}{-m \omega^2 + i \omega c + k}
    =
    \frac{1}{k} \left(
        1 - (\frac{\omega}{\omega_n})^2 + 2 i \zeta \frac{\omega}{\omega_n}
    \right)^{-1}
    \\
    \mathcal L[x(t)] = F(t)
    \rightarrow 
    x(t) = \int^{\infty}_{-\infty} F(t') G(t - t') dt
    \\
    -m \omega^2 \tilde x(\omega) + 
    i c \omega \tilde x(\omega)  + k \tilde x(\omega) = 
    \tilde F(\omega)
    \rightarrow 
    \tilde x(\omega) = \tilde F(\omega) \tilde G(\omega)
}

If you want to find $x$ given the fourier transform, you 
just do the inverse fourier transform. 
\[
x(t) = \frac{1}{2 \pi} \int^{\infty}_{- \infty} 
\tilde x(\omega) e^{i \omega t} \, d \omega 
\]

\section{Interpretation}
FT (fourier tranform) variable $\omega$ is frequency, 

In acoustics, $\omega$ is vibration frequency. 

$\tilde x$ is a spectrum, or distribution of frequency. 

in quantum mechanics, $\omega$ is related to energy $E = \hbar \omega$ so $\tilde x(\omega)$ is an energy spectrum 

\subsection{Example}
\[
x(t) = \cos(3 t)
\]

recall euler identity and do FT 
\equations{
    \cos(\omega_0 t) = 
    \frac{1}{2} (e^{i \omega_0 t} + e^{-i \omega_0 t})
    \hp 
    \omega_0 = 3
    \\
    FT: \tilde x(\omega) =
    \int^{\infty}_{-\infty} \cos(3t) e^{-i \omega t}
    =
    \int^{\infty}_{-\infty} e^{-i \omega t}
    \frac{1}{2} (e^{i 3 t} + e^{-i 3 t})
    =
    \\
    \frac{1}{2} \int^{\infty}_{-\infty} e^{i (3 - \omega) t}
    +
    \frac{1}{2} \int^{\infty}_{-\infty} e^{-i (3 + \omega) t}
    =
    \tilde x(\omega)
    =
    \pi (\delta(\omega - 3) + \delta(\omega + 3))
}
2 spikes of size $\pi$ at $\omega = 3$ and $\omega = -3$

if $x(t)$ has a long duration then 
its FT $\tilde x(\omega)$ is narrow.

if $x(t)$ is of finite short length then the spikes will be 
wider. 

\subsection{Gaussian Wave Packet}
\[
f(t) = e^{i \alpha t} \exp(- \frac{(t - t_0)^2}{T^2})
\]
It's just an oscillation times a Gaussian (normal distribution)

centered around $t = t_0$ with a width of $T$ 

\[
FT \approx \exp(\frac{(\omega - \alpha)^2}{\Omega^2})
\hp 
\Omega = \frac{2 \pi}{T}
\]

\chapter{Noninertial Reference Frames}
In a regular inertial frame, Newton's 2nd Law works,
$F = ma$ and $F = \sum F_{real}$

non-inertial reference frames have either rotation or linear 
acceleration or both. 

Acceleration can be described as "fictitious" force

The mass in $F = ma$ is an inertial mass 
\equations{
    \vec F = m_I \vec a
    \hp 
    \vec F_a = m_g \vec g \textrm{ gravitational mass}
    \\
    m_I \approx m_g \hp |\frac{m_I}{m_g} - 1| \leq 10^{-15}
}
some relativity shenanigans

\section{Equivalence Principle}
inertial and gravitational mass are the same. 

consider 2 uniformly accelerating frames 
(1 in a gravitational field, 1 not near any field)

The 1st elevator is in NO gravity, so the acceleration is just 
upwards, so the force is downwards. with a force $F_N$

The 2nd elevator both accelerates upwards AND has a gravitational 
force, 
\equations{
    F_{net} = \vec F_N - m \vec g = 0 
    \rightarrow 
    \vec F_N = m \vec g
}
The physical descriptions are equivalent. A force and being in 
an accelerated reference frame are the same. 

Locally, an observer cannot distinguish an experiment done 
in a uniform gravitational field compared to a uniformly 
accelerated reference frame. 

\section{Non-rotating Kinematics}
Consider 2 frames accelerated relative to each other 

$O$ is the rest frame $(\vec e_x, \vec e_y, \vec e_z)$ with $\dot e = 0$

$O'$ is the accelerated frame $(\vec e_{x'}, \vec e_{y'}, \vec e_{z'})$

position of $A$ in the rest frame $O$ is given by 
$r_{OA} = r_{OO'} + r_{O' A}$

the position of $O'$ in $O$ and the position of $A$ in $O'$

The velocity of $A$ in the rest frame $O$, also known as the "true" velocity, 
is $v_{OA} = v_{O O'} + v_{O' A}$

$a_{O A} = a_{O O'} + a_{O' A}$

In rest frame, $F = m a_{O A} = m(a_{O O'} + a_{O' A})$

In accelerated frame 
\[
F_{true} - m a_{O O'} = m a_{O' A}
\]

N2L in accelerated frame 
\[
    F_{net} = F_{true} + F_{fict} = m a_{O' A} 
\]


\section{Rotating Frame}
rest frame $O$ and rotating frame $O'$ with a frequency $\omega$

position vector $L$ appears fixed in $O'$ is seen in 
$O$ with $\frac{dL}{dt} = \omega \times L$

\equations{
    \omega = \frac{d \theta}{dt}
    \\
    r_{OA} = r_{O O'} + r_{O' A}
    \\
    v_{OA} = v_{O O'} + v_{O' A}
    \\
    v_{O O'} = 
    \frac{d}{dt} r_{O O'} = \frac{d}{dt} 
    \left(
        r_{O O' x} e_x + 
        r_{O O' y} e_y + 
        r_{O O' z} e_z 
    \right)
    \\
    v_{OA} = \frac{d}{dt} r_{OA} =
    \frac{d}{dt}
    \left(
        x' e_{x'} + 
        y' e_{y'} + 
        z' e_{z'} 
    \right)
    \\
    =
    \left(
        \dot x' e_{x'} + 
        \dot y' e_{y'} + 
        \dot z' e_{z'} 
    \right)
    +
    \left(
        x' \dot e_{x'} + 
        y' \dot e_{y'} + 
        z' \dot e_{z'} 
    \right)
    \\
    =
    v_{apparent} + \omega \times r_{O' A}
    \hp 
    \dot e_{x'} = \omega \times e_{x'}
    \\
    V_{O A} = V_{O O'} + v_{apparent} + \omega \times r_{O' A}
}

the velocity in a rest frame equals the velocity of the rotating frame in the 
rest frame + the apparent velocity of $A$ in $O'$ + $\omega \times r_{O' A}$

\subsection{Example}
person is a radius $b$ away from the center of a turntable of frequency $\omega$ 
inside a moving train with linear velocity $v_{train}$

\[
v_{O A} = v_{train} + u + \omega \times b
\]
$u$ is the velocity of the person in the rotating frame (same-ish frame as person)

\subsection{Acceleration}

\equations{
    a_{O A} = \frac{d}{dt} v_{O O'} + \frac{d}{dt} v_{app} 
    + \frac{d}{dt} (\omega \times r_{O' A})
    \\
    \frac{d}{dt} v_{app} = 
    \frac{d}{dt} ( v_{app x'} e_{x'} + \ldots ) 
    =
    a_{app} + \omega \times v_{app}
    \\
    \frac{d}{dt} (\omega \times r_{O' A})
    =
    \dot \omega \times r_{O' A} + \omega \times \dot r_{O' A}
    =
    \\
    \dot \omega \times r_{O' A} + 
    \omega \times (v_{app} + \omega \times r_{O' A})
    \\
    a_{O A} = 
    a_{O O'} + a_{app} + \omega \times (\omega \times r_{o' A})
    + 2 \omega \times v_{app} + \dot \omega \times r_{O' A}
}

$a_{O O'}$ acceleration between $O$ and $O'$

$a_{app}$ apparent acceleration of object as seen in $O'$

$\omega \times (\omega \times r_{O' A})$ is centripetal acceleration. 
for example, if $\omega \perp r_{O' A}$, then $\omega \times r_{O' A} = \omega r_{O' A} e_{\perp}$.
$\omega \times (\omega \times r_{O' A}) = - \omega^2 r_{O' A} e_{O' A}$

$2 \omega \times v_{app}$ is the coriolis effect 

$\dot \omega \times r_{O' A}$ is the Euler Term 


\subsection{Example (cont.)}
recall $v_{O A} = v_{train} + u + \omega \times b$

let $u, v_{train}, \omega = \const$

\equations{
    u = \const \rightarrow a_{app} = 0
    \\
    v_{train} = \const \rightarrow a_{O O'} = 0
    \\
    \omega = \const \rightarrow \dot \omega = 0
    \\
    a_{O A} = 
    a_{O O'} + a_{app} + \omega \times (\omega \times r_{o' A})
    + 2 \omega \times v_{app} + \dot \omega \times r_{O' A}
    =
    \\
    a_{O A} = 
    0 + 0 + \omega \times (\omega \times r_{o' A})
    + 2 \omega \times v_{app} + 0
    \\
    v_{app} = u = \const \sim e_x
    \\
    r_{O' A} = b \sim e_x
    \\
    a_{O A} = - \omega^2 b \vec e_x + 2 \omega u \vec e_y
}
centripetal + coriolis effects 


\section{Dynamics}
N2L in non-inertial reference frames works by adding fictitious forces. 
\[
F_{true} = m a_{O A} - F_{fict}
\]
insert what you find and rearrange 
\[
F_{true} + F_{fict} = m a_{app}
\]


\equations{
    m a_{app} = 
    \\
    F_{true} - m a_{O O'} - m \omega \times (\omega \times r_{O' A})
    - 2 m \omega \times v_{app} - m \dot \omega \times r_{O' A}
    \\
    \textrm{\hspace{1.5cm} elevator, \hp centrifugal, \hp coriolis, \hp azimuthal}
}

\subsection{Puck on Icy Disk}
Rotating disk moving counterclockwise with speed $\omega$. Puck launched from perimeter towards $B$. 
Initial conditions $x(0) = 0$, $\dot x(0) = 0$, $y(0) = -R$, $\dot y(0) = u$

\equations{
    F_{true} = 0
    \hp 
    a_{O O'} = 0
    \hp 
    \dot \omega \times r_{O' A} = 0
    \\
    \textrm{centrifugal force } \omega \times (\omega \times r_{O' A})
    = -\omega^2 (x e_x + y e_y)
    \\
    \vec \omega = \omega \vec e_{z}
    \hp 
    r_{O' A} = x e_x + y e_y
    \\
    \textrm{coriolis force } \omega \times v_{app} = 
    \omega \times (\dot x e_x + \dot y e_y)
    =
    \omega \dot x e_y - \omega \dot y e_x
    \\
    m (\ddot x e_x + \ddot y e_y) = 
    m a_{app} = m \omega^2 (x e_x + y e_y) - 
    2 m \omega (\dot x e_y - \dot y e_x)
    \\
    \ddot x = \omega^2 x + 2 \omega \dot y 
    \hp
    \ddot y = \omega^2 y + 2 \omega \dot x
}
coupled 2nd order diff eq to solve 

use a complex function and it somehow works 
%
\equations{
    \ddot \psi + 2 i \omega \dot \psi - \omega^2 \psi = 0
    \hp
    \psi = x(t) + i y(t) \in \mathbb C
    \\
    \textrm{ansatz $e^{\lambda t}$}
    \\
    \lambda^2 e^{\lambda t} + 2 i \omega \lambda e^{\lambda t} 
    - \omega^2 e^{\lambda t} = 0
    \\
    \lambda = - \frac{2 i \omega}{2} \pm \sqrt{- \omega^2 + \omega^2} = - i \omega 
    \\
    \psi(t) = A e^{- i \omega t} + Bt e^{- i \omega t}
    \\
    \textrm{$A, B$ from initial conditions}
    \\
    A \rightarrow \psi(0) = 0 = x(0) + i y(0) = 0 - iR
    \\
    B \rightarrow \dot \psi(0) = 0 = \dot x(0) + i \dot y(0) = 0 = R \omega + iu
    \\
    \phi(t) = - i R e^{- i \omega t} + (R \omega + iu) e^{- i \omega t}
    \\
    x = \Re (\psi(t)) = R t \omega \cos(\omega t) + (ut - R) \sin(\omega t)
    \\
    y = \mathbb C (\psi (t)) = 
    (ut - R) \sin(\omega t) - Rt \omega \cos(\omega t)
}


\subsection{Motion on Earth}
Let the Earth be a sphere of radius $R$ spinning with radius $\omega$. 
Someone is an angle $\theta$ above the equator

\equations{
    \vec \omega = \omega \vec p 
    \hp
    v_{lab} =  \omega R \cos(\theta)
    \hp 
    \dot \omega = \frac{1.7 ms}{\textrm{century}}
    \approx 0
    \\
    \textrm{relative acceleration } 
    \vec a_{O O'} = \frac{d^2 R}{dt^2} = \omega \times (\omega \times R)
    \\
    |\vec a_{O O'}| = \omega^2 R \cos(\theta)
    =
    0.034 \frac{m}{s^2} \approx 0
    \\
    \textrm{centripetal force } \omega \times (\omega \times r_{O' A}) = \omega^2 r
    \\
    r << R
    \omega^2 r << \omega^2 R << g \hp \textrm{ neglect }
}
only need to consider coriolis force 
\[
m \vec a_{app} = F_{true} - 2 m \omega \times v_{app}
\]

\subsection{Coriolis Force on Earth}
use coordinates $(e, n, u)$ where $e$ is east, $n$ is north, $u$ is upwards 

$\hat p = \cos(\theta) \hat n + \sin(\theta) \hat u$

\equations{
    m \ddot r_{app} = \sum F_{true} + F_{fict}
    \\
    m \ddot r = F_{true} - 2 m \vec \omega \times \dot r 
    \hp 
    F_{true} = -mg \hat u 
    \\
    m \ddot r = - mg \hat u - 2 m \omega \hat p \times \dot r 
    \hp 
    \dot r = (\dot x \hat e + \dot y \hat n + \dot z \hat u)
    \\
    m \ddot r =
    m (\ddot x \hat e + \ddot y \hat n + \ddot z \hat u) = 
    \\
    - mg \hat u - 2 m \omega \left[
        (\dot z \cos(\theta) - \dot y \sin(\theta)) \hat e + 
        \dot x \sin(\theta) \hat n - \dot x \cos(\theta) \hat u
    \right]
    \\
    \ddot x = 2 \omega \dot z \cos(\theta) + 2 \omega \dot y \sin(\theta) 
    \\
    \ddot y = -2 \omega \dot x \sin(\theta) 
    \\
    \ddot z = -g + 2 \omega \dot x \cos(\theta)
}

\section{Perturbation Theory}
Consider something dropping down on the direction with intitial conditions,
$\dot x(0) = \dot y(0) = \dot z(0) = x(0) = y(0) = 0$ and $z(0) = h$

because $F_{cor} << F_{g}$, we can treat it as a small perturbation $\epsilon$
\equations{
    F_{total} = F_{g} + \epsilon F_{cor}
    \hp
    x(t) = x_0(t) + \epsilon x_1(t)
    \\
    y(t) = y_0(t) + \epsilon y_1(t)
    \hp
    y(t) = y_0(t) + \epsilon y_1(t)
    \\
    \textrm{ eom for $F_g + \epsilon F_{cor}$ }
    \\
    \ddot x = 
    2 \epsilon \omega \dot z \cos(\theta) + 2 \epsilon \omega \dot y \sin(\theta) 
    \\
    \ddot y = 
    -2 \epsilon \omega \dot x \sin(\theta) 
    \\
    \ddot z = 
    -g + 2 \epsilon \omega \dot x \cos(\theta)
    \\
    \ddot x_0 + \epsilon \ddot x_1 = 
    2 \epsilon \omega (\dot z_0 + \epsilon \dot z_1) \cos(\theta) + 
    2 \epsilon \omega (\dot y_0 + \epsilon \dot y_1) \sin(\theta) 
    \\
    \ddot y_0 + \epsilon \ddot y_1 = 
    -2 \epsilon \omega (\dot x_0 + \epsilon \dot x_1) \sin(\theta) 
    \\
    \ddot z_0 + \epsilon \ddot z_1 = 
    -g + 2 \epsilon \omega (\dot x_0 + \epsilon \dot x_1) \cos(\theta)
}
treat everything order-by-order in $\epsilon$

% because $\dot x(0) = \dot y(0) = \dot z(0)$, our equation is 

in $\epsilon^0$
\equations{
    \ddot x_0 + \epsilon \ddot x_1 = 0
    \hp
    \ddot y_0 + \epsilon \ddot y_1 = 0
    \hp
    \ddot z_0 + \epsilon \ddot z_1 = -g
    \\
    x_0(t) = 0
    \hp 
    y_0(t) = 0
    \hp 
    z_0(t) = h - \frac{1}{2} gt^2
}

in $\epsilon^1$
\equations{
    \ddot x_1 = 
    -2 \omega \dot z_0 \cos(\theta) +
    2 \omega \dot y_0 \sin(\theta)
    \\
    \ddot y_1 = -2 \omega \dot x_0 \sin(\theta)
    \\
    \ddot z_1 = 2 \omega \dot x_0 \cos(\theta)
    \\
    \dot x_0 = 0
    \hp 
    \dot y_0 = 0 
    \hp 
    \dot z_0 = -gt 
    \hp
    \textrm{ from $e^0$}
    \\
    \ddot x_1 = 2 \omega gt \cos(\theta)
    \hp
    \ddot y_1 = 0
    \hp 
    \ddot z_1 = 0
    \\
    \dot x_1 = \int 2 \omega gt \cos(\theta) dt = 
    \omega gt^2 \cos(\theta) + C_1
    \\
    x_1(t) = \frac{1}{2} \omega gt^3 \cos(\theta) + C_2
}

insert into initial equation (9.58 - 9.59) in $\epsilon^1$
\equations{
    x(t) = x_0(t) + \epsilon x_1(t) =
    0 + \frac{1}{2} \omega gt^3 \cos(\theta) 
    \\
    y(t) = 0
    \\
    z(t) = h - \frac{1}{2} gt^2
}

The thingy moves eastward.

This is because when the object is "up", its actually closer to the equator,
so as it falls to the ground, it goes closer to the axis of rotation while 
keeping its initially faster eastward inertia. 

\subsection{Foucault Pendulum}
If you have a pendulum that's not affected by the rotation of the earth, 
then the earth will rotate under the pendulum and it will go 
in a circle. 

Goal: solve 

Consider a pendulum of length $L$ with an angle $\beta$ and height $h$.

However, this pendulum is 3d so we have to consider every movement direction $d$
\equations{
    d = \sqrt{x^2 + y^2}
    \hp 
    \cos(\beta) = \frac{h}{L}
    \hp 
    \sin(\beta) = \frac{d}{L}
}

use fictitious forces 

\equations{
    m a_{app} = \sum F_{true} + F_{fict}
    \\
    \textrm{get } F_{net} = \sum F_{true} \textrm{ from } F_{net} = -\nabla U
    \\
    U = -mgh = -mgL \cos(\beta)
    \\
    \textrm{assume $\beta$ is small}
    \\
    U = \const + \frac{mg}{2L}(x^2 + y^2)
    \\
    F_{net} = - \nabla U = - \frac{mg}{L} (x \hat e + y \hat n)
    \\
    \textrm{fictitious forces}
    \\
    F_{cor} = -2m \vec \omega \times \vec v_{app}
    \hp 
    \vec \omega = \omega \hat p = 
    \omega (\cos(\theta) \hat n + \sin(\theta) \hat u)
    \\
    v_{app} = \dot x \hat e + \dot y \hat n
    \\
    F_{cor} = 
    -2m \omega 
    \left(
        -\dot x \cos(\theta) \hat u + 
        \dot x \sin(\theta) \hat n -
        \dot y \sin(\theta) \hat e
    \right)
    \\
    \textrm{N2L}
    \\
    \ddot x = - \frac{g}{L} x + 2 \omega \dot y \sin(\theta)
    \\
    \ddot y = - \frac{g}{L} y - 2 \omega \dot x \sin(\theta) 
    \\
    \ddot z = 0 + 2 \omega \dot x \cos(\theta) 
    \\
    \textrm{introduce } \omega_z = \omega \sin(\theta)
    \\
    \ddot x = 
    - \frac{g}{L} x + 2 \omega_z \dot y
    \\
    \ddot y = 
    - \frac{g}{L} y - 2 \omega_z \dot x
    \\
    \ddot z = 0 + 2 \omega \dot x \cos(\theta) 
}

Case 1: $\theta = 0$ (equator) meaning $\omega_z = 0$

simple harmonic oscillator 

\equations{
    \ddot x + \omega_n^2 x = 0
    \hp 
    \omega_n = \sqrt{\frac{g}{L}}
}

Case not 1: do some solving junk 

\equations{
    \psi = x + iy 
    \hp
    \dot \psi = \dot x + i \dot y 
    \hp
    \ddot \psi = \ddot x + i \ddot y 
    \\
    \ddot \psi = 
    \left(
        - \frac{g}{L} x + 2 \omega_z \dot y
    \right)
    + i
    \left( 
        - \frac{g}{L} y - 2 \omega_z \dot x
    \right)
    =
    - \frac{g}{L} \psi - 2 i \omega_z \dot \psi 
    \\
    \ddot \psi + \frac{g}{L} \psi + 2 i \omega_z \dot \psi 
    =
    \ddot \psi + 2 i \omega_z \dot \psi + \omega_n^2 \psi 
    = 0
}
ansatz $\psi = e^{\lambda t}$
%
\equations{
    \lambda^2 + 2 i \omega_z \lambda + \omega_n^2 = 0
    \hp 
    \lambda = 
    \frac{ -2i \omega_z \pm \sqrt{-4 \omega_z^2 - 4 \omega_n^2}}{2}
    =
    \\
    -i \omega_z \pm \sqrt{\omega_z^2 + \omega_n^2}
    \\
    \lambda = i \omega_z \pm i \omega_n
    \\ 
    \psi(t) = e^{i \omega_z \pm i \omega_n}
    =
    e^{-i \omega_z t} (C_1  e^{i \omega_n} + C_2 e^{-i \omega_n})
    \\
    x = iy = e^{-i \omega_z t} (x_{NR}(t) + i y_{NR}(t))
    \\
    e^{i \omega_z t} = \cos(\omega_z t) + i \sin(\omega_z t)
    \\
    \begin{pmatrix}
        x(t) \\ y(t)
    \end{pmatrix}
    =
    \begin{pmatrix}
        \cos(\omega_z t) && \sin(\omega_z t) \\
        - \sin(\omega_z t) && \cos(\omega_z t)
    \end{pmatrix}
    \begin{pmatrix}
        x_{NR}(t) \\ y_{NR}(t)
    \end{pmatrix}
}
rotates counterclockwise by $\phi = \omega_z$

\chapter{Analytical Mechanics}
Lagrangian and Hamiltonian shenanigans 

based on "least action" principle 

introduce configuration for "phase space" with a smooth function $\mathcal L$, 
the Lagrangian.

Define an action that is a functional- integral of the Lagrangian.

Find the extrema of the Lagrangian "stationary points" of the action 
("least action")

Get equations of motion from the stationary point. 

\subsection{Snell's Law and Fermat's Principle}
Snell's Law is how light refracts through a medium. 

Setup: consider 2 media with refraction indices from $A \to B$

Determine the path of light that minimizes the time from $A \to B$.
Special case of Fermat's Principle where light will take the path 
between 2 points that takes the elast time.

We need the relationship between refraction indices and the angles of incidence. 

Speed of light in a medium: $v = c / n$ where $n$ is the refraction index. and 
$c = 3 * 10^8$

\equations{
    t = t_1 + t_2 = \frac{l_1}{v_1} + \frac{l_2}{v_2} = 
    \frac{l_1n_1}{c} + \frac{l_2n_2}{c}
    \\
    l_1^2 = d^2 + h^2 
    \hp 
    l_2^2 = (x - d)^2 + (y - h)^2
}
Find $h$ that minimizes $t$
\equations{
    t = \frac{l_1}{v_1} + \frac{l_2}{v_2} = 
    \frac{l_1n_1}{c} + \frac{l_2n_2}{c}
    =
    \frac{\sqrt{d^2 + h^2} n_1}{c} + \frac{\sqrt{(x - d)^2 + (y - h)^2} n_2}{c}
    \\
    t' = 
    \frac{2h n_1}{c\sqrt{d^2 + h^2}} + 
    \frac{2 (y - h) n_2}{c\sqrt{(x - d)^2 + (y - h)^2}}
    = 0
    \\
    2h n_1 + 
    2 (y - h) n_2
    = 0
    \rightarrow 
    2h n_1 + 
    2 y n_2 - 2 h n_2
    = 0
    \rightarrow 
    \\
    2h n_1 + 
     - 2 h n_2
    = -2 y n_2
    \\
    h = \frac{y n_2}{n_2 - n_1}
}
plus some trig stuff that I took pictures of but did not 
put in my notes 

\section{More Stuff}
introduce generalized coordinates $\{ q_i \} = \{ q_1, q_2, q_3, \ldots \}$
and a generalized velocity $\dot q_i = \frac{d}{dt} q_i$

denote path $\vec q(t)$ with fixed endpoints $q(t_1)$ and $q(t_2)$.

introduce action 
\[
    S[q(t)] = 
    \int^{t_2}_{t_1} \mathcal L(q_i(t), \dot q_i(t),  t)
\]


\subsection{total vs partial derivatives}
$\del$ has explicit dependence.

total derivative of a function $f(q(t), \dot q(t), t)$
\equations{
    \frac{d}{dt} f = \frac{\del}{\del t} f(q, \dot q, t) + 
    \frac{\del q}{\del t} \frac{\del f}{\del q} +
    \frac{\del \dot q}{\del t} \frac{\del f}{\del \dot q}
}

\section{Finding the Least Path}

let $\{\vec q(t), \dot \vec q(t) \}$ be the extremal path. Consider 
small deviations $\delta \vec q(t)$. 

$S[\vec q]$ should be stationary- $\delta S[\vec q] = 0$
\[
\delta S[\vec q] = S[\vec q + \delta \vec q] - S[\vec q]
=
\int^{t_2}_{t_1} dt \mathcal L(q_i + \delta q_i, \dot q_i + \delta q_i, t) - 
\int^{t_2}_{t_1} dt \mathcal L(q_i, \dot q_i, t) 
\]
taylor expand term 1
\equations{
    \mathcal L(q_i + \delta q_i, \dot q_i + \delta q_i, t) 
    = 
    \\
    \mathcal L(q_i, \dot q_i, t) +
    \sum_i \frac{\del \mathcal L(q_i, \dot q_i, t)}{\del q_i} \delta q_i + 
    \sum_i \frac{\del \mathcal L(q_i, \dot q_i, t)}{\del q_i} \delta \dot q_i
    + O(\delta q^2)
    \\
    \delta S[\vec q] = \int^{t_2}_{t_1}
    dt \{
        \sum_i \frac{\del \mathcal L}{\del q} \delta q +      
        \sum_i \frac{\del \mathcal L}{\del \dot q} \delta \dot q 
    \}
    = 0
    \\
    \frac{d}{dt} \left(
        \frac{\del \mathcal L}{\del \dot q} \delta q
    \right)
    = 
    \frac{d}{dt}
    \left(
        \frac{\del \mathcal L}{\del \dot q} \delta q
    \right)
    +
    \frac{\del \mathcal L}{\del \dot q} \delta \dot q
    \\
    \int^{t_2}_{t_1}
    \frac{d}{dt} \left(
        \frac{\del \mathcal L}{\del \dot q} \delta q
    \right) 
    =
    \int^{t_2}_{t_1}
    dt 
    \frac{d}{dt}
    \left(
        \frac{\del \mathcal L}{\del \dot q} \delta q
    \right)
    -
    \int^{t_2}_{t_1}
    \frac{d}{dt}
    \left( 
    \frac{\del \mathcal L}{\del \dot q} \delta q
    \right)
    =
    \\
    \frac{\del \mathcal L}{\del \dot q} \delta q \Big|^{t_2}_{t_1} 
    - 
    \int^{t_2}_{t_1}
    dt 
    \left[
        \frac{d}{dt}
        \left(
            \frac{\del \mathcal L}{\del \dot q}
        \right)    
        \delta q
    \right]
    \hp
    \frac{\del \mathcal L}{\del \dot q} \delta q \Big|^{t_2}_{t_1} 
    = 0
    \\
    \int^{t_2}_{t_1}
    dt 
    \left[
        \frac{d}{dt}
        \left(
            \frac{\del \mathcal L}{\del \dot q}
        \right)    
        \delta q
    \right]
    \rightarrow 
}

idk im missing the derivation but 

\subsection{Euler-Lagrange Euquation}
\[
\frac{\del \mathcal L(q, \dot q, t)}{\del q} - 
\frac{d}{dt} \frac{\del L(q, \dot q, t)}{\del \dot q} = 0
\]

\section{New Lecture}
\equations{
    S[\vec q_i(t)]
    =
    \int^{t_2}_{t_1} dt \mathcal L(q_i(t), \dot q_i(t); t)
}
Where $S$ is defined as the path integral of the system, or the action 
of the system. 

\begin{itemize}
    \item
    Find the Lagrangian and set up the action/path integral 
    \item 
    Identify quantities $(q, \dot q, t, \mathcal L)$
    \item 
    Find and solve the E.L. equation
\end{itemize}
Euler-Lagrange Euquation (again lol)
\[
\frac{\del \mathcal L(q, \dot q, t)}{\del q} - 
\frac{d}{dt} \frac{\del L(q, \dot q, t)}{\del \dot q} = 0
\]

\subsection{Path between 2 points}
Setup:2 fixed points $A(x_1, y_1)$ and $B(x_2, y_2)$ in $\mathbb R^2$

Goal: Find a curve with the shortest length sonnecting $A$ and $B$ 

Consider an infinitesimally small path $dl = dx^2 + dy^2$

The length of the curve is an integral 
\equations{
    L = \int^B_A \sqrt{dx^2 + dy^2}
    \rightarrow  
    \int^B_A \sqrt{1 + (\frac{dy}{dx})^2} dx
    \rightarrow  
    \int^B_A \sqrt{1 + (y')^2} dx
    \\
    q_i = y, \hp \dot q_i = \frac{dy}{dx}, \hp t = x
    \hp 
    \mathcal L = 
    \sqrt{1 + (y')^2}
}
now we plug that into the E-L equation 
\equations{
    \frac{\del \mathcal L(q, \dot q, t)}{\del q} - 
    \frac{d}{dt} \frac{\del L(q, \dot q, t)}{\del \dot q} = 0
    \rightarrow 
    \frac{\del \mathcal L}{\del y} - 
    \frac{d}{dx} \frac{\del L}{\del y'} 
    =
    \\
    \frac{\del}{\del y}
    \sqrt{1 + (y')^2}
    -
    \frac{d}{dx} \frac{\del}{\del y'}  
    \sqrt{1 + (y')^2}
    =
    -
    \frac{d}{dx} \frac{\del}{\del y'}  
    \sqrt{1 + (y')^2}
    =
    \\
    -
    \frac{d}{dx} 
    \frac{y'}{\sqrt{1 + (y')^2}}
    = 0
    \rightarrow 
    \frac{y'}{\sqrt{1 + (y')^2}}
    = c
    \rightarrow 
    \\
    y'
    = c \sqrt{1 + (y')^2}
    \rightarrow 
    y'^2
    = c^2 + c^2 (y')^2
    \rightarrow 
    \\
    y'^2 - c^2 y'^2
    = c^2 
    \rightarrow
    y' = \frac{c}{\sqrt{1 - c^2}}
    = a = \const
    \\
    y = ax + b
}
a straight line woo 

\section{More Calculus of Variations}
functional 
\equations{
    S[\vec q_i]
    =
    \int^{t_2}_{t_1}
    \mathcal L(q_i, \dot q_i, t)
    dt 
}
fix the end points $q(t_1), q(t_2)$

extremize $s[\vec q_i]$ by the path $\vec q$ that 
satisfies the EL equation $\rightarrow$ leads you to 
a differential equation for the system $q(t)$ with boundary conditions.

if $q(t_1)$ is bounded but $q(t_2)$ is undefined, then 
the term not involving 
\[
"\int" : \frac{\del \mathcal L}{\del \dot q_i} 
\delta q_i \Big|^{t_2}_{t_1} \neq 0
\]
thus we need to require that $\frac{\del}{\del \dot q_i} \mathcal L = 0$

\subsection{Higher Derivatives}
what if $\mathcal L$ is $\mathcal L(q, \dot q, \ddot q, t)$?

then some math happens 
\equations{
    \frac{\del}{\del q} \mathcal L - \frac{d}{dt} \frac{\del L}{\del \dot q} + 
    \frac{d^2}{dt^2} \frac{\del L}{\del \ddot q} 
    = 0
}

\subsection{Brachisochrone}
A particle is sliding down a curve without friction but with gravity. 

Goal: Find the path that takes the least time 

First: set up the action/path integral (action)

Our function will be $T$ because we're trying to minimize $T$

\equations{
    dt = \frac{dl}{v} \hp dl^2 = dx^2 + dy^2
}
energy conservation 
\equations{
    \frac{1}{2} mv^2 = mgy \rightarrow 
    v = \sqrt{2gy}
    \\
    T
    =
    \int \frac{dl}{v} = 
    \int^B_A \frac{\sqrt{dx^2 + dy^2}}{\sqrt{2gy}}
    =
    \int^B_A \frac{\sqrt{1 + (\frac{dy}{dx})^2}}{\sqrt{2gy}} dx
    \\
    \mathcal L = \frac{\sqrt{1 + (\frac{dy}{dx})^2}}{\sqrt{2gy}}
    \hp 
    q = y, \hp \dot q = \frac{dy}{dx}, \hp t = x
}
ig that's actually wrong so im just gonna follow lecture 

\equations{
    \mathcal L =  
    \frac{\sqrt{1 + x'^2}}{\sqrt{2gy}}
    \\
    \frac{\del}{\del x}
    \frac{\sqrt{1 + x'^2}}{\sqrt{2gy}}
    -
    \frac{d}{dy}
    \frac{\del}{\del x'}
    \frac{\sqrt{1 + x'^2}}{\sqrt{2gy}}
    = 0
    \rightarrow 
    \\
    \frac{d}{dy}
    \frac{1}{\sqrt{2gy}} \frac{x'}{\sqrt{1 + x'^2}}
    = 0
    \rightarrow
    \frac{1}{\sqrt{2gy}} \frac{x'}{\sqrt{1 + x'^2}}
    = c
    \\
    \frac{x'}{\sqrt{2gy}}
    = c\sqrt{1 + x'^2}
    \rightarrow
    \frac{x'^2}{2gy}
    = c^2 + c^2 x'^2
    \rightarrow 
    x' = \frac{\sqrt{2gy} c}{\sqrt{1 - 2gy c^2}}
    \\
    x(y) = \frac{1}{2c} \sqrt{\frac{2}{gy} - 4c^2}
    +
    \frac{1}{2gc^2} \arctan(\sqrt{\frac{1}{2gyc^2} - 1})
}

\section{Lagrange Multipliers}
What is there is a constraint on your functional?

given a functional $S_1[q_i(t)]$ subject to a global (integral) restraint 
$S_2[q_i(t)]$

functional $S_1$ depending on $x(t), y(t)$ that have a local relationship. 

Goal: minimize $S_1$ while holding $S_2$ fixed 

\equations{
    S_1[q_i(t)] = 
    \int^{t2}_{t1} \mathcal L_1(q_i(t), \dot q_i(t), t) dt 
    \hp \textrm{ is minimized}
    \\
    S_2[q_i(t)] = 
    \int^{t2}_{t1} \mathcal L_2(q_i(t), \dot q_i(t), t) dt 
    =
    \gamma = \const 
}
Use these along with a constant to construct a compound functional 
\equations{
    C[q_i(t)] = S_1[q_i(t)] - \lambda S_2[q_i(t)]
    =
    \\
    \int^{t2}_{t1} (\mathcal L_1(q_i(t), \dot q_i(t), t) - 
    \lambda \mathcal L_2(q_i(t), \dot q_i(t), t)) dt 
    \\
    \delta C = \delta S_1 - \lambda \delta S_2 = 0
    \hp 
    \textrm{ make an E-L equation}
    \\
    \frac{\del}{\del q} (\mathcal L_1 - \lambda \mathcal L_2)
    -
    \frac{d}{dt} \frac{\del}{\del \dot q_i} 
    (\mathcal L_1 - \lambda \mathcal L_2)
    = 0
}
Solve the equation with $2$ integration constants $c$ and $\lambda$.

\subsection{Chain Example}
Consider a chain hanging between 2 points. How does the chain lie to 
minimize potential energy?

consider a linear mass density $\rho = \frac{dm}{dl}$

The chain has start and endpoints $(0, 0)$ and $(a, b)$

\equations{
    dl = \sqrt{dx^2 + dy^2} = \sqrt{(\frac{dx}{dy})^2 + 1} dy
    = \sqrt{x'^2 + 1} dy
    \\
    m = \int \rho dl 
    \rightarrow 
    \\
    U = -mgl = - \int g y \rho dl = - \int g \rho y \sqrt{x'^2 + 1} dy 
}
now make the compount function considering the length of the thingy 
\equations{
    \mathcal L[x(y)] = \int dl = \int^b_0 \sqrt{x'^2 + 1} dy 
    \\
    C[x(y)] = U - \tilde \lambda L = 
    U + g \rho \lambda \sqrt{x'^2 + 1} = 
    - g \rho (y - \lambda) \sqrt{x'^2 + 1}
}
plug it into the Euler-Lagrange equation and hope for the best 
\equations{
    \frac{\del \mathcal L}{\del x} - 
    \frac{d}{dy} \frac{\del \mathcal L}{\del \dot x}
    = 0
    \rightarrow 
    \\
    0 - 
    \frac{d}{dy} \frac{\del}{\del \dot x}
    - g \rho (y - \lambda) \sqrt{x'^2 + 1}
    \\
    - g \rho (y - \lambda) \frac{x'}{\sqrt{x'^2 + 1}}
    =
    \const = k
    \rightarrow 
    x' = \pm \frac{k}{\sqrt{(y - \lambda^2)} - k^2}
    \\
    x = \pm k \ln(\lambda - y + \sqrt{(y - \lambda)^2 - k^2})
    \rightarrow 
    y = k \cosh(\frac{x - k_2}{k}) + \lambda 
    \hp 
    k_2 = k \ln(\frac{1}{k})
}
solve and you'll get hyperbolic cosine 

\section{Local Constraints}
consider functionals of $2$ or more functions that are related to each other 

\equations{
    F[x(z), y(z); z] = \int f(x(z), x'(z) y(z), y'(z), z) dz 
    \\
    g(x(z), y(z); z) = 0
    \\
    \mathcal L = f - \lambda g 
    \\
    C = \int (f - \lambda g) dz
}
that's your new lagrangian, so plug that into the E-L equation and you're good 

\section{Lagrangian Mechanics}
consider a particle in a gravitational field 
\equations{
    \vec g = g \vec e_y 
    \hp 
    m \ddot y = -mg 
    \\
    \frac{\del \mathcal L}{\del q_i}
    - 
    \frac{d}{dt}
    \frac{\del \mathcal L}{\del \dot q_i}
    = 0
    \\
    \mathcal L = \frac{1}{2} m \dot y^2 - mgy 
    \rightarrow 
    \\
    \mathcal L = T - U
}
The Lagrangian for mechanical systems with conservative forces that determine the potential 

\subsection{Caution}
\[T - U \neq T + U\]

THE SIGN MATTERS ALOT 

\[T + U = E = \const\]

\[
T - U = \mathcal L
\]

\subsection{Spring}
Consider a mass $m$ on a spring with $k$ sliding along the $x$ axis 

Goal: Find the equation of motion using the Lagrangian 
\equations{
    F = m \ddot x = - kx 
    \\
    L = T - U
    \hp 
    T = \frac{1}{2} m \dot x^2 
    \hp 
    U = \frac{1}{2} kx^2
    \\
    \mathcal L =  \frac{1}{2} m \dot x^2 
    - \frac{1}{2} kx^2
}
Now just get the Euler-Lagrange equation 
\equations{
    \frac{\del L}{\del x} - \frac{d}{dt} \frac{\del L}{\del \dot x}
    =
    -kx - \frac{d}{dt} m \dot x = 0
    \rightarrow 
    m \ddot x + kx = 0
}
Look at that we got N2L from the Lagrangian 

You get the equation of motion from the Euler-Lagrange equation 




\end{document}