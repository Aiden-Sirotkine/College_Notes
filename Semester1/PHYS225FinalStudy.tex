\documentclass{report}
\usepackage{geometry}
\usepackage{amssymb}
\usepackage{fancyhdr}
\usepackage{multicol}
\usepackage{blindtext}
\usepackage{color}
\usepackage[fontsize=16pt]{fontsize}
\usepackage{lipsum}
\usepackage{pgfplots}
\usepackage{physics}
\usepackage{mathtools}

\setlength{\columnsep}{1cm}
\def\columnseprulecolor{\color{blue}}
\date{Fall 2023}

\newcommand{\textoverline}[1]{$\overline{\mbox{#1}}$}

\newcommand{\hp}{\hspace{1cm}}

\newcommand{\del}{\partial}

\newcommand{\pdif}[2]{ \frac{\partial #1}{ \partial #2} }

\newcommand{\pderiv}[1]{ \frac{\partial}{ \partial #1} }


\title{PHYS225 Final Studyguide}
\author{Aiden Sirotkine}

\begin{document}

\pagestyle{fancy}
\maketitle
\tableofcontents
\clearpage

\chapter{Relativity and Math Applications}

i think ill be fiiiiiiine

but how the fuck do I do Taylor Series huh

I also can make a 2 page formula sheet on top of the formula sheet that already exists

\section{Lorentz Transformations}
A nice equation that allows you to see position and time from a different reference frame. Takes into account the rear-clock-ahead affect, length contraction, and time dilation. The formula is on the sheet.

Proper time is in the frame where the events happen in the exact same location. 

proper length is where both sides of the things are in the rest frame.

\section{What the Fuck is Rapidity}
Rapidity is just a goofy way to make a lorentz transformation look like a rotation matrix just with hyperbolic trig functions. 

All of the regular identities of lorentz transformations are held under the goofy rapidity.

\section{Group Theory}
\begin{itemize}
\item
Identity

There exists an element in the group such that $gI = ig = g$

\item
Inverse

for every element, there exists another element such that $g^{-1}g = gg^{-1}$

\item
Closure

for all elements $g, h$ in the group, $gh$ is in the group.

\end{itemize}

\section{Fourier Transforms}
you guess for f(x, y, z)
\[
f = Ae^{i k_x x} e^{i k_y t} e^{i k_z t}
\]
then u derive that and hope for the best



\section{Midterm Fix}
How the fuck do I do the taylor series of the original thing
\[
\sqrt{\Delta s^2} = \sqrt{c^2t^2 - \Delta x^2} 
= 
(c^2t^2 - \Delta x^2)^{1/2}
=
( - \Delta x^2 + c^2t^2)^{1/2}
=
\]
\[
x(-1 + \frac{c^2 t^2}{x^2})^{1/2}
=
x(1 -  \frac{c^2 t^2}{2 x^2}) = x -  \frac{c^2 t^2}{2 x }
\]

Okay just always turn the thing into some form of $(1 + x)^n$ and use the shortcut

\section{Final Practice Problem}
\[
\frac{640}{600} = 
\sqrt{\frac{1 + \beta}{1 - \beta}}
\rightarrow
1.1377 -1.1377 \beta = 1 + \beta
\rightarrow
\]
\[
0.1377 = 2.1377 \beta
\beta = 
\]
Whoops i didnt use the approximation

\subsection{Energy of a particle}
\[
E = \gamma m = \frac{1}{\sqrt{1 - \beta^2}} m = (1 - \beta^2)^{-1/2}
=
m(1 + \frac{\beta^2}{2})
\rightarrow 
e
=
m
\]
that's all wrong specifically because it isnt beta thats going to 0 but the $1/200$.
\[
\beta = 1 - 1/200 \rightarrow \beta^2 = (1 - 1/200)^2 = 1 - 2/200 
\]




\end{document}