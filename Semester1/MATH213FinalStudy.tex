\documentclass{report}
\usepackage{geometry}
\usepackage{amssymb}
\usepackage{fancyhdr}
\usepackage{multicol}
\usepackage{blindtext}
\usepackage{color}
\usepackage[fontsize=16pt]{fontsize}
\usepackage{lipsum}
\usepackage{xcolor}
\usepackage{enumitem}
\usepackage{amsmath}

\setlength{\columnsep}{1cm}
\def\columnseprulecolor{\color{blue}}
\date{Fall 2023}

\newcommand{\textoverline}[1]{$\overline{\mbox{#1}}$}
\newcommand{\done}{\textbf{\checkmark}}

\newcommand{\hp}{\hspace{1cm}}


\title{MATH213 Final Study Guide}
\author{Aiden Sirotkine}

\begin{document}
\maketitle{}

\tableofcontents

\chapter{Lets FUcking Go EZ Final Ace}


\subsection{Pigeonhole Principle}
if you have $N$ objects in $k$ boxes, then one box contains at least the ceiling of $N/k$ objects.

WHAT THE FUCK IS A PERMUTATION RAAAAAAHHHHHHHHHH

https://www.youtube.com/watch?v=zWy77XbkRF8


\subsection{Binomial Theorem}
\[
(x + y)^n = \sum_{k = 1}^n {n \choose k} x^{n-k}y^k
\]


\subsection{Another useful theorem}
\[
\sum_{k=1}^n {n \choose k} = 2^n
\textrm{ power set identity basically}
\]



\section{Combinatorial Identities}

\subsection{Pascal's Identity}
\[
{n+ 1 \choose k} = {n \choose k-1} + {n \choose k}
\]
Combinatorial proof: youre choosing $n+1$ objects $\{ 1,2, 3 ,\ldots n, \alpha\}$.
 ${n \choose k-1}$ is choosing from the rest of the items because you've picked 
 $\alpha$ and ${n \choose k}$ is picking from the rest of the items because you definitely have not picked 
 $\alpha$
 
 
 \subsection{Vandermonde's Identity}
 \[
 {m + n \choose r} = \sum_{k = 0}^r {m \choose r - k} {n \choose k}
 \]
 Combinatorial proof: Just imagine the two different combinatorics things as different subsets of the same set $m + n$ and you're picking the same amount of stuff from the same set, just at two different times.
 
 \subsection{Hockey Stick Identity}
 \[
 {n+1 \choose r+1 } = \sum_{j=r}^n {j \choose r}
 \]
 Ive proved this before but ${n \choose r}$ is you definitely picked the $n+1$st item and ${n - 1 \choose r}$ is you definitely picked the $n$th item and not the $n+1$st and so on and so forth.
 
\subsection{Stars and Bars}
if given you have to pick $n$ total objects of $r$ different types of objects, then the total way is
\[
{n + r - 1 \choose r-1}
=
{n + r - 1 \choose n}
\]


\subsection{Distinguishable People in Undistinguishable Boxes}
if you have $n$ total objects with $a$ of type 1, $b$ of type 2, etc. the formula is
$\frac{n!}{a!b!c!\ldots}$

Just fucking figure it out permutations are not that difficult.

\subsection{Bernoulli Trials}
Probability of exactly $k$ successes for a given independent thingy and $n$ total trials with success chance $p$ and fail chance $1 - p$ is 
\[
{n \choose k} p^k (1 - p)^{n - k}
\]


\subsection{Expected Value and Variance}
\[
E(x) = \sum_{s \in S}p(s)X(s)
\]
Expected values have some nice properties

\[
V(x)
=
\sum_{s \in S}(X(s) - E(X))^2p(s)
\]
If $X$ is a random variable on a sample space, then
\[
V(X) = E(X^2) - E(X)^2
=
E(X - E(X))^2
\]
Both follow the law of superposition if the probabilities are independent


\subsection{Example Problems}
The amount of ways to permute the letters in SUCCESS is
\[
C(7, 3)C(4, 2)C(2, 1)C(1, 1)
\]
The (7, 3) is 7 open spots for 3 letters (S), then 4 spots for 2 (C), then 2 for 1 (E) and 1 for 1 (U)



\section{Linear Homogeneous Recurrence Relations}
Solve the things, get the solutions, put the solutions as things that go to the nth power. Solve for coefficients based off of initial conditions.

\section{Solving Recurrence Relations with Generating Functions}

\section{Power Series}
\[
1 + x + x^2 + \ldots = \frac{1}{1-x}
\]

Trust the process, or remember the youtube video where the guy put $x^n$ in each recurrence relation for each level $n$ and developed generating functions that way.
\[
a_n = 3a_{n-1}
\rightarrow
G(x) - 3xG(x) = a_0
\rightarrow
G(x) = \frac{a_0}{1 - 3x}
\]
and just solve

\subsection{MORE ON GENERATING FUNCTIONS}
\[
(1 + x + x^2 + \ldots + x^n) = \frac{1 - x^{n+1}}{1-x}
\]
useful for permutation stuff and then you can just use the stars n bars

\[
\left( \frac{1}{1-x} \right)^n = \sum_{k =1}^\infty {n + k - 1 \choose k - 1} x^k
\]
\[
1 + x^2 + x^4 \ldots = \frac{1}{1 - x^2}
\]

the derivative of a generating function is $\sum n x^n$

\newpage
\section{Fixing Midterm 2}
The last proof

Prove that
\[
\sum^n_{k = 1} 
k
{n \choose k}
=
n
2^{n-1}
\]
$2^{n-1}$ is the amount of subsets of a set of cardinality $n-1$. Let $N$ have cardinality $n$. $n 2^{n-1}$ is the total amount of ways to pick an element $n \in N$ and then a subset of $N$ that does NOT contain $n$.

Now we abuse the identity
\[
{n \choose r} = {n \choose n - r}
\]

Let $1 {n \choose 1}$ be the amount of ways you can pick a subset of size $n-1$ and then an element outside of that subset. $2 {n \choose 2}$ is the amount of ways you can pick a subset with cardinality $n -2$ and an element outside of that subset. So on and so forth until you reach $n {n \choose n}$ which is picking a subset of size $0$ and an element in $N$

\subsection{The Generating Function One}
\[
x_1 \leq 2, x_2 \leq 2, x_3 \geq 3, x_4 \geq 3
\rightarrow
x^6(1 - x^3)^2 \left( \frac{1}{1 - x} \right)^4
\rightarrow
\]
\[
x^6(1 - 2x^3 + x^6)\left( \frac{1}{1 - x} \right)^4
\rightarrow
{14 + 4 - 1 \choose 4 - 1} - 2{11 + 4 - 1 \choose 4 - 1} + {8 + 4 - 1 \choose 4 - 1}
\]


\newpage
\subsection{3d}
Find the coefficient of $x^{16}$ in $(2x - 5/x^3)^{100}$
\[
(2x - 5x^{-3})^{100} = 
\]
The binomial theorem is
\[
(x+y)^n = \sum^n_{j = 0} {n \choose j} x^{n-k}y^j
\]
I need to figure out for what $j$ does 
\[
x^{100 - j} + {x^{-3j}} = x^{16}
\rightarrow
100 - j - 3j = 16
\rightarrow
-4j = -84
\rightarrow
j = 21
\]
\[
{100 \choose 21}2^{79} (-5)^{21}
\]


\section{Review Stuff}
Figure out how to prove an isomorphism using a bijection between vertices

For $10.3.26$ just use the exact same bijection that makes the isomorphism between regular $G$ and $H$.


\chapter{Graphs}
\subsection{Handshaking Lemma}
\[
\sum \textrm{deg}(V) = 2 \times \textrm{edges}
\]

for a degree sequence to be graphic the sum of degrees needs to be even and nothing needs to break math.


\section{The Shit with Euler and the bridges}
\begin{center}
path $\subset$ trail $\subset$ walk
\end{center}
A trail is a walk with no edges repeated.

A path is a walk with no vertices or edges repeated
\begin{itemize}
\item
A circuit is a path that starts and ends at the same vertex.
\item
An Eulerian Circuit is a circuit that uses every edge exactly once (can repeat vertices)
\item
A graph has an Eulerian Circuit iff all its vertices have even degrees.

It has an Eulerian Path, but not an Eulerian Circuit iff 2 of its vertices have odd degrees.
\item
A Hamiltonian Circuit is a circuit that goes through every vertex exactly once.
\item
a graph is connected if there exists a path from any vertex to any other vertex.
\item
Strong connected and weakly connected is only important in di-graphs
\item
vertex and edge cuts are kind of self explanatory

\end{itemize}

\subsection{Dirac's Theorem}
Let $G$ be a simples graph with vertices $n \geq 3$. if each vertex has a degree greater than $n/2$ then there exists a hamiltonian circuit.

\subsection{Dijkstra's Algorithm}
Just know it to find the shortest path between 2 nodes in a weighted graph.

\section{Graph Colorings}
bipartite graphs are 2-colorable. A matching is the subset of edges that connect the 2 sides of a bipartite graph

COLORS ARE VERTICES NOT FACES I SWEAR TO FUCKING GOD IF YOU MAKE THIS MISTAKE AGAIN




\section{Graph Representations}

\subsection{Adjacency Matrices}
an adjacency matrix has 0 if there isnt an edge between two vertices and 1 if there is.

\subsection{Incidence Matrices}
rows are vertices, columns are edges. 1 if the edge is connected to a vertex, 0 if othewise.

\section{Isomorphism}
If there exists a 1 to 1 function between each vertex such that vertices in F are adjacent iff theyre adjacent in G.


\section{Planar Graphs}
Graphs that can lay on a plane without any overlaps

\subsection{Euler's Formula}
Let $G$ be a planar graph.
\begin{center}
vertices + faces - edges = 2
\end{center}


\subsection{Kuratowski's Theorem}
A graph is non-planar if it contains a subgraph of either $K_{3, 3}$ or $K_5$

\chapter{Trees}
I don't actually think there was much we went over in like any of the lectures that really covered trees but to be honest I don't give a shit im learning it all anyways.

A full m-ary tree with i internal vertices has mi + 1 total vertices.

Oh SHit figure out minimum weighted spanning trees.

\subsection{Full M-ary Tree}
\begin{itemize}
\item
if $n$ vertices, then $(n-1)/m$ interval vertices and 

$((m - 1)n + 1)/m$ leaves.

\item
if $i$ interval vertices then $mi + 1$ regular vertices and $(m-1)i + 1$ leaves

\item
if $l$ leaves then $(ml - 1)/(m - 1)$ vertices and $(l-1)/(m-1)$ internal leaves

\end{itemize}


\section{Spanning Trees}
A spanning tree of a graph $G$ is a subgraph of $G$ that is a tree and contains every vertex in $G$.

A simple graph is connected iff it has a spanning tree.

if spanning tree is unweighted just use depth first or breadth first search to find a spanning tree. 

\subsection{Minimum Spanning Trees}
in a connected weighted graph, it is a spanning tree that has the smallest possible sum of edge weights.

\subsection{Prim's Algorithm}
Start at one node. add the smallest possible edge every time.




\subsection{Kruskal's Algorithm}
Add the smallest weighted edges that do NOT contain a simple circuit. Stop after $n-1$ edges have been selectied (for $n$ vertices)

 
 \newpage
\section{Miscellaneous}
 
 \subsection{Bubble Sort}
 you know it
 
 \subsection{Selection Sort}
 find smallest, put at beginning, repeat.
 
 \subsection{Merge Sort}
 split up a bunch. merge function actually sort.

\section{Time Complexity of Divide n Conquer}
Let
\[
f(n) = af(n/b) + cn^d
\]
for $a \geq 1$, $n = b^k$ for integer $b$, $c, d$ are real numbers.
\[
f(n)
=
\begin{cases}
O(n^d) \rightarrow a < b^d \\ 
O(n^d\log n) \rightarrow a = b^d \\
O(n^{\log_ba}) \rightarrow a > b^d
\end{cases}
\]

\subsection{Relations}
Equivalence Classes are basically the range of the function

S is totally ordered if the partial ordering of S contains all elements in S. The partial ordering is well-ordered 

\subsection{Hasse Diagrams}
turn a relation into a di-graph



\subsection{Well-Ordering Property}
Every non-empty set of integers has a least element.



\subsection{Extra thing abt Generating Function}
\[
1 + x^2 + x^4 \ldots = \frac{1}{1 - x^2}
\]
















\end{document}