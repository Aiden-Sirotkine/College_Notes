\documentclass{report}
\usepackage{geometry}
\usepackage{amssymb}
\usepackage{fancyhdr}
\usepackage{multicol}
\usepackage{blindtext}
\usepackage{color}
\usepackage[fontsize=16pt]{fontsize}
\usepackage{lipsum}

\setlength{\columnsep}{1cm}
\def\columnseprulecolor{\color{blue}}
\date{Fall 2023}

\newcommand{\textoverline}[1]{$\overline{\mbox{#1}}$}



\title{Intro to Japanese Studies}
\author{Aiden Sirotkine}

\begin{document}

\pagestyle{fancy}
\maketitle
\tableofcontents
%Concepts of Japan, Japanese Culture and the Japanese
\clearpage


\chapter{Concepts of Japan, Japanese Culture, and the Japanese, by Harumi Befu}
Japanese culture is not an objective thing, instead Japanese culture is based off of a bajillion different variables.

People believed that every country had a specific neat culture bounded perfectly within their country.

Globalisation in the 60's changed that. \\ Land $\neq$ people $\neq$ culture $\neq$ society $\neq$ polity

\section*{Japan}
Japan is a big chain of islands vaguely related to each other 

Honsh$\overline{\mbox{u}}$, Ky$\overline{\mbox{u}}$sh$\overline{\mbox{u}}$, Shikoku, Hokkaid$\overline{\mbox{o}}$ and Ry$\overline{\mbox{u}}$ky$\overline{\mbox{u}}$ islands (Okinawa).

\section*{Stratified Japan}
Interpretations of Japan are geographical (for example, the "four seasons" come from mainly central Japan).

However, elites tried to paint japan as a homogeneous nation. Despite proclaiming homogeneity, they still considered non-mainland Japanese people as "others".  This is seen through the use of \textit{naichi} (proper Japan) and \textit{gaichi} (Japanese colonies). Colonies were expected to emulate real Japan. 

\section*{Otherness of Japan}
Other countries depictions of Japan come from their historical relations with the country. 

Frail, Feminine, Exotic as seen in French art.

Treacherous, Sneaky, Without Mercy, Backwards during WW2.

\section*{"Japanese Culture}
\subsection*{essentialism}
\textit{nihonjiron}
uniqueness of Japan, exceptionality.

flawed, failed to recognize heterogeneity of the islands. 

Also no mention of Chinese influence or Westernization. 

\subsection*{de/re-territorialisation}
Japanese emigrants have been found around the world since the 1800's bringing their cultures. 

these communities of emigrants are extensions of Japan. 

\section*{The Japanese}
"typical" Japanese must have 2 Japanese parents, be a native speaker, and embody the culture. Not having any of those is "suspect".

\subsection*{Koreans and Chinese in Japan}
Japanese legal status, but foreign social status. 
\textit{passing}

Don't live in Japan, but are some variety of actually Japanese.

\textit{Naturalized}

abandon their Japanese citizenship and completely identify with another country.

\textit{orphans}

Can become Japanese via legal shenanigans if they have 1 Japanese adoptee parent. Still socially very much foreigners though.

\section*{Conclusion}
Big group of varying peoples.

People try to fit Japan in a box. It does not fit in a box.

"Japan" as an idea is actually from only central Japan. Non-central Japanese places/people are considered "other".

Despite the Japanese elites trying to push the idea of homogeneity to the outside, there still exists mass discrimination in Japan against people not considered "typically" Japanese. 






The Japanese state forces upon the rest of the population "proper" japaneseness (Tokyo dialect).

Ainu (from Hokkaid$\overline{\mbox{o}}$) and Okinawans (from Okinawa) get discriminated against. 

\chapter{Turning Japanese by Andrew Curry}
Basically Japan learned agriculture in the 900's BC instead of the 200's BC and that was a big thing.


\chapter{Early Japan by Nancy K. Stalker}
Japan is very mountainous and 75\% uninhabitable.

\section*{Timeline}
Most of our knowledge of ancient Japan comes from only two places: the Kojiki and the Nihon Shogi.

\indent Paleolithic - 35,000 BC to 15,000 BC

Jom\textoverline{o}n - 15,000 BC to 900 BC (had poetry, but hunter gatherers)

Yayoi - 900 BC to 250 CE (agriculture)

Kofun - 250 CE to 600 CE (large burial mounds)

Modern Japanese have a mix of Jom\textoverline{o} and Yayoi genes

\section{Jomon}
Had preservation and very elaborate pottery

Dog\textoverline{u} = small lil clay figurines

\section{Yayoi}
\textbf{RICE}

decent metallurgy

more boring pottery

social distinctions in burials

D\textoverline{o}taku = funky bells

\section{Kofun}
MASSIVE key-shaped burial mounds

\section{Yamato}
Japan was a federation of a bunch of different states

ruled by mystical sorceress Himiko

\section{Shinto}
Recorded in both the Kojiki and the Nohin Shogi.

Theorized to have been created to legitimize the Yamato Clan's "divine right" to the throne.

Tales of Amaterasu

\subsection{Creation Myth}

Izanagi and Izanami make Japan and have kid

already eaten the food of the underworld??????????

\section{Buddhism}
idk

\chapter{The Kami Tradition by Kojiki}
\begin{itemize}
\item
Shinto is not necessarily completely related to the Japanese Kami, and it is debated among scholars how much Shinto actually is a part of Japanese culture compared to Buddhism. 
\item
Buddhism is very well defined religion, while Shinto is just a bunch of gods that people might or might not worship. 
\item
Buddhists shrines and Shinto temples also became so intertwined that they basically became the same thing: "shrine-temple multiplexes".
\end{itemize}

\section*{Shinto}
Shinto is a big bowl of unorganized practices. Japanese people rarely call themselves Shinto, yet the majority of them still interact in some way or another with shrines related to the kami and Shinto. Kami and buddhist divine people (bodhisattvas) also have become venerated in the same manner.

\section{Tradition as Illusion}
Religion is not always institutionalized.

\subsection*{Sacred Spaces}
\subsection*{Yayoi}
\begin{itemize}
\item
ritual sites were separate from settlements, often at sources of water. The kami would come down from the mountains and attend.
\item
The kami were often related to natural landmarks. Kami are also identified by their power (Mt. Fuji being a literal volcano).

\item
Initially worshipped in open-air, then buildings were buildings.

\item
Very related to growing rice. Religious power and political power were very interconnected.
\end{itemize}


\section{The Rule of Taste}
\begin{itemize}
\item
The Fujiwara gained control of the emperor through external influence
\item
Michinaga Fujiwara is said to be the guy referenced for Genji in the tale of Genji
\item
Japan has a culture of accepting a culture and then staunchly rejecting it for a bit.
\item
Heian's considered a good color palette to be noble
\item 
very gender binaried, but also male beauty was much more feminine than in Europe
\item
kana = "women's hand"
\item
patriarchy
\item
formal love affairs?
\end{itemize}
\subsection{The Tale of Genji}
\begin{itemize}
\item
first novel?
\item
1000AD
\item
PLOT

He's a bastard kid who gets kicked out and fucks his stepmom and grooms his stepmom's niece.


\end{itemize}


\section{Buddhism}
\begin{itemize}
\item
Brought to Japan by Koreans in the 700's AD to become political allies.
\item
source of political power by the elites

\end{itemize}


\chapter*{Week 5 Everything}
I missed both lectures because I am currently dying.






\end{document}
