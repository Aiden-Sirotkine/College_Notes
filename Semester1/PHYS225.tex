\documentclass{report}
\usepackage{geometry}
\usepackage{amssymb}
\usepackage{fancyhdr}
\usepackage{multicol}
\usepackage{blindtext}
\usepackage{color}
\usepackage[fontsize=16pt]{fontsize}
\usepackage{lipsum}
\usepackage{pgfplots}
\usepackage{physics}
\usepackage{mathtools}

\setlength{\columnsep}{1cm}
\def\columnseprulecolor{\color{blue}}
\date{Fall 2023}

\newcommand{\textoverline}[1]{$\overline{\mbox{#1}}$}

\newcommand{\hp}{\hspace{1cm}}

\newcommand{\del}{\partial}

\newcommand{\pdif}[2]{ \frac{\partial #1}{ \partial #2} }

\newcommand{\pderiv}[1]{ \frac{\partial}{ \partial #1} }


\title{PHYS225}
\author{Aiden Sirotkine}

\begin{document}

\pagestyle{fancy}
\maketitle
\tableofcontents
\clearpage

\chapter{Relativity and Math Applications}
Dr. Koptieva is Ukrainian, does gravity research

Venn Diagram Course Plan:

Very intense, might die. 

Special Relativity \hspace{1cm} Symmetries and other junk \hspace{1cm}

\textbf{OFFICE HOURS}

Wednesday 12:30pm-3:30pm in Loomis 276

or email TA

\section*{Course Goals}
\begin{itemize}
\item time dialation, length contraction, Doppler shifts
\item the postulates of relativity and how they imply the Lorentz transformations
\item 4-vectors and the spacetime metric
\item relativistic kinematics, using invariance an energy-momentum 

\item more stuff
\end{itemize}

\section{TEXTBOOKS}
REQUIRED

Special Relativity for Enthusiastic Beginners by David Morin

RECOMMENDED

Basic Training in Mathematics: A Fitness Program for Science Students by R. Shankar

\section{Homework}
Opens Thursday August 24th

Due Thursday August 31st

\section{Midterm}

October 10th, in-class


\chapter{Introduction to Special Relativity}
First written down in 1905 by Albert Einstein. Passed every experimental test for 100 years.

\textbf{OMP}
(One Minute Paper: Write down one topic you are excited to learn about in PHYS225:

I know that gravity is the curvature of spacetime, but I don't actually know anything about it, to learning about spacetime curvature in a more in-depth manner sounds fun.)
\newline

The speed of light is constant within all reference frames which means everything else gets goofy.

Let someone throw a ball on a moving train

\subsection*{Regular Relativity (Galilean Relativity):}

Newton's Laws of Motions are the same in all reference frames.

Time is universal and equal for all observers
\[
v_{ball} = v_{train} + v_{throw}
\]

\subsection{Special Relavity:}
\textbf{EXTREMELY IMPORTANT EQUATION}
\[
v_{ball} = 
\frac
{v_{train} + v_{throw}}
{1+v_{train}v_{throw}/c^2}
\]
Not much changes with small values, however, for large numbers, the equation creates an effective upper bound. No addition of velocities can ever reach the speed of light.

\subsection*{Electromagnets}
Maxwell's Equations 
%$$
%E_{loop} = \int_a^b \frac{d\phi}{dt} \textbf{idk fix later}
%$$
Lorentz Forces
\[
F = qv \times B
\]

These are wholly related to Einstein's special relativity

\subsection{Einstein's Theory of Relativity}
\begin{itemize}
\item
All laws of physics are the same in all inertial reference frames
\item
The speed of light in a vacuum is constant in all inertial frames

\end{itemize}

velocities vary depending on reference frames.

\textbf{OMP}
(One Minute Paper:
What does "special" mean in special relativity?)

Read Morin 1.1-1.2, Appendix F 6.6.1 - 6.6.3

\chapter{The Three Fundamental Effects}
\textbf{OMP}
List one or two examples of systems that you know from PHYS211 that can be used s clocks. In other words, their motion is cyclical with a definite period.

Answer: gravity based swinging pendulum and spring pendulum

\begin{itemize}
\item
define the concepts of reference frame, event, simultaneity, time, and length

\item
Use these concepts to see the three fundamental effects of relativity:

Rear Clock Ahead

Time Dialation

Lorentz (length) Contraction.

\end{itemize}

\subsection{goofy paradox}
You travel at the speed of light and someone shines a laser at you. How fast is it moving?
\[
r_{laser} = \frac{c-c}{1-(c^2/c^2)} = 0/0
\]
That illiegal, which shows how the speed of light is not a valid inertial reference frame.
If \(v < c\), then you will see that the laser's speed is in fact \(c\).

\section{Reference Frame}
What is a reference frame?

Imagine a network of synchronized clocks.

Everyone is the same reference frame sees the same clock/time.

\subsection{Event}
A thing that happens at a specific time and a specific place.

The reading on the clock in S at the location of the event.

\subsection{Simultaneity}
Two events that happen in different locations in space, but at the same time.

\section{How do we measure a time elapsed in S}
We look at the same clock at 2 different times and observe the difference of the readings.

\subsection{How do we measure length in S}
We measure two locations in S at different times

\subsection{Length given constant velocity}
Observe velocity over time.

\subsection*{fuck}
Events that are simultaneous in one reference frame are NOT necessarily simultaneous in another because of constant speed of light.

Imagine a light in the middle of two trains with clocks on either side of the train. to measure $\Delta t$, you do
\[
t_1 - t_2 = \frac{L(c+v)}{2c^2} - {L(c-v)}{2c^2} = \frac{Lv}{c^2} = \textrm{ difference in clocks time}
\]



\section{Time Dialation}
How to measure time elapsed in S' reference frame.
\[
\Delta t = \gamma \Delta \tau
\]
Where t is the time elapsed in S and 
\[
\gamma = \frac{1}{\sqrt{1 - v^2/c^2}} = \textrm{Gamma/Lorentz/boost factor}
\]
\(\gamma\) is always bigger than 1, which means time for someone with a velocity appears slower than someone who is stationary. But velocity is relative to reference frame, so who has slower time? 

They both are when they have to check the clock????? I don't get it. Acceleration fucks up special relativity.


\section{Length Contraction}
\[
\Delta t' = \frac{L}{v}
\]
\[
\Delta t' = \frac{L}{\gamma v}
\]
Planck's something
\[
L_A = v \Delta t = \frac{L}{\gamma}
\]

OMP

Which of the fundamental effects was most surprising?



\chapter{Vectors and Matrices}

OMP: Give an example of a mathematical tool that is useful in physics

Answer: sine and cosine functions with regards to harmonic motion.

\begin{itemize}
\item
Vector Manipulation

\item
Matrix Manipulation

\item 
How Linear Transformations are used in physics

\end{itemize}

\section{Vectors}
Object with magnitude and direction.

Can be decomposed into unit vectors
\[
\vec{v} = v_1\hat{x} + v_2\hat{y} + v_3\hat{z}, \hspace{1cm}
\vec{v} = \langle v_1, v_2, v_3 \rangle
\]
Vectors have no preferred origin.
\subsection*{Vector Functions}
\begin{itemize}
\item
addition
\item
scalar multiplication
\item
dot product
\item
cross product

\end{itemize}



\section{Matrix}
\begin{itemize}
\item
Rectangle of numbers. 
\item
Matrices allow us to transform vectors.
\item
consists of m rows and n columns
\end{itemize}
\[
A\vec{x} =
\begin{bmatrix}
a && b \\
c && d 
\end{bmatrix}
\begin{bmatrix}
v_1 \\ v_2
\end{bmatrix}
=
\begin{bmatrix}
av_1 + bv_2 \\
cv_1 + dv_2
\end{bmatrix}
\]

\subsection{Einstein Summation Notation}
Kronecker Delta $\delta^i_j$
\[
\delta^{ij}= 
\begin{bmatrix}
1 && 0 \\ 0 && 1 
\end{bmatrix}
\]
0 if $i \neq j$.
1 if $i = j$

write v = Mx
\[
M = \begin{bmatrix}
M^1_1 && M^2_1 \\
M^1_2 && M^2_2 
\end{bmatrix}
\]
\[
v_i = M^j_i x_j 
\hspace{1cm}
v_1 = M^1_1x_1 + M^2_1x_2
\hspace{1cm}
v_1 = M^1_2x_1 + M^2_2x_2
\]
Matrix is considered an operator to turn a vector into another vector.

dot product
\[
\vec{A} \cdot \vec{B} = \sum^2_{i, j=1} \delta^{ij}A_iB_j
\]


\chapter{Deriving Lorentz Transformations}
OMP: Name at least one way the Galilean Transformations of Newtonian Mechanics must be modified to be consistent with relativity.
\section{Two Important Postulates of Relativity}
\begin{enumerate}
\item
All physical laws must be the same in all reference frames $S$ and $S'$.
\item
The speed of light is constant in all reference frames.
\[
x=ct, x' = ct', \rightarrow x-ct = 0, x' - ct' = 0 \rightarrow x'-ct' = (x - ct)
\]

\end{enumerate}


Holy Fucking Shit I Fucking Give Up I have no idea what is happening


\chapter{Matrix Multiplication and Taylor Series}
OMP:

We often make approximations for small quantities, but "small" is a dimensionless concept. How would we define a "small' (dimensionless) velocity in relativity?

\begin{itemize}
\item
Learn how to multiply two matrices
\item
Relate matrix multiplication to linear transformations
\item
other stuff I didn't get in time

\end{itemize}
For a matrix $M$ and a vector $v$
\[
Mv = \begin{bmatrix} av_1 + bv_2 \\ cv_1 + dv_2 \end{bmatrix}
\]
Multiplying a $2 \times 2$ matrix is treating the second matrix as a bunch of vectors.
\[
M = 
\begin{pmatrix}
a && b \\ c && d
\end{pmatrix}
N = 
\begin{pmatrix}
p && q \\ r && s 
\end{pmatrix}
\]
\[
MN = 
\begin{pmatrix}
ap+br && aq + bs \\
cp + dr && cq + ds 
\end{pmatrix}
\hspace{1cm}
NM
\begin{pmatrix}
ap+cq && bp + dq \\
ar + cs && br + ds 
\end{pmatrix}
\]
Einstein notation is actually very useful here but I didnt fUCKING WRITE IT IN TIME
\[
MN = M^i_j N^j_i
\]

imagine person S'' with velocity $v_2$ on train $S'$ with velocity $v_1$ on ground S
\[
(x''^0, x'''^1) \overset{\Delta_v1}{\rightarrow} (x'^0, x''^1) \overset{\Delta_v2}{\rightarrow}(x^0, x^1)
\]
\[
\begin{pmatrix}
x^0 \\ x^1
\end{pmatrix}
=
\Delta_{v1} \Delta_{v2}
\begin{pmatrix}
x''^0 \\ x''^1
\end{pmatrix}
\]
In HW you will show how multiplying these two matrices gets you the velocity addition formula
\[
\frac{\beta_1 + \beta_2}{1 + \beta_1 \beta_2}
\]
\[
\begin{pmatrix}
x'^0 \\ x'^1 
\end{pmatrix}
=
\Delta_{v1}
\begin{pmatrix}
x''^0 \\ x''^1 
\end{pmatrix}
=
\begin{pmatrix}
\gamma_1 && \gamma_1 \beta_1 \\
 \gamma_1\beta_1 && \gamma_1
 \end{pmatrix}
\begin{pmatrix}
x''^0 \\ x''^1 
\end{pmatrix}
\]
\[
\begin{pmatrix}
x^0 \\ x^1 
\end{pmatrix}
=
\Delta_{v2}
\begin{pmatrix}
x'^0 \\ x'^1 
\end{pmatrix}
=
\begin{pmatrix}
\gamma_2 && \gamma_2 \beta_2 \\
 \gamma_2\beta_2 && \gamma_2
 \end{pmatrix}
\begin{pmatrix}
x'^0 \\ x'^1 
\end{pmatrix}
\]
\[
\begin{pmatrix}
x^0 \\ x^1
\end{pmatrix}
=
\begin{pmatrix}
\gamma_2 && \gamma_2 \beta_2 \\
 \gamma_2\beta_2 && \gamma_2
 \end{pmatrix}
\begin{pmatrix}
x'^0 \\ x'^1 
\end{pmatrix}
=
\]
\[
\begin{pmatrix}
\gamma_2 (\gamma_1x''^0 + \gamma_1 \beta_1 x''^1) + 
\gamma_2 \beta_2 (\gamma_1 \beta_1 x''^0) + \gamma_1 x''^1)
\\
\gamma_2 \beta_2 (\gamma_1 x''^0 + \gamma_1 \beta_1 x''^1 + 
\gamma_2 (\gamma_1 \beta_1 x''^0 + \gamma_1 ''^1))
\end{pmatrix}
\]
=
\[
\begin{pmatrix}
(\gamma_2 \gamma_1 + \gamma_2 \beta_2 \gamma_1 \beta_1) x''^0 + 
(\gamma_2 \gamma_1 \beta_1 + \gamma_2 \beta_2 \gamma_1) x''^1
\\
(\gamma_2 \beta_2 \gamma_1 + \gamma_2 \gamma_1 \beta_1)x''^0 +
(\gamma_2 \beta_2 \gamma_1 \beta_1 + \gamma_2 \gamma_1)x''^1 +
\end{pmatrix}
\]

\subsection{EINSTEIN NOTATION}
\[
x'^\mu = (\Delta_{v1})^\mu_\nu x''^\nu
\]
\[
x^\sigma = (\Delta_{v2})^\sigma_\mu ((\Delta_{v1})^\mu_\nu x''^\nu )
\]


\section{Taylor Series}
\[
f(x) = f(a) + f'(a)(x-a) + \frac{1}{2}f''(a)(x-a)^2 + \cdot = \sum^\infty_{n=0} \frac{f^{(n)}(a)}{n!}(x-a)^n
\]
Example
\[
e^x = 1 + x + \frac{x^2}{2!} + \frac{x^3}{3!} + \cdots + \frac{x^n}{n!}
\]
\[
\sin x = x - \frac{x^3}{3!} + \frac{x^5}{5!} - \frac{x^7}{7!} + \cdots
\]
\[
\cos x = 1 - \frac{x^2}{2!} + \frac{x^4}{4!} + \cdots
\]

What about things that arent defined everywhere such that $\ln 0$.
\[
\ln 1+x = x - \frac{x^2}{2} + \frac{x^3}{3} - \frac{x^4}{4} = \cdots
\]
Goofy approximation
\[
\sin x = x \textrm{ for small angles x (error proportional to $x^3$)}
\]
Can multiply series together
\[
\sin^2 x = \left(x - \frac{x^3}{3!} + \frac{x^5}{5!} - \frac{x^7}{7!} + \cdots \right)^2
=
x^2 - \frac{x^4}{3} + \frac{2x^6}{45} + \cdots
\]
Common variations
\[
\frac{1}{1+x} = \frac{d}{dx} \ln (1 + x) = \frac{d}{dx} \textrm{sum} = 1 - x + x^2 + \cdots
\]
\[
\frac{1}{1-x} = 1+x+x^2+x^3 + \cdots
\]
\[
(1+x)^\alpha = x + \alpha x + \cdots \textrm{ binomial expansion}
\]
OMP:

Differentiate the first two terms of the Taylor series for $\sin x$ to obtain the first two terms of the Taylor series for $\cos x$




\chapter{Symmetrics and Group Theory}
OMP:

How owuld you define the concept of symmetry?

\begin{itemize}
\item
Raise and lower indices with contractions

\item
Learn the concept of transpose, inverse, and determinant

\item
Define an orthogonal matrix as one which preserves the dot product

\item 
Define a mathematical group 

\item 
Show that Lorentz boosts along a single axis form a group.

\end{itemize}

\section{Types of Vectors}
Column vector:
\[
\vec{v} = \begin{pmatrix} v_1 \\ v_2 \end{pmatrix}
\]

Row vector:
\[
\vec{r} = \begin{pmatrix} r_1 && r_2 \end{pmatrix}
\]

Convention/Notation:
\[
M^i_j \textrm{ $i$ is column, $j$ is rows}
\]

Dot product is really just the product of a row vector and a column vector via matrix multiplication.

Matrix multiplication in the form of dot products
\[
w \cdot v = w_i v^i \hspace{1cm} Mv = M^i_{\,\,j} v^j
\]
Fancy way to write dot products
\[
w \cdot v = \delta*{ij}w_iv_j = w_i(v_j \delta^{ij})
\]
Theres some shenanigans about northeast that I missed.

can also raise the index on $w$
\[
w \cdot v = \delta*{ij}w_iv_j = v_i(w_j \delta^{ij}) = w^jv_j
\]
Other stuff

Dot product is commutative so $v \cdot w = w \cdot v$. and Kronecker delta is symmetric so $\delta^{ij} = \delta^{ji}$


You can also have delta with lower index
\[
w \cdot v = \delta_{ij}w^iv^j
\]


\chapter{Invariant Interval and Spacetime Metric}
OMP:

Draw a pair of coordinate axes to represent $x$ and $ct$, and draw the path of a light ray on those axes.

Everything gets fucked up in different reference frames, gee would it be nice to have a quantity thats the same in every reference frame.
\[
\Delta s^2 \equiv
c^2(\Delta t^2)
-
(\Delta x^2)
-
(\Delta y^2)
-
(\Delta z^2)
\]
Not actually square! Just a thing called an invariant interval.

Synchronize Coordinate Systems
\[
\Delta s^2 = 
(x^0)^2
-
(x^1)^2
-
(x^2)^2
-
(x^3)^2
\]
For Lorentz boosts along the x-axis

Try to find numbers such that the Lorentz boosted stuff equals the non-boosted stuff.
\[
ct' = \gamma ct - \gamma \beta x,
\hspace{1cm}
x' = \gamma \beta ct - \gamma x
\]
Do algebra to get
\[
(ct')^2 
- 
(x')^2 = 
(ct^2) 
- 
x^2
\]
Works in all combinations of all directions but the equation for just $x$ motion is convenient

Lorentz Transformations and rotations are similar and both have invariant thingies I think

\[
\Delta s^2 = (x^0)^2 - \vec{x} \cdot \vec{x}
\]


\[
\Delta s^2 > 0 \textrm{ timelike separation}
\]


You can do stuff and things to make locations the same and only vary in time. Pick $v = x/t$
\[
\Delta \tau = t' = \sqrt{\Delta s^2}/c
\]

If $c^2t^2 < x^2$ then you cannot do a Lorentz transformation because $x/t > c$

What we do is make them happen at the same time, but separated in space
\[
\Delta L = x' = \sqrt{- \Delta s^2}
\]
Space like events can never effect each other, so you can always find a frame such that one happened before the other (both happened before each other??)

$\Delta s^2 = 0$ = Light-like separation

\section{Minkowski Spacetime Diagram}


\chapter{MIDTERM SHENANIGANS}
Taylor series will be on the text pretty extensively 

probably something with Lorentz transformations forming a group (the lorentz transformation matrix)

Taylor Series for Lorentz transformations (because v/c is close to 0)

Office hours on Monday
\[
\gamma = \frac{1}{\sqrt{ 1 - \beta^2}}, \hspace{1cm} \beta << 1
\]
Find $\gamma$ for $\beta \approx 0$ Using a goofy Taylor Series.



\chapter{Relativistic Energy, Momentum, and Force}
OMP:

Define force and energy in classical mechanics in terms of momentum, mass, and/or time derivatives.

\section{Momentum}
Momentum is useful because it is conserved in the absence of external forces. We want to define momentum is special relativity so that it remains conserved at relativistic speeds.
\[
u_{B,S} = \frac{u_0}{\gamma}
\]
This means something surely. Define relativistic momentum as:
\[
p = \gamma m v
\]
I am so tired i genuinely could not get relativity at the moment if i wanted to
\[
\Delta p_{y,B} = \frac{2mu_0}{\gamma}
\]
Force is just the regular force equation but you have to use relativistic momentum
\[
F = \frac{d\mathbf{p}}{dt}
\]
\[
\frac{d}{dt}(\gamma m v) = m(\overset{\cdot}{\gamma}v + \overset{\cdot}{v}\gamma)
\]
Do some bullshit calculus
\[
F = \gamma^3 ma
\]
im gonna kms

\section{Relativistic Energy}
\[
\Delta E = \int \mathbf{F} \cdot d \mathbf{l}
\]
line integral
\[
\int^{v_f}_0 m \gamma^3 v \, dv
\]

Kinetic Energy??
\[
KE = (\gamma - 1)mc^2
\]
that $mc^2$ looks awfully familiar

sdfg';sdf
hjsiupkjf,chzgxndsfz;j,mnadsf;
\[
M = 2 \gamma_u m
\]

\section{Acceleration and Rapidity}
\[
v(r) + dv = \frac{v(r) + a \, d\tau}{1 + ad\tau v(r)/c^2}
\]



\chapter{Vector Shenanigans}
\[
p = \gamma m v
\hp
F = \frac{dp}{dt}
\]
You can do the same thing for multiple directions
\[
F = \langle F_x, F_y \rangle
\hp
\vec{F} = m \langle \gamma^3a_x, \gamma a_y \rangle
\]
She derived it on the blackboard using a bunch of funky calculus

Know gradient and curl which are both things you already know. Also use other coordinate systems. I am not terribly worried I'll be completely honest.



\chapter{4-Vector}
OMP:

Remind yourself of the Dopper Effect in classical mechanics. If a fire truck is driving towards you, do you hear the siren at a higher or lower frequency?

Answer:

Higher frequency.


\section{Time 4 vector}
\[
x'^\mu = \Lambda^\mu_\nu x^\nu \iff \nu'^\mu \Lambda^\mu_\nu
\]


\chapter{Relativistic Collisions}
OMP: 

Write down the energy-momentum 4-vector for a particle of mass $m$ in its own rest frame, in natural units where $c = 1$


\section{Energy Unit}
Electron volt
\[
E_k = qU \rightarrow 1 eV = 1.6 * 10^{-19} C \times 1 V = 1.6 * 10^{-19} J
\]
\[
E_0 = mc^2 \iff E_0 = m
\hp
m_e = 511 keV
\hp
m_p = 938 MeV
\]

Different types of decay (pion, beta, alpha)
\[
\pi^0 \rightarrow \gamma + \gamma 
\hp
n \rightarrow p + e^- + \overline{\nu}_e
\hp
U \rightarrow Th + He
\]
Smash particles together as well (matter anti-matter anhiilliation, Higgs boson)
\[
e^+ + e^- = \gamma + \gamma
\hp
p + \overline{p} = H
\]

Velocity is ugly and unfun, use energy and momentum instead and something good might happen. 

Use invariance of relativistic dot products. Compute the same quantity in 2 different frames, and set the result equal.

A charged pion of mass $M$ decays most of the time into a muon and a neutrino. The neutrino is so light that we can approximate it as massless. What is the energy of the outgoing neutrino in the rest frame of the pion?
\[
\pi^+ \rightarrow \mu^+ + \nu
\]
Step 1 i sassign energy-momentum 4-vector to each of the particles
\[
p^\mu_1 = p^\mu_2 + p^\mu_3
\]
\[
p_1 = (M, \vec{0})
\hp
p_2 = (E_2, \vec{p_2})
\hp
p_3 = (E_3, \vec{p_3})
\]
We don't care about the direction of momentum in the 4-vectors, just energy.

Do some math and get a dot product
\[
p_1 - p_3 = p_2
\]
\[
(p_1 - p_3) \cdot (p_1 - p_3) = p_2 \cdot p_2
\]
\[
p_1 \cdot p_1 + p_3 \cdot p_3 - 2p_1 \cdot p_3 = p_2 \cdot p_2
\]
\[
p_1 \cdot p_1 = M^2
\hp
p_2 \cdot p_2 = m^2
\hp
p_3 \cdot p_3 = 0 \textrm{ bc massless}
\]
\[
p_1 \cdot p_3 = ME_3 - 0 \cdot p_3 = ME_3
\]
\[
M^2 - 2ME_3 = m^2
\]
\[
E_3 = \frac{M^2 - m^2}{2M}
\]
Makes sense, right dimensions, if the particles are massless then $E_3 = M/2$ and if $m > M$ then you get a negative answer which makes sense because you'd be turning a particle into a heavier particle which breaks the laws of physics.

Two fast enough protons can create a new particle with enough energy
\[
p + p \rightarrow p + p + p - p
\]
The antiproton has the same mass $m_p$ as the proton. What is the minimum amount of energy required for the reaction to occur?

(
Key strategy: use both the laboratory frame and the center-of-momentum frame (CM), where the total momentum is zero)
)

\[
p^\mu_1 + p^\mu_2 \rightarrow k^\mu_1 + k^\mu_2 + k^\mu_3 + k^\mu_4
\]
\[
p_1 = (E_{lab}, \vec{p}_{lab} )
\hp
p_2 = (m_2, \vec{0})
\]
\[
p_1 \cdot p_1 = p_2 \cdot p_2 = m^2_p
\hp
p_1 \cdot p_2 = E_{lab}m - \vec{0}\vec{p} = E_{lab}m
\]
We could find the boost to figure out the CM frame but we dont have to
\[
\vec{p}_1 + \vec{p}_2 = 0
\hp
\vec{k}_1 + \vec{k}_2 + \vec{k}_3 + \vec{k}_4 = 0
\]
Because of conservation of momentum and the direction of the original protons is opposite so they cancel out.
\[
(p_1 + p_2)^2 = (k_1 + k_2 + k_3 + k_4)^2
\]
Evaluate left side in lab frame and right side in CM frame
\[
(p_1 + p_2) \cdot (p_1 + p_2) = p_1 \cdot p_1 + p_2 \cdot p_2 + 2p_1 \cdot p_2 = 2m^2_p + 2E_{lab}m_p
\]
\[
4m_p \cdot 4m_p - \vec{0} \cdot \vec{0} = 16m^2_p
\]
\[
2m^2_p + 2E_{lab}m_p = 16m^2_p \longrightarrow
E_{lab} = 7m_p
\]

If you have to know the velocity of a particle
\[
p^\mu = (E, \vec{p}) = (\gamma m, \gamma m \vec{v})
\Longrightarrow
v = p/E
\]
If you need to find the velocity of the CM frame in the rest frame
\[
p^\mu_{tot} = p^\mu_1 + p^\mu_2 + p^\mu_3 + \ldots = (E_{tot}, \vec{p}_{tot})
\Longrightarrow
v_{CM} = \frac{\vec{p}_{tot}}{E_{tot}}
\]

OMP:

In our first example of pion decay, how would you rearrange $p_1 = p_2 + p_3$ to find the energy of $p_2$ (the muon)?



\chapter{Vector Calculus and Maxwell's Equations}
OMP:

Which of Maxwell's equations involve a surface integral? Which ones involve a line integral?


Thank the lord I've already done calc 3 this shit ez.

\section{Line Integrals}
Imagine how a vector field acts upon an object as it moves along a line
\[
\lim_{\Delta t \to 0} \sum^{n}_{i = 1} \vec{F}(\vec{r}(t_i)) \cdot \Delta \vec{r}_i
=
\int_C \vec{F} \cdot d\vec{r}
\]
Use the fundamental theorem of calculus and some shenanigans with vector fields to get
\[
\int_C \nabla f \cdot d\vec{r}
\equiv
f(b) - f(a)
\textrm{ for a curve going from a to b}
\]
Line integrals are path independent. The line integral for a closed line along any conservative vector field is 0.

\section{Surface Integral}
\[
\lim_{\Delta u, v  \to 0} \sum^{}_{i} \vec{F}(u, v) \cdot \vec{n}(u_i, v_i)_i \Delta u \Delta v
\equiv
\int_S \vec{F} \cdot d\vec{S}
\]
\[
\int_S \vec{F} \cdot d\vec{S}
=
\iint
\vec{F}(\vec{r}(u, v)) 
\cdot
\left(
\pdif{\vec{r}}{u}
\times 
\pdif{\vec{r}}{v}
\right)
du \, dv
\]
Where the goofy cross product is a normal vector relative to the surface
\[
\int_R df = \oint_{\partial R} f
\]
\[
\int_V (\div \vec{F})d^3x
=
\oint_{\del V} \vec{F} \cdot dS
\]
\section{Stoke's Theorem}
\[
\int_S (\curl \vec{F}) \cdot d\vec{S}
 = 
 \oint_{\del S} \vec{F} d\vec{r}
\]
\[
\int_S (\curl (\nabla f)) \cdot d\vec{S} 
=
 \oint_{\del S} \nabla f \cdot d\vec{r} = 0
\]
The the curl of a gradient is ALWAYS 0

\section{More Index Notation}
\[
\epsilon^{ijk} = 
\begin{cases}
1, (ij) = \textrm{ even permutation of (1, 2, 3)} \\
-1, (ijk) = \textrm{ odd permutation of (1, 2, 3)} \\
0 \textrm{ otherwise}
\end{cases}
\]
This is called the (totally antisymmetric) Levi-Civita symbol. Swapping two adjacent number sturns a permutation from even to odd and vice versa.

Curl in index notation:
\[
\curl F
\equiv 
\epsilon^{ijk} \del_j \vec{F}_k
\]
\[
\int_V (\div (\curl A)) d^3x 
=
\oint_{\del V} (\curl A) \cdot d\vec{S}
=
\oint_{\del \del V} A???
=
0
\]
Divergence of a curl MUST be 0

Consider the vector field $F = \frac{1}{r^2} \hat{r}$ in spherical coordinates.

compute divergence
\[
\div F
=
\frac{1}{r^2}
\pdif{r^2F_r}{r}
=
\frac{1}{r^2}\pderiv{r}(1) 
= 
0
\]

but if we compute the divergence on a general ball $B_R$ of radius $R$, we get not 0
\[
\int_V \delta^{(3)}(x)d^3x 
=
\begin{cases}
1 \textrm{, V contains the origin} \\
0 \textrm{ otherwise}
\end{cases}
\]
\section{Gauss's Law}
\[
\frac{Q_{enc}}{\epsilon_0} = \oint E \cdot dS = \int \div E dV \rightarrow \div E = \frac{\rho}{\epsilon_0}
\]
\[
0 = \oint B \cdot dS
\]
\section{Faraday's Law}
\[
- \frac{d}{dt} \int B \cdot dS = \oint E \cdot dr = \int (\curl E) \cdot dS = -\pdif{B}{t} dS
\]
OMP:

To apply Stoke's Theorem, do we want to integrate a curl as a line integral or a surface integral?


\chapter{Complex Numbers and Fourier Transforms}
OMP:
In what areas of physics or math have you encountered complex numbers before, or is this your first time being introduced to them?


I know what a complex number is.
\[
z = x + iy
\]
Makes sure that polynomials have the sufficient amount of answers

You can add complex numbers by just combining like terms kinda similarly to a vector.

The best thing about multiplying complex numbers is that $i^2 = -1$ so you always get just a regular complex number instead of some shenanigans.

To divide complex numbers, you have to multiply the top and bottom of the fraction by the conjugate of the denominator to make is nice and good.
\[
\textrm{conjugate} \equiv z* \equiv \overline z
\]
multiplying a number by its conjugate gets you the modulus or the magnitude of z
\[
|z| = z* \times z = \sqrt{x^2 + y^2}
\]


if $|z| = 1$ then we get nice shenanigans
\[
z = x + iy = \cos \theta + i \sin \theta
\]
proof via taylor series nonsense on the blackboard

This gets you
\[
e^{i \pi} = -1 
\]
You can also use this to take fractional powers of $i$

\[
z = re^{i \pi} \hp z* = re^{-i \pi}
\]

And we can use this trig shenanigans to graph complex numbers in polar form

We can derive more goofy trig identities in complex forms which is currently being done on the whiteboard.

Can be used to solve PDE's???

Given some PDE
\[
\frac{ \del^2 f(t, x)}{\del t^2} 
=
v^2 
\frac{\del^2 f(t, x)}{\del x^2}
\]
Guess a solution
\[
f = Ae^{iwt}e^{ikx} ???
\]
Do some stuff and things to get something.

Get an integral of sdjilan j
\[
f(x, t) = \int^\infty_{-\infty} = dk \overset{\sim}{f} (k)e^{ikx} e^{ivkt}
\]
Also works in more dimensions????
\[
(\nabla^2 + C)f = 0
\]
Where that thing is the Lapacian operator.

OMP:
write $i$ in polar form


\chapter{Maxwell's Equations and Special Relativity}
OMP: 
name 1 topic from each category that you learned about in this course for the first time

Maxwell's Equations
\[
\div E = \frac{p}{\epsilon_0}
\hp
\div B = 0
\hp 
\curl E = -\pdif{B}{t}
\hp
\curl B = \mu_0 J + \mu_0 \epsilon_0 \pdif{E}{t}
\]
Take curl of both sides of third equation to get
\[
\curl \left( - \pdif{B}{t} \right) = - \pdif{}{t} (\curl B)
=
-\mu_0 \epsilon_0 \pdif{^2 E}{t^2}
\]


Fourier transform w 4d wave vector
\[
E(t, \vec x) = E_0 e^{-i \vec k_\mu \vec x^\mu}
\]
Einstein has done some goofy nonsense talking about how electric fields and magnetic fields are the same and everything is fucking goofy

OMP: What was your favorite topic covered in this class?










\end{document}