\documentclass[fleqn]{report}
\usepackage{geometry}
\usepackage{amssymb}
\usepackage{fancyhdr}
\usepackage{multicol}
\usepackage{blindtext}
\usepackage{color}
\usepackage[fontsize=16pt]{fontsize}
\usepackage{lipsum}
\usepackage{pgfplots}
\usepackage{physics}
\usepackage{mathtools}

\setlength{\columnsep}{1cm}
\def\columnseprulecolor{\color{blue}}
\date{Spring 2024}

\newcommand{\textoverline}[1]{$\overline{\mbox{#1}}$}

\newcommand{\hp}{\hspace{1cm}}

\newcommand{\del}{\partial}

\newcommand{\pdif}[2]{ \frac{\partial #1}{ \partial #2} }

\newcommand{\pderiv}[1]{ \frac{\partial}{ \partial #1} }

\newcommand{\equations}[1]{
\begin{gather*}
#1
\end{gather*}
}


\title{PHYS 214}
\author{Aiden Sirotkine}

\begin{document}

\pagestyle{fancy}
\maketitle
\tableofcontents
\clearpage

\chapter{Intro to Quantum Physics}
idk bro I skipped half the class to play violin

\chapter{Waves}
I know what a wave is
\[
y(t) = A \cos (kx - \omega t + \phi)
\]

\[
I = \frac{A^2}{2}
\]

\section{Phasor}
Use the imaginary projection of a wave and add them like vectors to figure out interference.


\chapter{Interferenec}
You can use phasors and the law of cosines 

$(A_{tot} = A_1 + A_2 + 2A_1 A_2 \cos \theta)$ to figure out the amplitude of the resultant waves.

The inside of the wave function $A \cos \alpha$ will be 
\[
\alpha_{tot} = \pi - (\alpha_2 - \alpha_1)
\Rightarrow
(\alpha_2 - \alpha_1)
\Rightarrow
kr_1 - kr_2 + \varphi_2 - \varphi_1
\]

\[
A_{tot}^2 = A_1^2 + A_2^2 + 2A_1 A_2 \cos (k(r_1 - r_2) + \varphi_2 - \varphi_1)
\]

\[
f_1 = A \cos (\frac{2 \pi r_1}{\lambda} -  \omega t)
\hp
f_2 = A \cos (\frac{2 \pi r_2}{\lambda} -  \omega t)
\]
The phase difference between $f_1$ and $f_2$ is
\[
2\pi \frac{(r_2 - r_1)}{\lambda} + \varphi_0
\]

\[
A_{tot} = A^2 + A^2 + 2 A^2 \cos ( \frac{2 \pi}{ \lambda} (r_2 - r_1)a - \omega t)
\]


\section{More Goofy Interference}
\[
\Delta \varphi = (kr_2 - \omega t + \varphi_2) - (kr_1 - \omega t + \varphi_1)
\]
\[
I = 2A^2 \cos^2( \varphi/2)
\]
\[
\varphi = \frac{2 \pi (r_2 - r_1)}{\lambda} + \varphi_2 - \varphi_1
\]
Don't trust any formula I put in these notes.

constructive interference when $(r_2 - r_1) = m \lambda, m \in \mathbb{Z}$


\chapter{Diffraction}
When light goes in a small slit, it starts bending around and then interfering with itself, making a wave pattern

"Diffraction is interference when the number of sources goes to infinity"

-Nikolas
\[
tan(\theta) = y/L
\hp
a \sin \theta = \lambda
\textrm{ (first wave minimum)}
\]

For circular apertures
\[
\sin \theta_0 = 1.22 \lambda / D
\]
where D is the diameter of the aperture



\chapter{Photons}
\[
E = hf
\hp
p = \frac{h}{\lambda}
\]
$p$ is momentum, $h$ is Planck's constant.

Energy is unit is usually the electron-volt $eV = 1.6 * 10^{-19} J$ bc 1 electron$ = 1.6 * 10^{-19}$ Coulombs.

Its the energy an electron gains/loses when its potential changes by 1 volt

\section{Photoelectric Effects}
Electrons boink off metal when it gets hit with enough light
\[
hf = \phi + KE
\]
$\phi$ is the work function (the energy it takes to release the electron)
\[
\hbar = \frac{h}{2 \pi}
\]

\chapter{Probabilities and Complex Numbers}
How to reconcile probability and intensity
\[
P = IA = Nhf
\]
That is the power absorbed by a detector

The absorbed number of photons is
\[
N = P_{\textrm{detector}} *N_{\textrm{emit}}
\]

\section{Probability Density}
Given a probability density function, you can find the probability a particle is in between a and b via
\[
P(a < x < b) = \int^b_a p(x) \, dx = \int^b_a | \Psi |^2 \, dx
\]

Linear Intensity = power per unit area.
\[
\hbar
\]


\section{Complex Numbers}
You know how complex numbers work.



\chapter{Wave Function}
$\Psi (x, t)$ is a complex number
\[
p(x) = |\Psi (x, t)| = \Psi (x, t)\Psi^* (x, t)
\]
Take the integral over an interval to find the probability the particle is in that interval.
\[
p(a < x < b) = \int^b_a |\Psi (x, t)|^2 dx
\]
particles have wavelengths as well because fuck you
\[
\Psi (x, t) = e^{i(kx - \omega t)} = \cos (kx - \omega t) + i \sin (kx - \omega t)
\]
\subsection{De Broglie Wavelength}
\[
P = \frac{h}{\lambda} = \hbar k
\]
It all has superposition as well
\[
\Psi (x, t) = A (e^{ikr_{1}} + e^{ikr_{2}})e^{-i \omega t}
\hp
\Psi^* (x, t) = A (e^{-ikr_{1}} + e^{-ikr_{2}})e^{i \omega t}
\]
\[
|\Psi(x, t)|^2 = A^2 \left( 1 + e^{ik (r_1 - r_2)} + e^{ik(r_2 - r_1)} + 1 \right)
\]
\[
 A^2 \left( 2 + e^{ik (r_1 - r_2)} + e^{-ik(r_1 - r_2)} \right)
 \rightarrow
 A^2 (2 + 2 \cos (k(r_1 - r_2))) \in \mathbb{R}
\]
\[
2A^2 (1 + 1 \cos (k(r_1 - r_2))) \textrm{ intensity formula}
\]
\[
E = hf = \hbar \omega
\]
Given particle with wavelength $\lambda = \frac{2 \pi}{k}$

ignore

\subsection{Probabiltity and Intensity}
\[
p_{avg}(y) \propto \cos^2 \frac{k(r_2 - r_1)}{2}
\hp
I_{avg}(y) \propto \cos^2 \frac{k(r_2 - r_1)}{2}
\]


\chapter{Momentum and Position}
\[
E = hf = \hbar \omega
\hp
p = \frac{h}{\lambda} = \hbar k
\]
\[
\Psi (x, t) = Ae^{i(kx - wt)}
=
A \cos (kx - wt) + iA \sin (kx - wt)
\]
\[
\Psi (x, t ) = Ae^{i (\frac{p}{\hbar} x - \frac{E}{\hbar} t)} \ 
\]
The wave function for $p = \hbar k$ is $e^{ikx}$

Probability function is a constant for $\Psi (x, t) = Ae^{i(kx - \omega t)}$
\[
\Psi (x, t) = Ae^{i(kx - \omega t)} Ae^{i(px - E t)/ \hbar}
\]
Sodmething about superpositions.

They have discrete positions with distinct probabilities
\[
\hbar k_1 \rightarrow \frac{|a|^2}{|a|^2 + |b|^2}
\hp
\hbar k_2 \rightarrow \frac{|b|^2}{|a|^2 + |b|^2}
\]

Localized wave functions are superpsoitions of moentum eigenstates. The mor elocalized ,the more eigenstates are required. This is summarize in the relationship 
\[
\Delta x \Delta p \geq \hbar / 2 
\]

Position can be anywhere but momentum has a discrete number of values depending on eigenstates and superpositions.

When we measure momentum the wave function collapses and turns from a superpositions of states to just
\[
\Psi_{k_1} = e^{ik_1x}
\]

\section{Localized Waves}
more eigenstates means more uncertainty


\chapter{Time Independent Schroedinger equation}
\[
\frac{- \hbar^2}{2m} \frac{\del^2 \Psi (x)}{\del x^2}
+
U(x) \Psi (x)
=
E \Psi (x)
\]
$\hat{h}$ = hamiltonian operator
\[
\hat{h} \Psi = E \Psi
\hp
A \vec{x} = \lambda \vec{x}
\]
The wave function must be an energy eigenstate for the hamiltonian operator to work

\section{Momentum Eigenstate}
\[
-i \hbar \frac{d \Psi(x)}{dx} = p \Psi(x)
\hp
\Psi = Ae^{ikx}
\]
\[
\hat{p} = -i \hbar \frac{d}{dx}
\longrightarrow
\hat{p} \Psi = p \Psi
\]
\[
\hat{p} \Psi 
=
 -i \hbar \frac{d}{dx}
\left(
Ae^{ikx}
\right)
\]
\[
=
-i \hbar A (ik) e^{ikx}
=
\hbar k A e^{ikx}
=
\hbar k \Psi(x)
\]
It all works with $E = p^2 / 2m$
\[r
E = \frac{p^2}{2m} = \frac{\hbar^2 k^2}{2m}
\]
\[
\frac{\hat{p}^2}{2m} = \frac{\hbar^2}{2m} \frac{d^2}{dx^2}
\]
and then you can use the eigenvalue properties of $\Psi(x)$ and $\hat{p}$ to get the k's

Waves with definite momentum have definite energy

\section{Schroedinger Equation}
Wave functions that satisfy this equation are energy eigenstates and have definite energy.

They are also stationary, so probability does not change with time

particles have discrete energy eigenstates and cannot take on all the energies.

\section{Infinite Square Well}
Potential Energy $U(x)$ is infinite on 2 sides and 0 in the middle

The wave function is 0 where $U(x)$ is infinite and $\Psi (x) = \sin(kx)$ where $U(x)$ is 0

\subsection{Boundary}
continuous function 
\[
k = \frac{n \pi}{L}
\]
goofy node shenanigans

\subsection{Normalization}
\[
\Psi(x) = C \sin (kx) 
\hp
k = \frac{n \pi}{L}
\]
What could $C$ be to make sure the integral over everywhere is 1?
\[
C = \sqrt{\frac{2}{L}}
\]


\chapter{Harmonic Oscillator}
\[
\Psi(x, t) = A(e^{i(kr_1 - \omega t)} + e^{i(kr_2 - \omega t)})
\]
That's not a harmonic oscillator
\[
U = \frac{1}{2} k x^2
\hp
kx = \ddot{x}
\hp
x(t) = A \cos (\omega t + \phi)
\hp
\omega = \sqrt{\frac{k}{m}}
\]
To have a quantum harmonic oscillator you need
\[
\frac{-\hbar^2}{2m} \pdif{^2 \Psi(x)}{x^2} + U(x) \Psi (x) = E \Psi (x)
\]
and $U = kx^2/2$. Potential is parabola shaped 
\[
\frac{-\hbar^2}{2m} \pdif{^2 \Psi(x)}{x^2} + \frac{kx^2}{2} \Psi (x) = E \Psi (x)
\]

\section{Ground State of a Harmonic Oscillator}
Do some Schroedinger nonsense with guess equation
\[
\Psi (x) = Ae^{-2ax^2}
\]
And get
\[
E = \frac{1}{2} \hbar \sqrt{ \frac{k}{m}}
=
\frac{ \hbar^2 a}{m}
\hp
a = \sqrt{ \frac{km}{2 \hbar^2}}
\]
\[
E_0 = \frac{1}{2} \hbar \omega = \frac{1}{2} \hbar \sqrt{ \frac{k}{m}}
\]
smaller mass = greater energies. larger spring constant = greater energy

Given a quantum number $n$
\[
E = \left( n + \frac{1}{2} \right) \hbar \omega
\hp
\omega = \sqrt{ \frac{k}{m}}
\]
Each energy state is equally spaced with spacing $E_n - E_{n-1} = \hbar \omega$

\section{Many Particles}
Literally just remember the shit u learned in AP chem.

Electron Spin = $\pm 1/2$

QHO = Quantum Harmonic Oscilattor

SINGLE Electron in QHO has energy $\Psi_n : E = (n + 1/2) \hbar \omega$

MANY ELECTRONS must have a unique wave functions

\subsection{Pauli Exclusion Principle}
2 electrons per level (one 1/2 spin one -1/2 spin)

\subsection{Aufbau Principle}
Electrons fill lowest energies first

total energy is just adding the energies of the individual electrons together.

\section{Atoms}
Some awful 3d wave function
\[
\Psi (r, \theta, \phi)
\]
Use Hamiltonian something or other
\[
H \Psi_{n, l, m} (r, \theta, \psi) = E_n \Psi_{n, l, m} (r, \theta, \psi)
\]
To get
\[
U(r) = \frac{-ke^2}{r}
\]
Goofy shapes depending on energy level which you remember from AP Chem

some fuckin s p d orbital nonsense

higher n means more nodes in the waves
\[
E_i = E_f + E_{\textrm{photon}}
\]


\chapter{Band Structure}
Infinite square wells have a discrete number of possible eigenstates inside the well and potential is just $\infty$ outside of the well so $\Psi (x)$ is 0
\equations{
\Psi(x) = \sqrt{\frac{2}{L}} \sin \left( \frac{n \pi x}{L} \right)
=
 \sqrt{\frac{2}{L}} \frac{1}{2i} \left(  e^{\frac{i n \pi x}{L}} - e^{\frac{-i n \pi x}{L}} \right)
\\
K_n = \frac{n \pi}{L}: n = 1, 2, 3 \ldots
\\
P(E = E_n) = 1
\hp
E_n = \frac{\hbar^2 n^2 \pi^2}{2m L^2}
\\
P(p = + \hbar k) = \frac{1}{2}
\hp
P(p = - \hbar k) = \frac{1}{2}
}

for a large enough square well, there are so many eigenstates that energy is basically continuous energy and momentum 
$ E = \frac{p^2}{2m}$

Electron density is electrons per unit length

Conductors have 0 band gap

Semi-conductors have some gap

Insulators have big gap

\subsection{Fermi Level}
Highest occupied level of a system by an electron
\[
E_{max} = \frac{\hbar^2 n^2_{max} \pi^2}{2 m L^2} 
= 
\frac{ \hbar^2 \pi^2}{2 m} \left( \frac{ N_e/2}{L} \right)^2
=
\frac{\hbar^2 \pi^2}{8m} \left( \frac{N_e}{L} \right)
\]

Electron density $n_e = \frac{N_e}{L}$ where $N_e$ is the total number of electrons

\[
E_{max} = \frac{\hbar^2 \pi^2}{8 m} n_e^2
\]

If we assume wiggles inside an infinite square well (from Coulomb Interaction of atomic ion something or other)
\[
U(r) \approx \frac{-1}{r}
\]
Approximate by adding corrugation
\[
V(x) = 2U \cos (gx)
\]
$U$ is interaction strength between electrons and atomic ions

$g$ is whatever constant is necessary such that the wavelength is $2 \pi / g$


If $U = 0$, then you just get a regular infinite square well

As $U$ become non-zero, when start getting funky something or other (energy gaps)

Our quadratic function has been split into two \textbf{bands} split by \textbf{band gaps}

The band gaps size is the exact distance between something in our crystal

$g \iff k$ is crystal momentum

Every important part of the band gap is within $\pm \hbar g$

The Fermi levels within a band gap chart are determined by electrons per atom

If the photon energy is larger than the band gap, then the photon can be absorbed by the material

For example, silicon has a band gap energy $E_g < \frac{hc}{750 nm}$


\chapter{Polarization and Spin}

\section{Polarization}
horizontal and vertical directions
\[
\Psi = a \Psi_v + b \Psi_h
\]
with chances
\[
P(v) = \frac{|a|^2}{|a|^2 + |b|^2}
\hp
P(h) = \frac{|b|^2}{|a|^2 + |b|^2}
\]
everything is in unit vectors

professor goofy something
\[
P_{i \to f} = |\Psi_i* \cdot \Psi_f|^2
\]
example
\[
P_{i \to f} = (0 \Psi_v + 1\Psi_f) \cdot (1 \Psi_v + 0Psi_f) = 0
\]


\subsection{Circular Polarization}
\[
\Psi = \frac{\Psi_v \pm i \Psi_h}{\sqrt{2}}
\]
imaginary polarization

always a 50\% chance that it goes through any filter

\section{Electron Spin}
Angular momentum of $\pm \hbar / 2$

stern-gerlach filters can block certain spins and x-axis orientations













\newpage
\chapter{Formulas}
\[
e^{ix} + e^{-ix} = 2 \cos (x)
\hp
e^{ix} - e^{-ix} = 2 i \sin (x)
\]

\section{Easiest Way to Integrate $\sin^2(x)$}
\[
\sin^2 (x) + \cos^2 (x) = 1
\hp
%\int^b_a sin^2(x) = \int^b_a \cos^2(x)
\]
\[
\int \sin^2 (x) = \int \frac{1 - \cos (2x)}{2}
=
\frac{1}{2} \left( x + \frac{\sin(2x)}{2} \right)
\]

\end{document}