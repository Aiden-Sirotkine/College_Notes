\documentclass[fleqn]{report}
\usepackage{geometry}
\usepackage{amssymb}
\usepackage{fancyhdr}
\usepackage{multicol}
\usepackage{blindtext}
\usepackage{color}
\usepackage[fontsize=16pt]{fontsize}
\usepackage{lipsum}
\usepackage{pgfplots}
\usepackage{physics}
\usepackage{mathtools}

\setlength{\columnsep}{1cm}
\def\columnseprulecolor{\color{blue}}
\date{Spring 2024}

\newcommand{\textoverline}[1]{$\overline{\mbox{#1}}$}

\newcommand{\hp}{\hspace{1cm}}

\newcommand{\del}{\partial}

\newcommand{\pdif}[2]{ \frac{\partial #1}{ \partial #2} }

\newcommand{\pderiv}[1]{ \frac{\partial}{ \partial #1} }

\newcommand{\equations}[1]{
\begin{gather*}
#1
\end{gather*}
}


\title{PHYS 213}
\author{Aiden Sirotkine}

\begin{document}

\pagestyle{fancy}
\maketitle
\tableofcontents
\clearpage

\chapter{Intro to Thermal Physics}

\section*{conservation of energy??????????????}
\[
W = \int \vec{F} \, dx
\]

sfhfjdvsddjvjsh

pressure is forc iver an area

\[
dU = dQ - pdV
\]

\subsection{Heat Transfer}
\[
\Delta U_{hot} < 0 \hp \Delta U_{cold} > 0
\]

\section{Some basic ass AP chem specific heat}

\section{Entropy}
Based on the amount of possible ways a thing can be arranged (permutations)

When rolling 2 dice, the sum of 7 has the highest entropy

\subsection{Process}
\begin{itemize}
\item
compute S as a function of macrostate variables
\item
Write down what can change and what cannot
\item
Maximize S with respect to the variables that can change
\end{itemize}


\subsection{Equilibrium Positiion of a particle}
\[
S = N k \ln V + f(U, N)
\]

\equations{
\frac{dS}{dV_1} = 0 = \frac{dS_1}{dV_1} + \frac{dS_2}{dV_2} \frac{dV_2}{dV_1}
}


\subsection{Equations/Relationships}
\equations{
\frac{1}{T} = (\frac{dS}{dU})_{V, N}
\\
C_V = (\frac{dQ}{dT})_{V, N}
\\
\Delta S = \int^{T_{f}}_{T_i} \frac{C_V(T)}{T} dT
\\
\textrm{important}
\\\
\Delta U = \int^{T_{f}}_{T_i}  C_V (T) dT
\\
dS = \frac{dU}{T} + \frac{P}{T} dV = \frac{dQ}{T}
}
\section{Ideal Gas Law}
\[
pV = NkT
\rightarrow
pV = N \frac{m(|v|^2)}{3}
\]
\section{Quasistatic Processes}
\begin{itemize}
\item
isothermal = constant temp
\item
isochoric = constant volume
\item
isobaric = constant pressure
\item
adiabatic = $\Delta S = 0$
\end{itemize}

Isothermal processes are reversable

Isobaric processes are not

\subsection{PV Diagrams}
pressure on y-axis, volume on x-axis
\[
W_{on} = - \int p(V) \, dV
\]

\section{Heat Engines}
heat gas to push piston, do work to put piston back.

\[
dS = \frac{dQ}{T} \rightarrow \Delta S = - \frac{Q_h}{T}
\]

\subsection{Efficiency}
\[
\epsilon = \frac{W_{by2} + W_{by4}}{Q_h}
\]

\equations{
\frac{Q_H}{Q_C} \leq \frac{T_H}{T_C}
\\
\epsilon = \frac{W_{by}}{Q_H} = \frac{Q_H - Q_C}{Q_H} \rightarrow
\\
\epsilon = \frac{W_{by}}{Q_H} \leq 1 - \frac{Q_C}{Q_H}
}

\subsection{Heat Pumps/Refrigerators}
COP is coefficient of performance 
Heat pump
\[
COP = \frac{Q_H}{W_{ON}} \leq \frac{1}{1 - \frac{T_C}{T_H}}
\]

Refrigerator
\[
COP = \frac{Q_C}{W_{ON}} \leq \frac{1}{\frac{T_H}{T_C} - 1}
\]

\section{Gibbs Free Energy}
Easier formula to figure out whether something is reversible
\[
G = U - T S + pV
\]
\[
G = \frac{dU}{dX} + p \frac{dV}{dX} - T\frac{dS}{dX}
\]

\subsection{Chemical Potential}
\[
\mu_A = \frac{dG_A}{dN_A}
\]
\[
G = \mu N
\]
Minimize Free Energy
\[
\frac{d(G_A + G_B)}{dN_A} = 0
\hp
N_A + N_B = N_{tot}
\]

\section{Something with Processes}
\equations{
dS = \frac{1}{T} dU + \frac{p}{T} dV - \frac{\mu}{T} dN
\\
G = U + pV - TS
\\
dU = TdS - pdV + \mu dN
\\
dG = Vdp + \mu dN - S dT
\\
d \mu_x = \frac{V_x}{N_x} dp - \frac{S_x}{N_x} dT
\\
\frac{V_x}{N_x} = \textrm{ density}
\hp
\frac{S_x}{N_x} = \textrm{ entropy per particle}
}
The reasoon that materials change in the order of solid-liquid-gas is because the density increases with each of those which increases chemical potential

\section{Ideal Solutions}
\[
\frac{N_{sugar}}{N_{water}} \propto e^{\frac{\Delta}{kT}}
\]

It costs energy to dissolve particles in a solution

its all microstates
\[
\Omega(N) = \frac{N_{water}}{N} 
\]
so if you add salt (not water molecule)

\[
\Delta S(N + 1) - \Delta S(N) = k \ln (\frac{N_{h2O}}{N})
\]

Assume solid molecule don't interact. Assume constant energy cost per salt added no matter the number of salt

\equations{
G(N + 1) - G(N) = \Delta_{sol} - Tk \ln (\frac{N_{H2O}}{N}) = 0
\\
\Delta_{sol} = Tk \ln (\frac{N_{H2O}}{N})
\rightarrow
\frac{\Delta_{sol}}{kT} = \ln (\frac{N_{H2O}}{N})
\rightarrow
e^{\frac{\Delta_{sol}}{kT}} = \frac{N_{H2O}}{N}
\rightarrow
\\
\frac{N}{N_{H2O}} = e^{\frac{\Delta_{sol}}{kT}}
}
doesnt actually work, but is proportional to
\[
\frac{N}{N_{H2O}} = n_s e^{\frac{\Delta_{sol}}{kT}}
\]
You have you figure out $n_S$ yourself

\subsection{Goofy Silly Graph}
slope of graph is proportional to $- \Delta{sol}$

\subsection{Effects on Phase Diagrams}
lowers freezing point, increases boiling point.

Basically just makes everything want to be a liquid

\section{Boltzmann Factors}
microstate probability is dependent on temperature
\[
P(e) \approx e^{\frac{-E}{k T}}
\]
equilibrium means minimized Gibbs free energy
\equations{
E = \frac{1}{2} mv^2 + mgy
\\
P(h) = \frac{e^{\frac{-E}{kT}}}{Z}
\\
P(h_1)/P(h_2) = e^{mg (h_1 - h_2)}{kT}
}
assume temperature does not change with height

\subsection{Centrifuge}
its literally the same as gravity, just sideways

\subsection{More Boltzmann Factor Stuff}
\equations{
f_i = e^{\frac{-E_i}{kT}}
}

\subsection{Quantum Mechanics}
\equations{
P(i) = \frac{ e^{\frac{-E_i}{kT}}}{ \sum e^{\frac{-E_i}{kT}}}
}

increase heat to make all possible microstates equal

max entropy is $k \ln 2$

\section{Semiconductors}
Remember Band Gaps from 214

That except semiconductors have a valence band that doesnt conduct electricity at all and a higher energy conduction band, so if you add energy and get electrons in the conduction band, then it conducts

\equations{
\rho = \frac{N_{conduction}}{N_{atoms}} = e^{\frac{- \Delta}{2kt}}
}

insulator = large $\Delta$

\subsection{Neutrality}
\equations{
N_e = N_C
\\
n_e = \frac{1}{Z} \exp(\frac{- \Delta}{2kt})
}


\chapter{Final Study}

\subsection{Changing Boiling Point}
\equations{
d \mu_{liquid} = d \mu_{gas}
\\
V_l dp - S_l dT = V_g dp - S_g dT
\\
(V - V) dp = S - S dT
}

\section{All Sortsa Boltzmann Factors}
\equations{
\textrm{resistivity } = e^{-\Delta / 2kT} / Z
\\
p = e^{-E/kT} / Z
\\
p = e^{-mgh/kT} / Z
}




\end{document}