\documentclass[fleqn]{report}
\usepackage{geometry}
\usepackage{amssymb}
\usepackage{fancyhdr}
\usepackage{multicol}
\usepackage{blindtext}
\usepackage{color}
\usepackage[fontsize=16pt]{fontsize}
\usepackage{lipsum}
\usepackage{pgfplots}
\usepackage{physics}
\usepackage{mathtools}
\usepackage[makeroom]{cancel}
\usepackage{ulem}
\usepackage{esint}

\geometry{a4paper, margin=2cm} % Set paper size and margins
\graphicspath{ {../Images/} }
\setlength{\columnsep}{1cm}
\addtolength{\jot}{0.1cm}
\def\columnseprulecolor{\color{blue}}
\date{Fall 2025}

\newcommand{\textoverline}[1]{$\overline{\mbox{#1}}$}

\newcommand{\hp}{\hspace{1cm}}

\newcommand{\const}{\textrm{const}}

\newcommand{\del}{\partial}

\newcommand{\pdif}[2]{ \frac{\partial #1}{ \partial #2} }

\newcommand{\pderiv}[1]{ \frac{\partial}{ \partial #1} }

\newcommand{\comment}[1]{}

\newcommand{\equations} [1] {
\begin{gather*}
#1
\end{gather*}
}

\newcommand{\numequations} [1] {
\begin{gather}
#1
\end{gather}
}

\newcommand{\twovec}[2]{ 
\begin{pmatrix}
#1 \\ 
#2
\end{pmatrix}
}

\title{ME 487}
\author{Aiden Sirotkine}

\begin{document}

\pagestyle{fancy}
\maketitle
\tableofcontents
\clearpage

\chapter{ME 487}
The lab safety training is already up and I have to do that before the first lab.

You'll make a pressure sensor, and then a microfluidic mixer (because mixing small
fluids is hard).

\textbf{THERE WILL BE POP QUIZZES IN LECTURE}

\section{LAB POLICY}
\begin{itemize}
    \item
    \textbf{NO SHORTS, CONTACTS, OPEN TOED SHOES}
    \item 
    Don't touch anything unless told so 
    \item 
    If you do anything stupid you get kicked out.
\end{itemize}

\section{Importance}
MEMS are important because they take advantage of forces that scale dramatically 
at small scales 

\begin{itemize}
    \item 
    Surface tension scales with $l$ 
    \item 
    Fluid/electrostatic forces scale with $l^2$
    \item 
    weight/inertia forces scale with $l^3$
    \item
    electromagnetic forces scale with $l^4$
\end{itemize}

So at very small scales, these forces act very differently than in 
macro-scale systems. You can almost completely ignore gravity.

\subsection{Cantilever}
Deflection under self weight goes down dramatically with $l^2$.

Resonant frequency \textbf{increases} with $l^{-1}$.

\subsection{Fabrication Scaling}
\begin{itemize}
    \item 
    Devices are on a single substrate, and you can make 
    thousands of devices on a single substrate. 
    \item 
    We can take infrastructure/technology from the semiconductor industry. 
\end{itemize}

\subsection{Function Integration}
Both electrical and mechanical functions can be obtained 
with the same materials and processes. 

Projectors use a MEM that has a mirror, tilting mechanism, 
and electronic controller all on 1 substrate. 

\subsection{Material Saving}
Because MEMS are on the atomic scale, you use close to 
nothing on material saving. There is functionally 
0 material cost. 

\section{Difficulties}
\begin{itemize}
    \item 
    Really small $(10^6)$
    \item 
    Only very very specific materials actually work in lithography
    \item 
    All tools are only planar- we can only edit from the 
    top/bottom of the wafer. 
    \item 
    You need very specific/expensive facilities and materials.
\end{itemize}

\section{Typical Process}
\begin{enumerate}
    \item 
    Deposition
    
    You put a target material onto a substrate
    \item 
    Lithography

    You put a photoresisting material onto that target, and you 
    use lasers to get rid of material to leave a pattern. 
    \item 
    Etching 

    You get rid of the target material with lasers, and all the target 
    material under the photoresist is untouched. 

    You then remove the photoresist material and then you're left with 
    just the target material in just the pattern you wish. 
\end{enumerate}


\subsection{MUMPs (Multi-User MEMS Processes) Sequence}
This is a 7 layer process of depositing, adding a photoresist, and etching. 

This esseentially allows a company to make multiple different MEMS blueprints 
on a single wafer. 

Often used by fab-less companies. They buy a small bit of a wafer from a 
fab company and have that company etch a specific blueprint onto that 
small portion of wafer. 

\subsection{Finish Fabrication}
\begin{itemize}
    \item 
Singulation: Cut the large wafer 
into little dies such that each individual die has 1 MEM.
    \item 
    Packaging: 

    Bond the functional pieces of the MEM with a wire so that it can 
    work in a larger system. 
    \item 
    Encapsulation

    seal the MEM so it becomes a functional black box with wires for 
    input/output.
\end{itemize}

\chapter{Cleanroom Procedure}
Something something something review.

We're usually going to be working in the $\sim 5 \mu m$ range for our 
MEMS building. 

\section{Wafer Contamination}
If you leave your wafer in the cleanroom uncovered, it will eventually 
get contaminated with dust particles. The amount of time it takes for 
your device to get contaminated is probabilistic. A more clean cleanroom
will mean the chances your device gets contaminated in a certain amount of 
time decreases.

As we consider smaller and smaller dust particles, they move more and 
more randomly due to the random motion of air particles (Cunningham 
Correction Factor). Particles move at a speed dependent on some large 
convoluted equation. 

Possible contaminants include dust particles, organic films, and atoms/ions. 

\subsection{Humidity}
Fluid condensation is bad. 

Water can contaminate your device, and as the water evaporates, the surface 
tension will warp your device (cantilever). 

You can calculate the forces caused by the surface tension of water stuck 
in your cantilever.

\section{Cleanroom Itself}
You have to wear a whole bunch of silly clothes 

The yellow room is for photolithography. 

The white room is for deposition and etching. 

\subsection{Chase}
A part of the cleanroom only available to staff. 

New device will be brought in through the chase.

\chapter{Chemicals}
All sorts of chemicals and they're all flammable and carcinogenic so 
don't touch anything or you'll die.

\begin{itemize}
    \item 
    Solvents
    \item 
    Photoresists (Carcinogen)
    \item 
    Developers (Weak base)
    \item 
    Strippers (acetone)
    \item 
    Etching (Acids)
\end{itemize}

\section{Materials Safety Data Sheet (MSDS)}
It gives you a bunch of information about certain chemicals 

\begin{itemize}
    \item 
    Chemical breakdown
    \item 
    Methods of exposure
    \item 
    Effects/risks of exposure
    \item 
    other 
    \item 
    other 
\end{itemize}

16 different important bits of information on every MSDS. 

There is a physical binder and a search engine that gives you the MSDS 
to every chemical in the cleanroom.

\section{NFPA Diamond}
It's the square with 4 parts that tells you everything about a certain chemical 
in a container.

Yellow is for reactivity, red is for flammability, blue is for health hazards, 
white is for miscellaneous. 

White: OXY, ACID, ALK, COR, W-, RAD 

\section{PPE}
You need certain equipment BEYOND THE CLEANROOM GOWN to work with certain 
chemicals. 

HF spills have to be treated with special care.

\subsection{Do's and Dont's}
Do 
\begin{itemize}
    \item 
    change gloves whenever dirty/broken
    \item 
    Use fresh gloves
    \item 
    use cleanroom paper
    \item 
    remove rings and bracelets
\end{itemize}

\chapter{Lithography}
Writing stuff on stone by using an insoluble material on top. 

You can then put an acid on top and the resist on top will not be dissolved, 
but the rock will be dissolved.

\subsection{Process (Etch)}
\begin{enumerate}
    \item 
    Prepare surface and apply photoresist 
    \item 
    Pre-bake oven 
    \item 
    aligner exposure (with mask)
    \item 
    develop, rinse, and dry 
    \item 
    Post-bake oven
    \item 
    Inspect and measure
    \item 
    Etch and Deposit
    \item 
    Strip the photoresist and clean 
    \item 
    Deposit or grow new layer 
    \item 
    repeat 
\end{enumerate}

Etching is putting down deposition material and then putting the photoresist 
on top and in a certain pattern and then removing all material \textbf{not}
under the photoresist.

\subsection{Lift-off}
You put the photoresist on and get rid of the unwanted resist, 
\textbf{then} you put on the deposition material, and then the material that 
is on top of the photoresist gets removed.

\subsection{Issues}
Photoresists etch at various rates, so you need a tall photoresist layer 
to make sure that the photoresist doesn't completely etch away before your 
unwanted material is fully gone. 

\section{Wafer Cleaning}
Degrease the wafer with acetone, alcohol (IPA), de-ionized (DI) water, 
dry with $N_2$ gas. 

\subsection{Standard Clean 1 (SC1, RCA1)}
DI water, ammonium hydroxide, hydrogen peroxide 

Removes light organic material. 


\subsection{Standard Clean 2 (SC2, RCA2)}
DI water, HCl, hydrogen peroxide. 

Good for removing metal ions. 

Pirhana solution = sulfuric acid + hydrogen peroxide. 
very very harsh etch. 
Must be done in a fume hood!!!!!
Reactive mixture must be allowed to deactivate.
Only really used if you're trying to salvage a wafer.

Surface tension determines how far into a crevass a solvent 
will go.

\subsection{Other Cleaning}
Ultrasonic Cleaning - 
use sound in the 20-400 kHz range to dislodge large particles $(> 2 \mu m)$ 

Megasonic cleaners use even faster waves (800-2000 kHz) to dislodge 
smaller particles $(< 0.5 \mu m)$.

Cryogenic Cleaning - immerse your substrate into liquid nitrogen to make 
your wafer (and debris) very brittle, which should break apart debris.

Supercritical cleaning/drying - 
Immerse in ethanol or methanol. 
Supercritical CO2 dissolves the solvent. When pressure is lowered 
the gas changes.

\section{Wafer Priming}
You want a very hydrophobic surface for best adhesion. You also 
want to minimize humidity. 

You can use a number of methods for priming 
\begin{itemize}
    \item 
    Oxygen plasma descum 
    \item 
    Adhesion Promoter AP8000 (chemical)
    \item 
    Increase soft bake 
    \item 
    Sputter the surface to induce micro-roughness. 
\end{itemize}

\section{Details Of Photolithography}
\begin{enumerate}
    \item 
    Surface preparation 
    \item 
    Spin coating 
    \item 
    Alignment and Exposure 
    \item 
    Post-exposure bake 
    \item 
    Develop, clean, and dry 
    \item 
    Inspection 
    \item 
    Descum and Hard Bake 
    \item 
    Resist Stripping (Removing the PR)
\end{enumerate}

\subsection{Photoresist}
PR's consist of 3 main materials 

\begin{itemize}
    \item 
    Polymer (resin)
    \item 
    sensitizer 
    \item 
    Solvent 
\end{itemize}

You want your PR to be as flat as possible so that you are 
at minimal risk of under-exposure or over-exposure.

\subsection{Spin Coating}
You just use centrifugal forces. 

edge beading is what happens at the edge of your wafer that comes from 
excess PR at the edge of the wafer coming back into the wafer once it is done 
spinning.

The equation for spin coating is given by 
\equations{
    T 
    =
    \frac{K C^{\beta} \eta^{\gamma}}{\omega^{\alpha}}
}
\begin{itemize}
    \item 
    T = thickness 
    \item 
    K = calibration constant 
    \item 
    C = polymer concentration 
    \item 
    $\eta$ = intrinsic viscosity 
    \item 
    $\omega$ = angular speed (rpm)
    \item 
    $\alpha, \beta, \gamma$ = empirically determined parameters
\end{itemize}

Features on the wafer should be no more than 20\% of the 
resist thickness, otherwise the resist may not have good coverage. 

To estimate the dispense volume, you just do basic geometry 
\equations{
    V 
    =
    \pi r^2 T 
}


\subsection{Other Methods of PR Deposition}
\begin{itemize}
    \item 
    Spray coating 
    \item 
    Electrostatic Spraying 
    \item 
    Meniscus Coating 
    \item 
    Electrodeposition 
    \item 
    Roller Coating 
    \item 
    Silkscreen Pirnting 
    \item 
    Dip Coating
    \item 
    Curtain Coating 
    \item 
    Extrusion Coating
\end{itemize}

\section{Soft Baking}
Depends on the type and the thickness of the PR. 
It is also called pre-bake or pre-exposure bake. 
Hotplates are used in the 60C to 110C range. 

Pre-baking is important because it 
\begin{itemize}
    \item 
    Relieves stresses in the film
    \item 
    Removes solvents
    \item 
    Promotes adhesion
\end{itemize}

Too much heat can degrade photosensitivity!! 

PR must be soft-baked so that it does not stick to your mask during alignment.

\section{Exposure}
Optical transfer of the pattern on the mask to the PR. 

The light wavelength of our exposure is mainly in the UV range because we 
need visible light to see. 

Lasers are monochromatic and coherent. 

PR's have a required "dose" of light to successfully be removed from the wafer. 

\section{Mask}
It's the device that blocks light at certain locations so that the PR 
is untouched underneath it. There are multiple types of masks 

Masks are made of optically flat glass coated with a patterned 
absorber layer which blocks the light.

\begin{itemize}
    \item
    lightfield mask

    exposed almost everywhere
    \item
    darkfield mask

    exposed only in features.
    \item 
    Binary Mask 

    either completely exposed or completely blocked 
    \item 
    Greyscale mask
    
    transparency varies so you can deposit material of varying 
    heights with only a single layer of PR. 
    \item 
    Phase Shift Mask 

    Work like holograms and odulate the phase-shift of the 
    light beam to create interference patterns in the 
    PR. Requires a coherent light-source for exposure. 
\end{itemize}

\subsection{Mask Materials}
\begin{itemize}
    \item
Chrome is opaque to both visible and UV light
    \item
iron oxide is clear to visible light but opaque to UV 
    \item
    emulsion based materials are transparent 

    Soda Lime glass, BK-7, Quartz, CaF$_2$
\end{itemize}

The emulsion based materials have various cutoff wavelengths, and you would 
use different mask materials depending on both your PR and your exposure 
wavelength. 

Mask production is another lithography step. 

\subsection{Exposure Tools}
\begin{itemize}
    \item 
Contact mask: put the mask on the wafer directly

Fast, cheap, and simple, but the wafer has to be specifically made to withstand 
contact, there's no magnification, and the mask is the same size as the wafer.
    \item 
    Proximity: put the mask as close as you can without touching 

    Fast, and no-contact, but slightly more complex, the separation 
    leads to diffraction blur, and the mask needs to be the same size as 
    the wafer (expensive).
    \item 
    Projection: use lenses to be able to accurately place the 
    light without proximity. 

    Masks are easier, laser is more expensive.
\end{itemize}

Projection is useful because the mask does not have to be the same size 
as the pattern because the lens will de-magnify the light.

The modulation transfer function is the function that determines 
the type of feature sizes that can be put into a system.

\subsection{Projection Lithography}
You have to worry about both resolution and depth of focus.


\chapter{SKIPPED OPTICS LECTURE}


\chapter{Polymers}

\section{Crosslinking}
This is what happens when a polymer changes into an insoluble 
product due to light exposure.

This is for negative photoresists

\section{Scission}
Polymers separate under light exposure and actually become soluble.

This is for positive photoresists

\subsection{Examples of Positive PR}
\begin{itemize}
    \item 
    PMMA

    sensitive at 220nm and useful with DUV but not mercury 
    \item 
    DNQ resists

    Have a DNQ ester and a resin 

    useful with mercury lamp lines 
\end{itemize}

\section{Pros and Cons}
Positive PR's are better for resolving isolated holes and trenches 

Negative PR's are better for resolving isolated lines.

Positive PR leaves PR everywhere except the illuminated parts 

Negative PR leaves PR only on the illuminated parts and not anywhere else.

In order to use PR, you put you material \textbf{where there is no 
photoresist} and then the PR removes all the deposited material on top 
of it.


\subsection{Permanent Resists}
These are PR's that become permanent components of the device. 
Thicker layers are achievable.
They use strong adhesion, which makes removal difficult.

\subsection{Dry Resists}
A film is directly laminated onto a surface. No liquids are used. 
Available thickness from $25\mu m$ to $100 \mu m$. It conforms 
to surfaces of different topologies, and is developed in sodium 
carbonate solutions.

\subsection{Image Reversal Resists}
You can use both heat and UV to make the PR act as both a 
positive PR and a negative PR. 

\section{Overexposure}
You use too much light and part of the PR that was meant to be 
not touched is touched. 

\subsection{Positive}
Too much PR has been removed 

\subsection{Negative}
Too much PR is solidified on the wafer.

\subsection{Liftoff}
Positive photoresists are better for liftoff when overexposed because 
the PR forms a concave structure that makes sure the deposition material 
is definitely disconnected. 

\section{Underexposure}
The opposite of overexposure. 

Negative PR is better when you underexpose.

\section{PR Resolution and Contrast}
Certain resists only change a certain amount depending 
on the dosage of light. 

The sensitivity = resist contrast = the slope of resist removed over dosage 

\equations{
    \gamma_p 
    =
    \frac{1}{
        \log(D_p)
        -
        \log(D_p^0)
    }
    =
    \left(\log(\frac{D_p}{D_p^0})\right)^{-1}
}

Having a PR with less contrast is useful if you have 2 masks. 


\section{Developers}
\begin{itemize}
    \item 
    MIF (Metal Ion Free) Developesr
    \item 
    Inorganic (metal ion based) Developer
\end{itemize}

\textbf{Do not mix these two developers}

\section{Choosing a PR}
Dependent on wavelength, feature size, thickness, positive/negative, 
lift-off, chemical/plasma resistance, removal, developer, stripper.

\subsection{Multi-layer Resists}
These are used when you have resolution issues but still need a very 
thick layer. 

A multi-layer resist could also be used if there are features already in the 
wafer. You use 1 layer to cover all your features and then use a 
second layer for your actual pattern.

The resist thickness should be $5 \times$ as big as the largest feature.

\subsection{LIGA}
It's some German word.

Essentially you use a very thick PR to make a mold, and then you 
plate the PR mold with metal electroplating.


\subsection{Lift-Off Resist}
Basically you put a resist on top of an overexposed resist so 
you can get the nice concavity while still having a straight line for the 
deposition process.It's some German word.

\subsection{Post-Exposure Treatment}
\textbf{Heavily Dependent on Resist}.
baking, radiation, reactive gas, vacuum, time.

\subsection{Stripping}
There's all sorts of ways to get PR's off of your wafer.









\end{document}