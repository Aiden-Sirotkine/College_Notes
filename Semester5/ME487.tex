\documentclass[fleqn]{report}
\usepackage{geometry}
\usepackage{amssymb}
\usepackage{fancyhdr}
\usepackage{multicol}
\usepackage{blindtext}
\usepackage{color}
\usepackage[fontsize=16pt]{fontsize}
\usepackage{lipsum}
\usepackage{pgfplots}
\usepackage{physics}
\usepackage{mathtools}
\usepackage[makeroom]{cancel}
\usepackage{ulem}
\usepackage{esint}

\geometry{a4paper, margin=2cm} % Set paper size and margins
\graphicspath{ {../Images/} }
\setlength{\columnsep}{1cm}
\addtolength{\jot}{0.1cm}
\def\columnseprulecolor{\color{blue}}
\date{Fall 2025}

\newcommand{\textoverline}[1]{$\overline{\mbox{#1}}$}

\newcommand{\hp}{\hspace{1cm}}

\newcommand{\const}{\textrm{const}}

\newcommand{\del}{\partial}

\newcommand{\pdif}[2]{ \frac{\partial #1}{ \partial #2} }

\newcommand{\pderiv}[1]{ \frac{\partial}{ \partial #1} }

\newcommand{\comment}[1]{}

\newcommand{\equations} [1] {
\begin{gather*}
#1
\end{gather*}
}

\newcommand{\numequations} [1] {
\begin{gather}
#1
\end{gather}
}

\newcommand{\twovec}[2]{ 
\begin{pmatrix}
#1 \\ 
#2
\end{pmatrix}
}

\title{ME 487}
\author{Aiden Sirotkine}

\begin{document}

\pagestyle{fancy}
\maketitle
\tableofcontents
\clearpage

\chapter{ME 487}
The lab safety training is already up and I have to do that before the first lab.

You'll make a pressure sensor, and then a microfluidic mixer (because mixing small
fluids is hard).

\textbf{THERE WILL BE POP QUIZZES IN LECTURE}

\section{LAB POLICY}
\begin{itemize}
    \item
    \textbf{NO SHORTS, CONTACTS, OPEN TOED SHOES}
    \item 
    Don't touch anything unless told so 
    \item 
    If you do anything stupid you get kicked out.
\end{itemize}

\section{Importance}
MEMS are important because they take advantage of forces that scale dramatically 
at small scales 

\begin{itemize}
    \item 
    Surface tension scales with $l$ 
    \item 
    Fluid/electrostatic forces scale with $l^2$
    \item 
    weight/inertia forces scale with $l^3$
    \item
    electromagnetic forces scale with $l^4$
\end{itemize}

So at very small scales, these forces act very differently than in 
macro-scale systems. You can almost completely ignore gravity.

\subsection{Cantilever}
Deflection under self weight goes down dramatically with $l^2$.

Resonant frequency \textbf{increases} with $l^{-1}$.

\subsection{Fabrication Scaling}
\begin{itemize}
    \item 
    Devices are on a single substrate, and you can make 
    thousands of devices on a single substrate. 
    \item 
    We can take infrastructure/technology from the semiconductor industry. 
\end{itemize}

\subsection{Function Integration}
Both electrical and mechanical functions can be obtained 
with the same materials and processes. 

Projectors use a MEM that has a mirror, tilting mechanism, 
and electronic controller all on 1 substrate. 

\subsection{Material Saving}
Because MEMS are on the atomic scale, you use close to 
nothing on material saving. There is functionally 
0 material cost. 

\section{Difficulties}
\begin{itemize}
    \item 
    Really small $(10^6)$
    \item 
    Only very very specific materials actually work in lithography
    \item 
    All tools are only planar- we can only edit from the 
    top/bottom of the wafer. 
    \item 
    You need very specific/expensive facilities and materials.
\end{itemize}

\section{Typical Process}
\begin{enumerate}
    \item 
    Deposition
    
    You put a target material onto a substrate
    \item 
    Lithography

    You put a photoresisting material onto that target, and you 
    use lasers to get rid of material to leave a pattern. 
    \item 
    Etching 

    You get rid of the target material with lasers, and all the target 
    material under the photoresist is untouched. 

    You then remove the photoresist material and then you're left with 
    just the target material in just the pattern you wish. 
\end{enumerate}


\subsection{MUMPs (Multi-User MEMS Processes) Sequence}
This is a 7 layer process of depositing, adding a photoresist, and etching. 

This esseentially allows a company to make multiple different MEMS blueprints 
on a single wafer. 

Often used by fab-less companies. They buy a small bit of a wafer from a 
fab company and have that company etch a specific blueprint onto that 
small portion of wafer. 

\subsection{Finish Fabrication}
\begin{itemize}
    \item 
Singulation: Cut the large wafer 
into little dies such that each individual die has 1 MEM.
    \item 
    Packaging: 

    Bond the functional pieces of the MEM with a wire so that it can 
    work in a larger system. 
    \item 
    Encapsulation

    seal the MEM so it becomes a functional black box with wires for 
    input/output.
\end{itemize}

\chapter{Cleanroom Procedure}
Something something something review.

We're usually going to be working in the $\sim 5 \mu m$ range for our 
MEMS building. 

\section{Wafer Contamination}
If you leave your wafer in the cleanroom uncovered, it will eventually 
get contaminated with dust particles. The amount of time it takes for 
your device to get contaminated is probabilistic. A more clean cleanroom
will mean the chances your device gets contaminated in a certain amount of 
time decreases.

As we consider smaller and smaller dust particles, they move more and 
more randomly due to the random motion of air particles (Cunningham 
Correction Factor). Particles move at a speed dependent on some large 
convoluted equation. 

Possible contaminants include dust particles, organic films, and atoms/ions. 

\subsection{Humidity}
Fluid condensation is bad. 

Water can contaminate your device, and as the water evaporates, the surface 
tension will warp your device (cantilever). 

You can calculate the forces caused by the surface tension of water stuck 
in your cantilever.

\section{Cleanroom Itself}
You have to wear a whole bunch of silly clothes 

The yellow room is for photolithography. 

The white room is for deposition and etching. 

\subsection{Chase}
A part of the cleanroom only available to staff. 

New device will be brought in through the chase.

\chapter{Chemicals}
All sorts of chemicals and they're all flammable and carcinogenic so 
don't touch anything or you'll die.

\begin{itemize}
    \item 
    Solvents
    \item 
    Photoresists (Carcinogen)
    \item 
    Developers (Weak base)
    \item 
    Strippers (acetone)
    \item 
    Etching (Acids)
\end{itemize}

\section{Materials Safety Data Sheet (MSDS)}
It gives you a bunch of information about certain chemicals 

\begin{itemize}
    \item 
    Chemical breakdown
    \item 
    Methods of exposure
    \item 
    Effects/risks of exposure
    \item 
    other 
    \item 
    other 
\end{itemize}

16 different important bits of information on every MSDS. 

There is a physical binder and a search engine that gives you the MSDS 
to every chemical in the cleanroom.

\section{NFPA Diamond}
It's the square with 4 parts that tells you everything about a certain chemical 
in a container.

Yellow is for reactivity, red is for flammability, blue is for health hazards, 
white is for miscellaneous. 

White: OXY, ACID, ALK, COR, W-, RAD 

\section{PPE}
You need certain equipment BEYOND THE CLEANROOM GOWN to work with certain 
chemicals. 

HF spills have to be treated with special care.

\subsection{Do's and Dont's}
Do 
\begin{itemize}
    \item 
    change gloves whenever dirty/broken
    \item 
    Use fresh gloves
    \item 
    use cleanroom paper
    \item 
    remove rings and bracelets
\end{itemize}

\chapter{Lithography}
Writing stuff on stone by using an insoluble material on top. 

You can then put an acid on top and the resist on top will not be dissolved, 
but the rock will be dissolved.

\subsection{Process (Etch)}
\begin{enumerate}
    \item 
    Prepare surface and apply photoresist 
    \item 
    Pre-bake oven 
    \item 
    aligner exposure (with mask)
    \item 
    develop, rinse, and dry 
    \item 
    Post-bake oven
    \item 
    Inspect and measure
    \item 
    Etch and Deposit
    \item 
    Strip the photoresist and clean 
    \item 
    Deposit or grow new layer 
    \item 
    repeat 
\end{enumerate}

Etching is putting down deposition material and then putting the photoresist 
on top and in a certain pattern and then removing all material \textbf{not}
under the photoresist.

\subsection{Lift-off}
You put the photoresist on and get rid of the unwanted resist, 
\textbf{then} you put on the deposition material, and then the material that 
is on top of the photoresist gets removed.

\subsection{Issues}
Photoresists etch at various rates, so you need a tall photoresist layer 
to make sure that the photoresist doesn't completely etch away before your 
unwanted material is fully gone. 

\section{Wafer Cleaning}
Degrease the wafer with acetone, alcohol (IPA), de-ionized (DI) water, 
dry with $N_2$ gas. 

\subsection{Standard Clean 1 (SC1, RCA1)}
DI water, ammonium hydroxide, hydrogen peroxide 

Removes light organic material. 


\subsection{Standard Clean 2 (SC2, RCA2)}
DI water, HCl, hydrogen peroxide. 

Good for removing metal ions. 

Pirhana solution = sulfuric acid + hydrogen peroxide. 
very very harsh etch. 
Must be done in a fume hood!!!!!
Reactive mixture must be allowed to deactivate.
Only really used if you're trying to salvage a wafer.

Surface tension determines how far into a crevass a solvent 
will go.

\subsection{Other Cleaning}
Ultrasonic Cleaning - 
use sound in the 20-400 kHz range to dislodge large particles $(> 2 \mu m)$ 

Megasonic cleaners use even faster waves (800-2000 kHz) to dislodge 
smaller particles $(< 0.5 \mu m)$.

Cryogenic Cleaning - immerse your substrate into liquid nitrogen to make 
your wafer (and debris) very brittle, which should break apart debris.

Supercritical cleaning/drying - 
Immerse in ethanol or methanol. 
Supercritical CO2 dissolves the solvent. When pressure is lowered 
the gas changes.

\section{Wafer Priming}
You want a very hydrophobic surface for best adhesion. You also 
want to minimize humidity. 

You can use a number of methods for priming 
\begin{itemize}
    \item 
    Oxygen plasma descum 
    \item 
    Adhesion Promoter AP8000 (chemical)
    \item 
    Increase soft bake 
    \item 
    Sputter the surface to induce micro-roughness. 
\end{itemize}

\section{Details Of Photolithography}
\begin{enumerate}
    \item 
    Surface preparation 
    \item 
    Spin coating 
    \item 
    Alignment and Exposure 
    \item 
    Post-exposure bake 
    \item 
    Develop, clean, and dry 
    \item 
    Inspection 
    \item 
    Descum and Hard Bake 
    \item 
    Resist Stripping (Removing the PR)
\end{enumerate}

\subsection{Photoresist}
PR's consist of 3 main materials 

\begin{itemize}
    \item 
    Polymer (resin)
    \item 
    sensitizer 
    \item 
    Solvent 
\end{itemize}

You want your PR to be as flat as possible so that you are 
at minimal risk of under-exposure or over-exposure.

\subsection{Spin Coating}
You just use centrifugal forces. 

edge beading is what happens at the edge of your wafer that comes from 
excess PR at the edge of the wafer coming back into the wafer once it is done 
spinning.

The equation for spin coating is given by 
\equations{
    T 
    =
    \frac{K C^{\beta} \eta^{\gamma}}{\omega^{\alpha}}
}
\begin{itemize}
    \item 
    T = thickness 
    \item 
    K = calibration constant 
    \item 
    C = polymer concentration 
    \item 
    $\eta$ = intrinsic viscosity 
    \item 
    $\omega$ = angular speed (rpm)
    \item 
    $\alpha, \beta, \gamma$ = empirically determined parameters
\end{itemize}

Features on the wafer should be no more than 20\% of the 
resist thickness, otherwise the resist may not have good coverage. 

To estimate the dispense volume, you just do basic geometry 
\equations{
    V 
    =
    \pi r^2 T 
}


\subsection{Other Methods of PR Deposition}
\begin{itemize}
    \item 
    Spray coating 
    \item 
    Electrostatic Spraying 
    \item 
    Meniscus Coating 
    \item 
    Electrodeposition 
    \item 
    Roller Coating 
    \item 
    Silkscreen Pirnting 
    \item 
    Dip Coating
    \item 
    Curtain Coating 
    \item 
    Extrusion Coating
\end{itemize}

\section{Soft Baking}
Depends on the type and the thickness of the PR. 
It is also called pre-bake or pre-exposure bake. 
Hotplates are used in the 60C to 110C range. 

Pre-baking is important because it 
\begin{itemize}
    \item 
    Relieves stresses in the film
    \item 
    Removes solvents
    \item 
    Promotes adhesion
\end{itemize}

Too much heat can degrade photosensitivity!! 

PR must be soft-baked so that it does not stick to your mask during alignment.

\section{Exposure}
Optical transfer of the pattern on the mask to the PR. 

The light wavelength of our exposure is mainly in the UV range because we 
need visible light to see. 

Lasers are monochromatic and coherent. 

PR's have a required "dose" of light to successfully be removed from the wafer. 

\section{Mask}
It's the device that blocks light at certain locations so that the PR 
is untouched underneath it. There are multiple types of masks 

Masks are made of optically flat glass coated with a patterned 
absorber layer which blocks the light.

\begin{itemize}
    \item
    lightfield mask

    exposed almost everywhere
    \item
    darkfield mask

    exposed only in features.
    \item 
    Binary Mask 

    either completely exposed or completely blocked 
    \item 
    Greyscale mask
    
    transparency varies so you can deposit material of varying 
    heights with only a single layer of PR. 
    \item 
    Phase Shift Mask 

    Work like holograms and odulate the phase-shift of the 
    light beam to create interference patterns in the 
    PR. Requires a coherent light-source for exposure. 
\end{itemize}

\subsection{Mask Materials}
\begin{itemize}
    \item
Chrome is opaque to both visible and UV light
    \item
iron oxide is clear to visible light but opaque to UV 
    \item
    emulsion based materials are transparent 

    Soda Lime glass, BK-7, Quartz, CaF$_2$
\end{itemize}

The emulsion based materials have various cutoff wavelengths, and you would 
use different mask materials depending on both your PR and your exposure 
wavelength. 

Mask production is another lithography step. 

\subsection{Exposure Tools}
\begin{itemize}
    \item 
Contact mask: put the mask on the wafer directly

Fast, cheap, and simple, but the wafer has to be specifically made to withstand 
contact, there's no magnification, and the mask is the same size as the wafer.
    \item 
    Proximity: put the mask as close as you can without touching 

    Fast, and no-contact, but slightly more complex, the separation 
    leads to diffraction blur, and the mask needs to be the same size as 
    the wafer (expensive).
    \item 
    Projection: use lenses to be able to accurately place the 
    light without proximity. 

    Masks are easier, laser is more expensive.
\end{itemize}

Projection is useful because the mask does not have to be the same size 
as the pattern because the lens will de-magnify the light.

The modulation transfer function is the function that determines 
the type of feature sizes that can be put into a system.

\subsection{Projection Lithography}
You have to worry about both resolution and depth of focus.


\chapter{SKIPPED OPTICS LECTURE}


\chapter{Polymers}

\section{Crosslinking}
This is what happens when a polymer changes into an insoluble 
product due to light exposure.

This is for negative photoresists

\section{Scission}
Polymers separate under light exposure and actually become soluble.

This is for positive photoresists

\subsection{Examples of Positive PR}
\begin{itemize}
    \item 
    PMMA

    sensitive at 220nm and useful with DUV but not mercury 
    \item 
    DNQ resists

    Have a DNQ ester and a resin 

    useful with mercury lamp lines 
\end{itemize}

\section{Pros and Cons}
Positive PR's are better for resolving isolated holes and trenches 

Negative PR's are better for resolving isolated lines.

Positive PR leaves PR everywhere except the illuminated parts 

Negative PR leaves PR only on the illuminated parts and not anywhere else.

In order to use PR, you put you material \textbf{where there is no 
photoresist} and then the PR removes all the deposited material on top 
of it.


\subsection{Permanent Resists}
These are PR's that become permanent components of the device. 
Thicker layers are achievable.
They use strong adhesion, which makes removal difficult.

\subsection{Dry Resists}
A film is directly laminated onto a surface. No liquids are used. 
Available thickness from $25\mu m$ to $100 \mu m$. It conforms 
to surfaces of different topologies, and is developed in sodium 
carbonate solutions.

\subsection{Image Reversal Resists}
You can use both heat and UV to make the PR act as both a 
positive PR and a negative PR. 

\section{Overexposure}
You use too much light and part of the PR that was meant to be 
not touched is touched. 

\subsection{Positive}
Too much PR has been removed 

\subsection{Negative}
Too much PR is solidified on the wafer.

\subsection{Liftoff}
Positive photoresists are better for liftoff when overexposed because 
the PR forms a concave structure that makes sure the deposition material 
is definitely disconnected. 

\section{Underexposure}
The opposite of overexposure. 

Negative PR is better when you underexpose.

\section{PR Resolution and Contrast}
Certain resists only change a certain amount depending 
on the dosage of light. 

The sensitivity = resist contrast = the slope of resist removed over dosage 

\equations{
    \gamma_p 
    =
    \frac{1}{
        \log(D_p)
        -
        \log(D_p^0)
    }
    =
    \left(\log(\frac{D_p}{D_p^0})\right)^{-1}
}

Having a PR with less contrast is useful if you have 2 masks. 


\section{Developers}
\begin{itemize}
    \item 
    MIF (Metal Ion Free) Developesr
    \item 
    Inorganic (metal ion based) Developer
\end{itemize}

\textbf{Do not mix these two developers}

\section{Choosing a PR}
Dependent on wavelength, feature size, thickness, positive/negative, 
lift-off, chemical/plasma resistance, removal, developer, stripper.

\subsection{Multi-layer Resists}
These are used when you have resolution issues but still need a very 
thick layer. 

A multi-layer resist could also be used if there are features already in the 
wafer. You use 1 layer to cover all your features and then use a 
second layer for your actual pattern.

The resist thickness should be $5 \times$ as big as the largest feature.

\subsection{LIGA}
It's some German word.

Essentially you use a very thick PR to make a mold, and then you 
plate the PR mold with metal electroplating.


\subsection{Lift-Off Resist}
Basically you put a resist on top of an overexposed resist so 
you can get the nice concavity while still having a straight line for the 
deposition process.It's some German word.

\subsection{Post-Exposure Treatment}
\textbf{Heavily Dependent on Resist}.
baking, radiation, reactive gas, vacuum, time.

\subsection{Stripping}
There's all sorts of ways to get PR's off of your wafer.

\section{MONDAY LITHOGRAPHY QUIZ (it should only take 5 minutes)}
Resist's have 3 components 
Sensitizer, resin, solvent.

Ferrera likes darkfield masks more than lightfield masks because he 
has more control. 

\subsection{Step and Repeat Process}
Used for high-volume fabrication. I wasnt paying attention 

\chapter{Alignment}
You need both in-plane and out-of-plane alignment. 

\subsection{Out of Plane}
Wedge error compensation. Necessary for good image transfer 

\subsection{In-Plane}
Overlay Registration. This is important for when we need to add new 
features to features already on the plane. 

\subsection{Top-Down}
Aligning features on the two faces of the wafer. 

There are multiple ways to do this of varying quality 
\begin{itemize}
    \item 
    Hole through the wafer 
    \item 
    IR Microscopy 
    \item 
    visible light with 2 masks 
    \item 
    back-side mask and top-side mask 
    \item 
    Wafer flats can also be used at reference points.
\end{itemize}

\section{Errors}
There are all sorts of errors that can occur due to misalignment in 
projection lithography.

\subsection{Misalignment Examples}
Thermal run-in and run-out is when your features move due to a change 
in temperature during lithography 

\equations{
    R 
    =
    r 
    \left(
        \Delta T_{mask}
        \alpha_{mask}
        -
        \Delta T_{Si}
        \alpha_{Si}
    \right)
}


\section{Alignment Markers}
\subsection{Fiducial Marks}
Little marks on the wafer that you can align your mask towards.
They are usually on the sides of the wafer where you are not 
putting devices.

\subsection{Contact Aligner}
It's just a mark with depth such that you can align the wafer with a 
physical process instead of just visual.

\subsection{Special Marks}
Vernier scales and Moire-style marks are used for very quick 
alignment.

\subsection{Vernier Scale}
Vernier scales use slightly misaligned steps such that you can 
see very small measurements more easily. Each misalignment is 1 step, 
and you can see the total misalignment by seeing which step is 
aligned with the wafer itself.

LC = least count, MSP = main scale pitch. 
\equations{
    O 
    =
    ZL * MSP + (LC \times MB)
}

If the vernier scale aligns with the number $10$, and the spacing 
difference is $0.2$ microns, then the alignment is $10 * 0.2 = 2$ micron 
misalignment.

\subsection{Moire Scale}
Uses constructive and destructive interferencec to essentially show 
a large wave pattern with a very small misalignment.

\section{Projection Lithography}
If your wafer is at an angle to the mask, the diffraction will cause 
blurry portions, and the dimensions will also be off because you are 
no longer in the focal plane.

\section{Resolution Enhancement Technology}
How do you improve resolution? 

You can use shadow printing 
\equations{
    R 
    =
    B_{min}
    =
    k \sqrt{\lambda \left( s + \frac{z}{2} \right)}
}

\begin{itemize}
    \item Resists

You can use better resists or multi-layer resists. You can use an 
anti-reflection coating. 
\item New Masks 

Phase shifting, optical proximity correction 

\item 
Exposure system
 
buy a new system
\end{itemize}

resists are the cheapest option of any of these. 
Masks are pretty expensive. Aligners are crazy expensive.

\section{Chemically Amplified Resists}
Quantum Efficiency 
=
$\frac{\textrm{Number of photons induced}}
{\textrm{number of photon absorbed}}$

They have improved efficiency compared to regular resists. 

They do have some issues 
\begin{itemize}
    \item 
    Poor stability 
    \item 
    dependent on process parameters 
    \item 
    Oxygen causes surface inhibition 
\end{itemize}

Single layer photoresists also cause issues

\subsection{Anti-Reflective Coatings}
A coating is applied to the wafer before the PR is put on. 

\begin{itemize}
    \item 
    reduce reflected light 
    \item 
    reduces standing waves 
    \item 
    planarized surface 
\end{itemize}

We don't worry about non-standing waves in our cleanroom because 
we use a mercury lamp that release non-coherent light. 

\subsection{Thin Film Imaging}
You use multiple layers of resists. Use a very thin top resist 
and only etch the top resist. Then the bottom resist acts to get 
rid of standing waves and interference, increasing resolution. 

\subsection{Wavelength}
Decreased wavelength means less diffraction and greater resolution.

\section{Mask/Wavefront Engineering}


\subsection{Phase Shifting Mask}
A phase shifting mask improves both resolution and DOF. 
To use a phase shifting masks, you need \textbf{coherent light} (lasers), 
so that the phase is consistent.

Controls light diffraction using interference.

This can be done with features etched into glass, known 
as a \textbf{hard shifter mask}. 

Soft shifter masks use a non-chrome material that allows only 
phase-shifted light to come through. 

\subsection{Levenson-type Mask}
Masks can have additional material patterned onto the surface 
that shifts the phase by 180 degrees to create interference 

\subsection{Mask Geometry Engineering}
When light goes through a pattern, some with defract and there will be 
noise in the design. 

\textbf{optical proximity correction} is encoding noise into the mask such 
that the light that goes through the mask will etch the pattern \textbf{and}
account for the noise created by the light. Add light where the pattern isn't 
fully etched and remove light where the pattern is over-etched.

Mask is no longer WYS / WYG (what you see is what you get) (lmao). 

\subsection{Improved Exposure Systems}
These improved systems only work with projective lithography. 

\textbf{Off-Axis Illumination} changes the diffraction pattern by 
angling the light 

\textbf{Kohler Illumination} focusing light at the entrance of the aperture. 
This minimizes glares, shadows, and contrast. The field is evenly illuminated.

\subsection{Resolution Example}
\equations{
    B_{min}
    =
    \frac{k_1 \lambda}{N A}
    \hp 
    DOF 
    =
    \pm 
    \frac{k_2 \lambda}{(NA)^2}
    \hp 
    NA 
    =
    n \sin(\theta)
}
The maximum that $NA$ can be is 1 (index of refraction of air), so 
even with the absolute maximum, with $\lambda = 193nm$ and $k_1 = 0.25$, we get 
\equations{
    R = B_{min}
    =
    \frac{0.25 (193)}{1}
    =
    48.25 nm
}
However, with this resolution, we were able to get 22nm features. 
The way that they increased their resolution was by increasing the 
refractive index $n$ with a liquid medium. You need a liquid 
that does not absord UV radiation, has a high refractive index, and is 
compatible with all of the materials on your wafer. 

This is called immersion lithography (or immersion imaging for microscopes). 

\section{Summary}
Wavefront engineering improves certain types of mask imaging. We 
have 3 major improvements
\begin{itemize}
    \item
    phase shifting masks
    \item
    optical proximity correction
    \item
    off-axis illumination 
\end{itemize}

\chapter{Optical Lithography}
What's new? Limit shapes and keep it simple. Advances beyond 
PSM, OPC, and OAI. 
New light sources ($F_2$ lasers at 157nm, possibly lasers at 126nm). 

Mask materials? Quartz cuts out at 190nm and CaF$_2$ at 150nm. Now what? 

Throughput is impressive at 200WPH (wafers per hour). What is this number 
when new technology has to be introduced?

At some point, photolithography reaches its limit, so we have to switch 
to UV or even x-ray lithography. The newest 3nm ASML machines use 
EUV lithography or soft x-ray lithography. We also 
have e-beam lithography and ion-beam lithography and nano-imprint lithography. 

\section{Extreme UV Lithography}
Wavelength range from 10nm to 14nm. These are also called soft x-rays 
or vacuum UV. They developed from laser produced plasmas (LPP) and 
synchrotrons. 
They are only surface resist and vacumm compatible. 

\subsection{Pros}
Capable of printing sub-100nm features. All optics also must be reflective 
(no lenses)

\subsection{Cons}
EUV is strongly absorbed in all materials, so it must be done in a vacuum. 
All optics must be reflective (no lenses). Masks must be 100\% defect free. 
New resists are need for this very low wavelength beam. 

\section{X-ray Lithography}
wavelength range from $0.4nm \to 5nm$. 
Also called deep x-ray lithography. Sources include electron impact tubes, 
laser based plasmas, and synchrotrons. PMMA can be used as a resist. 

\subsection{Pro}
\begin{itemize}
    \item 
    Large DOF 
    \item 
    Capable of printing sub-200nm 
    \item 
    Diffraction is not an issue
    \item 
    Dust is transparent
    \item 
    no optics involved
\end{itemize}

\subsection{Cons}
\begin{itemize}
    \item
    masks need ot be opaque and transparent to x-ray
    \item
    Mask reduction is not possible 
    \item
    Masks can only be written with e-beams 
\end{itemize}

We can theoretically get smaller features, but the biggest issue 
is the masks (tubes instead of mirrors/lenses). The masks also are all 
shadowless.

\section{E-Beam Lithography}
Energy range from $10 \to 50 keV$. Can be both narrow beam or flood exposure. 
PMMA and PGMA can both be used as a resist. 

\subsection{Pros}
\begin{itemize}
    \item
    Micron and sub-micron resist geometries
    \item
    Can be automated and precisely controlled
    \item
    Greater DOF
    \item
    Direct write without mask (narrow beam)
\end{itemize}

\subsection{cons}
\begin{itemize}
    \item 
    slow, low throughput (scanning a 4in wafer takes over an hour)
    \item 
    must be done in a vacuum to avoid absorption 
    \item 
    Sample and resist must be able to handle electrons 
\end{itemize}

\section{Ion Beam Lithography}
Energy range $> 100keV$. Uses positive ions (Gallium, Indium, Gold). 
Various methods for producing ion beams (both narrow and flood exposure).
PMMA and PGMA both can be used as a resist. 

\subsection{Pros}
\begin{itemize}
    \item
    Smallest beam size of anything $(< 8nm)$
    \item
    Higher resist sensitivity than e-beam
    \item
    negligible ion scanning 
\end{itemize}

\subsection{Cons}
\begin{itemize}
    \item 
    Slow, low throughput
    \item 
    Must be done in a vacuum
    \item 
    Ion contamination (Gallium can implant in your substrate)
    \item 
    can break your wafer 
\end{itemize}

\section{Nanoimprint Lithography}
Hot embossing applied to nanofabrication. A force presses a master 
against a polymer, so that it imprints with the desired patterns via 
UV, heat, or pressure. 
Essentially, you squish a polymer (liquid) onto your substrate with the mask 
in between. We're pressing a liquid through a very small channel at a very 
high pressure.

\subsection{Pros}
\begin{itemize}
    \item
    Simple and relatively cheap
    \item
    Single layers are economical
    \item
    Small features are easier than large features (less stamping)
\end{itemize}

\subsection{Cons}
\begin{itemize}
    \item
    Low throughput 
    \item
    Alignment can be costly 
    \item
    Mask can degrade 
    \item
    Release can be challenging 
\end{itemize}

\section{Maskless Lithography}
There is a machine called a Heidelberg that is a Spacial Light Modulator 
that produces patterns on the photoresist directly. 

\subsection{Pros}
\begin{itemize}
    \item 
    good for R\&D (no mask development)
\end{itemize}

\subsection{Cons}
\begin{itemize}
    \item
    Write speeds depend on the model and can take minutes to hours
\end{itemize}

\section{3D Nanoprinting}
\begin{enumerate}
    \item
    A negative PR is polymerized using 2 photon absorption
    \item
    The polymer itself is used to pattern the surface 
    \item
    A master can be made with molding
\end{enumerate}
The PR itself is not the material being used. Nanoprinting allows you to 
make 3D PR patterns, and then you get etch/deposit a 3d pattern based on that 
PR.

\subsection{Pros}
\begin{itemize}
    \item
    Simple and Low cost
    \item
    Simple layers are cheap 
    \item
    Small features are easier than large features (less printing)
\end{itemize}


\subsection{Cons}
\begin{itemize}
    \item 
    Stitching errors 
    \item 
    Limited size of patterns 
    \item 
    Limited polymer materials (must be a negative PR)
\end{itemize}

\section{Soft Lithography}
You use a stamp with a soft polymer and wack it into the thing. 


\chapter{Physical Vapor Deposition (PVD)}
Source/target to wafer/substrate.

The 2 things to do after you have a pattern on your wafer are deposition 
(adding stuff to the wafer), and etching (removing stuff from the wafer). 
There are 2 types of main deposition processes, PVD and CVD. 

PVD is physical vapor deposition, where you you plasma or boiling or any 
sort of non-chemical process to atomize a material and place it onto a 
substrate.

CVD could involve 2 gasses being mixed such that the precipitate is formed 
on your wafer. 

Epitaxial processes involve depositing only a single layer film onto a wafer. 
The way that this is done is with beams of molecules being placed (epitaxy). 

Atomic layer deposition involves a self-terminating reaction. 2 materials 
react on the substrate, but they are self-terminating, so they stop very quickly. 
You condition your surface with 1 chemical and add a 2nd. They react and only 
create a sinlge layer of reactant, and then you can purge your gasses to get 
only a very thin film.

Implantation and diffusion are types of doping (material modification)
    
\subsection{Types of PVD}
\begin{itemize}
    \item
    Evaporation
    \item
    Sputtering
    \item
    Laser
    \item
    Molecular beam epitaxy
\end{itemize}

No matter what, film growth is achieved via \textbf{condensation} 
of a vapor onto your substrate.
This can be done with evaporation, or bombardent/sputtering.

The deposition rate is proportional to the rate of mass transfer to the surface. 
PVD is usually done in a vacuum to increase your mean free path. Also, 
a very high vacuum keeps contamination to a bare minimum (ignore oxidation).
Lastly, a high vacuum also depressed the boiling point of materials.

We usually work in a high vacuum of $10^{-5} \to 10^{-7}$ torr.

\subsection{Evaporation}
Heat up your substance with an electron beam, heating coil, laser, RF induction, 
resistive heating, etc. 

\subsection{Sputtering}
Ion bombardment and laser ablation are used to wack atoms off of your target.
We usually use Argon or Nitrogen ions for sputtering because they are unreactive.

\section{Mean Free Path}
Defined as the average distance a molecule travels 
before colliding with another molecule.
The mean free path statistics result in the equation 

p = pressure, T = temperature, d = diameter.
\equations{
    L 
    = 
    \frac{k_B T}{\sqrt{2} \pi d^2 p}
    \hp 
    \frac{k_B T}{p}
    =
    \frac{V}{N}
}
We want to increase mean free path, so we decrease pressure. The math is 
in the lecture, but at a high vacuum, the mean free path is $L \approx 4m$.

\subsection{Types of Pumps}
Roughing pumps (rotary, etc.) can get you to a low/medium vacuum. 
Cryopumps, ion pumps, turbo pumps, etc. 
are able to get you to high vacuums.

\subsection{Vacuum System Operation}
Mechanical pump $\to$ Hi-vacuum pump, with lots of valves in between. 
If a Hi-vacuum pump reaches atmospheric pressure, it breaks.

\subsection{Pressure Gauges}
At regular pressures, we use a gauge that measures that resistance 
of a heated wire (less pressure means less heat loss from convection)

\subsection{Ion Gauge}
Operates by measuring thermionic current. We boil a material to make ions 
and then accelerate them with a coil. The amount of ions collected is 
proportional to the pressure. Less pressure means less ions and less current.

\subsection{Outgassing}
gasses emanate from surfaces inside the chamber. This is a limiting 
factor for vacuums below $10^{-7}$. There are a few potential solutions: 
\begin{itemize}
    \item
    Eliminate materials that absord gases
    \item
    bake the entire vacuum system at 150C to evaporate them.
\end{itemize}

\section{Evaporation}
Use a heat source to turn a material to a gas. There 
are many types of heat sources for evaporation.

\begin{itemize}
    \item
    Resistance heat (no radiation, but might contaminate)
    \item
    E-Beam (no contamination, but emits radiation)
    \item
    RF-induction (no radiation, but contamination)
    \item
    Laser (no radiation, but expensive)
\end{itemize}

The three steps of this type of PVD are evaporation, transport, condensation.

\subsection*{Very Long Tangent About Evaportation (Equilibrium Vapor 
Pressure and Stuff)}
If you pull a vacuum, things evaporate easier.

\subsection{Clausius-Calpeyron Equation}
Boiling point and vapor pressure 
\equations{
    \ln(P_v)
    =
    \frac{- \Delta H_{vap}}{R} \frac{1}{T}
    +
    C
}

\subsection{Hertz-Knudsen Equation}
\equations{
    N_e 
    =
    (2 \pi M k T)^{-1/2} P_v
}
$N_e$ is the number of molecules evaporating per unit area per time. M 
is the molecular mass, $P_v$ is the vapor pressure. 

\subsection{Thermal Evaporation}
Pass current through a tungsten boat to generate heat. 

\section{Deposition}
The deposition rate is more important than the evaporation rate, even 
if they are related. It depends on the evaporation rate, distance of the substrate, 
and the angle of deposition. The evaporated mass expands and spreads out from 
the source. We use Lambert's Cosine Law nad Knudsen's equation as well as a 
factor $n$ that is related to the energy that an atom has when it leaves 
the surface. $\phi$ is between the normal and the line of sight, and $\theta$ 
is is between the observer's line of sight and the normal of the surface.
\equations{
    \frac{dM_s}{dA_s}
    =
    \frac{M_e(n + 1) \cos^n(\phi) \cos(\theta)}{2 \pi r^2}
}

When $n=0$, we get 
\equations{
    \frac{dM_s}{dA_s}
    =
    \frac{M_e \cos(\theta)}{2 \pi r^2}
}
The angles aren't completely intuitive, but there is a picture in the lab 
slides. 

\subsection{Film Thickness Variation}
You get less deposition materials on the sides of the materials compared to the 
center depending on the angle of deposition and the distance difference. 
For a point source (n=0), we get a thickness of 

\equations{
    t 
    =
    \frac{M_e h}{4 \pi (h^2 + l^2)^{3/2} \rho_{density}}
    \hp 
    \frac{t_{max}}{t_{min}}
    =
    \frac{1}{\left(1 + \left(\frac{l}{h}\right)^{2}\right)^{3/2}}
}

This is found with the slightest amount of algebra. 
For a surface source (n=1), we get a different equation 

\equations{
    t 
    =
    \frac{M_e h}{4 \pi (h^2 + l^2)^{2} \rho_{density}}
    \hp 
    \frac{t_{max}}{t_{min}}
    =
    \frac{1}{\left(1 + \left(\frac{l}{h}\right)^{2}\right)^{2}}
}

We want a thickness ratio as close to 1 as possible, so the point source 
is the most ideal source for deposition (maximizes min/max). We 
want to keep our point source as far as possible to decrease variation, but 
that also dramatically decreases your deposition rate.

\subsection{Planetary Substrate Holder}
Making a point source material radiator is impossible, but my move our 
wafers around mid-deposition so that the thickness anywhere on the wafer 
is approximately constant. 

\subsection{Directionality and Local Step Coverage}
If you have features on your wafer, you have to worry about the fact that 
space under/behind a feature will not have material deposited on it. 

\section{Sources}
Regular sources just have like a chamber that gets hot 

\subsection{E-Beam}
Very locally heat material using E-Beams and a magnetic field that focuses 
and aligns it. The magnetic field can also raster scan the beam across the 
source material. The main benefit is you have evaporate high-melting-point 
materials without too much difficulty. 85\% of the electron energy goes 
into heating, with the resting going into x-rays and electron generation. 
x-ray damage can be an issue (secondary radiation). E-beams result is 
higher quality films and deposition rates than thermal deposition.

\section{Quartz Crystal Monitor (QCM)}
Use a crystal to measure the thickness of thin films. You put a crystal 
in your sputterer, and the deposited thin film changes the resonant 
frequency of the crystal (because of the increased mass). 
The resonant frequency can be measured, and that 
can be use to calculate the thickness of the film. 

\subsection{Math}
It's just a simple harmonic oscillator.
\equations{
    f_1 
    =
    \frac{1}{2 \pi}
    \sqrt{\frac{k}{m}}
    \Rightarrow 
    f_1^2 
    =
    \frac{1}{4 \pi^2}
    \frac{k}{m}
    \\
    f_2 
    =
    \frac{1}{2 \pi}
    \sqrt{\frac{k}{m + \Delta m}}
    \Rightarrow 
    f_2^2 
    =
    \frac{1}{4 \pi^2}
    \frac{k}{m + \Delta m}
    \\
    \Delta m 
    =
    \frac{k}{4 \pi^2}
    \left[
        \frac{1}{f_1^2 }
        -
        \frac{1}{f_2^2 }
    \right]
}

\section{Summary}
\begin{itemize}
    \item
    Evaporation is most suitable for metallic thin films
    \item
    Compounds and alloys don't deposit well
    \item
    Deposition is direction, so it does not do well will steps 
    \item
    E-Beam deposition usually produces better films because of higher 
    temperatures and less contamination
    \item
    High vacuum is required
\end{itemize}

Know a number of things about PVD. 
equilibrium vapor pressure. its relation to temperature (Clausius 
Clapeyron). Relation to evaporation rate (hertz-knudsen). Flux distribution. 
thickness variation. 

Dramatically raising the temperature will cause bigger thickness 
variation and your boat will start melting (Lambert's Cosine Law).

What if you want to deposit something like tungsten with a high melting point? 
Don't use thermal evaporation. Use sputtering instead. 

\chapter{Sputtering}
Whack a material real hard so it gets kicked off the boat. 
You can use ion bombardment (plasma) or laser ablation. 

\section{Process}
\begin{itemize}
    \item 
    The source and substrate are placed in a vacuum
    \item 
    Bombard the surface with high energy ions
    \item 
    Enough energy is added to a particle for it to escape the surface
    \item 
    The ejected atoms deposit on the substrate
\end{itemize}

After the vacuum is made, a sputtering gas (argon) is used to strike a plasma. 

\section{Generating Plasma}
Cathode (negative voltage) to Anode (grounded)

Apply a voltage to accelerate free electrons that collide with atoms to make 
ions.

\subsection{Why Low Pressure?}
At STP, the electrons would have a very small Mean Free Path, so they wouldn't 
have the energy to create Argon ions. 

With low pressure, the electron has time to accelerate, so it gains enough 
energy to be able to ionize the Argon.

The reason that we're in a medium vacuum and not a high vacuum is because 
we need a decent amount of ion flux to be able to actually bombard our 
target.

\subsection{Electron Origin?}
Field Emitted Electrons. The cathode emits electrons because of the electric 
field created from the high voltage. It reduces the Work Function of the metal 
electrons, so they are released. You also get free electrons from 
the generated ions (Townsend Avalanche). 

The plasma acts as a weakly charged gas in a quasi-neutral medium with 
some amount of influencing fields. The particles all influence each other 
because of their electric fields. 

\subsection{Glow Discharge}
The electrons can be absorbed by ions, and when they're absorbed, they 
release light of a certain wavelength.

The plasma is slightly positively charged because the ions have a higher 
mass, so they need more time to accelerate and thus collect in that conductive 
region (slightly). 

The cathode at the target. The anode is at the substrate holder.

\subsection{THE IONS ARE POSITIVE}

\subsection{Magnetron Sputterer}
Use an M field instead of an E field to accelerate electrons in circles. 
This increases their mean free path and thus makes a denser plasma. 
This is caused by the \textbf{Lorentz Force}

\section{Ion Bombardment}
Multiple things can happen with an ion hits a surface 

\begin{itemize}
    \item 
    Reflection
    \item 
    Absorption
    \item 
    Sputtering
    \item 
    Ion Implantation
    \item 
    Chemical Reactions
    \item 
    Electron and Photon Emission
\end{itemize}

\subsection{Levels of Ion Bombardment}
\begin{itemize}
    \item
    Very low energy = $< 5 ev$

    Reflection or physisorption
    \item
    Low energy = $5 - 10 ev$

    Surface damage and surface migration
    \item 
    Medium = $10 - 10000 eV$ 

    Lattice damage, atomic ejection, nad heat - Sputtering 
    \item 
    high = $> 10 eV$ 

    Ion implanatation into lattice (doping)
\end{itemize}

\section{Sputtering Regimes}


\subsection{Single Knock-on Regime}
Seen in the 10-30eV range. Mostly just recoil, but there are sometimes 
collisions. 

\subsection{Linear Collision Cascade Regime}
Recoil is minimal. Cascading collisions produce sputtering, Binary collisions 
(independent atom-atom collisions)


\subsection{Heat Spike Regime}
\begin{itemize}
    \item
Everything is happening all at the same time 
    \item
    Collisions cannot be considered independent 
    \item
    Energetic ion and dense material 
    \item
    feels like localized heating
\end{itemize}

\section{Sputtering Yield}
Defined as atoms ejected per incident ion. 

Dependent on a million things (incident angle, incident ion energy, 
bombarding species, target materials). Varies from 0.1 - 10. 

\subsection{Sputtered Atoms}
Tend to have small energies between 1-10 eV (evaporation would have just 0.1eV). 
Sputtering results in better film adhesion and density.
There is a very wide distribution in sputtered atom energies because the process 
is roughly stochastic.

\section{Contamination}
Contamination causes all sorts of issues and comes from a million different 
sources 

\subsection{Sources}
\begin{itemize}
    \item
    Insufficient vacuum
    \item
    Outgassing
    \item
    Oxygen Impingement rate 
\end{itemize}

Resistivity of film can increase tremendously from implanted oxygen. 

\subsection{Solution}
\begin{itemize}
    \item
    Pump down to lower pressures 
    \item
    fill chamber with argon and pump down again to dilute remaining gas 
\end{itemize}

\section{Reactive vs Non-Reactive Sputtering}
\subsection{Non-Reactive}
\begin{itemize}
    \item
    Sputtering gas is not incorporated into film 
    \item
    Common choices are noble gases (mainly Argon)
\end{itemize}

\subsection{Reactive Sputtering}
\begin{itemize}
    \item
    Sputtering gas will be incorporated into film
    \item
    This allows for certain alloys to be made 
    $(Al_2 O_3, SiO_2, \ldots)$
    \item
    Common choices are $O_2, N_2, NH_3$.
\end{itemize}

\section{Laser Ablation Deposition}
Instead of using a plasma to excite your target, you use a laser. 
Useful for complex compounds. 

\section{Molecular Beam Epitaxy (MBE)}
You try to deposit a single crystal film onto a substrate by matching 
lattice constants. 

\chapter{Chemical Vapor Deposition (CVD)}
Instead of providing mechanical or thermal energy, you provide chemical 
potential energy of atoms in a gas. 

\section{Mechanism}
\begin{enumerate}
    \item
    Mass transport of reactant gases
    \item
    Gas-phase reaction leading to film precursors (homogeneous)
    \item
    Mass transport of film precursors
    \item
    Adsorption of film precursors
    \item
    Surface migration of film forming materials to the growth site 
    \item
    surface reaction (heterogeneous products)
    \item
    Desorption of byproducts
    \item
    Mass transport of byproducts
\end{enumerate}

\section{CVD Process Family}
\begin{itemize}
    \item
CVD can occur in a million different ways. 
    \item
Can deposit metals, semiconductors, insulators, barriers, silicides, etc. 
    \item
Energy for CVD can be derived from thermal, plasma, and photon energy. 
    \item
There's a whole table in the slides
\end{itemize}

\section{CVD Reaction Types}
Thermal Decomposition (not common in semiconductors), Redox Reactions, 
Exchange Reactions, Coupled Reactions.

\subsection{Precursor}
\begin{itemize}
    \item
    Chemicals must be volatile/gaseous.
    \item
    Film formation must be thermodynamically predicted 
    (decrease Gibbs Free Energy).
    \item
    Byproducts must also be volatile.
\end{itemize}

\subsection{Atmospheric Pressure CVD}
Very fast reaction rate because there's a whole lot of gas. 
Main issue is contamination.



\section{Modeling CVD Processes}\
Reactions occur at the surface. The concentration of species in gas (N$_g$) drops 
to the contration of species at the surface $(N_s)$ across the boundary 
(or tagnant) layer of thickness $(\delta)$.

Flux to surface $(J_g)$ and flux consumed in film $(J_s)$ are controlled 
by the reaction-rate constant $(k_S)$ and the mass trnasport constant $(h_g)$.

\equations{
    J_g
    =
    \frac{D_g}{\delta}
    (N_g - N_s)
    =
    h_g 
    (N_g - N_s)
    \textrm{
    \hp \hp
        Flux to Surface}
}

\equations{
    J_s
    =
    k_s \times N_s
    \textrm{
        \hp \hp \hp \hp 
        Flux consumed in grown film}
}

At steady state, fluxes must be equal 
\equations{
    J_g 
    =
    J_s
    =
    \frac{N_g}{\frac{1}{h_g} + \frac{1}{k_s}}
    =
    \frac{h_g k_s N_g}{h_g + k_s}
}

The film growth rate is given by 
\equations{
    v 
    \propto 
    J_s 
    =
    \frac{h_g k_s N_g}{h_g + k_s}
    % \propto 
    % h_g N_g
}

If the film growth rate is mainly limited by mass transport, then you have a 
diffusion limited regime $h_g << k_s$.

If it's limited by reaction rate, then $h_g >> k_s$.

Those last 2 limitations are the important things to remember about CVD deposition 
rates. 

The velocity is 
\equations{
    v 
    \propto 
    h_g N_g
    \hp 
    \textrm{limited by mass transport}
    \\
    v 
    \propto 
    k_s N_g
    \hp 
    \textrm{limited by reaction rate}
}

\subsection{Flux to Surface}
When we are difficult (mass transport) limited, the film growth rate 
depends on the flux to the surface $(J_s)$. 

If we have a low vacuum to atmospheric pressure, then a stationary boundary layer 
will develop, and the precursor gas must diffuse through it (caused by 
laminar flow, not an issue if high vacuum because there are minimal gas 
interactions). 

If $delta(x)$ is the boundary layer thickness and is a function of distance 
along the substrate $x$, then 
\equations{
    \delta(s)
    =
    \left(
        \frac{\eta x}{\rho V}
    \right)^{1/2}
}
Where $\eta$ is the gas viscosity, $\rho$ is the gas density, and $V$ is the gas 
stream velocity.

The average thickness of the boundary layer is 
\equations{
    \delta 
    =
    \frac{1}{L}
    \int^{L}_{0}
    \delta(x) dx 
    =
    \frac{1}{L}
    \left(
        \frac{\eta}{\rho V}
    \right)^{1/2}
    \int^{L}_{0}
    x^{1/2} \, dx \, 
    =
    \frac{2}{3}
    L 
    \left(
        \frac{\eta}{\rho V L}
    \right)^{1/2}
    =
    \frac{2L}{3 \sqrt{Re}}
}

Where $Re$ is Reynold number which is the ratio of inertial forces to viscous 
forces ($< 2000$ is laminar and over is turbulent, CVD should be laminar).

Use Fick's first law of diffusion across the boundary layer 
\equations{
    J_g 
    =
    -D \frac{d \phi}{d x}
    \approx 
    - D 
    \frac{N_s - N_g}{\delta}
}
Where $D$ is the diffusivity or diffusion constant. Further, at steady state, 
then $J_g = J_s$. 

For a diffusion limited process, the molecules on the surface will be sconsumed 
as fast as they arrive, therefore the concentration at the surface 
\equations{
    N_s = 0
    \rightarrow J_s = J_g 
    =
    D 
    \frac{3 N_g \sqrt{Re}}{2L}
}

\subsection{Gas Flow Rate Dependence}
At low flow rates, the film growth rate shows a square root dependence. 
At high flow rates, the growth rate reachesa maximum and 
becomes independent of flow. At this point, the 
reaction rate controls the deposition.

The rate limiting phenomenon can dictate the design of the reactor. 

A vertical design is good for reaction-rate limited deposition, while a 
horizontal design that maximizes gas flow is best for a mass transport 
limited deposition. 


\section{Surface Adsorption and Reaction} 
It is worthwhile to study what occurs after a precursor molecule/atom reaches 
the surface. 

The must find a site to adsorb to the surface long enough for the decomposition/
reduction/exchange reaction to take place. This will depend on 
\begin{itemize}
    \item
    Precursor arrival rate
    \item
    Density of sites available for adsorption of precursor molecules 
    \item
    substrate temperature (that determines the residence time required for the 
    reaction)
\end{itemize}
(sites consist of defects/discontinuities on the surface). 

The elemental adatoms that result are not stationary but move in 
 the surface uintil they encounter high-energy binding sites 
 (vacancies, edges, kinks)


Trapped adatoms nucleate growth of islands by trapping other migrating adatoms, 
eventually developing a continuous film.

\begin{center}
    Nucleation 
    $\rightarrow$
    Growth of Nuclei 
    $\rightarrow$
    Coalescence
    $\rightarrow$
    Film Growth
    $\rightarrow$
    Continuous Film.
\end{center}
An adatom is a single atom on a crystal 


\subsection{Material Bonding}
Chemical bonds vs adhesive via coordinate bonds (Van der Vaals, hydrogen, 
ionic) which determines if layer can exist and not delaminate.

\subsection{Film Growth Rate vs Temperature}
At high temperatures, the reaction rate is sufficiently, we get a mass shortage. 

\equations{
    g 
    =
    C e^{\frac{- \Delta E}{k_b T}}
    \Rightarrow 
    \log(g)
    =
    \frac{- \Delta E}{k_b} \frac{1}{T}
}

Log and $1/T$ are linearly proportional to each other (If reaction rate 
limited regime). As $T$ increases you get back to a mass transport limited 
regime.

While higher temperatures produce higher growth rates, they also promote 
gas phase reactions that are generally undesirable in CVD because they produce 
particulates that fall onto the substrate and create film defects. To overcome 
this, we can 
\begin{itemize}
    \item
    Use a different precursor
    \item
    Reduce pressure (LPCVD) - increases diffusivity and reduces heat transfer.
\end{itemize}

\section{Film Growth Mechanisms}
What happens once the reactant sreach the surface where 
the film needs to grow? 

\subsection{Nucleation of Growth}
Generally, a stochastic process that strongly depends on the 
precursor arrival flux, adsorption site density, surface migration 
temperature, etc. 

\subsection{Type of Growth}
Island vs layer determines the grain size and boundaries while step 
vs cluster determines the roughness and grain orientation. 
These can be thought of as a thermodynamic competition between 
surface / intersurface energies. 

\begin{itemize}
    \item
    Frank-van der Merve mode (2 dimensional growth mode)

    $\gamma_{substrate} \geq \gamma_{film} + \gamma_{interface}$
    \item
    Volmer-Weber mode (Island growth mode)

    $\gamma_{substrate} < \gamma_{film} + \gamma_{interface}$
    \item
    Stranski-Krastanov mode

    Initially, we have 
    $\gamma_{substrate} \leq \gamma_{film} + \gamma_{interface}$

    but then it turns to
    $\gamma_{substrate} < \gamma_{film} + \gamma_{interface}$ 
    due to strain effect
\end{itemize}

$\gamma$ is the surface energy or interface energy. 

\subsection{Step Coverage}
This is dependent on several factors 

\begin{itemize}
    \item 
    Flux arriving at the surface: 
    local visibility angle nad precursor flux 
    $\theta = \int J_g d \theta$.
\end{itemize}

As the visibility angle is reduced, we see a thickness
$\theta = \tan^{-1}(\frac{w}{h})$.

Another thing that might affect this thickness given a hole/feature-y area is 
the fact that complex surfaces have far more surface area for the same amount 
of deposited material. SO basically, the two key factors are 

\begin{itemize}
    \item
    Mean Free Path
    \item
    Surface Diffusivity
\end{itemize}

The outside corners have an angle of $\theta = 270 \deg$, which is the largest 
anywhere, the thus those corners have the most material deposited on them, and 
they might actually completely cover the hole with material.

\subsection{Reactive Stiction (Sticking) $S_c$}
That is the sticking coefficient 

\equations{
    S_c 
    =
    \frac{\Gamma_{incident} - \Gamma_{re-emission}}{\Gamma_{incident}}
    =
    \frac{\Gamma_{reaction}}{\Gamma_{Incident}}
}

Direct diffusion is $S_c = 1$, and bouncing means that $S_c < 1$.

Low $S_c$ means that the film is conformal (it fills features evenly). 

\section{Thing to Keep In Mind}
Uniformity of the film: 
Thickness, composition, structure, and properties are all affected by 
all sorts of factors. 

\begin{itemize}
    \item
    Throughput esired
    \item
    Intra-wafer and inter-wafer uniformity
    \item
    Film composition/purity
    \item
    Conformity on wafer topography
\end{itemize}

\subsection{Stresses}
If there is a lattice mismatch or thermal expansion mismatch, there might be 
residual stresses in the film. 

Tensile stresses bring the sides up, and compressive stresses bring the center 
up. Both are bad. 

Types of material structures of deposited films include 

Polycrystalline, amorphous, epitaxial (single crystalline), 
columnar.

The type of resulting films affects basically everything.

\subsection{LPCVD}
LPCVD is usually reaction rate limited, so the amount of material just into 
the material is less important as much as the temperature of the substrate.

\subsection{PECVD}
PECVD is plasma-enhanced CVD. You use high energy ions to activate the precursor 
so that it is more reactive (increasing reaction rate), and the electrons 
also bombard the substrate to make it more reactive. 

A common application is a low temperature insulator over metals. 

typical process conditions include:
\begin{itemize}
    \item
    0.5 to 5 torr
    \item
    200-400C 
    \item
    Surface reaction rate limited 
    \item
    Most common hot wall furnace
\end{itemize}

PECVD results in a low sticking coefficient. This results in low stress films, 
but because of the extra reactions, there might be contamination.

pros include uniformity, good step coverage, and low temperatures 

cons include particle contamination, chemical contamination (high H content), 
and pinhole errors in the deposition.

\subsection{MOCVD}
This is known as metal-organic CVD (or metal-organic vapor phase epitaxy). 

Gases included arsine, phosphine, ammonia, disilane, and silane, 

and liquid sources are trmethylgallium, trymethyl indium, trymethyl aluminum 
that react to form films including GaAs.

This is mainly used in solar cells and VCSELs and other things.

It creates single layer growth. 

\subsection{ALD}
Atomic Layer Deposition. CVD is broken down into 2 half-reactions. 

\begin{itemize}
    \item
    Precursors are kept separate and introduced squentially to complete
    the reaction. 
    \item
    One cycle lays down one monolayer for the film in a well controlled manner
    \item
    self-terminating reaction.
\end{itemize}

pros include uniformity, monolyaer control, step coverage, and no gas 
phase reaction. 

The main con is that it is very very slow.

\section{Summary}
\begin{itemize}
    \item
Seen all the steps of the CVD mechanism. 
    \item
We classified different types of reactions 
with examples of their use. 
    \item
    Developed a simple intuitive model to identify 
    different process regimes 
    (mass transport limited and reaction rate limited regimes)
    \item
    film growth characteristics
    \item
    conformal coverages
    \item
    Different variants and types of CVD
\end{itemize}

\section{MIDTERM 1 PREP}
What's on the exam: Everything in Deposition (up to the end of Deposition). CVD 
will be on the exam, dry etching will not be on the exam. 

Exam is on monday during class, should take about 1 hour, but we'll have the room 
for a full 2 hours. NO formulas need to be memorized, we will be given a sheet. Just 
know how to use them. Not heavy calculation- if you set up the equation correctly, 
you can make estimates. YOU CAN BRING A CALCULATOR. The best way to prepare 
is go through the notes, read texts, go through the homeworks (best way). 

Know how cleanrooms are classified and clean room safety processes. 
Stokes settling and number of particles per hour that settle on a substrate . 

Sources and types of contamination. Problems caused by contamination adn humidity. 
Purpose of bunny suit and rules. 

Know an MSDS. 

NFPA and HMIS labels, how to read them, what they're meant for. 

\subsection{Lithography 1}
\begin{itemize}
    \item
    Process flow in microfabrication
    \item
    Typical patterning process flows (liftoff vs masking)
    \item
    Lithography process flows
    \item
    Cleaning processes (RCA1 vs RCA2, pihrana solution, O2 plasmas) and their purposes
    \item
    supercritical drying, ultrasonic cleaning, 
    promoting surface adhesion for lithography
    \item
    Different methods of applying PR to a substrate with 
    an emphasis on spinning. Know what happens during spinning 
    and what are edge beads and why they form, why they're a problem, and how to 
    alleviate the problem
    \item
    purpose of soft baking before exposure
    \item
    (me added) all baking processes and their purposes
\end{itemize}

\subsection{Lithography 2}
\begin{itemize}
    \item
    Radiation for exposures, mercury lines, types of masks, transparency of mask materials 
    at  different wavelengths
    \item
    Exposure systems, contact, proximity, projection, their resolution,
    advantages nad disadvantages
    \item
    Resolution, diffraction and its impact resolution, diffraction limit, 
    point spread function, numerical aperture, modulation transfer function. 
\end{itemize}


\subsection{Lithography 3}
\begin{itemize}
    \item
    Components of a PR< resist tone nad what light does to PR's of different tones, 
    different yypes of resists (single component, two-component, image reversal)
    \item
    What remains and what's gone with different tone resists and lightfield/darkfield 
    masks
    \item
    effects of over and under-exposure for different tones of resists 
    (Fresnel Zone, \textbf{other zone (google)})
    \item
    Resist sinsitivty and contrast; side-wall profiles 
    \item
    Development and post-development processing; resist stripping (acetone bath).
\end{itemize}

\subsection{Lithography 4}
\begin{itemize}
    \item
    Alignment errors and how they einfluence exposure
    \item
    Vernier and Moire fringes for allignment 
    \item
    Sequence of things to do if you want to enhance resolution
    \item
    Resists: anti-reflection coatings, multi-layer resists, lift-off resists
    \item
    Masks: phase shift masks, mask geometry engineering
    \item
    Exposure: concept of different illumination systems, immersion 
    lithography concepts
    \item
    Advanced lithography systems, particle-beams, maskless, soft.
\end{itemize}


\subsection{Deposition 1}
\begin{itemize}
    \item
    Vacuum concepts, mean-free path and its relation to pressure, 
    what does high vacuum do for deposition processes, technology ;
    pumps and gauges
    \item
    Evaporation, vapor pressure and oiling; Clasusium Clapeyron Equation and how it is used 
    to get the temperature fro a given 
    vapor pressure; 
    Hertz-Knudsen equation and how it is used 
    to calculate the mass evaporation rate 
    \item
    Lamberts Cosine law and its use in determining 
    evaporative flux 
    desnity; point sources and plane sources; 
    film thickness variation
    \item
    Evapoeation technology, e.g., e-beam evaporation, virtual point 
    sources; direactionality and coverage, film thickness measurement with quartz 
    crystal monitors - concept and calculations.
\end{itemize}

\subsection{Deposition 2}
\begin{itemize}
    \item
    Sputtering concept, how and why plasmas are used, 
    why is the pressrue in sputtering higher than that 
    for evaporation
    \item
    Rudimentary idea of plasmas, basic zones, voltage drops across a plasma
    \item
    Particle energy and sputtering regimes; sputtering yield and 
    factors affecting sputtering yield; Directionality, and sub-cosine 
    flux distributing
    \item
    Comparisons between sputtering and evaporation, reactive cputtering nad laser 
    ablation
\end{itemize}

Deposition is more conformal with sputtering because there are more collisions.

Know why the glow region is slightly positive (gas moves slower than electrons)

\subsection{Deposition 3}
\begin{itemize}
    \item
    CVD mechanisms and reaction types. Conditions for use of a reaction 
    in CVD, Diffusion across the boundary layer and surface reaction rates and how 
    they determine CVD process regime
    \item
    Diffusion vs reaction rate limited regimes, growth rate behavior in these 
    regimes
    \item
    Surface reaction and film growth mechanism adn how these determine 
    the type of film obtained 
    \item
    Different CVD processes, the presmises behind the variation nad their advantages 
    and disadvantages nad applications; ALD and the use of sequenced coupled reactions to get 
    atomic layers. 
    \item
    Coverage, Reactive Sticking coefficient and how its effects conformal 
    coverage.
\end{itemize}

Why use LPCVD vs just regular atmospheric CVD (if its reaction rate limited, you 
don't need much gas in the chamber. Also, the vacuum keeps the substrate hot)

\chapter{Removal - Dry Etching}
How to get crap off of your substrate.

\section{Subtractive Processes}
Dry Etching is generally more used than dry etching 

There are many many different etching techniques 
\begin{itemize}
    \item
    Wet and dry etching
    \item
    Focused ion beam (FIB) milling
    \item
    Laser Machining
    \item
    Ultrasonic drilling
    \item
    Electrical discharge machining (EDM)
    \item
    Tradition precision machining
\end{itemize}

\subsection{Dry Etching}
A solid surface is etched by the gas or vapor phase. 
Recall ashing (dry stripping of resists). 

In general, ry etching has fewer disposal problems, less corrosion of metal 
features in the structure, less undercutting and broadening of PR features, and 
cleaner surfaces. 

types of dry etching include 
\begin{itemize}
    \item
    Physically via ion bombardment
    \item
    Chemically by a reaction throuhg a reactive species at the surface 
    \item
    Combination of physical and chemical reactions
\end{itemize}

\begin{center}
    Source (CF4)
    $\to$
    Plasma 
    $\to$
    Reactant (F0)
    $\to$
    Substrate (Si) 
    $\to$
    product (SiF4)
\end{center}

Dry etching, CVD, and evaporation all use vapor phase reactants.
in CVD, the surface reaction gies rise to at least one solid product, 
but in etching one hopes that all the reaction products are vapors 
(having high vapor pressures). 
Dry Etching, sputtering, adn version of CVD (PECVD) use plasmas, though 
suttering uses a plasma for different purposes. 

\begin{center}
    Source (SiH4)
    $\to$
    Substrate (Si) 
    $\to$
    gas byproduct (2H2)
\end{center}

Dry etching is very similar to regular lithography because you put on 
the PR, and then dry etch, and the perform lift-off with the PR.

Anisotropic etching is vertical etching with no horizontal etching. 
Isotropic etching will etch underneath the mask, and that's bad.

dry etching is \textbf{highly directional} because the atoms hitting the substrate 
are charged, so they move along the path of the electric field. 

\subsection{SPUTTERING IS LESS DIRECTIONAL}









\end{document}