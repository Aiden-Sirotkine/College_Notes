\documentclass[fleqn]{report}
\usepackage{geometry}
\usepackage{amssymb}
\usepackage{fancyhdr}
\usepackage{multicol}
\usepackage{blindtext}
\usepackage{color}
\usepackage[fontsize=16pt]{fontsize}
\usepackage{lipsum}
\usepackage{pgfplots}
\usepackage{physics}
\usepackage{mathtools}
\usepackage[makeroom]{cancel}
\usepackage{ulem}
\usepackage{esint}

\geometry{a4paper, margin=1cm} % Set paper size and margins
\graphicspath{ {../Images/} }
\setlength{\columnsep}{1cm}
\addtolength{\jot}{0.1cm}
\def\columnseprulecolor{\color{blue}}
\date{Fall 2025}

\newcommand{\textoverline}[1]{$\overline{\mbox{#1}}$}

\newcommand{\hp}{\hspace{1cm}}

\newcommand{\const}{\textrm{const}}

\newcommand{\del}{\partial}

\newcommand{\pdif}[2]{ \frac{\partial #1}{ \partial #2} }

\newcommand{\pderiv}[1]{ \frac{\partial}{ \partial #1} }

\newcommand{\comment}[1]{}

\newcommand{\equations} [1] {
\begin{gather*}
#1
\end{gather*}
}

\newcommand{\numequations} [1] {
\begin{gather}
#1
\end{gather}
}

\newcommand{\twovec}[2]{ 
\begin{pmatrix}
#1 \\ 
#2
\end{pmatrix}
}

\title{PHYS 486}
\author{Aiden Sirotkine}

\begin{document}

\setlength{\headsep}{10pt}
\setlength{\topmargin}{-2cm}

\pagestyle{fancy}
\maketitle
\tableofcontents
\clearpage

\chapter{PHYS486}
Quantum Mechanics lol. This is apparently a math language more 
than actual physics.

\section{What is QM}
\begin{itemize}
    \item 
Classical Mechanics - $F = ma$, solve for $x$ and $p$, position and momentum.
    \item
Classical E \& M - $\ddot \vec E(\vec r, t) - \nabla^2 \vec E(\vec r , t) = 0$
\end{itemize}

You have observable entities and you work with just the observable entities to 
figure out the exact numbers. 

\subsection{The Quantum State $\ket \psi$ }
This is a state vector.

The quantum state by definition is \textbf{not} observable.

An observable operator is something like position that describes the system that 
we can actually see. 

\equations{
    \hat A \ket{\psi} 
    =
    a \ket \psi 
}
where $\hat A$ is an observable and $a$ is the measurement out.

This is just an eigenvalue equation.

\subsection{Schrodinger Equation}
\equations{
    i \hbar \frac{\del}{\del t} \ket{\psi(t)} = \hat H \ket{\psi(t)}
}

$\hat H$ is a matrix and $\ket{\psi(t)}$ is a vector.

\section{Summary of "Central Phenomena"}
Imagine an electron orbiting around a proton. 

The electron \textbf{cannot} take arbitrary orbits.

The electron can only have \textbf{certain discrete} orbits.

\subsection{Superpositions}
Consider two discrete states in which an electron can orbit around a proton. 

Because $0$ and $1$ are solutions for the electron, a superposition of $0$ and $1$ is 
also a solution for the electron.

\equations{
    \alpha \ket 0 + \beta \ket 1
}

This is the electron being in both discrete states at the same time.

\equations{
    \sim
    \left[ 
        \ket 0 \sim {1 \choose 0}, 
        \ket 1 \sim {0 \choose 1}, 
        SP {\alpha \choose \beta}
    \right]
}

\section{Probabilistic Interpretation}
Consider a system with the solution 
\equations{
    \alpha \ket 0 + \beta \ket 1
}

If we have a thing that measures the energy, we can get 
\textbf{either} $0$ \textbf{or} $1$ with probabilities depending on the 
values $\alpha$ and $\beta$.

\section{Entanglement}
Consider atom $A$ with state $\{ \ket{0}_A, \ket{1}_A \}$
and atom $B$ with state $\{ \ket{0}_B, \ket{1}_B \}$.

It is completely legal to have an entangled state with the form 

\equations{
    \alpha \ket{0}_A \ket{0}_B 
    +
    \beta \ket{1}_A \ket{1}_B 
}


\section{Course Outline}
\begin{enumerate}
    \item 
    Basic Rules 
    \item 
    Wave Mechanics ("toy models")
    \item 
    Formalism (Midterm 1)
    \item 
    Simplest Real System (Hydrogen Atom) (Midterm 2)
    \item 
    Intro to Multiparticle Descriptions
\end{enumerate}

\chapter{Actual Physics Now}

\section{Black Body Radiation}
Consider an object that perfectly absorbs radiation and turns it into heat. 

Place it in thermal equilibrium 
(finite $T$, constant $E$, absorption $\rightarrow$ emission).

\subsection{Classical Description}
Consider the system as a harmonic oscillator. 

\equations{
    \langle E \rangle = k_B T
    \Rightarrow 
    I(\omega)
    \sim 
    \frac{\omega^2 k_B T}{\pi^2 c^3}
    \hp
    \textrm{Rayleigh-Jeans Law}
}

This \textbf{does not} match experimental data. 

While the classical description says that temperature will go up forever, the 
experiments show that the temperature stops increasing for large frequencies.

\subsection{Plack, 1900}
Complete guess that maybe energy is discretized. Maybe the harmonic oscillator 
comes in chunks of size $\hbar \omega$.

Now, energy can be written in the form 
\equations{
    P(E)
    =
    \alpha e^{- \frac{-E}{k_B T}}
    \longrightarrow 
    \langle E \rangle 
    =
    \frac{\hbar \omega}{\exp((\hbar \omega / k_B T) - 1)}
    \\
    I(\omega)
    =
    \frac{\hbar \omega^3}{\pi^2 c^3 \exp((\hbar \omega / k_B T) - 1)}
}
This perfectly matches experimental data.

\section{Photoelectric Effect}
Consider a circuit that absorbs light with frequency $\omega$ and intensity $J$
to create a potential.

As light intensity $J$ increases, current $I$ increases, but $V_0$ stays the same. 

$V_0$ is dependent on $\omega$ and the properties of the material. 

Einstein figured out that there is an energy threshold such that light will 
not be absorbed if it does not have a high enough energy. 

The kinetic energy of the electron can be given in the form:
\equations{
    kE 
    =
    \hbar \omega 
    -
    W 
    > 
    - eV_0
}

\section{Wave Particle Duality}
Consider a photon with energy and momentum
\equations{
    E = \hbar \omega
    \hp 
    p = \hbar k 
    \left(
        = \frac{h}{2 \pi} \cdot \frac{2 \pi}{\lambda}
    \right)
}

The photon has no mass $(m = 0)$, so consider the relativistic mass of the photon 
\equations{
    E^2 = m^2 c^4 + p^2 c^2 
    \rightarrow 
    p^2 c^2
    \rightarrow 
    \\
    E = pc 
    \hp 
    \omega = kc 
    \hp 
    f \lambda = c
}

\subsection{De Broglie, 1924}
Everything has a wavelength and frequency

\equations{
    \lambda 
    =
    \frac{h}{p}
}

The wavelength for even electrons is extremely small, so this feature 
is very hard to observe even if its true for everything.

\subsection{Davisson and Germer, 1927}
Electrons act in the same wave-like nature as light when shot through 
a small slit (double slit experiment but for electrons).

This proved the De Broglie wavelength theory for particles.

\chapter{Beginning of QM}
The issue with quantum mechanics is that the math is derived only 
from experimental data. There is no classical physics basis or derivation 
behind quantum mechanics. 

Multiple textbooks will bring up quantum mechanics in different ways. 
The professor recommends the Townsend QM textbook 

\chapter{Ruleset}
Motivated from experimental observations. These are base postulates 
with no derivations. 

\section{System State}
The state of the physical system is a vector $\ket \Psi$ 
in Hilbert Space. 

A Hilbert space $(\equiv H^2)$ is a complex vector space with a well-behaved 
inner product.

(bra + ket = braket). You have to take the complex conjugate
\equations{
    \bra \Psi \ket{\Psi}
    \hp 
    \textrm{"bra"}
    =
    (\ket{\Psi}^*)^T
    \hp 
    \bra \alpha \ket{\beta}
    =
    \bra \beta \ket{\alpha}^*
}

\section{Observables}
Observables are operators in Hilbert Space, and they have to 
have real eigenvalues. 

The measurement outcomes of the observable are the eigenvalues.

\equations{
    \hat A \ket{\alpha_n}
    =
    a_n \ket \alpha_n
}

I think $\hat A$ is the observable of the state $\ket{\alpha_n}$ 
and $a_n$ is the $n$-th eigenvalue of the observable.

\subsection{Born's Rule}

The probability for an outcome $a_n$ is given by 
\equations{
    P(a_n)
    =
    |\bra{\alpha_n} \ket{\Psi}|^2
}
Where $\alpha_n$ is an eigenvector of the operator. 

\subsection{Expected Value}
Given an observable operator $\hat A$, the expected value is given by 

\equations{
    \langle \hat A \rangle 
    =
    \bra{\Psi} \hat A \ket{\Psi}
}

\subsection{Waveform "Collapse"}
Given a quantum state $\ket{\Psi}$, when you measure 
an observable $\hat A$, you return only a single result $a_n$

\equations{
    \ket{\Psi}
    \longrightarrow 
    \ket{\alpha_n}
}
The system state collapses to just the individual eigenvector state. 

\section{Time Evolution (Schrodinger Equation)}
\equations{
    i \hbar \frac{\del}{\del t} \ket{\Psi}
    =
    \hat H \ket{\Psi}
    \hp 
    \hat H = 
    \textrm{Hamiltonian (Energy Operator)}
}

The Eigenvectors of $\hat H$ are stationary states (do not change with 
time). 

This is the time-independent Schrodinger Equation 
\equations{
    \hat H \ket{\Psi}
    =
    E_h \ket{\Psi}
}

Now you know the entirety of quantum mechanics. Everything else 
can be derived. 

\chapter{Experiments}
\section{Double Slit}
If you shine light through a small slit, you will see a diffraction 
pattern. 

Imagine shining light at 2 slits and an attennuator at the other 
side that \textbf{measures} the location of each photon at the screen. 

If you just shoot a single photon, the location will be random, but if you
shoot multiple photons, there is a very distinct diffraction pattern that 
demonstrate the probability distribution of the list. 

\section{Stern-Gerlach}
Consider a beam of atoms with varying spins. This beam of atoms is 
put into a magnetic field gradient. 

If the magnetic moment is up, the atom will go up. If the magnetic 
moment is down, the atom will go down. 

If we put the beam through this field, then half the atoms will go up, 
and half the atoms will go down (probabilistically).

This makes sense. 

The quantum state of each atom is given by 
\equations{
    \ket{\Psi}
    =
    \frac{1}{\sqrt{2}}
    \left(
        \ket{\uparrow}
        +
        \ket{\downarrow}
    \right)
    \hat z
}

Consider a basis 
\equations{
    \ket{\uparrow}
    =
    \begin{pmatrix} 1 \\ 0 \end{pmatrix}
    \hp
    \ket{\downarrow}
    =
    \begin{pmatrix} 0 \\ 1 \end{pmatrix}
}

Let's call spin up $+1$ and spin down $-1$. 

Because of those numbers, our observable operator is 
\equations{
    \hat A 
    =
    \begin{bmatrix}
        1 && 0 \\
        0 && -1
    \end{bmatrix}
    \hp 
    \hat A \ket{\uparrow}
    =
    + 1 \ket{\uparrow}
    \hp 
    \hat A \ket{\downarrow}
    =
    - 1 \ket{\downarrow}
}

Now we have our quantum state and our observable operator matrix. 

Because we have two possible eigenvectors, the superposition 
of our quantum state can be written as 

\equations{
    \ket{\Psi}
    =
    \left(
        \sum_k \ket{k} \bra{k}
    \right)
    \ket{\Psi}
    \equiv 
    \mathbb{\lambda}_i
    , 
    \ket{k}
    \hp 
    \textrm{kth basis vector}
    \\
    =
    \sum_k 
    c_k \ket{k}
    \hp 
    c_k = \bra{k}\ket{\Psi}
}

\subsection{Born's Rule}
\equations{
    P(+1)
    =
    |\bra{\uparrow}\ket{\Psi}|^2
    =
    \frac{1}{2}
    |\bra{\uparrow}\ket{\uparrow}
    +
    \bra{\uparrow}\ket{\downarrow}|^2
}

Because we're using Dirac notation, we know exactly what inner products
are parallel and what inner products are orthonormal.


\equations{
    P(+1)
    =
    |\bra{\uparrow}\ket{\Psi}|^2
    =
    \frac{1}{2}
    |1
    +
    0|^2
    =
    \frac{1}{2}
}

\subsection{Expected Values}
\equations{
    \langle \hat A \rangle 
    =
    \bra{\Psi} \hat A \ket{\Psi}
    \hp 
    \hat A 
    =
    \sum_k 
    \lambda_k 
    \ket{k} \bra{k}
    \hp 
    \lambda_k 
    =
    \textrm{kth eigenvalue}
}

You can use the Spectral Theorem to get 
\equations{
    \hat A 
    =
    \ket{\uparrow}
    \bra{\uparrow}
    -
    \ket{\downarrow}
    \bra{\downarrow}
}

So to find the expected value, we use 

\equations{
    \langle \hat A \rangle 
    =
    \bra{\Psi} \left(
        \ket{\uparrow}
        \bra{\uparrow}
        -
        \ket{\downarrow}
        \bra{\downarrow}
    \right) \ket{\Psi}
    =
    \ldots 
    = 0
}

\section{Double Slit and the Wave Function}
we need $\ket{\Psi}$ in the position basis. 

Imagine if the space was discrete. 
\equations{
    \ket{\Psi}
    =
    \sum_k 
    c_k \ket{x}
}

But our space is continuous, so instead of taking a discrete sum, 
we take an integral. 

\equations{
    \bra{x_k} \ket{x_j}
    =
    \delta (k - j)
    \hp 
    \int dx 
    \bra{x_k} \ket{x_j}
    =
    1
    \\
    \ket{\Psi}
    =
    \int \, dx \,
    \ket{x} \bra{x} \ket{\Psi}
    =
    \int \, dx \, 
    \Psi(x) \ket{x}
}

That's the wave function 

implement Born's Rule to get 

\equations{
    P(x)
    =
    |\bra{x}\ket{\Psi}|^2
    =
    |\Psi(x)|^2
}

\subsection{Bra-Ket Notation}
\equations{
    \bra{x} \ket{y}
}
Is an inner product that yields just 1 number.

\equations{
    \ket{y} \bra{x} 
}
is an outer product that yields a matrix.

\subsection{Stationary State}
A stationary state is an eigenstate of the Hamiltonian.
It has a perfectly well-defined energy. 

\section{Interpreting the Double Slit}
You have a quantum state dependent probability density 

\equations{
    P(x)
    =
    |\Psi(x)|^2
}

That's a probability density function that needs to be normalized 

\equations{
    \int^\infty_{-\infty}
    dx |\Psi(x)|^2 
    =
    1
}

So to find the expected value, we get 

\equations{
    \langle x \rangle 
    =
    \int^\infty_{-\infty}
    dx \, 
    x 
    |\Psi(x)|^2 
    \hp
    \langle f(x) \rangle 
    =
    \int^\infty_{-\infty}
    dx \, 
    f(x) 
    |\Psi(x)|^2 
}

And the variance of that probability distribution is given by 
\equations{
    \sigma_x^2 
    =
    \langle x^2 \rangle 
    -
    \langle x \rangle^2
}

Given a quantum state, if we can measure the position, we 
can \textbf{only} estimate $|\Psi(x)|^2$ with finite accuracy. 
Each position sampled is just a single point of a probability distribution 
that cannot be directly measured. 

\chapter{Wave Mechanics}
Let's start with 

\equations{
    \ket{\Psi(x)} 
    =
    \Psi(x, t)
    \hp
    H 
    =
    - \frac{\hbar^2}{2 m}
    \left( \frac{\del}{\del x} \right)
    +
    V(x)
}

Where $V(x)$ is the potential. This is our wave function 
and our Hamiltonian. 

The Schrodinger Wave equation is 
\equations{
    i \hbar \frac{\del}{\del t} \Psi(x, t)
    =
    - \frac{\hbar^2 }{2m} \frac{\del^2}{\del x^2} \Psi(x, t) 
    +
    V(x) \Psi(x, t)
}

\section{Important Properties of the S.W.E.}
\subsection{ Unitary }
Conservation of probability.
Found is Griffiths Textbook chapter 1.4

\equations{
    0
    =
    \frac{d}{dt}
    \int^\infty_{-\infty}
    dx \, 
    |\Psi(x, t)|^2 
    =
    \int dx \, 
    \frac{\del}{\del t}
    |\Psi(x, t)|^2
    =
    \\
    \int dx \, 
    \left(
        \Psi^* \frac{\del}{\del t} \Psi 
        +
        \Psi \frac{\del}{\del t}
        \Psi^*
    \right)
    \rightarrow 
    \int dx \, 
    \left(
        \Psi^* \frac{\del^2}{\del x^2} \Psi 
        -
        \Psi \frac{\del}{\del x^2}
        \Psi^*
    \right)
    \frac{i \hbar}{2 m}
    \\
    =
    \int dx \, 
    \frac{d}{dx}
    \left(
        \Psi^* \frac{\del}{\del x} \Psi 
        -
        \Psi \frac{\del}{\del x}
        \Psi^*
    \right)
    \frac{i \hbar}{2 m}
}

So now we have to 

\equations{
    \left(
        \Psi^* \frac{\del}{\del x} \Psi 
        -
        \Psi \frac{\del}{\del x}
        \Psi^*
    \right)^\infty_{-\infty}
    \frac{i \hbar}{2 m}
    =
    0
    \hp 
    \Psi, \Psi' \to 0 
    \hp 
    |x| \to \infty
}

To first try and figure out the wave function, we start with a bunch of 
plane waves. 

A single plane wave will have equation 
\equations{
    \Psi_k 
    =
    A e^{i (kx - \omega t)}
}

Plug that into the S.W.E. to get 

\equations{
    \frac{\del}{\del t}
    \Psi_k 
    =
    -i \omega \Psi_k 
    \hp 
    \frac{\del}{\del x}
    \Psi_k 
    =
    i k \Psi_k 
    \hp 
    \frac{\del^2}{\del x^2}
    \Psi_k 
    =
    -k^2 \Psi_k 
    \\
    \hbar \omega \Psi_k 
    =
    \frac{\hbar^2 k^2}{2 m}
    \Psi_k
    \Rightarrow 
    E 
    =
    \frac{p^2}{2m}
}

I can express all wave functions as plane waves 

\equations{
    f(x)
    =
    \int \frac{dk}{2 \pi}
    e^{ikx}
    \tilde f(x)
}

Add time and 
\equations{
\omega = \frac{\hbar k^2}{2m}
}

And get equation

\equations{
    \Psi(x, t)
    =
    \int 
    \frac{dk}{2 \pi}
    e^{ikx - i \omega t}
    \ldots
    \ldots
    \ldots
    \ldots
    \ldots
    \ldots
}

Unbounded plane waves are not normalizable. 

\equations{
    \int^\infty_{-\infty}
    dx \, 
    |A e^{i(kx - \omega t)}
    |^2
    =
    |A|^2
    \int^\infty_{-\infty}
    dx 
    =
    \infty
}

The way to fix this is by putting the plane wave in a finite box 
that is much larger than the bounds of the problem 

\equations{
    1
    =
    \int^L_{-L}
    dx 
    |A|^2
    \rightarrow
    A 
    =
    \frac{1}{\sqrt{2} L}
}

So the SWE can be given as 

\equations{
    i \hbar \frac{\del}{\del t}
    \Psi 
    =
    - \frac{\hbar}{2 m}
    \frac{\del^2}{\del x^2}
    \Psi 
    \hp
    (+ V(x) \Psi)
}

In classical mechanics, we would have 
\equations{
    E 
    =
    \frac{p^2}{2m}
    +
    V(x)
    \rightarrow 
    p 
    \iff 
    \frac{\del}{\del x}
    \Psi
}

So what is $\hat p$? 

For plane waves 
\equations{
    p \Psi_k 
    =
    \hbar k \Psi_k
}
From De Broglie. We put that into our wave equation

\equations{
    p \Psi_k 
    =
    \hbar k \Psi_k 
    =
    -i \hbar 
    \frac{\del}{\del x}
    \Psi_k
}

Classically, we know that 
\equations{
    V = \dot x 
    \hp 
    p = mV 
}

The expected value is the value over many many quantum mechanical states 
\equations{
    \frac{d \langle x \rangle}{dt}
    =
    \frac{d}{dt}
    \int dx \, 
    x |\Psi(x, t)|^2
    =
    \int dx \, 
    x 
    \left(
        \left(
            \frac{\del}{\del t} \Psi^*
        \right)
        +
        \Psi^*
        \left(
            \frac{\del}{\del t} \Psi
        \right)
    \right)
}

This derivation is in Griffiths 1.5 

\equations{
    \frac{i \hbar }{2m}
    \int dx \,
    x \frac{\del}{\del x}
    \left(
        - \Psi 
        \frac{\del}{\del x}
        \Psi^*
        +
        \Psi^*
        \frac{\del}{\del x}
        \Psi
    \right)
}

We do integration by parts 

\equations{
    \frac{\del}{\del x}
    \left(
        - \Psi 
        \frac{\del}{\del x}
        \Psi^*
        +
        \Psi^*
        \frac{\del}{\del x}
        \Psi
    \right)
    =
    g'
    \hp 
    x = f 
}

\equations{
    - \frac{i \hbar }{2m}
    \int dx \,
    \left(
        \Psi^*
        \frac{\del}{\del x}\Psi 
        -
        \Psi 
        \frac{\del}{\del x}
        \Psi^*
    \right)
    =
    \frac{i \hbar}{m}
    \int dx \,
    \Psi^* \frac{\del}{\del x}
    \Psi
    \\
    =
    \frac{1}{m}
    \int dx \,
    \Psi^*
    \left(
        -i \hbar 
        \frac{\del}{\del x}
    \right)
    \Psi 
    =
    \frac{\del \langle x \rangle} {\del t}
}

Momentum should be 
\equations{
    p 
    \sim 
    m \frac{d}{dt}
    \langle x(t) \rangle
    \\
    \langle p \rangle 
    =
    \int dx \, 
    \Psi^*
    \left(
        -i \hbar \frac{\del}{\del x}
    \right)
    \Psi
}

The momentum operator (expressed in position), known as $\hat p$, is 
\equations{
    \hat p 
    =
    -i \hbar \frac{\del}{\del x}
}

\section{Recap}

Given a state vector $\ket{\Psi}$, you can make a probability distribution 
to describe the position of the particle at time $t$ 
\equations{
    P(x, t)
    =
    |\Psi(x, t)|^2
}

Plane waves cannot be normalized, so they are bad wave functions. 

Solve for $\Psi \rightarrow $ Plane waves (only for math)

The momentum of the particle can be written as 
\equations{
    \hat p 
    =
    - i \hbar 
    \frac{\del}{\del x}
}
That is specifically in the position basis. 

The ep
\equations{
    \langle \hat p \rangle 
    =
    \int dx \, 
    \Psi^* \hat p \Psi
}

That is true for any 
\equations{
    \hat A 
    \equiv
    \hat A(\hat x, \hat p)
    \hp
    \langle \hat A \rangle 
    =
    \Psi^* \hat A \Psi
}

\section{Uncertainty Relation}
Given a plane wave wavefunction. 
Because I have a single wave with a well defined wavelength, the momentum 
is well-defined. 

\equations{
    \sigma_p \to 0
}

However, because it's a plane wave, we cannot say anything about the 
position of the particle. 

\equations{
    \sigma_x \to \infty
}

If given a wave function that looks like a dirac delta with a single 
spike, then the position is well-defined but the momentum is not. 

If you perfectly measure position, the momentum is then a superposition of 
all possible momenta, so the particle will spread out over time because 
the particle has to be moving due the momentum uncertainty.

Generally, the bound for momentum and position variance is 
\equations{
    \sigma_x 
    \sigma_p
    \geq 
    \frac{\hbar}{2}
}

There are wave functions that miimize the uncertainty relation 
(Gaussian Wave Packets).

\section{Solving the SWE}
Given $V(x)$, how do we get $\Psi(x, t)$

The Schrodinger wave equation is given as 
\equations{
    i \hbar \frac{\del \Psi}{\del t}
    =
    - \frac{\hbar^2}{2m} 
    \frac{\del^2 \Psi}{\del x^2}
    +
    V \Psi
}

Here we assume that the state is \textbf{not} time-dependent 
\equations{
    \frac{\del V}{\del t}
    =
    0
}

Because of our time-independence, we can just separate the variables. 
\equations{
    \Psi(x, t)
    =
    \Psi(x) \varphi(t) 
}

Initial Condition: know what $\Psi(x, 0)$ is

\equations{
    i \hbar \frac{1}{\varphi}
    \frac{d \varphi}{d t}
    =
    -\frac{\hbar^2}{2m}
    \frac{1}{\Psi}
    \frac{d^2 \Psi}{d x^2} 
    + V
}

The separation constant is the energy $E$ 

\equations{
    i \hbar \frac{1}{\varphi}
    \frac{d \phi}{d t}
    =
    E 
    \rightarrow 
    \frac{d \varphi}{d t}
    =
    \frac{-i E}{\hbar} \varphi
    \rightarrow 
    \\
    \phi 
    =
    e^{\frac{-iE}{h}}
}

Now we consider the position from the state space 
\equations{
    - \frac{\hbar^2}{2m}
    \frac{d^2 \Psi}{d x^2}
    +
    V \Psi
    =
    E \Psi
}

The left side is the Hamiltonian operator on $\Psi$ 
and the right side is the energy.

The stationary solution is of the form 
\equations{
    \Psi(x, t)
    =
    \psi(x)
    \exp(\frac{-i E t}{\hbar})
}

The time independent and time dependent states are equal 

For any $\hat A$ 
\equations{
    \langle \hat A \rangle 
    =
    \int 
    \psi^* \hat A \psi \, dx
    =
    \textrm{const}
}

We can define the energy as 

\equations{
    H 
    =
    \frac{p}{2m}
    +
    V(x)
}
This is the total energy in classical mechanics (kinetic + potential energy)

Our quantum Hamiltonian operator is 
\equations{
    \hat H 
    =
    - \frac{\hbar^2}{2m}
    \frac{\del^2}{\del x^2}
    +
    V(x)
    \hp 
    \hat H \psi = E \psi 
    \hp 
    \langle \hat H \rangle 
    =
    E
    \\
    H^2 \psi 
    =
    E^2 \psi 
    \rightarrow 
    \langle H^2 \rangle 
    =
    E^2
    \rightarrow 
    \sigma_H^2
    = 0
}

So the general solution is a linear combination of the separated 
solutions. 

\equations{
    V(x)
    \rightarrow 
    \{ \psi_n \}
    \Rightarrow 
    \{ E_n \}
}

With an Initial Condition: 

\equations{
    \Psi(x, 0)
    =
    \sum_n c_n \psi_n(x)
}

And then we evolve that system 
\equations{
    \Psi(x, t)
    =
    \sum_n 
    c_n \psi_n(x) \exp(\frac{-i E_n t}{\hbar})
}

make sure that everything is normalized 
\equations{
    \sum_n |c_n|^2 = 1
    \hp 
    \langle H \rangle 
    =
    \sum_n
    |c_n|^2 E_n
}

\subsection{Question}
Given a stationary state $\{ \psi_n(x) \}$ for some $\hat H$, how 
can we explain motion?

To explain motion, we need a superposition of $\psi_n(x)$.

\equations{
    \Psi(x, 0)
    =
    c_1 \psi_1(x)
    +
    c_2 
    \psi_2(x)
    \rightarrow 
    \Psi(x, t)
    \\
    =
    c_1 \psi_1(x) \exp(\frac{-i E_1 t}{\hbar})
    +
    c_2 \psi_2(x) \exp(\frac{-i E_2 t}{\hbar})
    \\
    |\Psi(x, t)|^2
    =
    \psi^* \psi
}
Use Euler's equation

\equations{
    c_1^2 \psi_1^2 
    +
    c_2^2 \psi_2^2 
    +
    2 c_1 c_2 \psi_1 \psi_2 
    \cos(\frac{(E_2 - E_1) t}{\hbar})
}

You have two stationary states in position. They interfere and something 
happens in time. 

\section{Infinite Square Well}
Horrendously artificial, but one of the few problems we can actually solve. 

Consider a particle with mass $m$ and velocity $v$ in a valley of height 
$h$.

The particle is trapped in the well because $mv^2/2 < mgh$.

We can consider the valley as $h \to \infty$, because the particle can 
quantum tunnel out of it.

The width of the well goes from $0 \to a$. The potential of the 
particle can be stated as 

\equations{
    V(x)
    =
    \begin{cases}
        0: 0 \leq x \leq a 
        \\
        \infty : \textrm{elsewhere}
    \end{cases}
}

At $V = \infty$, we can say that the wave function $\psi = 0$

\subsection{Boundary Conditions}
\begin{itemize}
    \item
    $\psi(x)$ is always continuous.
    \item
    $\psi'(x)$ is always continuous but not if $|V(x)| = \infty$.
\end{itemize}

So now all we have to do is solve the SWE inside the well 
\equations{
    - \frac{\hbar^2}{2m}
    \frac{\del^2}{\del x^2}
    \psi 
    =
    E \psi 
    \iff 
    \psi''(x)
    =
    -k^2 \psi 
    \hp 
    k
    =
    \frac{\sqrt{2mE}}{\hbar}
}

Everything is a simple harmonic oscillator 

\equations{
    \psi=
    A e^{ikx}
    +
    B e^{-ikx}
}

Now we have to consider the boundary conditions. 

\equations{
    \psi(x=0)
    =
    \psi(x=k)
    =
    0
    \\
    A 
    + 
    B 
    = 
    0
    \rightarrow 
    A 
    =
    -B
    \\
    \psi(x)
    =
    A(e^{ikx} - e^{-ikx})
    =
    A \sin(kx)
    \\
    psi(x=0) = 0 
    \rightarrow 
    \sin(kx) = 0
    \\
    \textrm{shit}
}

\subsection{Energies}

\equations{
    k^2 
    =
    \frac{2mE}{\hbar^2}
    \rightarrow 
    E_n 
    =
    \frac{\hbar^2 k_n^2}{2ma^2}
}

\subsection{Normalization}
\equations{
    \int_0^{a}
    A^2 \sin^2(kx)
    \, dx 
    =
    1
    \rightarrow 
    A 
    =
    \sqrt{\frac{2}{a}}
    \\
    \psi_n(x)
    =
    \sqrt{\frac{2}{a}}
    \sin(\frac{n \pi x}{a})
    \hp
    E_h 
    =
    \frac{k^2 \pi^2 \hbar^2}{2ma^2}
}

\section{Discussion 1}
\subsection{The Classical Recipe}
\equations{
    F = ma 
    \hp 
    \frac{\del L}{\del \dot x}
    =
    \frac{\del}{\del x}
    \frac{\del L}{\del x}
}

\subsection{Quantum Reality}

Input is ket. Output is bra.

Ket to bra is input to output based off of the inner product. 

The wavefunction gives the probability density of finding a particle 
in a volume element $dx$

\equations{
    \rho(x)
    =
    \Psi^*(x, t) 
    \Psi(x, t) 
    \\
    P(a < x < b)
    =
    \int^b_a
    \Psi^*(x, t) 
    \Psi(x, t) 
    \, dx
}

The wave function is also normalized

\equations{
    \int^\infty_{-\infty}
    \Psi^*(x, t) 
    \Psi(x, t) 
    \, dx
    =
    1
}

Given an input state $\ket{\Psi_1}$ and an output state 
$\bra{\Psi_2}$. The probability amplitude is given in the form 

\equations{
    \bra{\Psi_2} \ket{\Psi_1}
    =
    \Psi_2^*(x, t) 
    \Psi_1(x, t) 
}

The output state comes first, and it is the one that's the conjugate.

Given a superposition 
\equations{
    \Psi_{in}
    =
    \frac{1}{\sqrt{3}}
    \Psi_1
    +
    \frac{\sqrt{2}}{\sqrt{3}}
    \Psi_2
}

So to find the probability of the output being 2

\equations{
    P(2)
    =
    |\bra{\Psi_{out}} \ket{Psi_{in}}|^2
    =
    \Psi_2
    \frac{1}{\sqrt{3}}
    \Psi_1
    +
    \Psi_2
    \frac{\sqrt{2}}{\sqrt{3}}
    \Psi_2
}
something something I'll look at the lecture notes later 

\subsection{Questions}

Consider a wave function 
\equations{
    \psi(x)
    =
    A(a^2 - x^2)
}

inside interval $\{-a, a\}$

Determine the normalization constant $A$ 

\equations{
    \int \psi(x)^2 \, dx \, = 1
    \\
    A^2
    \int^a_{-a} 
    (a^2 - x^2)^2
    \, dx 
    \hp 
    A^2 
    \left(
        a^4 x - (2 a^2 x^3)/3 + x^5/5
    \right) \Big|^a_{-a}
    =
    \\
    A^2 
    \left(
        a^4 a - (2 a^2 a^3)/3 + a^5/5
    \right)
    -
    \left(
        -a^4 a + (2 a^2 a^3)/3 - a^5/5
    \right)
    \\
    =
    2A^2 
    \left(
        a^5 - \frac{2}{3}a^5 + a^5/5
    \right)
    =
    \frac{16}{15} A^2 a^5 = 1
    \rightarrow 
    A 
    =
    \sqrt{\frac{15}{16 a^5}} 
    \\
    \frac{15}{15} - \frac{10}{15} + \frac{3}{15}
    =
    \frac{8}{15}
}

What is the probability of finding the particle at $x = a/2$. It is 0 
because that's a miniscule-ly small point. The range $-a/2 \to a/2$ is different. 

\equations{
    P(-a/2 < x < a/2)
    =
    \int^{a/2}_{-a/2}
    \Psi^*(x) \Psi(x)
    \\
    =
    2 \frac{15}{16 a^5}
    \left(
        a^4 x - (2 a^2 x^3)/3 + x^5/5
    \right)
    \Big|^{a/2}_{-a/2}
    \\
    =
    \frac{15}{4 a^5}
    a^4 (\frac{a}{2}) - (2 a^2 (\frac{a}{2})^3)/3 + (\frac{a}{2})^5/5
    =
    \frac{15}{4 }
    \left(
    (\frac{1}{2}) - (2  (\frac{1}{8}))/3 + (\frac{1}{32})/5
    \right)
    \\
    \frac{15}{4 }
    \left(
        \frac{1}{2}
        -
        \frac{1}{12}
        +
        \frac{1}{160}
    \right)
}

For what potential is the state an eigenfunction?

The energy eigenfunctions are determined from the Hamiltonian
\equations{
    \hat H 
    =
    - \frac{\hbar^2}{2m}
    \frac{\del^2}{\del x^2}
    \psi
    +
    V(x)
    =
    E \psi
}

So I need to take the double derivative 

\equations{
    \frac{\del^2}{\del x^2}
    Aa^2 - Ax^2
    =
    -2A
    \\
    \frac{\hbar^2}{2m}
    2A
    +
    V(x) ( Aa^2 - Ax^2)
    =
    E A (a^2 - x^2)
}

Set energy to 0 

\equations{
    -\frac{\hbar^2}{2m}
    2 \sqrt{\frac{15}{16 a^5}}
}

\subsection{bra-ket}
Consider a three dimension vector space spanned by an orthonormal basis 
\equations{
    \ket{1}
    \hp
    \ket{2}
    \hp
    \ket{3}
}

$\ket{\alpha}$ and $\ket{\beta}$ are defined as 

\equations{
    \ket{\alpha}
    =
    i \ket{1} - 2\ket{2} - i \ket{3}
    \hp
    \ket{\beta}
    =
    i \ket{1} + 2 \ket{3}
}

Construct $\ket{\alpha}$ and $\ket{\beta}$ in terms of the dual basis 
$\bra{1}$, $\bra{2}$, $\bra{3}$

Something that might be useful 
\equations{
    A_{ij} 
    =
    \bra{i} \hat A \ket{j}
    =
    \bra{i} (\ket{\alpha} \bra{\beta}) \ket{j}
    =
    \bra{i}\ket{\alpha}
    \cdot
    \bra{\beta}\ket{j}
}

\chapter{Solving the SWE}
\section{Recap}
Our starting point is 
\equations{
    \hat H 
    =
    - \frac{\hbar}{2 m}
    \frac{\del^2}{\del x^2}
    +
    V(x)
}
with initial conditions. 

The time-indepedent SWE can be written as 
\equations{
    \hat H \psi(x)
    =
    E_n \psi(x)
    \hp 
    \psi(x, 0) 
    =
    \sum_n c_n \psi_n(x) 
    \hp 
    \psi(x, t) 
    =
    \sum_n c_n \psi_n(x) 
    e^{\frac{-i}{\hbar} E_n t}
}

And infinite square well can be written as 
\equations{
    \psi(x \leq 0) = \psi(x \geq a) = 0 
}

\subsection{Stationary State}
If we consider a probability distribution 
\equations{
    |\psi(x, t)|
    =
    \psi^*(x, t) \psi(x, t)
}
So for an infinite square well. 

Consider the general statement 
\equations{
    \frac{d^2 \psi}{dx^2}
    =
    \frac{2m}{\hbar^2}
    (V(x) - E) \psi
}
if $E < V(x)$, then $\psi''$ and $\psi$ always have the same 
sign. The issue with this is the wave will not be normalizable
(because for positive x, the concavity has to be positive, so it has to increase).

The eigenstates are given by 
\equations{
    E_1
    =
    \frac{\pi^2 \hbar^2}{2 m a^2}
    \hp 
    E_2 = 4 E_1
    \hp
    E_3 = 9 E_2
}

In a stationary state, $\{ \psi_n\}$ forms a complete orthonormal eigenbasis. 
\equations{
    \int \, dx \, 
    \psi^*_m(x)
    \psi_n(x)
    =
    \delta_{mn}(m = n)
    \\
    f(x)
    =
    \sum_n c_n \psi_n(x)
    =
    \sqrt{\frac{2}{a}} 
    \sum_n c_n \sin(\frac{n \pi x}{a})
    \hp 
    c_n = \bra{\psi_n} \ket{f(x)}
}
\subsection*{Question} Given a wave function $\psi(x, t=0)$ 
and a stationary state, how can we find $c_n$ such that 
$\psi = \sum_n c_n \psi_n(x)$?
\equations{
    \int \, dx \, 
    \psi^*_m(x) f(x)
    =
    \int \, dx \, 
    \psi^*_m(x) f(x)
    \sum_n c_n \psi_n(x)
    \\
    =
    \sum_n c_n 
    \int 
    \psi^*_m(x)
    \psi_n(x)
    =
    \sum_n 
    c_n f_{mn}
    =
    c_m
}

\section{Time-Dependence}
\equations{
    \psi(x, t=0)
    =
    \sum_n c_n \psi_n(x)
    \rightarrow 
    \psi(x, t)
    =
    \sum_n c_n \psi_n e^{\frac{-i}{\hbar} E_n t}
}

This is just summing over all of the possible standing waves within 
an infinite square well. 

\subsection{THIS WILL BE ON THE MIDTERN1}
The thing above 

\subsection{Quantum Number}
It's a fake number that is just an index to show the different eigenvectors 
\equations{
    E_n
    \hp 
    n = 
    \textrm{quantum number}
}

\subsection*{Question}
How can we describe a "ball" bouncing between the walls 
of the infinite square well? 

\equations{
    \psi(x, t=0)
    =
    \sum_{odd}
    c_n \psi_n(x)
}
sum odd means that the function that we're summing over has to be an odd function. 

An even function will just have an expected value of 0.

\subsection{Free Particle}
\equations{
    V= 0 
    \rightarrow 
    \hat H 
    =
    -
    \frac{\hbar^2}{2m}
    \frac{\del^2}{\del x^2}
}
and the time-indepedent SWE is 
\equations{
    -
    \frac{\hbar^2}{2m}
    \frac{\del^2}{\del x^2}
    \psi
    =
    E \psi
    \rightarrow 
    \psi 
    =
    A e^{ikx} + B e^{-ikx}
    \hp 
    E = \frac{\hbar^2 k^2}{2m}
    \hp 
    kc=
    \sqrt{\frac{2mE}{\hbar^2}}
}

and the time-\textbf{dependent} SWE is 


\equations{
    \psi(x, t)
    A e^{ikx - \frac{i}{\hbar} E t}
    +
    B e^{-ikx - \frac{i}{\hbar} E t}
    =
    A e^{i(kx - \omega t)}
    +
    B e^{-i(kx + \omega t)}
}

The wave function is a superposition of a left-propagating and a 
right-propagating wave both with constant kintetic energy.

allow $k$ to be
\equations{
    k 
    =
    \pm 
    \sqrt{\frac{2mE}{\hbar^2}}
    \rightarrow 
    A e^{i(kx - \omega t)}
}

The velocity can be written as 
\equations{
    e^{-i(kx - \omega t)}
    =
    e^{-ik(x - \frac{\omega}{k} t)}
    \rightarrow 
    v 
    =
    \frac{\hbar |k|}{2m}
    =
    \sqrt{\frac{E}{2m}}
}

This is different from the classical definition of velocity by a factor of 2
\equations{
    \frac{1}{2} mv^2 = E 
    \rightarrow 
    v 
    =
    \sqrt{\frac{2E}{m}}
}
The reason this is so is because the quantum definition of velocity 
is \textbf{not} for a particle.

Consider a Guassian wave packet 
\equations{
    \psi(x, t)
    =
    \frac{1}{\sqrt{2 \pi}}
    \int 
    \Phi(k)
    e^{i(kx - \omega t)}
    \, dk
}
For some initial conditions $\psi(x, t=0)$, we find the weight function with 

\equations{
    \Phi(k)
    =
    \frac{1}{\sqrt{2 \pi}}
    \int 
    \psi(x, t=0)
    e^{-ikx}
    \, dx 
}

\section{MISSED LECTURE}

\section{Discussion}
Time-Evolving superpositions. We use a Hamiltonian basis 
\equations{
    \hat H \psi_n 
    =
    E_n \psi_n 
    \hp 
    \textrm{Basis }
    =
    \{
        \psi_1, \psi_2, \ldots
    \}
}

This basis is orthonormal 
\equations{
    \bra{\psi_m} \ket{\psi_n} = \delta_{mn}
}

And the basis is complete 
\equations{
    \Psi(x, 0)
    =
    c_1 \psi_1(x)
    +
    c_2 \psi_2(x)
    +
    c_3 \psi_3(x)
    +
    \ldots
}

And we have a time-evolution operator 

\equations{
    \Psi(x, 0)
    =
    c_1 \psi_1(x)
    +
    c_2 \psi_2(x)
    +
    c_3 \psi_3(x)
    +
    \ldots
    \\
    \Psi(x, t)
    =
    e^{\frac{-i \hat H t}{\hbar}}
    \left(
    c_1 \psi_1(x)
    +
    c_2 \psi_2(x)
    +
    c_3 \psi_3(x)
    +
    \ldots
    \right)
}

And this effects each individual state differently 
\equations{
    e^{\frac{-i \hat H t}{\hbar}}
    c_3 \psi_3(x)
    =
    e^{\frac{-i E_3 t}{\hbar}}
    c_3 \psi_3(x)
}

\subsection{Questions}
Find the normalized wave function at time $t$ for a particle in an 
infinite square with a wave function given by 
\equations{
    \psi(x, 0)
    =
    A 
    \left(
        \psi_1(x)
        +
        e^{i \theta}
        \psi_2(x)
    \right)
    \\
    P(x)
    =
    |\psi(x)|^2 
    =
    A^2 
    |\left(
        \psi_1(x)
        +
        e^{i \theta}
        \psi_2(x)
    \right)|^2
}

The time dependent portion of the wave function is 
\equations{
    e^{-i \frac{E}{\hbar}}
}

So our equation will be 

\equations{
    P(x)
    =
    e^{-i \frac{E}{\hbar}}
    |\psi(x)|^2 
    =
    A^2 
    |\left(
        \psi_1(x)
        +
        e^{i \theta}
        \psi_2(x)
    \right)|^2
}

For an infinite square well, the solution can be written as 
\equations{
    \psi(x, 0)
    =
    \sin(\frac{n \pi x}{L})
    \\
    \ket{\psi(x, 0)}
    =
    A 
    \left(
        \ket{\psi_1}
        +
        e^{i \theta}
        \ket{\psi_2}
    \right)
    \\
    |\psi(x)|^2
    =
    A^2 
    \left(
        \bra{\psi_1}
        +
        e^{-i \theta}
        \bra{\psi_2}
    \right)
    \left(
        \ket{\psi_1}
        +
        e^{i \theta}
        \ket{\psi_2}
    \right)
    =
    \\
    A^2 
    \left(
        \bra{\psi_1}\ket{\psi_1}
        +
        \ket{\psi_1}
        e^{i \theta}
        \ket{\psi_2}
        +
        \bra{\psi_2}\ket{\psi_2}
    \right)
    =
    2A^2
    =1
}

The probability density is just something 

\subsection{Infinite Square Well}
Given a unique wave function that's put in the paper and on the website, 
expand the wave function in terms of its eigenvectors using 
a Fourier series.

Go to chapter 11 of townsend for braket notation or chapter 4 for 
time-evolution.

\section{Finite Square Well}
Consider a well with potential 
\equations{
    V(x)
    =
    \begin{cases}
        -V_0 
        : -a < x < a
        \\
        % V(x) 
        0
        : x < -a, x > a
    \end{cases}
}

The wave function can extend out of the well now, and the 
wave function \textbf{must be continuous}.
Exact solutions are typically numerical bound states $(\geq 1)$

\subsection{Scattering Stationary States $(E > 0)$}

Consider that we're starting with a plane wave (coming from the left)
\equations{
    \psi_{I}(x, t=0)
    =
    A e^{ikx}
    +
    B e^{-ikx}
}

In the middle, we have 2 boundary conditions, and both the wavefunction 
itself and its derivative have to be continuous. 

\equations{
    \psi_{II}(x, t=0)
    =
    C \sin(lx )
    +
    D \cos(lx )
}

On the right side, there is still 1 boundary condition, but the 
other side is unbounded

\equations{
    \psi_{III}(x, t=0)
    =
    F e^{ikx}
}

You do some math (Griffith's 2.6). Solve for Boundary Conditions 
and make sure $x$ is continuous at the important parts
\equations{
    \frac{|F|^2}{|A|^2}
    =
    T 
    \hp 
    R 
    =
    1-T 
    \hp 
    k = \frac{\sqrt{2m E}{\hbar}}
    \hp 
    l 
    =
    \frac{\sqrt{2m (E + V_0)}}{\hbar}
    \\
    A e^{-ika}
    +
    B e^{ika}
    =
    -C \sin(la)
    + 
    D \cos(la)
    \\
    \Rightarrow
    \Rightarrow
    \Rightarrow
    \Rightarrow
    \Rightarrow
    \Rightarrow
    \\
    B = i \frac{\sin(2la)}{2kl}
    (l^2 - k^2)
    F
    \hp 
    F 
    =
    A e^{-2ika} 
    \left(
        \cos(2la)
        -
        i \frac{(k^2 + l^2)}{2kl}
        \sin(2la)
    \right)^{-1}
}

Those are the answer given 
\equations{
    \psi(x, 0)
    \propto 
    \int \, dk \, 
    \psi(k) e^{-ikx}
}

An interesting case is $T=1$. This can be fulfilled for 

\equations{
    \frac{2a}{\hbar}
    \sqrt{2m (E_n + V_0)}
    =
    n \pi
    \hp 
    (E_n + V_0)
    =
    \frac{n^2 \pi^2 \hbar^2}{2m (2a)^2}
}

We are matching $k$ to the "infinite well". 

\equations{
    B 
    =
    i \frac{\sin(2la)}{2kl}
    (l^2 - k^2) F 
    \\
    F 
    =
    (e^{-2ika} A)
    * 
    \left(
        \cos(2la)
        -
        i \frac{k^2 + l^2}{2kl}
        \sin(2la)
    \right)^{-1}
}

\section{Quantum Tunneling}
Consider the opposite of a finite square well 
\equations{
    V =
    \begin{cases}
        V_0 : 0 < x < a 
        \\
        0 : x < 0, x > a
    \end{cases}
}

The stationary state consists of a plane wave on the left side.
\equations{
    \psi_{I}
    =
    A^{ikx}
    +
    B^{-ikx}
}

We have exponential decay inside the "well".
The particle can also be reflected, so the wave function is 
given by 
\equations{
    \psi_{II}
    =
    Ce^{-kx}
    +
    De^{kx}
}

And then we have another plane wave on the 2nd side 
\equations{
    \psi_{III}(x, 0)
    =
    F e^{ikx}
}


























































\end{document}