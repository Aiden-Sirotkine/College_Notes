\documentclass[fleqn]{report}
\usepackage{geometry}
\usepackage{amssymb}
\usepackage{fancyhdr}
\usepackage{multicol}
\usepackage{blindtext}
\usepackage{color}
\usepackage[fontsize=16pt]{fontsize}
\usepackage{lipsum}
\usepackage{pgfplots}
\usepackage{physics}
\usepackage{mathtools}
\usepackage[makeroom]{cancel}
\usepackage{ulem}
\usepackage{esint}

\geometry{a4paper, margin=1cm} % Set paper size and margins
\graphicspath{ {../Images/} }
\setlength{\columnsep}{1cm}
\addtolength{\jot}{0.1cm}
\def\columnseprulecolor{\color{blue}}
\date{Fall 2025}

\newcommand{\textoverline}[1]{$\overline{\mbox{#1}}$}

\newcommand{\hp}{\hspace{1cm}}

\newcommand{\const}{\textrm{const}}

\newcommand{\del}{\partial}

\newcommand{\pdif}[2]{ \frac{\partial #1}{ \partial #2} }

\newcommand{\pderiv}[1]{ \frac{\partial}{ \partial #1} }

\newcommand{\comment}[1]{}

\newcommand{\equations} [1] {
\begin{gather*}
#1
\end{gather*}
}

\newcommand{\numequations} [1] {
\begin{gather}
#1
\end{gather}
}

\newcommand{\twovec}[2]{ 
\begin{pmatrix}
#1 \\ 
#2
\end{pmatrix}
}

\title{PHYS 486}
\author{Aiden Sirotkine}

\begin{document}

\setlength{\headsep}{10pt}
\setlength{\topmargin}{-2cm}

\pagestyle{fancy}
\maketitle
\tableofcontents
\clearpage

\chapter{PHYS486}
Quantum Mechanics lol. This is apparently a math language more 
than actual physics.

\section{What is QM}
\begin{itemize}
    \item 
Classical Mechanics - $F = ma$, solve for $x$ and $p$, position and momentum.
    \item
Classical E \& M - $\ddot \vec E(\vec r, t) - \nabla^2 \vec E(\vec r , t) = 0$
\end{itemize}

You have observable entities and you work with just the observable entities to 
figure out the exact numbers. 

\subsection{The Quantum State $\ket \psi$ }
This is a state vector.

The quantum state by definition is \textbf{not} observable.

An observable operator is something like position that describes the system that 
we can actually see. 

\equations{
    \hat A \ket{\psi} 
    =
    a \ket \psi 
}
where $\hat A$ is an observable and $a$ is the measurement out.

This is just an eigenvalue equation.

\subsection{Schrodinger Equation}
\equations{
    i \hbar \frac{\del}{\del t} \ket{\psi(t)} = \hat H \ket{\psi(t)}
}

$\hat H$ is a matrix and $\ket{\psi(t)}$ is a vector.

\section{Summary of "Central Phenomena"}
Imagine an electron orbiting around a proton. 

The electron \textbf{cannot} take arbitrary orbits.

The electron can only have \textbf{certain discrete} orbits.

\subsection{Superpositions}
Consider two discrete states in which an electron can orbit around a proton. 

Because $0$ and $1$ are solutions for the electron, a superposition of $0$ and $1$ is 
also a solution for the electron.

\equations{
    \alpha \ket 0 + \beta \ket 1
}

This is the electron being in both discrete states at the same time.

\equations{
    \sim
    \left[ 
        \ket 0 \sim {1 \choose 0}, 
        \ket 1 \sim {0 \choose 1}, 
        SP {\alpha \choose \beta}
    \right]
}

\section{Probabilistic Interpretation}
Consider a system with the solution 
\equations{
    \alpha \ket 0 + \beta \ket 1
}

If we have a thing that measures the energy, we can get 
\textbf{either} $0$ \textbf{or} $1$ with probabilities depending on the 
values $\alpha$ and $\beta$.

\section{Entanglement}
Consider atom $A$ with state $\{ \ket{0}_A, \ket{1}_A \}$
and atom $B$ with state $\{ \ket{0}_B, \ket{1}_B \}$.

It is completely legal to have an entangled state with the form 

\equations{
    \alpha \ket{0}_A \ket{0}_B 
    +
    \beta \ket{1}_A \ket{1}_B 
}


\section{Course Outline}
\begin{enumerate}
    \item 
    Basic Rules 
    \item 
    Wave Mechanics ("toy models")
    \item 
    Formalism (Midterm 1)
    \item 
    Simplest Real System (Hydrogen Atom) (Midterm 2)
    \item 
    Intro to Multiparticle Descriptions
\end{enumerate}

\chapter{Actual Physics Now}

\section{Black Body Radiation}
Consider an object that perfectly absorbs radiation and turns it into heat. 

Place it in thermal equilibrium 
(finite $T$, constant $E$, absorption $\rightarrow$ emission).

\subsection{Classical Description}
Consider the system as a harmonic oscillator. 

\equations{
    \langle E \rangle = k_B T
    \Rightarrow 
    I(\omega)
    \sim 
    \frac{\omega^2 k_B T}{\pi^2 c^3}
    \hp
    \textrm{Rayleigh-Jeans Law}
}

This \textbf{does not} match experimental data. 

While the classical description says that temperature will go up forever, the 
experiments show that the temperature stops increasing for large frequencies.

\subsection{Plack, 1900}
Complete guess that maybe energy is discretized. Maybe the harmonic oscillator 
comes in chunks of size $\hbar \omega$.

Now, energy can be written in the form 
\equations{
    P(E)
    =
    \alpha e^{- \frac{-E}{k_B T}}
    \longrightarrow 
    \langle E \rangle 
    =
    \frac{\hbar \omega}{\exp((\hbar \omega / k_B T) - 1)}
    \\
    I(\omega)
    =
    \frac{\hbar \omega^3}{\pi^2 c^3 \exp((\hbar \omega / k_B T) - 1)}
}
This perfectly matches experimental data.

\section{Photoelectric Effect}
Consider a circuit that absorbs light with frequency $\omega$ and intensity $J$
to create a potential.

As light intensity $J$ increases, current $I$ increases, but $V_0$ stays the same. 

$V_0$ is dependent on $\omega$ and the properties of the material. 

Einstein figured out that there is an energy threshold such that light will 
not be absorbed if it does not have a high enough energy. 

The kinetic energy of the electron can be given in the form:
\equations{
    kE 
    =
    \hbar \omega 
    -
    W 
    > 
    - eV_0
}

\section{Wave Particle Duality}
Consider a photon with energy and momentum
\equations{
    E = \hbar \omega
    \hp 
    p = \hbar k 
    \left(
        = \frac{h}{2 \pi} \cdot \frac{2 \pi}{\lambda}
    \right)
}

The photon has no mass $(m = 0)$, so consider the relativistic mass of the photon 
\equations{
    E^2 = m^2 c^4 + p^2 c^2 
    \rightarrow 
    p^2 c^2
    \rightarrow 
    \\
    E = pc 
    \hp 
    \omega = kc 
    \hp 
    f \lambda = c
}

\subsection{De Broglie, 1924}
Everything has a wavelength and frequency

\equations{
    \lambda 
    =
    \frac{h}{p}
}

The wavelength for even electrons is extremely small, so this feature 
is very hard to observe even if its true for everything.

\subsection{Davisson and Germer, 1927}
Electrons act in the same wave-like nature as light when shot through 
a small slit (double slit experiment but for electrons).

This proved the De Broglie wavelength theory for particles.

    \section{Beginning of QM}
The issue with quantum mechanics is that the math is derived only 
from experimental data. There is no classical physics basis or derivation 
behind quantum mechanics. 

Multiple textbooks will bring up quantum mechanics in different ways. 
The professor recommends the Townsend QM textbook 

\chapter{Ruleset}
Motivated from experimental observations. These are base postulates 
with no derivations. 

\section{System State}
The state of the physical system is a vector $\ket \Psi$ 
in Hilbert Space. 

A Hilbert space $(\equiv H^2)$ is a complex vector space with a well-behaved 
inner product.

(bra + ket = braket). You have to take the complex conjugate
\equations{
    \bra \Psi \ket{\Psi}
    \hp 
    \textrm{"bra"}
    =
    (\ket{\Psi}^*)^T
    \hp 
    \bra \alpha \ket{\beta}
    =
    \bra \beta \ket{\alpha}^*
}

\section{Observables}
Observables are operators in Hilbert Space, and they have to 
have real eigenvalues. 

The measurement outcomes of the observable are the eigenvalues.

\equations{
    \hat A \ket{\alpha_n}
    =
    a_n \ket \alpha_n
}

I think $\hat A$ is the observable of the state $\ket{\alpha_n}$ 
and $a_n$ is the $n$-th eigenvalue of the observable.

\subsection{Born's Rule}

The probability for an outcome $a_n$ is given by 
\equations{
    P(a_n)
    =
    |\bra{\alpha_n} \ket{\Psi}|^2
}
Where $\alpha_n$ is an eigenvector of the operator. 

\subsection{Expected Value}
Given an observable operator $\hat A$, the expected value is given by 

\equations{
    \langle \hat A \rangle 
    =
    \bra{\Psi} \hat A \ket{\Psi}
}

\subsection{Waveform "Collapse"}
Given a quantum state $\ket{\Psi}$, when you measure 
an observable $\hat A$, you return only a single result $a_n$

\equations{
    \ket{\Psi}
    \longrightarrow 
    \ket{\alpha_n}
}
The system state collapses to just the individual eigenvector state. 

\section{Time Evolution (Schrodinger Equation)}
\equations{
    i \hbar \frac{\del}{\del t} \ket{\Psi}
    =
    \hat H \ket{\Psi}
    \hp 
    \hat H = 
    \textrm{Hamiltonian (Energy Operator)}
}

The Eigenvectors of $\hat H$ are stationary states (do not change with 
time). 

This is the time-independent Schrodinger Equation 
\equations{
    \hat H \ket{\Psi}
    =
    E_h \ket{\Psi}
}

Now you know the entirety of quantum mechanics. Everything else 
can be derived. 

\chapter{Experiments}
\section{Double Slit}
If you shine light through a small slit, you will see a diffraction 
pattern. 

Imagine shining light at 2 slits and an attennuator at the other 
side that \textbf{measures} the location of each photon at the screen. 

If you just shoot a single photon, the location will be random, but if you
shoot multiple photons, there is a very distinct diffraction pattern that 
demonstrate the probability distribution of the list. 

\section{Stern-Gerlach}
Consider a beam of atoms with varying spins. This beam of atoms is 
put into a magnetic field gradient. 

If the magnetic moment is up, the atom will go up. If the magnetic 
moment is down, the atom will go down. 

If we put the beam through this field, then half the atoms will go up, 
and half the atoms will go down (probabilistically).

This makes sense. 

The quantum state of each atom is given by 
\equations{
    \ket{\Psi}
    =
    \frac{1}{\sqrt{2}}
    \left(
        \ket{\uparrow}
        +
        \ket{\downarrow}
    \right)
    \hat z
}

Consider a basis 
\equations{
    \ket{\uparrow}
    =
    \begin{pmatrix} 1 \\ 0 \end{pmatrix}
    \hp
    \ket{\downarrow}
    =
    \begin{pmatrix} 0 \\ 1 \end{pmatrix}
}

Let's call spin up $+1$ and spin down $-1$. 

Because of those numbers, our observable operator is 
\equations{
    \hat A 
    =
    \begin{bmatrix}
        1 && 0 \\
        0 && -1
    \end{bmatrix}
    \hp 
    \hat A \ket{\uparrow}
    =
    + 1 \ket{\uparrow}
    \hp 
    \hat A \ket{\downarrow}
    =
    - 1 \ket{\downarrow}
}

Now we have our quantum state and our observable operator matrix. 

Because we have two possible eigenvectors, the superposition 
of our quantum state can be written as 

\equations{
    \ket{\Psi}
    =
    \left(
        \sum_k \ket{k} \bra{k}
    \right)
    \ket{\Psi}
    \equiv 
    \mathbb{\lambda}_i
    , 
    \ket{k}
    \hp 
    \textrm{kth basis vector}
    \\
    =
    \sum_k 
    c_k \ket{k}
    \hp 
    c_k = \bra{k}\ket{\Psi}
}

\subsection{Born's Rule}
\equations{
    P(+1)
    =
    |\bra{\uparrow}\ket{\Psi}|^2
    =
    \frac{1}{2}
    |\bra{\uparrow}\ket{\uparrow}
    +
    \bra{\uparrow}\ket{\downarrow}|^2
}

Because we're using Dirac notation, we know exactly what inner products
are parallel and what inner products are orthonormal.


\equations{
    P(+1)
    =
    |\bra{\uparrow}\ket{\Psi}|^2
    =
    \frac{1}{2}
    |1
    +
    0|^2
    =
    \frac{1}{2}
}

\subsection{Expected Values}
\equations{
    \langle \hat A \rangle 
    =
    \bra{\Psi} \hat A \ket{\Psi}
    \hp 
    \hat A 
    =
    \sum_k 
    \lambda_k 
    \ket{k} \bra{k}
    \hp 
    \lambda_k 
    =
    \textrm{kth eigenvalue}
}

You can use the Spectral Theorem to get 
\equations{
    \hat A 
    =
    \ket{\uparrow}
    \bra{\uparrow}
    -
    \ket{\downarrow}
    \bra{\downarrow}
}

So to find the expected value, we use 

\equations{
    \langle \hat A \rangle 
    =
    \bra{\Psi} \left(
        \ket{\uparrow}
        \bra{\uparrow}
        -
        \ket{\downarrow}
        \bra{\downarrow}
    \right) \ket{\Psi}
    =
    \ldots 
    = 0
}

\section{Double Slit and the Wave Function}
we need $\ket{\Psi}$ in the position basis. 

Imagine if the space was discrete. 
\equations{
    \ket{\Psi}
    =
    \sum_k 
    c_k \ket{x}
}

But our space is continuous, so instead of taking a discrete sum, 
we take an integral. 

\equations{
    \bra{x_k} \ket{x_j}
    =
    \delta (k - j)
    \hp 
    \int dx 
    \bra{x_k} \ket{x_j}
    =
    1
    \\
    \ket{\Psi}
    =
    \int \, dx \,
    \ket{x} \bra{x} \ket{\Psi}
    =
    \int \, dx \, 
    \Psi(x) \ket{x}
}

That's the wave function 

implement Born's Rule to get 

\equations{
    P(x)
    =
    |\bra{x}\ket{\Psi}|^2
    =
    |\Psi(x)|^2
}

\subsection{Bra-Ket Notation}
\equations{
    \bra{x} \ket{y}
}
Is an inner product that yields just 1 number.

\equations{
    \ket{y} \bra{x} 
}
is an outer product that yields a matrix.

\subsection{Stationary State}
A stationary state is an eigenstate of the Hamiltonian.
It has a perfectly well-defined energy. 

\section{Interpreting the Double Slit}
You have a quantum state dependent probability density 

\equations{
    P(x)
    =
    |\Psi(x)|^2
}

That's a probability density function that needs to be normalized 

\equations{
    \int^\infty_{-\infty}
    dx |\Psi(x)|^2 
    =
    1
}

So to find the expected value, we get 

\equations{
    \langle x \rangle 
    =
    \int^\infty_{-\infty}
    dx \, 
    x 
    |\Psi(x)|^2 
    \hp
    \langle f(x) \rangle 
    =
    \int^\infty_{-\infty}
    dx \, 
    f(x) 
    |\Psi(x)|^2 
}

And the variance of that probability distribution is given by 
\equations{
    \sigma_x^2 
    =
    \langle x^2 \rangle 
    -
    \langle x \rangle^2
}

Given a quantum state, if we can measure the position, we 
can \textbf{only} estimate $|\Psi(x)|^2$ with finite accuracy. 
Each position sampled is just a single point of a probability distribution 
that cannot be directly measured. 

\chapter{Wave Mechanics}
Let's start with 

\equations{
    \ket{\Psi(x)} 
    =
    \Psi(x, t)
    \hp
    H 
    =
    - \frac{\hbar^2}{2 m}
    \left( \frac{\del}{\del x} \right)
    +
    V(x)
}

Where $V(x)$ is the potential. This is our wave function 
and our Hamiltonian. 

The Schrodinger Wave equation is 
\equations{
    i \hbar \frac{\del}{\del t} \Psi(x, t)
    =
    - \frac{\hbar^2 }{2m} \frac{\del^2}{\del x^2} \Psi(x, t) 
    +
    V(x) \Psi(x, t)
}

\section{Important Properties of the S.W.E.}
\subsection{ Unitary }
Conservation of probability.
Found is Griffiths Textbook chapter 1.4

\equations{
    0
    =
    \frac{d}{dt}
    \int^\infty_{-\infty}
    dx \, 
    |\Psi(x, t)|^2 
    =
    \int dx \, 
    \frac{\del}{\del t}
    |\Psi(x, t)|^2
    =
    \\
    \int dx \, 
    \left(
        \Psi^* \frac{\del}{\del t} \Psi 
        +
        \Psi \frac{\del}{\del t}
        \Psi^*
    \right)
    \rightarrow 
    \int dx \, 
    \left(
        \Psi^* \frac{\del^2}{\del x^2} \Psi 
        -
        \Psi \frac{\del}{\del x^2}
        \Psi^*
    \right)
    \frac{i \hbar}{2 m}
    \\
    =
    \int dx \, 
    \frac{d}{dx}
    \left(
        \Psi^* \frac{\del}{\del x} \Psi 
        -
        \Psi \frac{\del}{\del x}
        \Psi^*
    \right)
    \frac{i \hbar}{2 m}
}

So now we have to 

\equations{
    \left(
        \Psi^* \frac{\del}{\del x} \Psi 
        -
        \Psi \frac{\del}{\del x}
        \Psi^*
    \right)^\infty_{-\infty}
    \frac{i \hbar}{2 m}
    =
    0
    \hp 
    \Psi, \Psi' \to 0 
    \hp 
    |x| \to \infty
}

To first try and figure out the wave function, we start with a bunch of 
plane waves. 

A single plane wave will have equation 
\equations{
    \Psi_k 
    =
    A e^{i (kx - \omega t)}
}

Plug that into the S.W.E. to get 

\equations{
    \frac{\del}{\del t}
    \Psi_k 
    =
    -i \omega \Psi_k 
    \hp 
    \frac{\del}{\del x}
    \Psi_k 
    =
    i k \Psi_k 
    \hp 
    \frac{\del^2}{\del x^2}
    \Psi_k 
    =
    -k^2 \Psi_k 
    \\
    \hbar \omega \Psi_k 
    =
    \frac{\hbar^2 k^2}{2 m}
    \Psi_k
    \Rightarrow 
    E 
    =
    \frac{p^2}{2m}
}

I can express all wave functions as plane waves 

\equations{
    f(x)
    =
    \int \frac{dk}{2 \pi}
    e^{ikx}
    \tilde f(x)
}

Add time and 
\equations{
\omega = \frac{\hbar k^2}{2m}
}

And get equation

\equations{
    \Psi(x, t)
    =
    \int 
    \frac{dk}{2 \pi}
    e^{ikx - i \omega t}
    \ldots
    \ldots
    \ldots
    \ldots
    \ldots
    \ldots
}

Unbounded plane waves are not normalizable. 

\equations{
    \int^\infty_{-\infty}
    dx \, 
    |A e^{i(kx - \omega t)}
    |^2
    =
    |A|^2
    \int^\infty_{-\infty}
    dx 
    =
    \infty
}

The way to fix this is by putting the plane wave in a finite box 
that is much larger than the bounds of the problem 

\equations{
    1
    =
    \int^L_{-L}
    dx 
    |A|^2
    \rightarrow
    A 
    =
    \frac{1}{\sqrt{2} L}
}

So the SWE can be given as 

\equations{
    i \hbar \frac{\del}{\del t}
    \Psi 
    =
    - \frac{\hbar}{2 m}
    \frac{\del^2}{\del x^2}
    \Psi 
    \hp
    (+ V(x) \Psi)
}

In classical mechanics, we would have 
\equations{
    E 
    =
    \frac{p^2}{2m}
    +
    V(x)
    \rightarrow 
    p 
    \iff 
    \frac{\del}{\del x}
    \Psi
}

So what is $\hat p$? 

For plane waves 
\equations{
    p \Psi_k 
    =
    \hbar k \Psi_k
}
From De Broglie. We put that into our wave equation

\equations{
    p \Psi_k 
    =
    \hbar k \Psi_k 
    =
    -i \hbar 
    \frac{\del}{\del x}
    \Psi_k
}

Classically, we know that 
\equations{
    V = \dot x 
    \hp 
    p = mV 
}

The expected value is the value over many many quantum mechanical states 
\equations{
    \frac{d \langle x \rangle}{dt}
    =
    \frac{d}{dt}
    \int dx \, 
    x |\Psi(x, t)|^2
    =
    \int dx \, 
    x 
    \left(
        \left(
            \frac{\del}{\del t} \Psi^*
        \right)
        +
        \Psi^*
        \left(
            \frac{\del}{\del t} \Psi
        \right)
    \right)
}

This derivation is in Griffiths 1.5 

\equations{
    \frac{i \hbar }{2m}
    \int dx \,
    x \frac{\del}{\del x}
    \left(
        - \Psi 
        \frac{\del}{\del x}
        \Psi^*
        +
        \Psi^*
        \frac{\del}{\del x}
        \Psi
    \right)
}

We do integration by parts 

\equations{
    \frac{\del}{\del x}
    \left(
        - \Psi 
        \frac{\del}{\del x}
        \Psi^*
        +
        \Psi^*
        \frac{\del}{\del x}
        \Psi
    \right)
    =
    g'
    \hp 
    x = f 
}

\equations{
    - \frac{i \hbar }{2m}
    \int dx \,
    \left(
        \Psi^*
        \frac{\del}{\del x}\Psi 
        -
        \Psi 
        \frac{\del}{\del x}
        \Psi^*
    \right)
    =
    \frac{i \hbar}{m}
    \int dx \,
    \Psi^* \frac{\del}{\del x}
    \Psi
    \\
    =
    \frac{1}{m}
    \int dx \,
    \Psi^*
    \left(
        -i \hbar 
        \frac{\del}{\del x}
    \right)
    \Psi 
    =
    \frac{\del \langle x \rangle} {\del t}
}

Momentum should be 
\equations{
    p 
    \sim 
    m \frac{d}{dt}
    \langle x(t) \rangle
    \\
    \langle p \rangle 
    =
    \int dx \, 
    \Psi^*
    \left(
        -i \hbar \frac{\del}{\del x}
    \right)
    \Psi
}

The momentum operator (expressed in position), known as $\hat p$, is 
\equations{
    \hat p 
    =
    -i \hbar \frac{\del}{\del x}
}

\section{Recap}

Given a state vector $\ket{\Psi}$, you can make a probability distribution 
to describe the position of the particle at time $t$ 
\equations{
    P(x, t)
    =
    |\Psi(x, t)|^2
}

Plane waves cannot be normalized, so they are bad wave functions. 

Solve for $\Psi \rightarrow $ Plane waves (only for math)

The momentum of the particle can be written as 
\equations{
    \hat p 
    =
    - i \hbar 
    \frac{\del}{\del x}
}
That is specifically in the position basis. 

The ep
\equations{
    \langle \hat p \rangle 
    =
    \int dx \, 
    \Psi^* \hat p \Psi
}

That is true for any 
\equations{
    \hat A 
    \equiv
    \hat A(\hat x, \hat p)
    \hp
    \langle \hat A \rangle 
    =
    \Psi^* \hat A \Psi
}

\section{Uncertainty Relation}
Given a plane wave wavefunction. 
Because I have a single wave with a well defined wavelength, the momentum 
is well-defined. 

\equations{
    \sigma_p \to 0
}

However, because it's a plane wave, we cannot say anything about the 
position of the particle. 

\equations{
    \sigma_x \to \infty
}

If given a wave function that looks like a dirac delta with a single 
spike, then the position is well-defined but the momentum is not. 

If you perfectly measure position, the momentum is then a superposition of 
all possible momenta, so the particle will spread out over time because 
the particle has to be moving due the momentum uncertainty.

Generally, the bound for momentum and position variance is 
\equations{
    \sigma_x 
    \sigma_p
    \geq 
    \frac{\hbar}{2}
}

There are wave functions that miimize the uncertainty relation 
(Gaussian Wave Packets).

\section{Solving the SWE}
Given $V(x)$, how do we get $\Psi(x, t)$

The Schrodinger wave equation is given as 
\equations{
    i \hbar \frac{\del \Psi}{\del t}
    =
    - \frac{\hbar^2}{2m} 
    \frac{\del^2 \Psi}{\del x^2}
    +
    V \Psi
}

Here we assume that the state is \textbf{not} time-dependent 
\equations{
    \frac{\del V}{\del t}
    =
    0
}

Because of our time-independence, we can just separate the variables. 
\equations{
    \Psi(x, t)
    =
    \Psi(x) \varphi(t) 
}

Initial Condition: know what $\Psi(x, 0)$ is

\equations{
    i \hbar \frac{1}{\varphi}
    \frac{d \varphi}{d t}
    =
    -\frac{\hbar^2}{2m}
    \frac{1}{\Psi}
    \frac{d^2 \Psi}{d x^2} 
    + V
}

The separation constant is the energy $E$ 

\equations{
    i \hbar \frac{1}{\varphi}
    \frac{d \phi}{d t}
    =
    E 
    \rightarrow 
    \frac{d \varphi}{d t}
    =
    \frac{-i E}{\hbar} \varphi
    \rightarrow 
    \\
    \phi 
    =
    e^{\frac{-iE}{h}}
}

Now we consider the position from the state space 
\equations{
    - \frac{\hbar^2}{2m}
    \frac{d^2 \Psi}{d x^2}
    +
    V \Psi
    =
    E \Psi
}

The left side is the Hamiltonian operator on $\Psi$ 
and the right side is the energy.

The stationary solution is of the form 
\equations{
    \Psi(x, t)
    =
    \psi(x)
    \exp(\frac{-i E t}{\hbar})
}

The time independent and time dependent states are equal 

For any $\hat A$ 
\equations{
    \langle \hat A \rangle 
    =
    \int 
    \psi^* \hat A \psi \, dx
    =
    \textrm{const}
}

We can define the energy as 

\equations{
    H 
    =
    \frac{p}{2m}
    +
    V(x)
}
This is the total energy in classical mechanics (kinetic + potential energy)

Our quantum Hamiltonian operator is 
\equations{
    \hat H 
    =
    - \frac{\hbar^2}{2m}
    \frac{\del^2}{\del x^2}
    +
    V(x)
    \hp 
    \hat H \psi = E \psi 
    \hp 
    \langle \hat H \rangle 
    =
    E
    \\
    H^2 \psi 
    =
    E^2 \psi 
    \rightarrow 
    \langle H^2 \rangle 
    =
    E^2
    \rightarrow 
    \sigma_H^2
    = 0
}

So the general solution is a linear combination of the separated 
solutions. 

\equations{
    V(x)
    \rightarrow 
    \{ \psi_n \}
    \Rightarrow 
    \{ E_n \}
}

With an Initial Condition: 

\equations{
    \Psi(x, 0)
    =
    \sum_n c_n \psi_n(x)
}

And then we evolve that system 
\equations{
    \Psi(x, t)
    =
    \sum_n 
    c_n \psi_n(x) \exp(\frac{-i E_n t}{\hbar})
}

make sure that everything is normalized 
\equations{
    \sum_n |c_n|^2 = 1
    \hp 
    \langle H \rangle 
    =
    \sum_n
    |c_n|^2 E_n
}

\subsection{Question}
Given a stationary state $\{ \psi_n(x) \}$ for some $\hat H$, how 
can we explain motion?

To explain motion, we need a superposition of $\psi_n(x)$.

\equations{
    \Psi(x, 0)
    =
    c_1 \psi_1(x)
    +
    c_2 
    \psi_2(x)
    \rightarrow 
    \Psi(x, t)
    \\
    =
    c_1 \psi_1(x) \exp(\frac{-i E_1 t}{\hbar})
    +
    c_2 \psi_2(x) \exp(\frac{-i E_2 t}{\hbar})
    \\
    |\Psi(x, t)|^2
    =
    \psi^* \psi
}
Use Euler's equation

\equations{
    c_1^2 \psi_1^2 
    +
    c_2^2 \psi_2^2 
    +
    2 c_1 c_2 \psi_1 \psi_2 
    \cos(\frac{(E_2 - E_1) t}{\hbar})
}

You have two stationary states in position. They interfere and something 
happens in time. 

\section{Infinite Square Well}
Horrendously artificial, but one of the few problems we can actually solve. 

Consider a particle with mass $m$ and velocity $v$ in a valley of height 
$h$.

The particle is trapped in the well because $mv^2/2 < mgh$.

We can consider the valley as $h \to \infty$, because the particle can 
quantum tunnel out of it.

The width of the well goes from $0 \to a$. The potential of the 
particle can be stated as 

\equations{
    V(x)
    =
    \begin{cases}
        0: 0 \leq x \leq a 
        \\
        \infty : \textrm{elsewhere}
    \end{cases}
}

At $V = \infty$, we can say that the wave function $\psi = 0$

\subsection{Boundary Conditions}
\begin{itemize}
    \item
    $\psi(x)$ is always continuous.
    \item
    $\psi'(x)$ is always continuous but not if $|V(x)| = \infty$.
\end{itemize}

So now all we have to do is solve the SWE inside the well 
\equations{
    - \frac{\hbar^2}{2m}
    \frac{\del^2}{\del x^2}
    \psi 
    =
    E \psi 
    \iff 
    \psi''(x)
    =
    -k^2 \psi 
    \hp 
    k
    =
    \frac{\sqrt{2mE}}{\hbar}
}

Everything is a simple harmonic oscillator 

\equations{
    \psi=
    A e^{ikx}
    +
    B e^{-ikx}
}

Now we have to consider the boundary conditions. 

\equations{
    \psi(x=0)
    =
    \psi(x=k)
    =
    0
    \\
    A 
    + 
    B 
    = 
    0
    \rightarrow 
    A 
    =
    -B
    \\
    \psi(x)
    =
    A(e^{ikx} - e^{-ikx})
    =
    A \sin(kx)
    \\
    psi(x=0) = 0 
    \rightarrow 
    \sin(kx) = 0
    \\
    \textrm{shit}
}

\subsection{Energies}

\equations{
    k^2 
    =
    \frac{2mE}{\hbar^2}
    \rightarrow 
    E_n 
    =
    \frac{\hbar^2 k_n^2}{2ma^2}
}

\subsection{Normalization}
\equations{
    \int_0^{a}
    A^2 \sin^2(kx)
    \, dx 
    =
    1
    \rightarrow 
    A 
    =
    \sqrt{\frac{2}{a}}
    \\
    \psi_n(x)
    =
    \sqrt{\frac{2}{a}}
    \sin(\frac{n \pi x}{a})
    \hp
    E_h 
    =
    \frac{k^2 \pi^2 \hbar^2}{2ma^2}
}

\section{Discussion 1}
\subsection{The Classical Recipe}
\equations{
    F = ma 
    \hp 
    \frac{\del L}{\del \dot x}
    =
    \frac{\del}{\del x}
    \frac{\del L}{\del x}
}

\subsection{Quantum Reality}

Input is ket. Output is bra.

Ket to bra is input to output based off of the inner product. 

The wavefunction gives the probability density of finding a particle 
in a volume element $dx$

\equations{
    \rho(x)
    =
    \Psi^*(x, t) 
    \Psi(x, t) 
    \\
    P(a < x < b)
    =
    \int^b_a
    \Psi^*(x, t) 
    \Psi(x, t) 
    \, dx
}

The wave function is also normalized

\equations{
    \int^\infty_{-\infty}
    \Psi^*(x, t) 
    \Psi(x, t) 
    \, dx
    =
    1
}

Given an input state $\ket{\Psi_1}$ and an output state 
$\bra{\Psi_2}$. The probability amplitude is given in the form 

\equations{
    \bra{\Psi_2} \ket{\Psi_1}
    =
    \Psi_2^*(x, t) 
    \Psi_1(x, t) 
}

The output state comes first, and it is the one that's the conjugate.

Given a superposition 
\equations{
    \Psi_{in}
    =
    \frac{1}{\sqrt{3}}
    \Psi_1
    +
    \frac{\sqrt{2}}{\sqrt{3}}
    \Psi_2
}

So to find the probability of the output being 2

\equations{
    P(2)
    =
    |\bra{\Psi_{out}} \ket{Psi_{in}}|^2
    =
    \Psi_2
    \frac{1}{\sqrt{3}}
    \Psi_1
    +
    \Psi_2
    \frac{\sqrt{2}}{\sqrt{3}}
    \Psi_2
}
something something I'll look at the lecture notes later 

\subsection{Questions}

Consider a wave function 
\equations{
    \psi(x)
    =
    A(a^2 - x^2)
}

inside interval $\{-a, a\}$

Determine the normalization constant $A$ 

\equations{
    \int \psi(x)^2 \, dx \, = 1
    \\
    A^2
    \int^a_{-a} 
    (a^2 - x^2)^2
    \, dx 
    \hp 
    A^2 
    \left(
        a^4 x - (2 a^2 x^3)/3 + x^5/5
    \right) \Big|^a_{-a}
    =
    \\
    A^2 
    \left(
        a^4 a - (2 a^2 a^3)/3 + a^5/5
    \right)
    -
    \left(
        -a^4 a + (2 a^2 a^3)/3 - a^5/5
    \right)
    \\
    =
    2A^2 
    \left(
        a^5 - \frac{2}{3}a^5 + a^5/5
    \right)
    =
    \frac{16}{15} A^2 a^5 = 1
    \rightarrow 
    A 
    =
    \sqrt{\frac{15}{16 a^5}} 
    \\
    \frac{15}{15} - \frac{10}{15} + \frac{3}{15}
    =
    \frac{8}{15}
}

What is the probability of finding the particle at $x = a/2$. It is 0 
because that's a miniscule-ly small point. The range $-a/2 \to a/2$ is different. 

\equations{
    P(-a/2 < x < a/2)
    =
    \int^{a/2}_{-a/2}
    \Psi^*(x) \Psi(x)
    \\
    =
    2 \frac{15}{16 a^5}
    \left(
        a^4 x - (2 a^2 x^3)/3 + x^5/5
    \right)
    \Big|^{a/2}_{-a/2}
    \\
    =
    \frac{15}{4 a^5}
    a^4 (\frac{a}{2}) - (2 a^2 (\frac{a}{2})^3)/3 + (\frac{a}{2})^5/5
    =
    \frac{15}{4 }
    \left(
    (\frac{1}{2}) - (2  (\frac{1}{8}))/3 + (\frac{1}{32})/5
    \right)
    \\
    \frac{15}{4 }
    \left(
        \frac{1}{2}
        -
        \frac{1}{12}
        +
        \frac{1}{160}
    \right)
}

For what potential is the state an eigenfunction?

The energy eigenfunctions are determined from the Hamiltonian
\equations{
    \hat H 
    =
    - \frac{\hbar^2}{2m}
    \frac{\del^2}{\del x^2}
    \psi
    +
    V(x)
    =
    E \psi
}

So I need to take the double derivative 

\equations{
    \frac{\del^2}{\del x^2}
    Aa^2 - Ax^2
    =
    -2A
    \\
    \frac{\hbar^2}{2m}
    2A
    +
    V(x) ( Aa^2 - Ax^2)
    =
    E A (a^2 - x^2)
}

Set energy to 0 

\equations{
    -\frac{\hbar^2}{2m}
    2 \sqrt{\frac{15}{16 a^5}}
}

\subsection{bra-ket}
Consider a three dimension vector space spanned by an orthonormal basis 
\equations{
    \ket{1}
    \hp
    \ket{2}
    \hp
    \ket{3}
}

$\ket{\alpha}$ and $\ket{\beta}$ are defined as 

\equations{
    \ket{\alpha}
    =
    i \ket{1} - 2\ket{2} - i \ket{3}
    \hp
    \ket{\beta}
    =
    i \ket{1} + 2 \ket{3}
}

Construct $\ket{\alpha}$ and $\ket{\beta}$ in terms of the dual basis 
$\bra{1}$, $\bra{2}$, $\bra{3}$

Something that might be useful 
\equations{
    A_{ij} 
    =
    \bra{i} \hat A \ket{j}
    =
    \bra{i} (\ket{\alpha} \bra{\beta}) \ket{j}
    =
    \bra{i}\ket{\alpha}
    \cdot
    \bra{\beta}\ket{j}
}

\chapter{Solving the SWE}
\section{Recap}
Our starting point is 
\equations{
    \hat H 
    =
    - \frac{\hbar}{2 m}
    \frac{\del^2}{\del x^2}
    +
    V(x)
}
with initial conditions. 

The time-indepedent SWE can be written as 
\equations{
    \hat H \psi(x)
    =
    E_n \psi(x)
    \hp 
    \psi(x, 0) 
    =
    \sum_n c_n \psi_n(x) 
    \hp 
    \psi(x, t) 
    =
    \sum_n c_n \psi_n(x) 
    e^{\frac{-i}{\hbar} E_n t}
}

And infinite square well can be written as 
\equations{
    \psi(x \leq 0) = \psi(x \geq a) = 0 
}

\subsection{Stationary State}
If we consider a probability distribution 
\equations{
    |\psi(x, t)|
    =
    \psi^*(x, t) \psi(x, t)
}
So for an infinite square well. 

Consider the general statement 
\equations{
    \frac{d^2 \psi}{dx^2}
    =
    \frac{2m}{\hbar^2}
    (V(x) - E) \psi
}
if $E < V(x)$, then $\psi''$ and $\psi$ always have the same 
sign. The issue with this is the wave will not be normalizable
(because for positive x, the concavity has to be positive, so it has to increase).

The eigenstates are given by 
\equations{
    E_1
    =
    \frac{\pi^2 \hbar^2}{2 m a^2}
    \hp 
    E_2 = 4 E_1
    \hp
    E_3 = 9 E_2
}

In a stationary state, $\{ \psi_n\}$ forms a complete orthonormal eigenbasis. 
\equations{
    \int \, dx \, 
    \psi^*_m(x)
    \psi_n(x)
    =
    \delta_{mn}(m = n)
    \\
    f(x)
    =
    \sum_n c_n \psi_n(x)
    =
    \sqrt{\frac{2}{a}} 
    \sum_n c_n \sin(\frac{n \pi x}{a})
    \hp 
    c_n = \bra{\psi_n} \ket{f(x)}
}
\subsection*{Question} Given a wave function $\psi(x, t=0)$ 
and a stationary state, how can we find $c_n$ such that 
$\psi = \sum_n c_n \psi_n(x)$?
\equations{
    \int \, dx \, 
    \psi^*_m(x) f(x)
    =
    \int \, dx \, 
    \psi^*_m(x) f(x)
    \sum_n c_n \psi_n(x)
    \\
    =
    \sum_n c_n 
    \int 
    \psi^*_m(x)
    \psi_n(x)
    =
    \sum_n 
    c_n f_{mn}
    =
    c_m
}

\section{Time-Dependence}
\equations{
    \psi(x, t=0)
    =
    \sum_n c_n \psi_n(x)
    \rightarrow 
    \psi(x, t)
    =
    \sum_n c_n \psi_n e^{\frac{-i}{\hbar} E_n t}
}

This is just summing over all of the possible standing waves within 
an infinite square well. 

\subsection{THIS WILL BE ON THE MIDTERN1}
The thing above 

\subsection{Quantum Number}
It's a fake number that is just an index to show the different eigenvectors 
\equations{
    E_n
    \hp 
    n = 
    \textrm{quantum number}
}

\subsection*{Question}
How can we describe a "ball" bouncing between the walls 
of the infinite square well? 

\equations{
    \psi(x, t=0)
    =
    \sum_{odd}
    c_n \psi_n(x)
}
sum odd means that the function that we're summing over has to be an odd function. 

An even function will just have an expected value of 0.

\subsection{Free Particle}
\equations{
    V= 0 
    \rightarrow 
    \hat H 
    =
    -
    \frac{\hbar^2}{2m}
    \frac{\del^2}{\del x^2}
}
and the time-indepedent SWE is 
\equations{
    -
    \frac{\hbar^2}{2m}
    \frac{\del^2}{\del x^2}
    \psi
    =
    E \psi
    \rightarrow 
    \psi 
    =
    A e^{ikx} + B e^{-ikx}
    \hp 
    E = \frac{\hbar^2 k^2}{2m}
    \hp 
    kc=
    \sqrt{\frac{2mE}{\hbar^2}}
}

and the time-\textbf{dependent} SWE is 


\equations{
    \psi(x, t)
    A e^{ikx - \frac{i}{\hbar} E t}
    +
    B e^{-ikx - \frac{i}{\hbar} E t}
    =
    A e^{i(kx - \omega t)}
    +
    B e^{-i(kx + \omega t)}
}

The wave function is a superposition of a left-propagating and a 
right-propagating wave both with constant kintetic energy.

allow $k$ to be
\equations{
    k 
    =
    \pm 
    \sqrt{\frac{2mE}{\hbar^2}}
    \rightarrow 
    A e^{i(kx - \omega t)}
}

The velocity can be written as 
\equations{
    e^{-i(kx - \omega t)}
    =
    e^{-ik(x - \frac{\omega}{k} t)}
    \rightarrow 
    v 
    =
    \frac{\hbar |k|}{2m}
    =
    \sqrt{\frac{E}{2m}}
}

This is different from the classical definition of velocity by a factor of 2
\equations{
    \frac{1}{2} mv^2 = E 
    \rightarrow 
    v 
    =
    \sqrt{\frac{2E}{m}}
}
The reason this is so is because the quantum definition of velocity 
is \textbf{not} for a particle.

Consider a Guassian wave packet 
\equations{
    \psi(x, t)
    =
    \frac{1}{\sqrt{2 \pi}}
    \int 
    \Phi(k)
    e^{i(kx - \omega t)}
    \, dk
}
For some initial conditions $\psi(x, t=0)$, we find the weight function with 

\equations{
    \Phi(k)
    =
    \frac{1}{\sqrt{2 \pi}}
    \int 
    \psi(x, t=0)
    e^{-ikx}
    \, dx 
}

\section{MISSED LECTURE}

\section{Discussion}
Time-Evolving superpositions. We use a Hamiltonian basis 
\equations{
    \hat H \psi_n 
    =
    E_n \psi_n 
    \hp 
    \textrm{Basis }
    =
    \{
        \psi_1, \psi_2, \ldots
    \}
}

This basis is orthonormal 
\equations{
    \bra{\psi_m} \ket{\psi_n} = \delta_{mn}
}

And the basis is complete 
\equations{
    \Psi(x, 0)
    =
    c_1 \psi_1(x)
    +
    c_2 \psi_2(x)
    +
    c_3 \psi_3(x)
    +
    \ldots
}

And we have a time-evolution operator 

\equations{
    \Psi(x, 0)
    =
    c_1 \psi_1(x)
    +
    c_2 \psi_2(x)
    +
    c_3 \psi_3(x)
    +
    \ldots
    \\
    \Psi(x, t)
    =
    e^{\frac{-i \hat H t}{\hbar}}
    \left(
    c_1 \psi_1(x)
    +
    c_2 \psi_2(x)
    +
    c_3 \psi_3(x)
    +
    \ldots
    \right)
}

And this effects each individual state differently 
\equations{
    e^{\frac{-i \hat H t}{\hbar}}
    c_3 \psi_3(x)
    =
    e^{\frac{-i E_3 t}{\hbar}}
    c_3 \psi_3(x)
}

\subsection{Questions}
Find the normalized wave function at time $t$ for a particle in an 
infinite square with a wave function given by 
\equations{
    \psi(x, 0)
    =
    A 
    \left(
        \psi_1(x)
        +
        e^{i \theta}
        \psi_2(x)
    \right)
    \\
    P(x)
    =
    |\psi(x)|^2 
    =
    A^2 
    |\left(
        \psi_1(x)
        +
        e^{i \theta}
        \psi_2(x)
    \right)|^2
}

The time dependent portion of the wave function is 
\equations{
    e^{-i \frac{E}{\hbar}}
}

So our equation will be 

\equations{
    P(x)
    =
    e^{-i \frac{E}{\hbar}}
    |\psi(x)|^2 
    =
    A^2 
    |\left(
        \psi_1(x)
        +
        e^{i \theta}
        \psi_2(x)
    \right)|^2
}

For an infinite square well, the solution can be written as 
\equations{
    \psi(x, 0)
    =
    \sin(\frac{n \pi x}{L})
    \\
    \ket{\psi(x, 0)}
    =
    A 
    \left(
        \ket{\psi_1}
        +
        e^{i \theta}
        \ket{\psi_2}
    \right)
    \\
    |\psi(x)|^2
    =
    A^2 
    \left(
        \bra{\psi_1}
        +
        e^{-i \theta}
        \bra{\psi_2}
    \right)
    \left(
        \ket{\psi_1}
        +
        e^{i \theta}
        \ket{\psi_2}
    \right)
    =
    \\
    A^2 
    \left(
        \bra{\psi_1}\ket{\psi_1}
        +
        \ket{\psi_1}
        e^{i \theta}
        \ket{\psi_2}
        +
        \bra{\psi_2}\ket{\psi_2}
    \right)
    =
    2A^2
    =1
}

The probability density is just something 

\subsection{Infinite Square Well}
Given a unique wave function that's put in the paper and on the website, 
expand the wave function in terms of its eigenvectors using 
a Fourier series.

Go to chapter 11 of townsend for braket notation or chapter 4 for 
time-evolution.

\section{Finite Square Well}
Consider a well with potential 
\equations{
    V(x)
    =
    \begin{cases}
        -V_0 
        : -a < x < a
        \\
        % V(x) 
        0
        : x < -a, x > a
    \end{cases}
}

The wave function can extend out of the well now, and the 
wave function \textbf{must be continuous}.
Exact solutions are typically numerical bound states $(\geq 1)$

\subsection{Scattering Stationary States $(E > 0)$}

Consider that we're starting with a plane wave (coming from the left)
\equations{
    \psi_{I}(x, t=0)
    =
    A e^{ikx}
    +
    B e^{-ikx}
}

In the middle, we have 2 boundary conditions, and both the wavefunction 
itself and its derivative have to be continuous. 

\equations{
    \psi_{II}(x, t=0)
    =
    C \sin(lx )
    +
    D \cos(lx )
}

On the right side, there is still 1 boundary condition, but the 
other side is unbounded

\equations{
    \psi_{III}(x, t=0)
    =
    F e^{ikx}
}

You do some math (Griffith's 2.6). Solve for Boundary Conditions 
and make sure $x$ is continuous at the important parts
\equations{
    \frac{|F|^2}{|A|^2}
    =
    T 
    \hp 
    R 
    =
    1-T 
    \hp 
    k = \frac{\sqrt{2m E}{\hbar}}
    \hp 
    l 
    =
    \frac{\sqrt{2m (E + V_0)}}{\hbar}
    \\
    A e^{-ika}
    +
    B e^{ika}
    =
    -C \sin(la)
    + 
    D \cos(la)
    \\
    \Rightarrow
    \Rightarrow
    \Rightarrow
    \Rightarrow
    \Rightarrow
    \Rightarrow
    \\
    B = i \frac{\sin(2la)}{2kl}
    (l^2 - k^2)
    F
    \hp 
    F 
    =
    A e^{-2ika} 
    \left(
        \cos(2la)
        -
        i \frac{(k^2 + l^2)}{2kl}
        \sin(2la)
    \right)^{-1}
}

Those are the answer given 
\equations{
    \psi(x, 0)
    \propto 
    \int \, dk \, 
    \psi(k) e^{-ikx}
}

An interesting case is $T=1$. This can be fulfilled for 

\equations{
    \frac{2a}{\hbar}
    \sqrt{2m (E_n + V_0)}
    =
    n \pi
    \hp 
    (E_n + V_0)
    =
    \frac{n^2 \pi^2 \hbar^2}{2m (2a)^2}
}

We are matching $k$ to the "infinite well". 

\equations{
    B 
    =
    i \frac{\sin(2la)}{2kl}
    (l^2 - k^2) F 
    \\
    F 
    =
    (e^{-2ika} A)
    * 
    \left(
        \cos(2la)
        -
        i \frac{k^2 + l^2}{2kl}
        \sin(2la)
    \right)^{-1}
}

\section{Quantum Tunneling}
Consider the opposite of a finite square well 
\equations{
    V =
    \begin{cases}
        V_0 : 0 < x < a 
        \\
        0 : x < 0, x > a
    \end{cases}
}

The stationary state consists of a plane wave on the left side.
\equations{
    \psi_{I}
    =
    A^{ikx}
    +
    B^{-ikx}
}

We have exponential decay inside the "well".
The particle can also be reflected, so the wave function is 
given by 
\equations{
    \psi_{II}
    =
    Ce^{-kx}
    +
    De^{kx}
}

And then we have another plane wave on the 2nd side 
\equations{
    \psi_{III}(x, 0)
    =
    F e^{ikx}
}

\section{Recap}
Given a function in the position basis 
\equations{
    \ket{\psi(x)}
    =
    \int \, dx \, 
    \psi(x) \ket{x}
}
We are able to turn it into the energy basis 
\equations{
    \ket{\psi}
    =
    \sum c_n \ket{n}
    \hp 
    \bra{x}\ket{\psi} 
    \rightarrow 
    \psi_n(x)
}

\section{Basis Vectors}
Consider the notation $\ket 1 , \ket 2 , \ket 3$, where each of those corresponds
to the $\vec i, \vec j, \vec k = x, y, z$ unit vectors.

\subsection{Hilbert Space}
A complex space for $N = \ldots ? (\to \infty)$. 

Pick a basis $\ket{n}$ in that space. The column vectors are written in the form. 
\equations{
    \vec \alpha
    =
    \ket{\alpha}
    =
    \sum_n \alpha_n \ket{n}
}
This is just saying that $\vec \alpha$ can be written as a sum of the orthonormal 
basis vectors. 
The row vectors are written in the form
\equations{
    \vec \alpha
    =
    \bra{\alpha}
    =
    \sum_n \bra{n} \alpha_n
    ????????????????????????
}

And the inner product is defined as 
\equations{
    \bra{\alpha} \ket{\beta}
    =
    \sum_n \alpha^* \beta 
    =
    \bra{\beta} \ket{\alpha}^*
}
It's just a dot product. 

The norm is defined as 
\equations{
    \bra{\alpha} \ket{\alpha}
    =
    \sum_n \alpha_n^* \alpha_n
    =
    \sum_n |\alpha_n|^2
}

If the basis vectors are normalized, then 
\equations{
    \bra{k} \ket{n} = \delta_{kn}
    \hp 
    \bra{k} \ket{k} = 1
}

\subsection{Development in a Basis}
\equations{
    \ket{\alpha}
    =
    \sum_k \ket{k} \bra{k} \ket{\alpha}
    =
    \sum_k \alpha_k \ket{k}
    \hp 
    \alpha_k 
    =
    \bra{k} \ket{\alpha}
    \\
    \ket{f}
    =
    \sum_x \ket{x} \bra{x} \ket{f}
    =
    \sum_x \bra{x} \ket{f} \ket{x}
    \hp 
    f_x 
    =
    \bra{x} \ket{f}
    \\
    \sum_k c_k 
    \rightarrow 
    \int \, dx \, \rho 
    \hp 
    \rho = \textrm{ density}
    \Rightarrow 
    \int \, dx \, \ket{x} \bra{x} \ket{f} 
    =
    \int \, dx \, f(x) \ket{x}
    \\
    \bra{f} \ket{g}
    =
    \int \, dx \, \int \, dy \, 
    \bra{x} f^*(x) g(y) \ket{y}
    \hp
    \bra{x} \ket{y} 
    =
    \delta(x - y)
    \\
    \bra{f} \ket{g} 
    =
    \int \, dx \,
    f^*(x) g(x) 
    \bra{g} \ket{f}^*
}

The norm of a function can be written as 
\equations{
    \bra{f} \ket{f} 
    =
    \int \, dx \, 
    f^*(x) f(x)
}
And in a Hilbert Space, 
\equations{
    |\bra{f} \ket{f}|^2
    =
    1
}

\subsection{Operators}
An operator acts on a state 
\equations{
    \hat A \ket{\psi}
}
In the discrete case, operators can be written as matrices 
\equations{
    \hat A \ket{\alpha}
    =
    \ket{\alpha'}
    \hp 
    \bra{\alpha'}
    =
    (\hat A \ket{\alpha})^{*^T}
    =
   \bra{\alpha} \hat A^{\dagger}
   \hp 
   x^\dagger 
   =
   (x^*)^T
}
The basis of matrix elements can be written as 
\equations{
    \bra{1} \hat A \ket{3}
    =
    \alpha_{13}
    \hp 
    \textrm{1st row, 3rd column}
}

\subsection{Projections}
\equations{
    \bra{k} \ket{\alpha}
    =
    \bra{k} (\alpha_1 \ket{1} + \ldots)
    =
    \alpha_k
}

The projection operator can be written as 
\equations{
    \hat P_i 
    =
    \ket{i} \bra{i}
}
everything is 0 except for the ith row and ith column
\equations{
    \hat P_1 \ket{\alpha}
    =
    \ket{1} \bra{1}
    \left(
        \alpha_1 \ket{1}
        +
        \alpha_2 \ket{2}
        +
        \ldots 
    \right)
    =
    \alpha_1 \ket{1}
}

The sum of all projections can be written as 
\equations{
    \sum_i P_i 
    =
    \sum_i \ket{i} \bra{i}
    =
    1
}

Generally 
\equations{
    \ket{k} \bra{i}
    =
    \hp 
    \textrm{Matrix with 1 at kth row, ith column}
    \\
    \hat A 
    =
    \sum_k \sum_n 
    \alpha_{kn} 
    \ket{k} \bra{n}
}

That is just a fancy way to fill a matrix 

\section{Observables}
Observables are operators that are "measurable" (have a real value)
\equations{
    \langle \hat A \rangle 
    =
    \bra{\psi} \hat A \ket{\psi}
    \rightarrow
    \langle \hat A \rangle^* 
    =
    \langle \hat A \rangle
    \rightarrow
    \bra{\hat A \psi} \ket{\psi}
    =
    \bra{\psi} \ket{\hat A \psi}
}

This is equivalent to the Hermitian Conjugate $\hat A^\dagger = (\hat A^*)^T$.
This means that Observables are Hermitian Operators $\hat A^{\dagger} = \hat A$.
Eigenvalues of Hermitian Operators are real. 
\equations{
    \hat A \ket{v} 
    =
    \alpha \ket{v} 
    \rightarrow 
    \bra{v} \hat A \ket{v} 
    =
    \alpha \bra{v} \ket{v} 
    =
    \alpha 
    \hp
    \alpha 
    \in 
    \mathbb{R} 
    \rightarrow 
    \alpha 
    =
    \alpha^*
    \\
    \bra{v} A \ket{v} 
    =
    \bra{v} \ket{A v} 
    =
    \bra{v} \ket{A v}^*
    =
    \bra{A v} \ket{v} 
    =
    \bra{v} A^{\dagger} \ket{v} 
}

We can also solve for the expectation values of the operator 
\equations{
    \langle \hat A \rangle 
    =
    \bra{\psi} \hat A \ket{\psi}
    =
    \left(
        \sum_k c_k^* \bra{\alpha_k}
    \right)
    \hat A 
    \left(
        \sum_n c_n \ket{\alpha_n}
    \right)
    =
    \sum_{k, n}
    c_k^* c_n \bra{\alpha_k} \hat A \ket{\alpha_n}
    \\
    =
    \sum_{k, n}
    c_k^* c_n \alpha_k \bra{\alpha_k} \ket{\alpha_n}
    =
    \sum_{k}
    c_k^* c_n \alpha_n \delta_{kn}
    =
    \sum_{k}
    |c_k|^2 \alpha_k 
}
The kronecker delta is important because that is the operator that 
removes all of the $n$ terms in the expression.

\subsection{is $\hat p$ Hermitian?}
\equations{
    \bra{f} \ket{\hat p g}
    =
    \int f^* (- ih \frac{d}{dx}) g dx 
    =
    - i \hbar f^* g |^{\infty}_{-\infty}
    +
    \int (-ih \frac{d f^*}{dx}) g dx 
    =
    \bra{\hat p f} \ket{g}
}

\subsection{Relationships Between Operators}
The commutator is an operator that is 
\equations{
    [\hat A , \hat B]
    =
    \hat A \hat B 
    -
    \hat B \hat A 
    =
    -[\hat B, \hat A]
    \hp 
    (\hat A , \hat B)^T
    =
    B^T A^T
    \hp 
    (\hat A \hat B)^{\dagger}
    = 
    B^{\dagger}
    A^{\dagger}
}

For $\hat A, \hat B$, Hermitian Operators. 
$[\hat A, \hat B] = 0$ iff there is a basis in which \textbf{both} $\hat A$ 
and $\hat B$ are diagonal.
\equations{
    \bra{k} \hat A \ket{k'} = 0
    \hp
    \bra{k} \hat B \ket{k'} = 0
    \hp 
    \forall k \neq k'
}

\section{Discussion (The Quantum Recipe)}
\begin{enumerate}
    \item 
    Identify the Hamiltonian 
    \item
    Establish the basis (eigenstates)
    \item
    Is your state in the basis?
    \item
    Decompose your state to the basis (Often a fourier transform)
    \item
    Time evolve each element in the decomposition 
    \item
    Compute whatever you want 
\end{enumerate}

\subsection{Example}
We have a wave function of a free particle given by 
\equations{
    \Psi(x, 0)
    =
    A \cos(2kx) 
    +
    B \sin(kx)
}

Identify the Hamiltonian, idk what this means. 
Because it's a free particle, it has a Hamiltonian given by 
\equations{
    - \frac{\hbar}{2m}
    \frac{\del^2}{\del x^2}
    \psi(x, 0)
    =
    \hat H 
    \psi(x, 0)
}

Now establish the basis. The particle is a bunch of waves, so it can 
be written in the basis 
\equations{
    \psi_n(x, 0)
    =
    A e^{inx}
    \hp 
    k \in \mathbb{Z}
}

Is your state in the basis? NO because it is not a sum of eigenvectors.
Decompose your state to the basis 
\equations{
    \Psi(x, 0)
    =
    \frac{A}{2}
    \left(
        e^{i2kx}
        +
        e^{-i2kx}
    \right)
    +
    \frac{B}{2i}
    \left(
        e^{ikx}
        -
        e^{-ikx}
    \right)
}

Time-evolve your solution 
\equations{
    \Psi(x, 0)
    =
    \frac{A}{2}
    \left(
        e^{\frac{-i E_k t}{\hbar}}
        e^{i2kx}
        +
        e^{\frac{-i E_{-k} t}{\hbar}}
        e^{-i2kx}
    \right)
    +
    \frac{B}{2i}
    \left(
        e^{\frac{-i E_k t}{\hbar}}
        e^{ikx}
        -
        e^{\frac{-i E_{-k} t}{\hbar}}
        e^{-ikx}
    \right)
}

\subsection{Example 2}
You have a Hamiltonian and a particle of the form 
\equations{
    \hat H 
    =
    \begin{bmatrix}
        2 && 0 \\
        0 && 1 
    \end{bmatrix}
    \hp 
    \Psi
    =
    \begin{bmatrix}
        1 \\ 1 
    \end{bmatrix}
}

Identify the Hamiltonian (given)

Identify the basis (take the eigenvalues of the matrix)
\equations{
    \Psi_1 = \begin{bmatrix} 1 \\ 0 \end{bmatrix}
    \hp
    \Psi_2 = \begin{bmatrix} 0 \\ 1 \end{bmatrix}
}

Is your state in the basis? no (it is not one of the eigenvectors)

Decompose the state as a form of eigenvectors 
\equations{
    \Psi = \Psi_1 + \Psi_2
}

Time evolve each element in the decomposition.
Because the two states have different energy values (eigenvalues),
the decomposed portions will have different frequencies 
\equations{
    \Psi(t)
    =
    e^{\frac{-i E_1 t }{\hbar}} \Psi_1 
    +
    e^{\frac{-i E_2 t }{\hbar}} \Psi_2
}

\subsection{DO YOU NEED TO MEMORIZE THE BASES (YES)}
Study properties, behaviors, and form. 

\subsection{Questions (2)}
You have a Hamiltonian and a system defined by 
\equations{
    \hat H 
    =
    \begin{bmatrix}
        a && 0 && b \\
        0 && c && 0 \\
        b && 0 && a \\
    \end{bmatrix}
    \hp 
    \ket{S(0)}
    =
    \begin{bmatrix} 0 \\ 1 \\ 0 \end{bmatrix}
}

Find the time-dependent system 
\equations{
    \begin{bmatrix}
        a - \lambda && 0 && b \\
        0 && c- \lambda  && 0 \\
        b && 0 && a- \lambda  \\
    \end{bmatrix}
    =
    \begin{bmatrix}
        a && 0 && 0 \\
        0 && c && 0 \\
        0 && 0 && (a-\lambda) - b^2/(a - \lambda) \\
    \end{bmatrix}
    \\
    (a - \lambda)
    (c - \lambda)
    ((a-\lambda) - b^2/(a - \lambda))
    =
    0
    \\
    ((a-\lambda) - b^2/(a - \lambda)) = 0
    \rightarrow
    (a-\lambda)^2 - b^2 = 0
    \rightarrow 
    a - \lambda = b 
    \rightarrow 
}
eigenvalues are $c, (a - b), (a + b)$. Plug the eigenvalues back into the matrix 
to get the eigenvectors 
\equations{
    \begin{bmatrix}
        a - c && 0 && b \\
        0 && c- c  && 0 \\
        b && 0 && a- c \\
    \end{bmatrix}
    \rightarrow 
    \begin{bmatrix}
        a - c && 0 && b \\
        0 && 0  && 0 \\
        b && 0 && a- c \\
    \end{bmatrix}
    \\
    =
    \begin{bmatrix}
        a - c && 0 && b \\
        0 && 0  && 0 \\
        0 && 0 && a- c - \frac{b^2}{a-c} \\
    \end{bmatrix}
    =
    \begin{bmatrix}
        1 && 0 && 0 \\
        0 && 0  && 0 \\
        0 && 0 && 1 
    \end{bmatrix}
    \rightarrow 
    \\
    \begin{bmatrix} v_1 \\ v_2 \\ v_3 \end{bmatrix}
    =
    \begin{bmatrix} 0 \\ 1 \\ 0 \end{bmatrix}
}
I'm so unbelievably stupid 
\equations{
    \begin{bmatrix}
        a - (a + b) && 0 && b \\
        0 && c- (a + b)  && 0 \\
        b && 0 && a- (a + b) \\
    \end{bmatrix}
    =
    \begin{bmatrix}
        -b && 0 && b \\
        0 && c- a - b  && 0 \\
        b && 0 && -b \\
    \end{bmatrix}
    \\
    =
    \begin{bmatrix}
        -b && 0 && b \\
        0 && c- a - b  && 0 \\
        0 && 0 && 0 
    \end{bmatrix}
}
Im so sos os oso so so so stupid oh my god 
\equations{
    -1 v_1 + v_3 = 0
    \hp 
    v_2 = 0 
    \rightarrow 
    v_1 - v_3
    \\
    \begin{bmatrix} v_1 \\ v_2 \\ v_3 \end{bmatrix}
    =
    \begin{bmatrix} v_1 \\ 0 \\ v_1 \end{bmatrix}
    =
    \begin{bmatrix} 1 \\ 0 \\ 1 \end{bmatrix}
}
ajgfilagjiladfngjilasdnfjasdj;fkasdjfasdfnasdfnsadklfnasd;fnasdfs 





\subsection{3 (Diagonalizing Matrices)}

\section{Formalism}
This is a discrete basis 
\equations{
    \ket{\psi}
    =
    \sum_k 
    c_k \ket{\alpha_k}
    \hp 
    c_k 
    =
    \bra{\alpha_k} \ket{\psi}
    \hp 
    \bra{\alpha_k} \ket{\alpha_j}
    =
    \delta_{kj}
}

This is a continuous basis 
\equations{
    \ket{\psi}
    =
    \int \, dx \, 
    \psi(x) \ket{x} 
    \hp 
    \psi(x)
    =
    \bra{x} \ket{\psi}
    \hp 
    \bra{x'} \ket{x}
    =
    \delta(x - x')
}

An observable is a Hermitian operator with real eigenvalues 
\equations{
    \hat A 
    =
    \hat A^{\dagger}
    \rightarrow 
    \textrm{real eigenvalues}
    \\
    A_{jk}
    =
    \bra{k} \hat A \bra{j}
}

Consider matrix elements of $\hat x$ in the position basis 
\equations{
    \bra{x'} \hat x \ket{x}
    =
    \bra{x'} x \ket{x}
    =
    x \bra{x'} \ket{x}
    ====================
}

A commutator is an operator such that 
\equations{
    [\hat A, \hat B]
    =
    \hat A \hat B 
    -
    \hat B \hat A 
}

If $[\hat A, \hat B] = 0$, then a simultaneous eigenbasis exists.
This means that 
\equations{
    \exists \{ \ket{j} \}:
    \hat A \ket{j} = \alpha_j \ket{j}
    \hp 
    \hat B \ket{j} = \beta_j \ket{j}
}

\subsection{Eigenvalues/vectors}
Consider the momentum operator 
\equations{
    \hat p 
    =
    - i \hbar 
    \frac{\del}{\del x}
    \rightarrow 
    - i \hbar 
    \frac{\del}{\del x}
    \psi 
    =
    p \psi
    \rightarrow 
    \psi_p
    =
    A e^{i \frac{p}{\hbar} x}
}

\subsection{Energy Eigenstates}
consider the time independent schrodinger equation 
\equations{
    \hat H \psi = E \psi 
    \hp 
    \hat H 
    =
    \frac{\hat p^2}{2m }
    =
    - \frac{\hbar^2}{2m}
    \frac{\del^2}{\del x^2}
    \\
    - \frac{\hbar^2}{2m}
    \frac{\del^2}{\del x^2}
    \psi 
    =
    E \psi 
    \rightarrow 
    \left(
        \frac{\del}{\del x}
        -
        \frac{i \sqrt{2m E}}{\hbar}
    \right)
    \left(
        \frac{\del}{\del x}
        +
        \frac{i \sqrt{2m E}}{\hbar}
    \right)
    \psi 
    =
    0
}

This gives you two degenerate solutions.
\equations{
    \psi_1 = A e^{ikx} \hp \psi_2 =  B e^{-ikx}
    \hp 
    \psi_k 
    =
    A e^{ikx} + B e^{-ikx}
}

\section{Basis Transformations}
Suppose we know $\ket{\psi}$ in some basis $\{ \ket{\alpha_k} \}$, and 
we want to know it in some different basis $\{ \ket{\beta_k} \}$. 
Know that $c_k = \bra{\alpha_k} \ket{\psi}$
and $d_k = \bra{\beta_k} \ket{\psi}$
\equations{
    d_n = \bra{\beta_n} \ket{\psi}
    =
    \bra{\beta_n} \sum_k c_k \ket{\alpha_k}
    =
    \sum_k c_k \bra{\beta_n} \ket{\alpha_k}
    \\
    ================================
}

What about in the continuous case?
\equations{
    \hat p \ket{p} 
    =
    p \ket{p}
    \rightarrow 
    \textrm{position basis?}
    \\
    \int \, dx \, 
    \ket{x} \bra{x} \hat p \ket{p}
    =
    p \int \, dx \, 
    \ket{x} \bra{x} \ket{p}
    =
    p \int \, dx \, 
    \ket{x} f_p(x)
    \\
    \int \, dx \, dx' \, 
    \ket{x} \bra{x} \hat p \ket{x'} \bra{x'} \ket{p}
    \\
    \bra{x} \hat p \ket{\psi}
    =
    \int \, dx' \, 
    \bra{x} \hat p \ket{\psi} \bra{x'} \ket{\psi}
    =
    \int \, dx' \, 
    \bra{x} \hat p \ket{x'} \psi(x')
    \\
    =
    \int \, dx' \, 
    \hbar \frac{\del}{\del x} f(x - x') \psi(x')
    \Rightarrow 
    - i \hbar \frac{\del \psi}{\del x}
    \\ 
    \int \, dx \, 
    \ket{x} i \hbar \frac{\del f_p(x)}{\del x} 
    =
    \hat p \int \, dx \, 
    \ket{x} f_p(x)
}
$\bra{x} \hat p \ket{x'}$ is the matrix elements of $\hat p$ in the 
x-basis.

\subsection{What is $\psi(p)$?}
\equations{
    \ket{\psi} 
    =
    \int \, dp \, 
    \bra{p} \ket{\psi} \ket{p}
    =
    \int \, dp \, dx \,
    \bra{p} \ket{x} \bra{x} \ket{\psi} \ket{p}
    \\
    =
    \int \, dx \, dp \,
    A \cdot e^{-ipx / \hbar} \psi(x) \ket{p}
    =
    \int \, dp \, 
    \psi(p) \ket{p} \, dp
}

\subsection{Infinite Square Well}
The energy basis is described as 
\equations{
    \bra{n} \ket{\psi}
    =
    c_n 
    \hp 
    \ket{\psi}
    =
    \sum_n c_n \ket{n}
}

and the position basis is described as 
\equations{
    \bra{x} \ket{\psi_n}
    =
    \psi_n(x)
    =
    \sqrt{\frac{2}{a}}
    \sin(\frac{n \pi x}{a})
}

And the momentum basis is 
\equations{
    \bra{p} \ket{\psi_n} 
    =
    \psi_n(p)
}

In the position basis, our wave function is made by a superposition of planar 
waves. The higher the energy of the planar wave, the more defined is the 
corresponding momentum. Each eigenstate is described by 
\equations{
    \psi_n(p)
    =
    \frac{1}{\sqrt{2 \pi \hbar}} 
    \sqrt{\frac{2}{a}}
    e^{-iE_n t / \hbar}
    \int^a_0
    e^{-ipx / \hbar}
    \sin(\frac{n \pi x}{a}) 
    \, dx \,
}
This is solved with a calculator because why would you go through the trouble.
\equations{
    \frac{4 \pi a}{\hbar}
    \frac{n^2}{
        \left[
            (n \pi)^2
            -
            \left(
                \frac{a p}{\hbar}
            \right)^2
        \right]^2
    }
    *
    \begin{cases}
        \cos^2(\frac{ap}{2\hbar}) : n = odd
        \\
        \sin^2(\frac{ap}{2\hbar}) : n = even
    \end{cases}
}
\subsection{Know the theory, but don't worry about the calculations}

\section{Important Example: 2 Level System}
\equations{
    H 
    =
    \begin{bmatrix}
        1 && 0 \\
        0 && -1
    \end{bmatrix}
    =
    \hbar \omega \sigma_z 
}
$\sigma_z$ is the pauli matrix operator with eigevectors
\equations{
    +1 , \ket{0}
    \equiv 
    \begin{bmatrix}
        1 \\ 0 
    \end{bmatrix}
    \hp 
    -1, \ket{1}
    =
    \begin{bmatrix}
        0 \\ 1 
    \end{bmatrix}
    \hp 
    \textrm{Energy Basis}
}
The general state is of the form 
\equations{
    \alpha \ket{0}
    +
    \beta \ket{1}
    \hp 
    p(\psi = 0)
    =
    |\alpha|^2
}

Consider an operator 
\equations{
    \hbar 
    \begin{bmatrix}
        0 & 1 \\
        1 & 0
    \end{bmatrix}
    \equiv 
    \hbar \sigma_x 
    =
    \hbar 
    \left(
        \ket{0} \bra{1}
        +
        \ket{1} \bra{0}
    \right)
}

This operator \textbf{is observable} because it is Hermitian.
The eigenvectors are
\equations{
    \ket{+}
    =
    \frac{1}{\sqrt{2}}
    \left(
        \ket{0} + \ket{1}
    \right)
    \hp
    \ket{-}
    =
    \frac{1}{\sqrt{2}}
    \left(
        \ket{0} - \ket{1}
    \right)
    \hp 
    \begin{pmatrix}
        1 & 0 \\
        0 & -1 
    \end{pmatrix}
}
with an energy value of $+\hbar$

% The eigenvectors are 
% \equations{
%     \begin{pmatrix} 1 \\ 0 \end{pmatrix}
%     \hp 
%     \begin{pmatrix} 0 \\ 1 \end{pmatrix}
%     \hp 
%     \textrm{eigevalues =}
%     \pm 1
% }

% Consider another observable 
% \equations{
%     \sigma_x 
%     =
%     \begin{pmatrix}
%         0 & 1 \\
%         1 & 0 
%     \end{pmatrix}
%     \hp 
%     \ket{+}
%     =
%     \frac{1}{\sqrt{2}}
%     \left(
%         \ket{0}
%         +
%         \ket{1}
%     \right)
%     \hp
%     \ket{-}
%     =
%     \frac{1}{\sqrt{2}}
%     \left(
%         \ket{0}
%         -
%         \ket{1}
%     \right)
% }

\subsection{Double Well Potential}
Imagine two wells from $-b \to -a$ and $b \to a$ with potential $-V_0$ and then 
potential $0$ everywhere else (in between them is $-b \to b$).

In this system, we imagine measuring the left well vs the right well. 
Consider the tunneling operator 
\equations{
    \hat T \ket{0} = \ket 1
    \hp
    \hat T \ket{1} = \ket 0
    \hp 
    \bra{1} \hat T \ket{1} = 0
    \hp
    \bra{0} \hat T \ket{0} = 1
    \\
    \bra{0} \hat T \ket{1}
    =
    \bra{1} \hat T \ket{0} = 1
    \hp 
    \hat T 
    =
    \sigma_x 
    =
    \begin{pmatrix}
        0 & 1 \\
        1 & 0 
    \end{pmatrix}
    \hp 
    \ket{\pm}
    =
    \frac{1}{\sqrt{2}}
    \left(
        \ket{0}
        \pm
        \ket{1}
    \right)
}

Starting with quantum mechanics with waves and with two levels are 
functionally the same. 

\subsection{The Uncertainty Principle}
Incompatible Observables: measuring $\hat A$ disturbs the possible 
results of $\hat B$. 
\equations{
    \sigma_A \sigma_B 
    \geq 
    \frac{\hbar}{2}
}

The variance of an observable can be written in the form 
\equations{
    \sigma^2_A 
    = 
    \langle
    \left(
        \hat A - \langle \hat A \rangle 
    \right)^2
    \rangle 
    =
    \bra{\psi}
    \left(
        \hat A - \langle \hat A \rangle 
    \right)^2
    \ket{\psi}
    =
    \bra{
    (\hat A - \langle \hat A \rangle )
    \psi
    }
    (\hat A - \langle \hat A \rangle )
    \ket{\psi
    }
    \\
    \equiv 
    \bra{f} \ket{f}
    ============================
}

\subsection{Schwarz Inequality}
\equations{
    |\bra{f} \ket{g}|^2
    = 
    \bra{f} \ket{f}
    \bra{g} \ket{g}
    \\
    \sigma_A^2 \sigma_B^2 
    =
    \bra{f} \ket{f}
    \bra{g} \ket{g}
    \geq 
    |\bra{f} \ket{g}|^2
}

For some complex number $z = z' + iz''$, we know that 
\equations{
    \frac{1}{2i}
    \left(
        z - z^*
    \right)^2
}

So now we have to find the inner products of $f$ and $g$ 
\equations{
    \bra{f} \ket{g}
    =
    \bra{\psi}
    (\hat A - \langle A \rangle)
    (\hat B - \langle B \rangle)
    \ket{\psi}
    \\
    =
    \bra{\psi}
    \hat A \hat B 
    - 
    \hat A \langle B \rangle 
    -
    \hat B \langle A \rangle 
    +
    \langle A \rangle
    \langle B \rangle
    \ket{\psi}
}

Something with commutativity
\equations{
    =
    \bra{\psi}
    \langle \hat A \hat B \rangle
    -
    \langle A \rangle
    \langle B \rangle
    \ket{\psi}
    = 
    \bra{g} \ket{f}
}

Now we put everything together 
\equations{
    \bra{f} \ket{g}
    -
    \bra{g} \ket{f}
    =
    \langle 
        [\hat A, \hat B]
    \rangle
    \Rightarrow 
    \sigma_A^2 
    \sigma_B^2 
    \geq 
    \left(
    \frac{1}{2i}
    \langle 
        [\hat A, \hat B]
    \rangle
    \right)^2
}

\subsection{Back to 2 Layer System (TLS)}

The Pauli matrices are known as 
\equations{
    \sigma_z 
    =
    \ket{0}
    \bra{0}
    -
    \ket{1}
    \bra{1}
    \hp 
    \sigma_x
    =
    \ket{0}
    \bra{1}
    -
    \ket{1}
    \bra{0}
    \hp
    \sigma_y
    =
    \ket{1}
    \bra{0}
    -
    \ket{0}
    \bra{1}
}

Permutations of these matrices are given in the form 
\equations{
    [ \sigma_i, \sigma_j ]
    =
    2i * 
    asdfasdfasdfasdfasdfadfasdfasdfasdfasdfdasfasdf
}

\section{Block Sphere}
This is a graphical representation of all the possible states that 
a wave function can hold. The 6 axes are given by 
\equations{
    \{ \ket{0}, \ket{1} \},
    \{ \ket{+}, \ket{-} \},
    \{ \ket{+i}, \ket{-i} \}
    \\
    \ket{\psi}
    =
    \alpha \ket{0}
    +
    \beta \ket{1}
    =
    \cos(\frac{\theta}{2})
    \ket{0}
    + 
    \sin(\frac{\theta}{2})
    e^{i \phi}
    \ket{1}
}

\section{Discussion 1}
A couple midterm sanity checks. 

Express in wavefunction notation the dirac inner product 
between a ket and a bra.
\equations{
    \bra{\psi_i} \ket{\psi}
    =
    \int^{\infty}_{-\infty} \, dx \,
    \psi_i^* \psi 
}
It's just a convolution integral.

Express the orthonormal condition of a basis in both 
dirac and wavefunction notation 
\equations{
    \bra{e_1} \ket{e_2}
    =
    \delta_{ij}
    \hp
    \int^{\infty}_{-\infty} \, dx \,
    e_1^* e_2
    =
    \delta_{ij}
}
That is the kronecker delta. Basically, when $i \neq j$, its just 0. 

You should be very keen on the equivalence between Dirac notation 
and integral notation. 

\subsection{Adjoint Operators}
The hermitian adjoint is the complex conjugate transpose of the original 
operator. An operator is hermitian (or self-adjoint) if its hermitian adjoint 
is the same operator.

\equations{
    \bra{\psi_i} \ket{\hat L \psi_j}
    =
    \bra{L^{\dagger} \psi_i} \ket{\psi_j}
}

All observables are hermitian (This is because the conjugate of a real 
eigenvalue is just itself).

\subsection{SPOILER}
Show that $\hat p$ is self-adjoint (hermitian)

\subsection{The Quantum Recipe}
\begin{enumerate}
    \item
    identify your Hamiltonian 
    \item
    Establish your basis (eigenstates of the Hamiltonian)
    \item
    Is your state a basis vector (in the basis)
    \item
    If not, decompose your state to basis vectors
    \item
    Time evolve your state 
\end{enumerate}

\subsection{Spin}
Spin is like the zodiac of particles. 
Spin is an intrinsic property that determines the behavior of the particle. 

Up and Down are spins that determine how they go through a Stern-Gerlach 
filter.

\subsection{Reorienting Spin}
Particles don't by themselves have definite spin. Spin up and spin down 
are only determined by the orientation of the Stern-Gerlach filter. 
The Hamiltonian of the spin experiment changes basis. 

The \textbf{Spin Hamiltonian} gives you the orientation of the Stern 
Gerlach filters. The eigenstates of the Hamiltonian will be the two 
states parallel and anti-parallel to the sptial 
orientation described by the Hamiltonian. 

The visual for the direction of the orientation is called 
the \textbf{Bloch Sphere}.

\subsection{Spin Decomposition}
We start off in the $z$-basis, and then we decompose as states in the z-basis. 
So, given the Hamiltonian 
\equations{
    \hat A 
    =
    \hat \sigma_z
}
Looking at it, the Hamiltonian is in the $z$-direction. The eigenstates are 
\equations{
    \ket{\uparrow_H}
    =
    \ket{\uparrow_x}
    \hp
    \ket{\downarrow_H}
    =
    \ket{\downarrow_x}
}

If the Hamiltonian changes from $\hat \sigma_z$ to $\hat \sigma_x$, then 
our eigenstates change, and then they need to be written in terms 
of the $z$-basis.

Now, we get 
\equations{
    \hat H 
    =
    \hat \sigma_z 
    +
    \hat \sigma_x 
}
The direction of the Hamiltonian is 45 degrees in the xz plane, and 
the eigenstates are 
\equations{
    \ket{\uparrow_H}
    =
    \frac{1}{\sqrt{2}}
    \ket{\uparrow_z}
    +
    \frac{1}{\sqrt{2}}
    \ket{\uparrow_x}
}
But then $x$ needs to be decomposed to the $z$-basis.

\subsection{Show $\hat p$ is Hermitian}
Show that $\hat p^\dagger = \hat p$
\equations{
    \bra{g} \ket{\hat p f}
    =
    \int^{\infty}_{-\infty} \, dx \, 
    g^* (\hat p f)
    \rightarrow 
    \int^{a}_{0} \, dx \, 
    g^* \hat p f
    =
    \int^{a}_{0} \, dx \, 
    g^* (- i \hbar \frac{\del}{\del x}) f
    \\
    dv = \frac{\del}{\del x} f
    \hp 
    v 
    =
    f 
    \hp 
    u = (-i \hbar) g^* 
    \hp 
    du = \frac{d}{dx} (-i \hbar) g^* 
    \rightarrow 
    \\
    g^* (- i \hbar) f
    \Big|^{a}_{0} \, dx \, 
    -
    \int^{a}_{0} \, dx \, 
     \frac{d}{dx} (-i \hbar) g^* f
    =
    \int^{a}_{0} \, dx \, 
    (i \hbar) \frac{d}{dx} g^* f
    =
    \int 
    p^* g^* f
    \\
    =
    \int 
    (\hat p g)^* f
    =
    \bra{\hat p g} \ket{f}
}

\subsection{b)}
\equations{
    g^* (- i \hbar) f
    \Big|^{a}_{0} \, dx \, 
    -
    \int^{a}_{0} \, dx \, 
    \frac{d}{dx} (-i \hbar) g^* f
    \rightarrow 
    \\
    g^* (- i \hbar) f
    \Big|^{a}_{0} \, dx \, 
    =
    g^* (- i \hbar) f(0)
    -
    g^* (- i \hbar) \lambda f(0)
    =
    g^* (- i \hbar)
    (1 - \lambda)
}

\subsection{Skip 2 go to 3}
\equations{
    \hat H 
    =
    \hbar \omega 
    \left(
        \alpha \hat \sigma_z
        +
        \beta \hat \sigma_x
    \right)
}
Write $H$ as both a matrix and bra-ket form. 

\equations{
    \hbar \omega
    \begin{pmatrix}
        \alpha & \beta \\
        \beta & -\alpha
    \end{pmatrix}
    =
    \hbar \omega 
    \left(
        \alpha 
        \ket{1} \bra{1}
        +
        \beta
        \ket{0} \bra{1}
        +
        \beta
        \ket{1} \bra{0}
        -
        \alpha 
        \ket{1} \bra{1}
    \right)
}

All observables are \textbf{Hermitian}.

Draw the quantization axis in a Bloch Sphere Picture.

\chapter{Harmonic Oscillator}
\section{MISSED LECTURE}
Now our Hamiltonian is of the form 

\equations{
    \hat H 
    =
    - 
    \frac{\hbar^2}{2m}
    \frac{\del^2}{\del x^2}
    +
    \frac{1}{2}
    m \omega^2 x^2
}

And we have a new operator 
\equations{
    a_{\pm}
    =
    \frac{1}{\sqrt{2 \hbar \omega}}
}
$a_{+}$ and $a_{-}$ are known as the ladder operators. 

\equations{
    \hat H \ket{0}
    =
    \frac{\hbar \omega}{2} \ket 0
}

\equations{
    a_{+} a_{-} 
    =
    \frac{H}{\hbar \omega} - \frac{1}{2}
    =
    \hat n
}
This is known as the number operator.

\equations{
    \bra{n} \hat H \ket{n} 
    =
    \frac{1}{2} \hbar \omega 
    +
    n * \hbar \omega 
}

So we do some shenanigans 
$\bra{n} a_{+}$ is a stationary state and $a_{-} \ket{n}$ is a stationary state, 
so 
\equations{
    \bra{n}
    a_{+} * a_{-}
    \ket{n}
    \neq 0 
    \rightarrow 
    (a_{-})^{\dagger}
    =
    a_{+}
}
both those operators are Hermitian 

\section{Stationary States in the Position Basis}
\equations{
    \hat a_{-} \psi_0(x)
    =
    (2 \hbar m \omega)^{-1/2}
    (\hbar \frac{\del}{\del x} + m \omega x) \psi_0(x) 
    =
    0
    \\
    \frac{\del}{\del x} \psi_0(x)
    =
    - \frac{m \omega x}{\hbar} \psi_0(x)
    \rightarrow 
    \int \, dx \, 
    \frac{1}{\psi_0(x)}
    \frac{\del}{\del x} \psi_0(x)
    =
    \int \, dx \, 
    - \frac{m \omega x}{\hbar}
    ===
    \\
    \psi_0(x)
    =
    A e^{-m \omega x^2 / 2 \hbar}
}
It is a Gaussian distribution. The fourier transform of a Gaussian is 
also a Gaussian. Normalize the function to get 

\equations{
    1 
    =
    \bra{\psi} \ket{\psi}
    = 
    \int \, dx \, 
    \psi_0^*(x) \psi_0(x)
    =
    [-] 
    \Rightarrow 
    A 
    =
    \left(
        \frac{m \omega}{\pi \hbar}
    \right)^{1/4}
}

And now we try to find $\psi_1(x)$.
\subsection{I SKIPPED WHOOPS}

\equations{
    \psi_n(x)
    =
    \left(
        \prod_{k=1}^{n} 
        \frac{a_+}{\sqrt{k}}
    \right)
    \psi_0 
    =
    \frac{1}{\sqrt{n!}} a_{+}^{n} \psi_0 
    \\
    \psi_n(x)
    =
    \left(
        \frac{m \omega}{\pi \hbar}
    \right)^{1/4}
    \frac{1}{\sqrt{2^n n!}}
    H_n(x) e^{-x^2/2}
    \\
    H_n(\zeta)
    =
    (-1)^{n}
    e^{\zeta^2}
    \frac{\del^n}{\del \zeta^n}
    e^{-\zeta^2}
}

\section{Discussion}

is a physically valid wavefunction of a free particle 

If energy is definite, it cannot bounce :(

If you put enough unphysical states together, you get a physical state.


\subsection{Recap}
\equations{
    V(x)
    =
    \frac{1}{2} m \omega^2 x^2 
    \Rightarrow 
    \left(
        \frac{-\hbar^2}{2m} \frac{\del^2}{\del x^2}
        + 
    \frac{1}{2} m \omega^2 x^2 
    \right)
    \psi(x)
    =
    E \psi(x)
    \rightarrow 
    \\
    a_{\pm}
    =
    \frac{1}{\sqrt{2 \hbar m \omega}}
    \left(
        \pm i \hat p 
        +
        m \omega \hat x
    \right)
    \hp
    \hat a_{+}
    =
    (\hat a^{\dagger})
    \hp
    \hat a_{-}
    =
    (\hat a)
    \\
    a_{+} a_{-}
    = 
    \hat n 
    \hp 
    \hat a_{+} \ket{n}
    =
    \sqrt{n + 1} \ket{n + 1}
    \\
    \hat a_{-} \ket{n}
    =
    \sqrt{n} \ket{n - 1}
    \hp 
    [a_{-}, a_{+}] = 1 
    \frac{\hat H}{\hbar \omega} = a_{+} a_{-} + \frac{1}{2}
    \\
    a_{-}(0) = \ket{0}
    \hp 
    a_{-} \ket{\psi}
    =
    \psi_0(x)
    =
    A e^{\frac{-m \omega x^2}{2 \hbar}}
}

Stationary States/Eigenfunctions are Gaussians centered at 0. 
$a_{+} \ket{0} = \ket{1}$ has energy $\hbar \omega(1 + 1/2)$, but it 
contains noise carrying an energy of $1/2 \hbar \omega$.

\section{Relation to Classical S.H.O.}
You have a mass oscillating around with $x \propto \cos(\omega t)$
and $p(t) \propto \sin(\omega t)$. The relationship between $x$ and $p$ 
looks like a circle. 

What do the stationary states of $\hat H$ look like in the phase space?
The wavefunctions of the QSHO are of the form 
\equations{
    \psi_0(x)
    =
    A_0 e^{\frac{-m \omega x^2}{2 \hbar}}
    \hp
    \psi_1(x)
    =
    A_1 x e^{\frac{-m \omega x^2}{2 \hbar}}
}

Take a fourier transform to get the momentum space 
\equations{
    \psi_0(p)
    =
    \frac{1}{\sqrt{2 \pi \hbar}}
    \int e^{-i p x / \hbar } \psi_0(x)  \, dx \, 
    =
    B_0 e^{\frac{- p^2}{2 \hbar m \omega^2}}
    \hp
    \psi_1(p)
    =
    B_1 p e^{\frac{- p^2}{2 \hbar m \omega^2}}
}

You can Guassians of some width and height that I didn't write down :(

For a classical SHO, you have 
\equations{
    x = x_0 \cos(\omega t) 
    \hp 
    E_{tot}
    =
    \frac{1}{2} m \omega^2 x_0^2 
    \hp 
    p(x) dx 
    =
    2 \frac{dt}{T}
    \hp 
    T 
    =
    \frac{2 \pi}{\omega}
    \\
    dx 
    =
    -x_0 \omega \sin(\omega t) dt 
    =
    -x_0 \omega 
    \sqrt{1 - \left( \frac{x}{x_0} \right)^2} dt 
    \Rightarrow 
    p(x)
    =
    \frac{1}{\pi x_0 \sqrt{1 - \left( \frac{x}{x_0} \right)^2}}
}

for the quantum world, we solve for $x$ and $p$ in terms of 
$a_{+}$ and $a_{-}$
\equations{
    a_{\pm}
    =
    \frac{1}{\sqrt{2 \hbar m \omega }}
    (\pm i \hat p + m \omega \hat x)
    \Rightarrow 
    \hat x 
    =
    \sqrt{\frac{\hbar}{2m}} 
    \frac{\hat a_{-} + \hat a_{+}}{\sqrt{2}}
    \hp 
    \hat p 
    =
    \sqrt{\hbar m \omega}
    \frac{-i (\hat a_{-} - \hat a_{+})}{\sqrt{2}}
    \\
    \langle x \rangle 
    =
    \bra{n} \hat x \ket{n}
    \propto
    \bra{n} (a_- + a_+) \ket{n}
    =
    0
    \\
    \langle \Delta x \rangle^2
    =
    \langle x^2 \rangle 
    =
    \frac{\hbar}{2 m \omega}
    \bra{n}
    a_{-}^2 + a_{-} a_{+} + a_{+} a_{-} + a_{+}^2 \ket{n} 
    \\
    =
    \bra{n}
    a_{+} a_{-} + \frac{1}{2} \ket{n} 
    =
    \frac{\hbar}{m \omega} (n + \frac{1}{2})
    \\
    \langle p \rangle = 0 
    \hp 
    \langle p^2 \rangle 
    =
    \hbar m \omega (n + \frac{1}{2})
    \hp 
    \Delta x \Delta p 
    =
    (n + \frac{1}{2}) \hbar
}

We graph $\tilde{x} \equiv \frac{a_{-} + a_{+}}{\sqrt{2}}$
over $\tilde{p} \equiv \frac{-i (a_{-} - a_{+})}{\sqrt{2}}$.
We get concentric circles 
\equations{
    \tilde{x}^2 + 
    \tilde{p}^2
    =
    2 \hat n
}

\section{Coherent State}

Let's attempt $n$ such that $\bra{n} x \ket{n} = 0$. 

\equations{
    a_{-} \ket{\psi_\alpha} = \alpha \ket{\psi_{\alpha}} 
    \hp 
    \alpha \in \mathbb{C}
    \\
    \bra{\psi_{\alpha}} a_{-} + a_{+} \ket{\psi_\alpha}
    =
    \alpha^* + \alpha = 2 Re(\alpha)
    \\
    \bra{n} \ket{\psi_{\alpha}} 
    =
    \frac{1}{\sqrt{n!}} \bra{0} a_{-}^n \ket{\psi_\alpha}
    =
    \frac{\alpha^n}{\sqrt{n!}} \bra{0} \ket{\psi_{\alpha}} = 
    \frac{\alpha^n}{\sqrt{n!}} A
    \\
    \ket{\psi_{\alpha}}
    =
    \sum_{n = 0}^{\infty}
    \ket{n} \bra{n} \ket{\psi_{alpha}}
    =
    A
    \sum_{n = 0}^{\infty}
    \frac{\alpha^{n}}{\sqrt{n!}} \ket{n}
    \equiv
    A
    \sum_{n = 0}^{\infty}
    \frac{(\alpha a_{+})^{n}}{\sqrt{n!}} \ket{0}
    \\
    A 
    =
    e^{- |\alpha|^2 / 2}
}
$A$ is the normalization constant form 
$\bra{\psi_{\alpha}} \ket{\psi_{\alpha}} = 1$.
This is the most coherent quantum harmonic oscillator that we can get 
and this is the wave function of lasers. We can do all sorts 

\subsection{Time Evolution}
\equations{
    \ket{\alpha(t)}
    =
    e^{-|\alpha|^2 / 2}
    \sum_{n=0}^{\infty}
    \frac{\alpha^n}{\sqrt{n!}}
    e^{-i \omega t n} 
    \ket{n}
}

\chapter{3D Quantum Mechanics}
Trying to figure out the wave equation with the Coulomb Potential. 

\subsection{Assumptions}
Only observe the wave function of the electron (because the nucleus is heavy
so we can fix the position). Maybe something else I didnt pay attention. 

\section{Quantum Numbers}
Because we're now in 3d space, we have 4 quantum numbers (x, y, z, spin).

Because we're working with a central potential, angular momentum is conserved, 
and angular momentum is \textbf{quantized}.

If an operator is conserved, it commutes with the Hamiltonian, which means 
it is \textbf{simultaneously diagonalizable}.

The magnitude of angular momentum is conserved, but the direction is not 
necessarily. 

\equations{
    i \hbar \frac{\del}{\del t} \ket{\psi}
    =
    \hat H \ket{\psi}
    \hp
    \hat H 
    =
    \frac{\hat p^2}{2m} + V
    =
    \frac{1}{2m} (\hat p_x^2 + \hat p_y^2 + \hat p_z^2) + V(x, y, z)
    \\
    \hat p 
    =
    -i \hbar \nabla 
    \rightarrow 
    \frac{\vec{p^2}}{2m}
    =
    - \frac{\hbar^2}{2m} \nabla^2
    \\
    i \hbar \frac{\del}{\del t} \ket{\psi}
    =
    - \frac{\hbar^2}{2m} \nabla^2 \psi(\vec r, t)
    +
    V \psi(\vec r, t)
}

\section{Spherical Coordinates}
Just google it every time to need to use it. 

\equations{
    \nabla^2 
    =
    \frac{1}{r^2}
    \frac{\del}{\del r}
    \left(
        r^2 
        \frac{\del}{\del r}
    \right)
    +
    \frac{1}{r^2 \sin(\theta)}
    \frac{\del}{\del \theta}
    \left(
        \sin(\theta)
        \frac{\del}{\del \theta}
    \right)
    +
    \frac{1}{r^2 \sin^2(\theta)}
    \frac{\del^2}{\del \phi^2}
    \\
    - \frac{\hbar^2}{2m} \nabla^2 \psi(\vec r, t)
    +
    V \psi(\vec r, t)
    =
    E \psi
}

Use separation of variables to find solutions 
\equations{ 
    \psi(r, t)
    =
    R(r)
    Y(\theta, \phi)
    \\
    - \frac{\hbar^2}{2m} 
    \left(
        \frac{Y}{r^2}
        \frac{\del}{\del r}
        r^2 \frac{\del R}{\del r}
        +
        \frac{R}{r^2 \sin(\theta)}
        \frac{\del}{\del \theta}
        \left(
            \sin(\theta)
            \frac{\del Y}{\del \theta}
        \right)
        +
        \frac{R}{r^2 \sin^2(\theta)}
        \frac{\del^2 Y}{\del \phi^2}
    \right)
    \\
    +
    V(\vec r, t)
    R Y 
    =
    E * RY
    \\
    \left(
        \frac{1}{R}
        \frac{\del}{\del r}
        r^2 \frac{\del R}{\del r}
        -
        \frac{2 m r^2}{\hbar^2}
        \left[
            V(r) - E
        \right]
    \right)
    \\
    +
    \frac{1}{Y}
    \left(
        \frac{1}{\sin(\theta)}
        \frac{\del}{\del \theta}
        \left(
            \sin(\theta)
            \frac{\del Y}{\del \theta}
        \right)
        +
        \frac{1}{\sin^2(\theta)}
        \frac{\del^2 Y}{\del \phi^2}
    \right)
    =
    0
}

We now have an angular equation and a radial equation, with a separation 
constant $l (l + 1)$.

\section{Angular Momentum}
\equations{
    \hat L 
    =
    \hat r 
    \times 
    \hat p 
    =
    -i \hbar 
    \left(
        \hat r 
        \times 
        \vec \nabla
    \right)
    \Rightarrow 
    \\
    L^2 
    =
    - \hbar^2 
    \left[
        \frac{1}{\sin(\theta)}
        \frac{\del}{\del \theta}
        \left(
            \sin(\theta)
            \frac{\del}{\del \theta}
        \right)
        +
        \frac{1}{\sin^2(\theta)}
        \frac{\del^2}{\del \phi^2}
    \right]
}

The eigenstates are going to be $R(r) * Y(\theta, \phi) = $ an eigenfunction 
of angular momentum. 

Because angular momentum is conserved, $[\hat H, \hat L^2] = 0$.

\equations{
    \hat L = \hat{\vec r} \times \hat{\vec p}
    \Rightarrow 
    L_x 
    =
    y p_z - z p_y
    \hp
    L_y 
    =
    z p_x - x p_z
    \hp
    L_z 
    =
    x p_y - y p_x
    \\
    [L_x, L_y]
    =
    [y p_z - z p_y, z p_x - x p_z]
    \\
    =
    [y p_z, z p_x]
    -
    [y p_z,x p_z]
    -
    [z p_y, z p_x]
    +
    [z p_y, x p_z]
    \\
    =
    [y p_z, z p_x]
    -
    0
    -
    0
    +
    [z p_y, x p_z]
    =
    y p_x [p_z, z]
    +
    x p_y [z, p_z]
    \\
    =
    i \hbar 
    \left(
        x p_y -y p_z
    \right)
    =
    i \hbar L_z
}
This looks awfully similar to the pauli matrices. 

\equations{
    [L_x, L_y]
    =
    i \hbar L_z
    ,
    [L_y, L_y]
    =
    i \hbar L_x
    ,
    [L_z, L_x]
    =
    i \hbar L_y
    \hp
    [L_j, L_k]
    =
    i \hbar \epsilon_{jkm} L_m
}

The magnitude squared of angular momentum can be written as 
\equations{
    L^2
    =
    L_x^2
    +
    L_y^2
    +
    L_z^2
    \hp 
    [L^2, L_x]
    =
    [L_x^2, L_x]
    +
    [L_y^2, L_x]
    +
    [L_z^2, L_x]
    \\
    =
    L_y [L_y, L_x]
    +
    [L_y, L_x] L_y
    +
    L_z [L_z, L_x]
    +
    [L_z, L_x] L_z
    =
    0
}

\section{Discussion 6: Navigating the Gaussian (HALF MISSED)}
The ground state is a Gaussian. 

The raising operator has a $\sqrt{n+1}$ and the lowering 
operator has $\sqrt{n}$. 

If you start at state $12$ and go to state $14$, you'll get 
change of 
\equations{
    \hat a^{\dagger}
    \hat a^{\dagger}
    \psi_{12}
    =
    \hat a^{\dagger}
    \sqrt{13}
    \psi_{13}
    =
    \sqrt{14}
    \sqrt{13}
    \psi_{14}
}

The harmonic oscillator gas \textbf{linear} energy dependence. 

\subsection{Questions}

Find $x^2$ in terms of $a_+$ and $a_-$

\equations{
    a_{+}
    =
    \frac{1}{\sqrt{2 \hbar m \omega}}
    \left(
        + i \hat p 
        +
        m \omega \hat x 
    \right)
    \hp
    a_{-}
    =
    \frac{1}{\sqrt{2 \hbar m \omega}}
    \left(
        - i \hat p 
        +
        m \omega \hat x 
    \right)
    \\ 
    a_{+} + a_{-}
    =
    \frac{1}{\sqrt{2 \hbar m \omega}}
    2 \hat x
    \Rightarrow 
    x 
    =
    \sqrt{2 \hbar m \omega}
    \frac{ a_{+} + a_{-} } {2}
    \\
    x^2 = 
    \hbar m \omega
    \frac{ (a_{+} + a_{-})^2 } {2}
    \\
    \bra{n} x^2 \ket{n}
    =
    \frac{\hbar m \omega}{2}
    \bra{n}
    (a_{+} + a_{-})^2
    \ket{n}
    =
    \frac{\hbar m \omega}{2}
    \bra{n}
    a_{+}^2
    +
    a_{-}^2
    +
    a_{+} a_{-}
    \ket{n}
    \\
    \bra{n}
    a_{+}^2
    \ket{n}
    +
    \bra{n}
    a_{-}^2
    \ket{n}
    +
    \bra{n}
    a_{+} a_{-}
    \ket{n}
    =
    \\
    \bra{n}
    \sqrt{(n+2)(n+1)}
    \ket{n+2}
    +
    \bra{n}
    \sqrt{(n)(n-1)}
    \ket{n-2}
    \\
    +
    n
    \bra{n}
    \ket{n}
    +
    (n+1)
    \bra{n}
    \ket{n}
    =
    0 + 0
    +
    n
    \bra{n}
    \ket{n}
    +
    (n+1)
    \bra{n}
    \ket{n}
    =
    2n + 1
}

\subsection{}
Given a state 
\equations{
    \ket{\psi}
    =
    \frac{1}{\sqrt{2}}
    \left(
    \ket{0}
    e^{-i \omega t/2}
    +
    \ket{1}
    e^{-3i \omega t/2}
    \right)
}

And we know the average position 
\equations{
    \langle \hat x \rangle 
    \sqrt{\frac{\hbar}{2 m \omega}}
    \cos(\omega t)
}

do shenanigans to find $\langle p \rangle$ 
\equations{
    \hat p 
    =
    \frac{\sqrt{2 \hbar m \omega}}{2}
    (a_{+} - a_{-})
    \\
    \bra{\psi} \hat p \ket{\psi}
    =
    \frac{\sqrt{2 \hbar m \omega}}{2}
    \bra{\psi}
    (a_{+} 
    \ket{\psi}
    - a_{-}
    \ket{\psi}
    )
    \\
    a_{+} \ket{\psi} 
    =
    \frac{1}{\sqrt{2}}
    \left(
    1
    \ket{1}
    e^{-i \omega t/2}
    +
    \sqrt{2}
    \ket{2}
    e^{-3i \omega t/2}
    \right)
    \\
    a_{-} \ket{\psi} 
    =
    \frac{1}{\sqrt{2}}
    \left(
    0
    +
    1
    \ket{0}
    e^{-3i \omega t/2}
    \right)
    \\
    \frac{1}{\sqrt{2}}
    \left(
    \bra{0}
    e^{i \omega t/2}
    +
    \bra{1}
    e^{3i \omega t/2}
    \right)
    \frac{1}{\sqrt{2}}
    \left(
    1
    \ket{1}
    e^{-i \omega t/2}
    +
    +
    \sqrt{2}
    \ket{2}
    e^{-3i \omega t/2}
    1
    \ket{0}
    e^{-3i \omega t/2}
    \right)
    \\
    =
    \frac{1}{2}
    \left(
        e^{-3 i \omega t /2} e^{i \omega t /2}
        +
        e^{3 i \omega t /2} e^{- i \omega t /2}
    \right)
    =
    \frac{1}{2i}
    \left(
        e^{-i \omega t /2}
        +
        e^{i \omega t /2}
    \right)
    =
    \sin(\omega t / 2)
}
there's an i is the thing for $\langle p \rangle$ trust me bro 


\subsection{3D harmonic oscillator}
\equations{
    \frac{- \hbar}{2m }
    \nabla^2
    \ket{\psi}
    + 
    \frac{m \omega^2}{2}
    \left(
        x^2 + y^2 + z^2
    \right)
    \ket{\psi}
    =
    E \ket{\psi}
    \\
    \frac{\hbar}{2m }
    \nabla^2
    \ket{\psi}
    =
    \frac{m \omega^2}{2}
    \left(
        x^2 + y^2 + z^2
        - E
    \right)
    \ket{\psi}
}

Consider just x 
\equations{
    \frac{\hbar}{2m }
    \frac{\del^2}{\del x^2}
    \ket{\psi}
    =
    \frac{m \omega^2}{2}
    \left(
        x^2
        - E
    \right)
    \ket{\psi}
    \Rightarrow 
    \psi 
    =
    e^{x^2}
    \hp 
    \frac{\del}{\del x}
    e^{x^2}
    =
    2x
    e^{x^2}
    \\
    \frac{\del}{\del x}
    2x
    e^{x^2}
    =
    2
    e^{x^2}
    +
    4x^2
    e^{x^2}
    \hp 
    \psi
    =
    X(x) Y(y) Z(z)
    \\
    \frac{- \hbar}{2m}
    \left(
        2
        e^{x^2}
        +
        4x^2
        e^{x^2}
    \right)
    +
    \frac{m \omega^2}{2}
    e^{x^2}
    =
    E
    e^{x^2}
    \Rightarrow 
    \psi_{0}
    =
    A_0 
    e^{\frac{- m \omega x^2}{2 \hbar}}
    \hp 
    \psi_{1}
    =
    A_1 
    x
    e^{\frac{- m \omega x^2}{2 \hbar}}
}


\section{3D Stationary States}
\subsection{Recap}
Imagine we have 2 particles, one of which with a central potential $V(r)$. 
Because it is central, we can say that angular momentum is conserved 
$\vec L = \vec r \times \vec p = - i \hbar \left( \vec r \times \vec \nabla \right)$.

The SWE can be written as 
\equations{
    i \hbar 
    \frac{\del \psi(r, \theta, \phi)}{\del t}
    =
    - \frac{\hbar}{2m}
    \nabla^2 \psi(r, \theta, \phi)
    +
    V(r) \psi(r, \theta, \phi)
    \hp
    \psi 
    =
    R(r) Y(\theta, \phi)
    \Rightarrow 
    \\
    \left(
        \frac{1}{R}
        \frac{\del}{\del r}
        r^2 \frac{\del R}{\del r}
        -
        \frac{2 m r^2}{\hbar^2}
        \left[
            V(r) - E
        \right]
    \right)
    \\
    +
    \frac{1}{Y}
    \left(
        \frac{1}{\sin(\theta)}
        \frac{\del}{\del \theta}
        \left(
            \sin(\theta)
            \frac{\del Y}{\del \theta}
        \right)
        +
        \frac{1}{\sin^2(\theta)}
        \frac{\del^2 Y}{\del \phi^2}
    \right)
    =
    0
}
That is both the radial and angular equation that solve for our 
wave function. 

The expected value of angular momentum to be conserved is 
\equations{
    [\hat H, \hat L^2] = 0
}

Eigenstates and values of angular momentum can be written as 
\equations{
    [L_j, L_k]
    =
    i \hbar \epsilon_{jkl} L_l 
    \hp 
    j, k, l = \{x, y, z\}
    \\
    [L^2, L_j] = 0 
    \Rightarrow 
    L^2 \psi 
    =
    \lambda \psi 
    \hp
    L_z \psi 
    =
    \mu \psi
}

Let's define a thing and solve some stuff 
\equations{
    L_{\pm}
    =
    L_x \pm i Ly
    \\
    [L_z, L_{\pm}]
    =
    \pm \hbar L_{\pm}
    \hp 
    [L^2, L_{\pm}] = 0
}

If $\psi$ is an eigenfunction of $L^2$ and $L_z$, then $L_{\pm} \psi$ is also 
an eigenfunction of those. 

\equations{
    L^2 (L_{\pm} \psi)
    =
    L_{\pm} L^2 \psi 
    =
    L_{pm} \lambda \psi 
    =
    \lambda (L_{pm} \psi) 
    \\
    L_z (L_{pm} \psi)
    =
    \left(
        L_z L_{\pm}
        -
        L_{\pm} L_z
    \right)
    \psi 
    +
    L_{\pm} L_z \psi
    =
    \left(
        \mu \pm \hbar 
    \right) \psi
}

$L_{+} (L_{-})$ raises (or lowers) the Eigenvalue of $L_z$ by $\hbar$, 
but \textbf{do not change} $L^2$.
\equations{
    L_{\pm} \psi 
    =
    \psi' 
    \hp 
    L^2 \psi 
    =
    \lambda \psi
}

\section{Bounds of Angular Momentum}
\textbf{(z-component, given fixed $\lambda$)}

The angular momentum looks like a ladder with rungs $\hbar$ apart. 
The top of $\psi_t$ and the bottom is $\psi_b$
\equations{
    \psi = \mu 
    \Rightarrow 
    L_{+} \psi = \mu + \hbar 
    \Rightarrow 
    L_{-} \psi = \mu - \hbar 
    \hp 
    L_{-} \psi_{b} = 0
    \hp
    L_{+} \psi_{t} = 0
    \\
    L_z \psi_t
    =
    \hbar l \psi_t
    \hp
    L_z \psi_b
    =
    \hbar \bar{l} \psi_b
    \hp 
    L^2 
    \psi_{t/b}
    =
    \lambda
    \psi_{t/b}
}
$l$ and $\bar{l}$ are just the maximum and minimum values of $\mu$.

I can do some identity shenanigans 
\equations{
    L_{\pm}
    L_{\mp}
    =
    (L_x \pm i L_y)
    (L_x \mp i L_y)
    =
    L_x^2 + L_y^2
    \mp 
    (L_x L_y - L_y L_x)
    \\
    =
    L_2 - L_z^2
    \mp 
    i (i \hbar L_z)
    \Rightarrow 
    L^2 
    =
    L_{\pm}
    L_{\mp}
    + 
    L_z^2 
    \mp 
    \hbar L_z
    \\
    L^2 \psi_t
    =
    \left(
        L_{-} L_{+}
        +
        L_z^2 
        +
        \hbar L_z 
    \right) \psi_t
    =
    \hbar^2 l (l + 1) \psi_t
    =
    \lambda \psi_t
}

These look suspiciously like Legendre Polynomials? eh we'll find out. 
That is how the top of the ladder looks, but the bottom is different. 

\equations{
    \lambda 
    \equiv 
    \hbar^2 
    \bar{l}
    (\bar{l} - 1)
}

\section{Index Wavefunctions}
Some idiot decided to make the index $m$ so now we have another $m$ that isn't 
mass. 
\equations{
    L^2 
    \psi_{l}^{m}
    =
    \hbar^2 l(l+1)
    \psi_{l}^{m}
    =
    \lambda
    \psi_{l}^{m}
    \hp 
    L_z
    \psi_{l}^{m}
    =
    \hbar m 
    \psi_{l}^{m}
    =
    \mu
    \psi_{l}^{m}    
}
$L_{\pm}$ raises/lowers the z-projection by $\pm \hbar$
$m_{min} = -l(-l+1)$ and it increments by $\hbar$ until to get 
$m_{max} = l(l+1)$. $l$ is an integer or an integer $+ 1/2$.

\subsection{Griffiths Fig. 4.12}
If we imagine a sphere, a quantized value $L_z$ could be a circular trajectory 
that lies near the top of the sphere. The largest value of $L_z$ is 
the trajectory that lies on the diameter of the sphere, which is $\sqrt{L^2}$. 

\section{Recap}
Consider the SWE in 3D with a central potential $V = V(r)$. You write 
it all out in spherical coordinates and do separation of variables and then 
you get a radial equaation plus an angular equation = 0. The angular equation 
became equal to $l(l+1)$ and the radial = $-l(l+1)$.

Angular momentum is conserved. We should look for eigenfunctions of 
$L^2$ and $L_z$ $([L^2, L_z] = 0)$. Use the structure of orthogonal 
projections. 

Introduce raising and lowering functions $L_{+}$ and $L_{-}$, that 
raise and lower the $z$-projection of angular momentum. 
\equations{
    L^2 \psi_{l}^{m}
    =
    \hbar^2 l (l + 1)  \psi_{l}^{m}
    \hp 
    L_z  \psi_{l}^{m}
    =
    \hbar m \psi_{l}^{m}
    \hp 
    l 
    =
    \textrm{integer or integer + 1/2}
    \\ 
    m - -l, -l+1, \ldots, l+1, l
}

It's like a ladder with $\# m = zl + 1$ values. 
The magnitude of angular momentum is conserved \textbf{and} 
the z-projection \sout{are conserved} is quantized. 

\section{Connecting $L$ to the angular equation}
\equations{
    \vec L 
    =
    -i \hbar 
    \left(
        \vec r 
        \times 
        \vec \nabla
    \right)
    \hp 
    \nabla 
    =
    \hat r 
    \frac{\del}{\del r}
    +
    \hat \theta 
    \frac{1}{r}
    \frac{\del}{\del \theta}
    +
    \hat \phi
    \frac{1}{r \sin(\theta)}
    \frac{\del}{\del \phi}
    \hp 
    \vec r 
    =
    r 
    \hat r 
}

We cannot use just spherical coordinates because we know that only the 
$z$-projection of the angular momentum is quantized. 
\equations{
    \vec L 
    =
    -i \hbar 
    \left[
        r 
        \left(
            \hat r \times \hat r 
        \right) 
        \frac{\del}{\del r}
        +
        \left(
            \hat r \times \hat \theta
        \right)
        \frac{\del}{\del \theta}
        +
        \left(
            \hat r + \hat \phi
        \right)
        \frac{1}{\sin(\theta)}
        \frac{\del}{\del \phi}
    \right]
    \\
    =
    -i \hbar 
    \left[
        \hat \phi
        \frac{\del}{\del \theta}
        -
        \hat \theta
        \frac{1}{\sin(\theta)}
        \frac{\del}{\del \phi}
    \right]
    \hp 
    \hat \theta 
    =
    \cos(\theta)
    \cos(\phi)
    \hat i
    \hp 
    \vec r 
    =
    x \hat i 
    + 
    y \hat j
    +
    z \hat k
    \\
    L_x 
    =
    - i \hbar 
    \left[
        - \sin(\phi)
        \frac{\del}{\del \theta}
        -
        \cos(\phi)
        \cot(\theta)
        \frac{\del}{\del \phi}
    \right]
    \\
    L_y 
    =
    - i \hbar 
    \left[
        \cos(\phi)
        \frac{\del}{\del \theta}
        -
        \sin(\phi)
        \cot(\theta)
        \frac{\del}{\del \phi}
    \right]
    \hp
    L_z 
    =
    - i \hbar 
    \frac{\del}{\del \phi}
    \\
    L_{+}
    =
    L_x 
    \pm 
    i L_y 
    ====
    \pm \hbar 
    e^{\pm i \phi}
    \left(
        \frac{\del}{\del \theta}
        \pm i \cot(\theta)
        \frac{\del}{\del \phi}
    \right)
    \\
    L^2 
    =
    L_{+}
    L_{-}
    +
    L_z^2 
    -
    \hbar L_z
    =
    -\hbar^2
    \left[
        \frac{1}{\sin(\theta)}
        \frac{\del}{\del \theta}
        \left(
            \sin(\theta)
            \frac{\del}{\del \theta}
        \right)
        +
        \frac{1}{\sin^2(\theta)}
        \frac{\del^2}{\del \phi^2}
    \right]
    \\
    L_{+}
    L_{-}
    =
    - \hbar^2 
    \left(
        \frac{\del^2}{\del \theta^2}
        +
        \cot(\theta)
        \frac{\del}{\del \theta}
        +
        \cot^2(\theta)
        \frac{\del^2}{\del \phi^2}
        +
        i \frac{\del}{\del \phi}
    \right)
}

With all this shenanigans, no derivations are given, our eigenvalue equation is 
\equations{
    \hbar^2 l (l + 1) \gamma 
    =
    L^2 \gamma 
}

\subsection{Angular Equation (Y)}
\equations{
    x \sin^2(\theta) * Y 
    \hp 
    Y 
    =
    \Omega(\theta)
    \Lambda(\phi)
    \\
    \sin(\theta)
    \frac{\del}{\del \theta}
    \left(
        \sin(\theta) 
        \frac{\del Y}{\del \theta}
    \right)
    +
    \frac{\del^2 Y}{\del \phi^2}
    =
    -l(l+1) \sin^2(\theta) Y 
    \\
    =
    \left(
        \frac{1}{\Omega}
        \left[
            \sin(\theta)
            \frac{\del}{\del \theta}
            \left(
            \sin(\theta)
            \frac{\del \Omega}{\del \theta}
            \right)
        \right]
        +
        l(l+1) \sin^2(\theta)
    \right)
    +
    \frac{1}{\Lambda}
    \frac{\del^2}{\del \phi^2}
    \Lambda 
    =
    0
}

Say the first part equals $m^2$ and the second part equals $-m^2$.
\equations{
    \frac{1}{\Lambda}
    \frac{\del^2 \Lambda}{\del \phi^2}
    =
    -m^2 
    \Rightarrow 
    \Lambda(\phi)
    =
    e^{i m \phi}
    \Rightarrow 
    \Lambda(\phi)
    \overset{!}{=}
    \Lambda(\phi + 2 \pi)
    \Rightarrow 
    m 
    \in 
    \mathbb{Z}
    !!!!
}

We also figured out earlier than $m = -l, -l + 1, \ldots, l-1, l$
and $l$ doesn't have to be an integer (can be integer + 1/2).

in principle, Angular momentum allows half integer. 
but for \textbf{orbital} angular momentum, only integers 
are allowed.

\subsection{$\Omega(\theta)$ equation}
\equations{
    \sin(\theta)
    \frac{\del}{\del \theta}
    \left(
    \sin(\theta)
    \frac{\del \Omega}{\del \theta}
    \right)
    +
    \left(
        l(l+1) \sin^2(\theta)
        -
        m^2
    \right)
    \Omega
    =
    0
}

You get the Legendre Polynomials (remember from PHYS435). You will not have 
to derive it thank god. 
\equations{
    \Omega(\theta)
    =
    A P_{l}^{m}(\cos(\theta))
    \\
    P_{l}^{m}(x)
    =
    (-1)^m
    (1 - x^2)^{m/2}
    \left(
        \frac{\del}{\del x}
    \right)^m
    P_{l}(x)
    \hp 
    P_{l}(x)
    =
    \frac{1}{2^{l} l!}
    \left(
        \frac{\del}{\del x}
    \right)^l
    (x^2 - 1)^{l}
}

With normalization, you get 
\equations{
    Y_{l}^{m}
    (\theta, \phi)
    =
    \sqrt{
        \frac{
            (2l + 1)(l-m)!
        }
        {
            4 \pi (l + m)!
        }
    }
    e^{im \phi}
    P_{l}^{m}(\cos(\theta))
}

This is known as "Spherical Harmonics"
\equations{
    \bra{ Y_{l}^{m} }
    \ket{ Y_{l'}^{m'} }
    =
    \delta_{ll'}
    \delta_{mm'}
}

$l$ and $m$ have to equal each other. 

\section{Summary}
$V = V(r)$ leads to 
\equations{
    L^2 Y(\theta, \phi)
    =
    \hbar^2 l(l + 1)
    Y(\theta, \phi)
    \hp 
    L_z 
    Y(\theta, \phi)
    =
    \hbar m 
    Y(\theta, \phi)
}

Where $l$ is the total angular momentum and $m$ is something. 
Imagine the Griffiths sphere. 

\section{Discussion}
\subsection{MIDTERM 1 KNOW EVERYTHING ABOUT THE QUANTUM HARMONIC OSCILLATOR}

\subsection{3D QM}
a 1D operator is now a 3D operator. 

\equations{
    \hat p \psi 
    =
    \begin{bmatrix}
        \del_x \psi \\
        \del_y \psi \\
        \del_z \psi 
    \end{bmatrix}
    =
    3D
    \hp 
    p^2 
    =
    \hat p 
    \cdot 
    \hat p 
    =
    1D 
    \hp 
    \hat p 
    \cdot 
    \hat p 
    \neq 
    \hat p
    \hat p
}

\subsection{Degeneracy}
Different Quantum States can have the same energy. 

The degeneracy of $E_1$ is $3$ because each of $n_{x, y, z}$ can have 
$n=1$. The degeneracy of $E_2$ is 6 because of permutation shenanigans. 

\subsection{New Quantum Numbers}

$n$ = principle quantum number (energy level)

$l$ = angular momentum number. $0 < l < n$, so $n$ bounds $l$

$m$ = angular momentum $z$-component. $m \in [-l, l]$ because it is essentially 
just the $z$ projection of $\vec L$.

$|L|
=
\sqrt{l(l+1)} \hbar
$

\subsection{new notation}
$\ket{l, m}$ is a vector that tells you about your conserved angular 
momentum properties. 

\equations{
    L_{+}
    \ket{l, m}
    =
    C_{+}
    \ket{l, m+1}
    \hp
    L_{-}
    \ket{l, m}
    =
    C_{-}
    \ket{l, m-1}
}

When the angular momentum gets "raised" or "lowered", the energy level 
\textbf{does not change}, which means angular momentum magnitude also 
doesn't change. 

\subsection{Commutator Trick}
Taking the commutator of 2 operators. 
\equations{
    [\hat A, \hat B] f 
    =
    \left(
        \ldots
    \right)
    f
    \hp 
    \ldots 
    =
    [A, \hat B]
}

\textbf{ALWAYS USE A TEST FUNCTION}

\subsection{Questions}
Confirm the 3D Ehrenfest's Theorem

\equations{
    \frac{d}{dt}
    \langle \hat p \rangle 
    =
    \langle - \nabla V \rangle 
}

Where $V$ is potential is not velocity. 

I think we solved the expected value of momentum and 
\equations{
    \nabla V(r) 
    =
    \frac{1}{r}
    \sin(\theta) 
    asdfjaisfjasdfjasfadsfadksp[fkasdjfasdfnasdfnsadklfnasdfas
    defasf
    asdfsdf
    sad
    fdasdfasdfasdfasdfadfasdfasdfasdfasdfdasfasdfasdfasdfasdfasdfadfasdfasdfasdfasdfdasfasdf]
}

Show that it's all 0 and you're chilling. 

\subsection{Uncertainty Principle}
Formula Heisenberg's uncertainty principle, showing that there actually 
is not a correlation between dimensions. 

You jus yoi jsut do ou tuju standing

\subsection{$\ket{l, m}$}
Show that $L_{+} \ket{l, m}$ is an eigenstate of $L_z$ with eigenvalue 
$\hbar (m+1)$.

\equations{
    L^2 
    L_{+} \ket{l, m}
    =
    L^2 
    \left(
        L_x 
        +
        i L_y
    \right)
    \ket{l, m}
}



\section{Recap}

You do a bunch of absolutely horrendous spherical coordinates math to 
get a radial equation and an angular equation. 
\equations{
    L^2 Y(\theta, \phi)
    =
    \hbar l (l + 1) Y(\theta, \phi)
}

\section{Solving the Radial Equation}

Let $u(r) = r R(r)$, so then our equation can be written as 
\equations{
    \frac{dR}{dr} 
    =
    \left(
        r
        \left(
            \frac{du}{dr}
        \right)
        -
        u
    \right)
    \frac{1}{r^2}
    \Rightarrow 
    \\
    - \frac{\hbar^2}{2m}
    \frac{\del^2 u}{\del r^2}
    +
    \left[
        V 
        +
        \frac{\hbar^2}{2m}
        \frac{l(l+1)}{r^2}
    \right]
    u
    =
    Eu
}

We can reduce the whole potential to just a $V_{\textrm{effective}}$

\subsection{Spherical Infinite Square Well}
\equations{
    V 
    =
    \begin{cases}
        0 : r \leq a
        \\
        \infty : r > a
    \end{cases}
}

Inside the well, the SWE can be written as 
\equations{
    \frac{d^2 u}{dr^2}
    =
    \left[
        \frac{l(l+1)}{r^2}
        -
        k^2
    \right]
    u
    \hp 
    k 
    =
    \frac{\sqrt{2mE}}{\hbar}
}
With boundary condition $u(a) = 0$

for $l=0$ 
\equations{
    \frac{d^2 u}{dr^2}
    =
    k^2
    u
    \Rightarrow 
    u 
    =
    A \sin(kr)
    +
    B \cos(kr)
    \Rightarrow
    A \sin(kr)
    \hp 
    ka 
    =
    \pi N
    \\
    R 
    =
    A \frac{\sin(kr)}{r}
    +
    B \frac{\cos(kr)}{r}
    \Rightarrow 
    A \frac{\sin(kr)}{r}
}
The cosine disappears because the limit of cos(r)/r means that $B=0$

\subsection{$l > 0$}
Look in the big book of functions and see that spherical harmonic functions 
solve this equation 
\equations{
    u(r)
    =
    A \cdot r j_l(kr)
    +
    B r u_l(kr)
    \\
    j_l(x)
    =
    (-x)^{l}
    \left(
        \frac{1}{x}
        \frac{d}{dx}
    \right)^{l}
    \frac{\sin(x)}{x}
    =
    \textrm{Bessel Function}
    \\
    u_l(x) 
    =
    \textrm{Spherical Neumann Function}
}

Bessel functions just look like a damped oscillator 

The entire wave function looks like 
\equations{
    \psi(r, \theta, \phi)
    =
    A_{nl}
    j_l
    \left(
        B
    \right)
    asfasdkjlfjadfghadflkhadsj
}

It looks like a bunch of step functions but with slopes between them. 

$n$ is the principle quantum number (indexes energy)

\section{MIDTERM}
Everything up to HW9 - Discussion 8

Don't worry about spherical harmonics that much. Know what's up 
with the angular momentum operator. Know $\hat L$, but not $Y(\theta, \phi)$.

The harmonic oscillator will be the most important part of the midterm. 




\section{The Hydrogen Atom}
You have a coulomb potential $V(r) = \frac{-e^2}{4 \pi \epsilon_0} \frac{1}{r}$

The radial equation is of the form ($R = u / r$)
\equations{
    -
    \frac{\hbar^2}{2m}
    \frac{\del u^2}{\del r^2}
    +
    \left[
        -
        \frac{e^2}{4 \pi \epsilon_0}
        \frac{1}{r}
        +
        \frac{\hbar^2}{2m_e}
        \frac{l(l+1)}{r^2}
    \right]
    u 
    =
    E u 
    \hp 
    \psi 
    =
    R(r) Y_{l}^{m}(\theta, \phi)
}
You get a nice central potential with a central potential well 
like we saw in classical mechanics and E\&M a million times over. Everything 
in $[]$ is the effective potential, or $V_{eff}$

Let's clean up the function 
\equations{
    \kappa 
    \equiv 
    \frac{\sqrt{-2 m_e E}}{\hbar}
    \Rightarrow 
    \frac{1}{\kappa^2}
    \frac{d^2 u}{d r^2}
    =
    \left[
        1 
        -
        \frac{m_e e^2}{2 \pi \epsilon_0 \hbar^2 \kappa}
        \frac{1}{\kappa r}
        +
        \frac{l (l + 1)}{(\kappa r^2)}
    \right]
    \kappa
    \\
    \rho = \kappa \cdot r 
    \hp 
    \rho_0 
    =
    \frac{m_e e^2}{2 \pi \epsilon_0 \hbar^2 \kappa}
    \Rightarrow 
    \frac{d^2 u}{d \rho^2}
    =
    \left[
        1 
        -
        \frac{\rho_0}{\rho}
        +
        \frac{l (l + 1)}{\rho^2}
    \right]
    u
}

This is absurd to solve analytically so we can instead observe the behavior at the 
edges. 

\subsection{Asymptotic Behavior}
\equations{
    \lim_{\rho \to \infty}
    \Rightarrow 
    \frac{d^2 u}{d \rho^2}
    =
    u 
    \Rightarrow 
    u(\rho)
    =
    A e^{-\rho}
    +
    B e^{\rho}
    =
    A e^{-\rho}
    \\
    \lim_{\rho \to 0}
    \Rightarrow 
    \frac{d^2 u}{d \rho^2}
    =
    \frac{l(l + 1)}{\rho^2}
    u
    \Rightarrow 
    u(\rho)
    =
    C \rho^{l+1}
    +
    D \rho^{-l}
    =
    C \rho^{l+1}
}
Remove terms that lead to the equation being non-normalizable. 

This means that our ansatz given the asymptotes can be written as 
\equations{
    u(\rho)
    =
    \rho^{l+1}
    e^{-\rho}
    v(\rho)
    \hp
    \frac{d u}{d \rho}
    =
    \rho^{l}
    e^{-\rho}
    \left(
    (l + 1 - \rho)
    v(\rho)
    +
    \rho \frac{dv}{d \rho}
    \right)
    \\
    \frac{d^2 u}{d \rho^2}
    =
    \rho^{l}
    e^{-\rho}
    \left[
        \left(
            -2l - 2 + \rho + \frac{l(l+1)}{\rho}
        \right)
        v(\rho)
        +
        2(l + 1 - \rho)
        \frac{dv}{d \rho}
        +
        \rho
        \frac{d^2 v}{d \rho^2}
    \right]
}

The way to solve this is by finding a recursive formula such that 
\equations{
    \rho
    \frac{d^2 v}{d \rho^2}
    +
    \rho^{l}
    e^{-\rho}
    \left[
        \left(
            -2l - 2 + \rho + \frac{l(l+1)}{\rho}
        \right)
        v(\rho)
        +
        2(l + 1 - \rho)
        \frac{dv}{d \rho}
    \right]
    =
    0
    \\
    jfaklsdjfksadlfasklfklasdjflska;jfaldskjflksda
}

Develop $v(\rho)$ as a power series 
\equations{
    v(\rho)
    =
    \sum_{j=0}^{\infty}
    c_{j} \rho^{j}
    \hp 
    \frac{d v(\rho)}{d \rho}
    =
    \sum_{j=0}^{\infty}
    j
    c_j \rho^{j-1}
    =
    \sum_{j=0}^{\infty}
    (j+1)
    c_{j+1} \rho^{j}
    \\
    \frac{d^2 v}{d \rho^2}
    =
    \sum_{j=0}^{\infty}
    j(j+1)
    c_{j+1} \rho^{j-1}
}

Insert that into the original equation to get 
\equations{
    \sum_{j=0}^{\infty}
    j(j+1)
    c_{j+1} \rho^{j}
    +
    2(l+1)
    \sum_{j=0}^{\infty}
    (j+1)
    c_{j+1} \rho^{j}
    \\
    -
    2
    \sum_{j=0}^{\infty}
    j c_{j} \rho^{j}
    +
    \left[
        \rho_0 
        -
        2(l+1)
    \right]
    \sum_{j=0}^{\infty}
    c_{j} \rho^{j}
    =
    0
}

Now that we have this giant gross equation, it has to be true that 
we get an equation for \textbf{each} power of $\rho$
\equations{
    j(j+1)
    c_{j+1}
    +
    2(l+1)
    (j+1)
    c_{j+1}
    -
    2
    j c_{j}
    +
    \left[
        \rho_0 
        -
        2(l+1)
    \right]
    \sum_{j=0}^{\infty}
    c_{j}
    =
    0
}

We now have a relation of $c_{j+1}$ to $c_j$
\equations{
    c_{j+1}
    =
    \frac{2(j + l + 1) - \rho_0}{(j+1)(j + 2l + 2)}
    c_j
}

Now we have to find some end to the series. 
\equations{
    \textrm{large } \rho 
    \rightarrow 
    u(\rho)
    =
    e^{-\rho}
    \hp
    \textrm{large } j 
    \rightarrow 
    c_{j+1}
    \approx 
    \frac{2}{j+1}
    c_j
    \Rightarrow 
    c_j 
    =
    \frac{2j}{j!} c_0
}

But we know this cant be true for all $j$ because it leads to an incorrect answer
\equations{
    v(\rho)
    =
    c_0 
    \sum_j 
    \frac{2j}{j!}
    \rho^j 
    =
    c_0 e^{2 \rho}
    \Rightarrow 
    u(\rho)
    =
    c_0 \rho^{l+1}
    e^{\rho}
    \neq
    e^{-\rho}
}

\subsection{I zoned out whoops}
The series has to terminate somewhere 
\equations{
    c_{N-1} \neq 0 : c_{N} = 0
    \\
    2(N+ l )
    -
    \rho_0
    =
    0
    \hp 
    N = n + l
    \\
    \rho_0 = 2n 
    \Rightarrow 
    E 
    =
    - \frac{\hbar^2 \kappa^2}{2m_e}
    =
    - \frac{m_e e^4}{8 \pi^2 \epsilon_0^2 \hbar^2 \rho_0^2}
    \hp 
    E_n 
    =
    -
    \left[
        \frac{m_e}{2 \hbar^2}
        \left(
            \frac{e^2}{4 \pi \epsilon_0}
        \right)^2
    \right]
    \frac{1}{n^2}
    =
    \frac{E_1}{n^2}
    \\
    E_1 
    =
    -13.6 meV
}
Bohr actually found this result (Bohr Model 1913) 
without quantum mechanics which is kind of insane if you think about it. 

$N$ control the series termination and it controls the amount of nodes 
in the radial equation (similar to in other models with $n$).

If we observe earlier definitions 
\equations{
    \kappa 
    =
    \left(
    \frac{m_e e^2}{4 \pi \epsilon_0 \hbar^2}
    \right)
    \frac{1}{n}
    =
    \frac{1}{an}
    \hp 
    a 
    \equiv
    \frac{4 \pi \epsilon_0 \hbar^2}{m_e e^2}
    =
    0.53 \cdot 10^{-10} m
    \\
    \rho 
    =
    \frac{r}{an}
    \hp 
    \psi_{nlm}(r, \theta, \phi)
    =
    R_{nl}(r) Y_{l}^{m} (\theta, \phi)
    \hp 
    R_{nl}
    =
    \frac{1}{r}
    \rho^{l+1}
    e^{-\rho}
    v(\rho)
}
Where $v$ is a polynomial of degree $(n-l-1)$ and $c_j$ is known from 
the recursion formula. 

The wave function now fits in the neat potential well who would 
have guessed 
\equations{
    \psi_{1, 0, 0}
    =
    R_{1, 0}
    (r)
    Y_{0}^{0}
    (\theta, \phi)
    \hp 
    R_{1, 0}(r)
    =
    \frac{c_0}{a}
    e^{-r/a}
    \hp 
    v(\rho)
    =
    c_0 
    =
    \const
    \\
    \int^{\infty}_{0} dr \, 
    |R_{1, 0}|^2 r^2
    =
    1
    \rightarrow 
    c_0 
    =
    \frac{2}{\sqrt{a}}
    Y_{0}^{0}
    =
    \frac{1}{\sqrt{4\pi}}
    \\
    \psi_{1, 0, 0}(r, \theta, \phi)
    =
    \frac{1}{\sqrt{4 a^3}} e^{-r/a}
}
$v(\rho) = c_0$ is found from the recursive formula. 

\subsection{Importance}
We now have a nice graph of energy levels related to quantum number that 
I have definitely seen before. 

As $N$ increases, degeneracy also goes up. 
\equations{
    d(n)
    =
    \sum_{l=0}^{n-1}
    (2l+1)
    =
    n^2
}

If I add anything more than 13.6meV when its in the lowest state the hydrogen 
atom will just be ionized. 

This is as far as you can get with analytical math in quantum mechanics. 
Everything else involves approximations and computation.

\section{Discussion}
\subsection{Potentials You Know}
\begin{itemize}
\item
free particle = 0 
\item
infinite square is 0 or infinite 
\item
oscillator is $\frac{1}{2} m \omega^2 x^2$ 
\item
Now we study the hydrogen atom $V = \frac{1}{r}$
\end{itemize}

The SWE can be solved for the H atom and nothing else 
\equations{
    \hat H 
    =
    - \frac{\hbar^2}{2 \mu}
    \nabla^2
    -
    \frac{Z e^2}{4 \pi \epsilon_0 r}
}

And it has solution that can be derived from separation of variables 
\equations{
    \psi_{nlm}(r, \theta, \phi)
    =
    R_{n}(r) Y_{l, m}(\theta, \phi)
}

$m$ gives us the z-projection of the angular momentum quantum number 
which is given to us by $l$. $n$ gives us energy, and $n$ is independent 
from $l$ and $m$. 

\subsection{Normalizing in Space}
\equations{
    1
    =
    \int^{\infty}_{0} dr \,
    |R(r)|^2 r^2
    \int^{2 \pi}_{0}
    \int^{\pi}_{0} \, d \theta \, d \phi
    |Y(\theta, \phi)|^2
    \sin(\theta)
}

\subsection{Spherical Harmonics}
These are a class of well behaved functions that define our eigenstates 
\equations{
    \psi_{nlm}
    =
    \ket{n, m, l}
}

And $l$ and $m$ are coupled. $m$ has bounds $\pm l$ in unit steps. 

\subsection{Chemical Orbitals}
This is the consequence of spherical harmonics in quantum shenanigans. 

The orbital images are surfaces of equal probability density. 
(Technically, electrons are in a superposition of a bunch of the lowest states, 
but if we observe all of them, we get the chemists POV where they take 
up the lowest states in a neat ordered shelf).

\subsection{New Eigenvalues}
We have $\hat L_z$ operator, which projects the angular momentum to the z-axis. 
\equations{
    \hat L_z \ket{l, m}
    =
    \hbar m \ket{l, m}
    \hp 
    L^2 \ket{l, m}
    =
    l(l+1) \ket{l, m}
    \hp 
    Y_{l}^{m}
}
L is on the bottom and $m$ is on the top. 

\subsection{Questions}
\equations{
    -\frac{\hbar^2}{2m} \nabla^2 \psi 
    + 
    \frac{1}{2} m \omega^2 r^2 \psi 
    =
    E \psi 
    \Rightarrow 
    \\
    -\frac{\hbar^2}{2m}
    \left(
        \frac{\del^2}{\del x^2}
        +
        \frac{\del^2}{\del y^2}
        +
        \frac{\del^2}{\del z^2}
    \right)
    + 
    \frac{1}{2} m \omega^2 (x^2 + y^2 + z^2) \psi 
    =
    E \psi 
    \Rightarrow 
    \\
    -
    \frac{\hbar^2}{2m}
        \frac{\del^2}{\del x^2}
    -
    \frac{\hbar^2}{2m}
        \frac{\del^2}{\del y^2}
    -
    \frac{\hbar^2}{2m}
        \frac{\del^2}{\del z^2}
    \\
    =
    \frac{1}{2} m \omega^2 (x^2 + E_x) 
     \psi 
    + 
    \frac{1}{2} m \omega^2 (y^2 + E_y)  \psi 
    + 
    \frac{1}{2} m \omega^2 (z^2 + E_z) \psi 
    \\
    -
    \frac{\hbar^2}{2m}
        \frac{\del^2}{\del x^2}
        =
    \frac{1}{2} m \omega^2 (x^2 + E_x) 
    \Rightarrow 
    \frac{\del^2}{\del x^2}
    =
    -\frac{1}{2} m \omega^2 \frac{2m}{\hbar^2} (x^2 + E_x) 
}

This is just the 1d harmonic oscillator 3 times. 

Find the degeneracy 
\equations{
    E 
    =
    \hbar \omega 
    \left(
        n_x 
        +
        n_y 
        +
        n_z
        +
        \frac{3}{2}
    \right)
}

The degeneracy you just get from stars and bars. 

Show that $L_x$ is conserved. 
\equations{
    [\hat H, L_x]
    =
    [\hat H, y p_x]
    +
    [\hat H, x p_x]
}

The previous part implied that the eigenstates of the 3d SHO. 

$L_x$ and $L_y$ are different and not gonna also be eigenstates of $L_z$ 

\subsection{Other Question}
You have a rigid rotor in a magnetic field

If energy is measured, what are the possible results? 

\equations{
    \frac{L^2}{2I}
    \psi 
    +
    \omega_0 L_z \psi 
    =
    E
    \phi 
    \hp 
    L^2 \psi 
    =
    \hbar^2 l (l+1) \psi 
    \hp 
    L_z \psi 
    =
    \hbar m \psi 
    \\
    \frac{\hbar^2 l(l+1)}{2I}
    \psi 
    +
    \omega_0 \hbar m 
    \psi 
    =
    E
    \psi 
    \\
    \sqrt{\frac{3}{4 \pi}} 
    \sin(\theta) \sin(\phi)
    =
    \sqrt{\frac{3}{4 \pi}} 
    \sin(\theta)
    \left(
        \frac{e^{i \psi} - e^{-i \psi}}{2i}
    \right)
}

You turn the thing into 2 eigenstates. 

\section{MIDTERM 2 REVIEW}

Commutators $[\hat A, \hat B] = 0$ can be simultaneously diagonalized. 

But there is a certain amount of uncertainty in that. 
\equations{
    \sigma_A^2 \sigma_B^2 
    \geq
    \left(
    \frac{
    \langle 
    [\hat A, \hat B]
    \rangle
    }{2i}
    \right)^2
}

A quantity is conserved if $[\hat H, \hat A] = 0$

The generalized Ehrenfest Theorem is 
\equations{
    \frac{d}{dt}
    \langle A \rangle 
    =
    [\hat H, \hat A]
    +
    \langle 
    \frac{d A}{dt}
    \rangle 
}
Where the 2nd term is typically 0. The G.E.T. also leads to energy-time uncertainty
$(\Delta E \Delta t)$.

Basically, given some observable, you'll probably have to figure out the uncertainty 
and if the observable is conserved. 

\subsection{Simple Harmonic Oscillator} 
Just a wave function with a potential $V(x) = \frac{1}{2} m \omega^2 x^2$ 
\equations{
    H
    =
    \hbar \omega 
    \left(
    a_{+}
    a_{-}
    +
    \frac{1}{2}
    \right)
    =
    -
    \frac{\hbar^2}{2m}
    \frac{\del^2}{\del x^2}
    +
    \frac{1}{2}
    m \omega^2 x^2 
    \hp 
    [ a_{-}, a_{+} ]
    =
    1
    \\
    a_{-} 0 = 0
    \hp 
    a_{+} \ket{n} = \sqrt{n+1} \ket{n+1}
    \hp
    a_{-} \ket{n} = \sqrt{n} \ket{n-1}
    \hp 
    \bra{n} \hat A \ket{n}
    =
    \hbar \omega 
    \left(
    n + \frac{1}{2}
    \right)
}

The stationary states are some sort of special polynomial with a fancy name, but that 
won't come up on the midterm. 

You give us an observable made of +'s and -'s and a wave function made of stationary 
states and you have to calculate something. There is also time evolution but the 
time evolution is very simple. 

\subsection{Angular Momentum} 
Given some angular momentum $l$, there is basically a ladder for the quantum number $m$ 
that spans from $+l$ to $-l$.
\equations{
    L_{\pm}
    =
    L_x 
    +
    iL_y
}
I'm not 100$\%$ sure that this is true, double check the angulular momentum math. 

For the radial equation/3D harmonic oscillator, just solve for each dimension.

\subsection{Know 3D QM} 

\subsection{Know Coherent States}

\subsection{Know Pauli Matrices}

\chapter{Spin}
Classically, orbiting things have angular momentum $\vec L = \vec r \times \vec p$
and spinning things also have that $\vec s = \vec I \cdot \vec \omega$

Quantum mechanical particles have integer spin
\begin{itemize}
    \item
    Boson: integer spin (photons have s=1)
    \item
    Fermions: Integer + 1/2
\end{itemize}

Spin is comparable to orbital angular momentum 
\equations{
    \vec s 
    =
    \begin{bmatrix}
        s_x \\ s_y \\ s_z
    \end{bmatrix}
    \hp
    [\hat s_i, \hat s_j]
    =
    i \hbar \hat s_k
}

And we have ladder operators for spin the same way as angular momentum 
\equations{
    \hat s^2 \ket{s, m}
    =
    \hbar^2 s(s+1) \ket{s, m}
    \hp 
    \hat s_z =
     asfjklagjkoa;djfklsdajf;
}

\section{Spin 1/2}
The eigenstates of $\hat s_z$ are known as spinors 
\equations{
    \ket{0}
    \equiv
    \chi_{+}
    \equiv
    \begin{bmatrix}
        1 \\ 0
    \end{bmatrix}
    \equiv
    \ket{\frac{1}{2}, \frac{1}{2}}
    \equiv
    \ket{\uparrow}
    \hp
    \ket{1}
    \equiv
    \chi_{-}
    \equiv
    \begin{bmatrix}
        0 \\ 1
    \end{bmatrix}
    \equiv
    \ket{\frac{1}{2}, \frac{1}{2}}
    \equiv
    \ket{\downarrow}
}

And the eigenvalues of these states can be written as 
\equations{
    \hat s^2 \chi_{+}
    =
    \frac{3}{4} \hbar^2 \chi_{+}
    \hp
    \hat s^2 \chi_{-}
    =
    \frac{3}{4} \hbar^2 \chi_{-}
    \\
    \hat s^2 
    =
    \frac{3}{4} \hbar^2 
    \begin{bmatrix}
        1 & 0 \\ 0 & 1
    \end{bmatrix}
    =
    \frac{3}{4} \hbar^2 
    \left(
        \ket{0}
        \bra{0}
        +
        \ket{1}
        \bra{1}
    \right)
    \\
    \hat s_z \chi_{+}
    =
    \frac{\hbar}{2}
    \chi_{+}
    \hp
    \hat s_z \chi_{-}
    =
    -
    \frac{\hbar}{2}
    \chi_{-}
}

\subsection{Time Evolution}
$\mu$ rotates under an external magnetic field $\vec B$ 

There's a gyromagnetic ratio $\vec \mu = \gamma \cdot \vec L$. So spin in 
a magnetic field experiences: 
\equations{
    \dot{\vec L}
    =
    \vec \mu \times \vec B
    \Rightarrow 
    \dot{\vec \mu} 
    =
    \gamma \vec \mu \times \vec B
}

\section{Hamiltonian}
This can be derived with some shenanigans 
\equations{
    H 
    =
    \vec \mu 
    \cdot 
    \vec B
    \hp 
    \vec \mu 
    =
    \gamma \vec s
}

\subsection{Example}
\equations{
    \vec B = B_0 \hat z 
    \hp 
    \hat H 
    =
    - \gamma \vec B \cdot \vec s
    =
    - 
    \frac{\gamma B_0 \hbar}{2} \hat \sigma_z
}
$\sigma$ is a pauli matrix

idk something something I'm texting my family 

At $t=0$ 
\equations{
    \alpha \ket{0} 
    +
    \beta \ket{1} 
    =
    \cos(\frac{\theta}{2}) 
    \ket{0}
    +
    \sin(\frac{\theta}{2}) 
    e^{i \phi}
    \ket{1}
    \\
    \chi(t)
    =
    \alpha \chi_{+}
    e^{-i \frac{E_{+}}{\hbar} t}
    +
    \beta \chi_{-}
    e^{-i \frac{E_{-}}{\hbar} t}
    \\
    \langle s_x \rangle 
    =
    \bra{\chi(t)} s_x \ket{\chi(t)}
    =
    \frac{\hbar}{2} 
    \sin(\theta) \cos(\gamma B_0 t)
}

$\omega = \gamma B_0$ is the Larmor frequency or something.

\subsection{Bloch Spheres}
They can be used to represent spin but I can't draw in this notebook.

Magnetic resonance is just making spin states go from $\ket{\uparrow}$ 
to $\ket{\downarrow}$

\section{Recap}
Spin is a vector with a commutator that acts like Pauli matrices and there's 
also a ladder function that goes with it 
\equations{
    \hat s^2 \ket{s, m}
    =
    \hbar^2 s(s+1) \ket{s, m}
    \hp 
    \hat s_z \ket{s, m}
    =
    \hbar m \ket{s, m}
    \\
    s_{\pm} \ket{s, m}
    =
    \hbar 
    \sqrt{s(s+1) - m(m \pm 1)} \ket{s, m \pm 1}
}

For spin 1/2 particles, the operator is 
\equations{
    \vec s 
    =
    \frac{\hbar}{2} \vec \sigma 
    =
    \frac{\hbar}{2} \begin{bmatrix} \sigma_x \\ \sigma_y \\ \sigma_z \end{bmatrix}
}

The general state of a spin particle can be written as 
\equations{
    \chi 
    =
    \alpha \chi_{+}
    +
    \beta \chi_{-}
}

\section{1/2 Spin in B Field}
\equations{
    \hat H 
    =
    - \gamma \vec B \cdot \vec s 
    =
    - \frac{\gamma B_0 \hbar}{2}
    \hat \sigma_z
}

\section{Stern-Gerlach}
\equations{
    \vec F 
    =
    \vec \nabla 
    (\vec \mu \cdot \vec B)
    \hp
    \chi 
    =
    \frac{1}{\sqrt{2}}
    (
        \ket{\uparrow}
        +
        \ket{\downarrow}
    )
}

You just use a magnetic field, and when the particle is the wrong spin it gets 
pushed away. 

\equations{
    \ket{\pm}
    =
    \frac{1}{\sqrt{2}}
    (
        \ket{\uparrow}
        \pm
        \ket{\downarrow}
    )
}

$\ket{x, +}$ means that 1st one is direction, and the 2nd is spin. 

\subsection{Angular Momentum}
Consider 2 separate spin 1/2 particles. 
We have new operators 
\equations{
    \hat S_{x, y, z}^{(j)}
    \hp 
    \hat S^{(j)^2}
}

You can take the composite of 2 states 
\equations{
    \ket{s_1, s_2, m_1, m_2}
    \equiv 
    \ket{s_1, m_1} 
    \otimes 
    \ket{s_2, m_2}
}
Where $\otimes$ is the kronecker product. 

You take all the operators of all the things and get either $s(s+1)$ or $\hbar m$.

To measure the total angular momentum, you use a new operator 
\equations{
    \hat s 
    =
    \hat s^{(1)}
    +
    \hat s^{(2)}
}

What happens if we measure $\hat s_z$? 
\equations{
    \hat s_z 
    \ket{s_1, s_2, m_1, m_2}
    =
    s_z^{(1)}
    \ket{s_1, s_2, m_1, m_2}
    +
    s_z^{(2)}
    \ket{s_1, s_2, m_1, m_2}
    \\
    =
    \hbar(m_1 + m_2)
    \ket{s_1, s_2, m_1, m_2}
    \hp
    \ket{\uparrow \uparrow}
    \ket{\frac{1}{2}, \frac{1}{2},\frac{1}{2},\frac{1}{2}}
    \rightarrow
    m=1
    \\
    \ket{\uparrow \downarrow}
    \rightarrow m=0
    \hp
    \ket{\downarrow \uparrow}
    \rightarrow m=0
    \hp 
    \ket{\downarrow \downarrow}
    \rightarrow m=-1
}

This doesn't work because we now have 2 separate states with the same quantum 
number. 

\subsection{Joint Ladder Operators}
the top state is $\ket{\uparrow \uparrow}$, and if we apply the lowering 
operator to it, we get 
\equations{
    s_{-}
    \ket{\uparrow \uparrow}
    =
    \left(
    s_{-}^{(1)}
    \ket{\uparrow}
    \right)
    \ket{\uparrow}
    +
    \ket{\uparrow}
    \left(
    s_{-}^{(2)}
    \ket{\uparrow}
    \right)
    =
    \hbar 
    \ket{\downarrow}
    \ket{\uparrow}
    +
    \hbar 
    \ket{\uparrow}
    \ket{\downarrow}
    \\
    s_{-} \ket{10}
    =
    2 \hbar \ket{\downarrow \downarrow} \equiv \ket{1, -1}
}

And then you normalize it 
\equations{
    s = 1: 
    \begin{cases}
        dfhdaf;jfjkadsljflsdj
    \end{cases}
}

We have a missing basis vector that is 
\equations{
    \frac{1}{\sqrt{2}} 
    \left(
        \ket{\uparrow \downarrow}
        -
        \ket{\downarrow \uparrow}
    \right)
    \hp 
    s = 0 
    \hp 
    m=0
    \\
    \hat s^2 
    \frac{1}{\sqrt{2}} 
    \left(
        \ket{\uparrow \downarrow}
        -
        \ket{\downarrow \uparrow}
    \right)
    =
    ? 
    \\
    \hat s^2 
    =
    s^{(1)^2}
    +
    s^{(2)^2}
    +
    2
    s^{(1)}
    s^{(2)}
    \\
    s^{(1)^2}
    \ket{\psi_[-]}
    =nadfjkladjfkadsnfjkls
    \\
    s^{(1)}
    \cdot
    s^{(2)}
    \ket{\psi_{-}}
    =
    -\frac{3 \hbar^2}{\psi}
    \ket{\psi_{-}}
    \\
    \hat s^2 \ket{\psi} = 0 
    \Rightarrow 
    \frac{1}{\sqrt{2}} 
    \left(
        \ket{\uparrow \downarrow}
        -
        \ket{\downarrow \uparrow}
    \right)
    =
    \ket{00}
}

\section{Discussion}
Write down the matrices for $S_{+}$ and $S_{-}$. Write 

\section{Identical Particles}
a 2 electron system is not the state as 2 combined 1-electron systems. 

The hamiltonian of the system is 
\equations{
    H 
    =
    - \frac{\hbar^2}{2m_1}
    \nabla^2_1
    - 
    \frac{\hbar^2}{2m_2}
    \nabla^2_2
    +
    V(\vec r_1, \vec r_2)
}

We can use separation of variables 
\equations{
    V(\vec r_1, \vec r_2)
    =
    V_1(\vec r_1)
    +
    V_2(\vec r_2)
}

Now we can make an ansatz for the wave function 
\equations{
    \psi(\vec r_1, \vec r_2)
    =
    \psi_a(\vec r_1) \psi_b(\vec r_2)
    \hp
    H_{1, 2}
    =
    - \frac{\hbar^2}{2m_{1/2}}
    \nabla_{1/ 2}^2
    +
    V_{1/2}(\vec r_{1/2})
    \\
    H_{1/2} \psi_{a//b} 
    =
    E_{a/b} \psi_{a/b} 
    H_{tot}
    =
    H_{1}
    +
    H_{2}
    \hp 
    E_{tot}
    =
    E_a
    +
    E_b
    \\
    \psi
    =
    \psi_{a}
    \psi_{b}
    =
    e^{-i E_a/\hbar t}
    e^{-i E_b /\hbar t}
}

We can have solutions that are linear combinations 
\equations{
    \psi
    =
    A 
    \psi_a(\vec r_1) \psi_c(\vec r_2)
    +
    B
    \psi_b(\vec r_1) \psi_d(\vec r_2)
}

\section{Bosons are Fermions}
There are two ways to do something in 3D 
\equations{
    \psi_{\pm}(\vec r_1, \vec r_2)
    =
    A
    \left(
    \psi_a(\vec r_1)
    \psi_b(\vec r_2)
    \pm
    \psi_a(\vec r_2)
    \psi_b(\vec r_1)
    \right)
}
+ is Bosons and - is Fermions.

\subsection{Particle Exchange}
Bosons are symmetric and need exchange and Fermions are anti-symmetric 
\equations{
    \psi_{+}(\vec r_1, \vec r_2)
    =
    \psi_{+}(\vec r_2, \vec r_1)
    \hp
    \psi_{-}(\vec r_1, \vec r_2)
    =
    -\psi_{-}(\vec r_2, \vec r_1)
}

Consider 2 particles in the same state 
\equations{
    \psi_{+}(\vec r_1, \vec r_2)
    =
    \left(
    \psi_a(\vec r_1)
    \psi_a(\vec r_2)
    +
    \psi_a(\vec r_2)
    \psi_a(\vec r_1)
    \right)
    =
    2A
    \psi_a(\vec r_2)
    \psi_a(\vec r_1)
    \\
    \psi_{-}(\vec r_1, \vec r_2)
    =
    \left(
    \psi_a(\vec r_1)
    \psi_a(\vec r_2)
    -
    \psi_a(\vec r_2)
    \psi_a(\vec r_1)
    \right)
    =
    0
}

Pauli exclusion principle! Fermions can't be in the same quantum state. 

\section{2 Particles in a Well}
If we imagine indistinguishable bosons, we take the sum of the energies 
from each particle being in each state. 

If we imagine 2 indistinguishable Fermions. 
\equations{
    \psi_{12}
    =
    \psi_a(x_1) 
    \psi_b(x_2) 
}

But if the fermions/bosons are distinguishable, then 
\equations{
    \psi_{12}
    =
    A
    \left(
    \psi_a(x_1)
    \psi_b(x_2)
    \pm
    \psi_b(x_1)
    \psi_a(x_2)
    \right)
}

Consider spin, the total wave is 
\equations{
    \psi 
    =
    \psi(x) \chi 
    \hp 
    \chi 
    =
    \textrm{spinor}
}

For bosons:
\equations{
    \psi_{+}(\vec r_1, \vec r_2)
    =
    \psi_{+}(\vec r_2, \vec r_1)
}

For Fermions
\equations{
    \psi_{-}(\vec r_1, \vec r_2)
    =
    -\psi_{-}(\vec r_2, \vec r_1)
}

We can do some math shenanigans with Bosons
\equations{
    \psi(x_1, x_2)
    =
    \psi_a(x_1)
    \psi_b(x_2)
    =
    \ket{a}
    \ket{b}
    \Rightarrow 
    \psi_B
    =
    \frac{1}{\sqrt{2}}
    \left(
    \ket{a} 
    \ket{b} 
    +
    \ket{a} 
    \ket{b} 
    \right)
}

and Fermions 
\equations{
    \psi_F
    =
    \frac{1}{\sqrt{2}}
    \left(
    \ket{a} 
    \ket{b} 
    +
    \ket{a} 
    \ket{b} 
    \right)
}

We can consider a distance operator for distinguishable particles
\equations{
    \langle d^2 \rangle 
    =
    \langle (x_1 - x_2)^2 \rangle 
    =
    \langle x_1^2 \rangle
    +
    \langle x_2^2 \rangle
    +
    \langle x_1 x_2 \rangle
    \\
    \langle x_1^2 \rangle
    =
    \bra{a} x_1^2 \ket{a}
    \bra{b} \ket{b}
    =
    \langle x^2 \rangle_a
    \hp 
    \langle x_2^2 \rangle
    ===
    \langle x^2 \rangle_b
    \\
    \langle x_1 x_2 \rangle 
    =
    \bra{a} x_1 \ket{a}
    \bra{b} x_2 \ket{b}
    =
    \langle x \rangle_a
    \langle x \rangle_b
}

And for Bosons/Fermions, we can calculate the same thing 
\equations{
    \langle x_1^2 \rangle 
    =
    \frac{1}{2}
    \left[
	\bra{a} x_1^2 \ket{a} 
	\bra{b} \ket{b}
	+
	\bra{b} x_1^2 \ket{b} 
	\bra{a} \ket{a}
	\pm
	\bra{a} x_1^2 \ket{b} 
	\bra{b} \ket{a}
	\pm
	\bra{b} x_1^2 \ket{a} 
	\bra{a} \ket{b}
    \right]
    \\
    =
    \frac{1}{2}
    \left[
	\langle x^2 \rangle_a
	+
	\langle x^2 \rangle_b
    \right]
    =
    \langle x_2^2 \rangle
    \\
    \langle x_1 x_2 \rangle 
    ===
    \langle x \rangle_a
    \langle x \rangle_b
    \pm
    |\langle x \rangle_{ab} |^2
    \hp 
    \langle x \rangle_{ab}
    =
    \bra{a} x \ket{b} 
    \\
    \langle d^2 \rangle 
    =
    \langle x^2 \rangle_a 
    +
    \langle x^2 \rangle_b 
    -
    2
    \langle x \rangle_a
    \langle x \rangle_b
    \mp 
    2 |\langle x \rangle_{ab} |^{2}
}

This whole phenomenon is known as the Exchange Force 

\subsection{that's the end of final material, only 1 lecture left}

\section{DISCUSSION}
Imagine a particle with spin. If spin=1, then there are 3 possible spin states. 

Spin 1/2 is up and spin -1/2 is down.

For each spin value $S$, we have 2S + 1 spin states.

Consider an in state with up and down and out state with up and down. This gives us 
4 possible input and output combinations, which gives us a spin matrix with 4 entries. 

Consider 2 particles, each with an input and output, each with 2 possible states up 
and down. 

Now we work in a new basis where 
\equations{
    \ket{1}
    =
    \ket{\uparrow \uparrow}
    \hp 
    \ket{2}
    =
    \ket{\uparrow \downarrow}
    \hp
    \ket{3}
    =
    \ket{\downarrow \uparrow}
    \hp
    \ket{4}
    =
    \ket{\downarrow \downarrow}
    \\
    S_{ij}
    =
    \bra{i} S \ket{j}
}

This then means that $S$ is a 4x4 matrix with 16 entries. 

We can do some goofy tensor product shenanigans 
\equations{
    S 
    =
    S_1 
    \otimes 
    S_2
    \hp 
    S_1 
    =
    \begin{pmatrix}
	a_1 & b_1 \\
	c_1 & d_1 
    \end{pmatrix}
    \\
    S 
    =
    \begin{bmatrix}
	a_1 S_2 & b_1 S_2 \\
	c_1 S_2 & d_1 S_2 
    \end{bmatrix}
}

This shows how S is a 4x4 because it's the tensor product of each particle which 
are 2x2 matrices. 

We can't add two spins together because they live in different vector spaces for each 
particle 
\equations{
    \vec S_1 
    +
    \vec S_2
    =
    S_1 \otimes 1_2 
    +
    1_2 \otimes S_2
}

Where $1_2$ is the identity for the particle. Think of addition as 2 particles communicating 
with each other in different vector spaces. 

\subsection{Degeneracy}
given a 2 particle system, the maximum total spin is $S_1 + S_2$. The minimum total 
spin magnitude is $|S_1 - S_2|$. 

For example, if both spins are 1, then $0 < S < 2$, so there are 3 possible spin 
magnitudes. The degeneracy, however, is 2S+1 for each S, so its 1, 3, 5 for 0, 1, 2.
This means the total degeneracy is 9.

\subsection{Questions}
Consider 2 spin 1/2 particles (2 possible states up and down) 

Find the total spin $S^2$ and z component $S_z$ in terms of the spin m
operators for the individual particles $(S_1, S_2)$.

\equations{
    \vec S_x 
    =
    \frac{\hbar}{2} \sigma_z
}

The only possible values for total spin are 0 and 1. Because the state itself 
has spin states 0 and 1, the eigenvalues of that vector must also be 0 or 1.
\equations{
    \hat S \ket{S_{1+2}}
    =
    \lambda  \ket{S_{1+2}}
    =
    1 \ket{S_{1+2}}
    \hp
    0 \ket{S_{1+2}}
    \\
    \hat S^2 \ket{S_{1+2}}
    =
    \lambda^2 \ket{S_{1+2}}
    =
    1 \ket{S_{1+2}}
    \hp
    0 \ket{S_{1+2}}
}

The degeneracy comes from the total spin values. For $S^2 = 0$ there is only 
degeneracy 1 which leads to $S_z = 0$. For $S^2 = 1$ we have $2s+1 = 3$ degeneracy, 
which means $S_z = -1,0,1$. -1 is connected to $\ket{\downarrow \downarrow}$, 
you can correspond $S_z = 0$ to 
$\frac{1}{\sqrt{2}} (\ket{\downarrow \uparrow} + \ket{\uparrow \downarrow})$.

Consider the hamiltonian for 2 interactive spin 
\equations{
    \hat H 
    =
    \alpha 
    \hat S_1 
    \cdot 
    \hat S_2
}

Use your answers from question 1 to show that 
\equations{
    \hat S_1 
    \cdot 
    \hat S_2
    =
    \frac{1}{2}
    \left(
    \hat S^2 
    -
    \hat S_1^2
    -
    \hat S_2^2
    \right)
}

\chapter{Extra Homeworks}
\section{Hw10}
\subsection{1}
Let 
\equations{
    L_{+} f^m_l 
    =
    A^m_l f^{m+1}_{l}
    \hp
    L_{-} f^m_l 
    =
    A^m_l f^{m-1}_{l}
}

What are $A$ and $B$ if the eigenfunctions $f$ are normalized? 
I think we learned about this and it's just $m+1$ or whatever.

First show that $L_{\pm}$ is the Hermitian Conjugate of $L_{\mp}$ and then uh 
something. Since $L_x$ and $L_y$ are observables, you may assume they are 
Hermitian, but you can prove it if you want. 

\equations{
    L_{\pm}
    =
    L_x 
    \pm
    i L_y
}
Proof by obvious that it's Hermitian $(-i^* = i)$.
We use a specific identity. 
\equations{
    L_{\pm} L_{\mp}
    =
    L^2 - L_z^2 \pm \hbar L_z
    \\
    \bra{f^m_l}
    L^2 - L_z^2 \pm \hbar L_z
    \ket{f^m_l}
    =
    \bra{f^m_l}
    L^2
    \ket{f^m_l}
    -
    \bra{f^m_l}
    L_z^2
    \ket{f^m_l}
    \pm
    \bra{f^m_l}
    \hbar L_z
    \ket{f^m_l}
    \\
    =
    \hbar^2 l(l+1)
    \bra{f^m_l}
    \ket{f^m_l}
    -
    \hbar^2 m^2
    \bra{f^m_l}
    \ket{f^m_l}
    \pm
    \hbar^2 m
    \bra{f^m_l}
    \ket{f^m_l}
    \Rightarrow 
    \\
    \bra{L_{\mp} f^m_l}
    \ket{L_{\mp} f^m_l}
    =
    \left(
    \hbar^2 l(l+1)
    -
    \hbar^2 m^2
    \pm
    \hbar^2 m
    \right)
    \bra{f^m_l}
    \ket{f^m_l}
    \\
    =
    \hbar^2
    \left(
    l(l+1)
    -
    m^2
    \pm
    m
    \right)
    \bra{f^m_l}
    \ket{f^m_l}
    \\
    B
    =
    \hbar^2
    \sqrt{
    l(l+1)
    -
    m^2
    +
    m
    }
    \hp
    A
    =
    \hbar^2
    \sqrt{
    l(l+1)
    -
    m^2
    -
    m
    }
}
The $\pm$ and $\mp$ are important, I fucked that up on last midterm as well lol.

At the top and bottom of the ladder you get 0 which makes sense but we actually see
the math math out because $l = m$.

\subsection{2}
Do a hundred million commutation relations lmao. 
\equations{
    \\
    L 
    =
    \vec r \times \vec p 
    =
    \hat x (yp_z - zp_y)
    -
    \hat y (x p_z - z p_x)
    +
    \hat z (x p_y - y p_x)
    \hp 
    L_z 
    =
    L \hat z
    \\
    [L_z, x]
    =
    (x p_y - y p_x) x 
    -
    x (x p_y - y p_x)
    \hp 
    L_z
    =
    xp_y - yp_x
    =
    \\
    [L_z, p_x]
}
Okay I can figure out the rest but I gotta know my commutation relations. 

\subsection{3}
Get a rotational analog of Ehrenfest's theorem 
\equations{
    \frac{d}{dt} \langle L \rangle 
    =
    \langle N \rangle 
    =
    \langle 
    r
    \times (- \nabla V)
    \rangle
}

I have no clue how to solve this right off the bat 
\equations{
    \frac{d}{dt} \langle L \rangle 
    =
    \frac{d}{dt} \langle r \times p \rangle 
    =
    \langle r \times \frac{d}{dt} p \rangle 
    =
    \langle r \times - \nabla V \rangle 
}

Show that angular momentum is conserved for any spherical potential (idk)

\subsection{4}
What is $L_{+} Y^{+}_{+}$? It's 0 lmao 

Do some other stuff. Okay I feel better and I'll definitely be better after writing 
everything on a cheat sheet 

\section{Homework 12}
\subsection{1}
Compute the expected values of $\langle S_{x, y, z} \rangle$ for a normalized 
spinor $\chi$. Spinors are eigenfunctions of spin I think.

\equations{
    \chi
    =
    \begin{bmatrix} a \\ b \end{bmatrix} 
    =
    a \chi_{+}
    +
    b \chi_{-}
    =
    a
    \begin{bmatrix} 1 \\ 0 \end{bmatrix} 
    +
    b
    \begin{bmatrix} 0 \\ 1 \end{bmatrix} 
    \\
    S_z \chi_{+} 
    =
    \frac{\hbar}{2} 
    \chi
    \\
    S_{xyz}
    i \hbar \sigma_{xyz} 
}

This can be calculated by just knowing all of the matrices
\subsection{2}
An electron is in a magnetic field 
\equations{
    B 
    =
    B_0 \cos(\omega t) \hat k
}

Construct the Hamiltonian 
\equations{
    H
    =
    -\gamma \vec B 
    \cdot \vec S 
    =
    -\gamma B_0 \cos(\omega t) S_z
    =
    E 
}

Find $\chi(t)$

This can be done by solving the time dependent SWE directly. 
\equations{
    \chi(t)
    =
    \begin{pmatrix} \alpha(t) \\ \beta(t) \end{pmatrix}
    \\
    H \chi 
    =
    i \hbar \frac{\del}{\del t} \chi 
    \Rightarrow 
    \frac{- \gamma B_0 \hbar}{2}
    \cos(\omega(t))
    \begin{pmatrix}
	1 & 0 \\
	0 & -1 
    \end{pmatrix}
    \begin{pmatrix} \alpha(t) \\ \beta(t) \end{pmatrix}
    =
    i \hbar 
    \begin{pmatrix} \alpha'(t) \\ \beta'(t) \end{pmatrix}
}

Just solve the diff eq by separating the variables. 

Find $p(S_x = \hbar/2)$

The two states $S_x = \pm \hbar/2$ correspond to eigenstates $\ket{\pm}$, so we 
specifically need to find the probability that the state collapses to state $\ket{-}$.

Find the necessary $B$ to guarantee a spin flip. This is just math, if you solved
 the last part correctly then this is easy. 

\section{Hw 13}
Write the Hamiltonian for 2 non-interacting particles in an infinite square well. 
Verify that the fermion ground state is an eigenfunction of $H$, and find the 
eigenvalue. 

The fermion ground state is 
\equations{
    \psi_n(x)
    =
    \sqrt{\frac{2}{a}}
    \sin(\frac{n \pi}{a} x)
    \hp 
    E_n = n^2 K
    \\
    \psi_{n1, n2}
    =
    \psi_{n1}(x_1)
    \psi_{n2}(x_2)
    \hp 
    E_{n1, n2}
    =
    (n_1^2 + n_2^2) K
}
It's not actually that bad you can solve it with separation of variables. 

\subsection{1b}
Find the next two excited states (beyond the ground state). Wave functions, energies, 
and degeneracies. For each of the three cases (distinguishable, identical 
bosons, identical fermions).

you just like go up in energy and keep thinking about the particle. 

distinguishable particles will have non-zero degeneracy if the energies are different 
because you just swap the particles. This cannot be done for identical particles. 
Identical fermions (same spin) must follow the Pauli exclusion principle, and thus 
can't have the same energy.

You do a buncha bullshit

\chapter{FINAL PREP}
$l$ can only be an integer and $S$ can be either an integer or integer + 1/2
\begin{itemize}
    \item
	Know the SWE for time-independent and dependent lmao
    \item
	Know/derive every ensemble 
	(square wells, free particle, oscillators, 2-level, radial, spin? etc)
    \item
	Know the quantum formula (energy eigenstates, time operator, etc)
    \item
	KNOW MATH (commutators, operators, observables, eigenvalues, eigenstates, 
	pauli matrices, permutation symbol $\epsilon$
    \item
	know eigenbasis transformations (you saw on homework)
    \item
	Ehrenfest theorem and inequalities and Heisenberg Uncertainty Principle
    \item
	Hermitian stuff
    \item
	OPERATORS (momentum, S, L, $S_{xyz}$, $L_{xyz}$, $L^2$, $S^2$, etc)
    \item
	Separation of variables
    \item
	degeneracy 
    \item
	angular momentum 
    \item
	Bosons vs Fermions lmao
    \item
	multiple particles? (yea just in case)
\end{itemize}

\section{Multi-Particle Stuff}
Electrons are indistinguishable, so we do some bullshit to show this off 
\equations{
    \psi_{\pm}(r_1, r_2)
    =
    A
    \left(
    \psi_a(r_1)
    \psi_b(r_2)
    \pm
    \psi_b(r_1)
    \psi_a(r_2)
    \right)
}

The plus sign is bosons and the minus sign is fermions. You can clearly see 
the pauli exclusion principle because if 2 fermions are identical, your wave function 
is just 0.

Bosons are symmetric under interchange, and fermions are antisymmetric under 
interchange. So bosons are integer spin and fermions are half integer spin.





























\end{document}
