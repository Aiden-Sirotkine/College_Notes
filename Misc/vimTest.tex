\documentclass[fleqn]{report}
\usepackage{geometry}
\usepackage{amssymb}
\usepackage{fancyhdr}
\usepackage{multicol}
\usepackage{blindtext}
\usepackage{color}
\usepackage[fontsize=16pt]{fontsize}
\usepackage{lipsum}
\usepackage{pgfplots}
\usepackage{physics}
\usepackage{mathtools}
\usepackage[makeroom]{cancel}
\usepackage{ulem}

\setlength{\columnsep}{1cm}
\addtolength{\jot}{0.1cm}
\def\columnseprulecolor{\color{blue}}
\date{Fall 2024}

\newcommand{\textoverline}[1]{$\overline{\mbox{#1}}$}

\newcommand{\hp}{\hspace{1cm}}

\newcommand{\del}{\partial}

\newcommand{\pdif}[2]{ \frac{\partial #1}{ \partial #2} }

\newcommand{\pderiv}[1]{ \frac{\partial}{ \partial #1} }

\newcommand{\comment}[1]{}

\newcommand{\equations} [1] {
\begin{gather*}
#1
\end{gather*}
}

\newcommand{\twovec}[2]{ 
\begin{pmatrix}
#1 \\ 
#2
\end{pmatrix}
}

\title{MATH 257}
\author{Aiden Sirotkine}

\begin{document}

\pagestyle{fancy}
\maketitle
\tableofcontents
\clearpage

\chapter{MATH 257}
My laptop died and I skipped some lectures to go to a part time 
job fair but I know every basic thing about matrices and vectors 
so I should be fine 

\chapter{Column Vectors and Basis Vectors} 
If you take the columns of a vector, then you get a couple vectors 
that span a space.

Solving a linear system is the same as finding the linear combinations 
that equal a certain result

\section{Matrix Vector Multiplication}
\[
\begin{bmatrix}
    c_1 && c_2 && c_3
\end{bmatrix}
\begin{bmatrix} a \\ b \\ c \end{bmatrix} =
a c_1 + b c_2 + c c_3
\]

\section{Transformations}
You can multiply a vector by a matrix to transform it in a certain 
way

\subsection{Rotation}
\[
\begin{bmatrix}
\cos \theta && - \sin \theta \\
\sin \theta && \cos \theta
\end{bmatrix}
\]

\section{3d Rotation Matrix}
the way to figure out transformations is just to think about how the matrix
transforms the 3 unit vectors
\equations{
\begin{bmatrix}
1 \\ 0 \\ 0 
\end{bmatrix}
\rightarrow
\begin{bmatrix}
\cos \theta \\ \sin \theta \\ 0
\end{bmatrix}
\\
\begin{bmatrix}
0 \\ 1 \\ 0
\end{bmatrix}
\rightarrow
\begin{bmatrix}
-\sin \theta \\ \sin \theta \\ 0
\end{bmatrix}
\\
\begin{bmatrix}
0 \\ 0 \\ 1
\end{bmatrix}
\rightarrow
\begin{bmatrix}
\cos \theta \\ \sin \theta \\ 0
\end{bmatrix}
}

















\end{document}
