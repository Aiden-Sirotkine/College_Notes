\documentclass[fleqn]{report}
\usepackage{geometry}
\usepackage{amssymb}
\usepackage{fancyhdr}
\usepackage{multicol}
\usepackage{blindtext}
\usepackage{color}
\usepackage[fontsize=16pt]{fontsize}
\usepackage{lipsum}
\usepackage{pgfplots}
\usepackage{physics}
\usepackage{mathtools}
\usepackage[makeroom]{cancel}
\usepackage{ulem}

\setlength{\columnsep}{1cm}
\addtolength{\jot}{0.1cm}
\def\columnseprulecolor{\color{blue}}
\date{Spring 2025}

\newcommand{\textoverline}[1]{$\overline{\mbox{#1}}$}

\newcommand{\hp}{\hspace{1cm}}

\newcommand{\const}{\textrm{const}}

\newcommand{\del}{\partial}

\newcommand{\pdif}[2]{ \frac{\partial #1}{ \partial #2} }

\newcommand{\pderiv}[1]{ \frac{\partial}{ \partial #1} }

\newcommand{\comment}[1]{}

\newcommand{\equations} [1] {
\begin{gather*}
#1
\end{gather*}
}

\newcommand{\numequations} [1] {
\begin{gather}
#1
\end{gather}
}

\newcommand{\twovec}[2]{ 
\begin{pmatrix}
#1 \\ 
#2
\end{pmatrix}
}

\title{PHYS 435}
\author{Aiden Sirotkine}

\begin{document}

\pagestyle{fancy}
\maketitle
\tableofcontents
\clearpage

\chapter{PHYS435}
This goddamn professor is half retired im so cooked 

\section{Coulomb's Law}
\equations{
    \vec E(\vec r) = \frac{1}{4 \pi \epsilon_0} \frac{q}{r^2} \hat r 
    \\
    \vec E(\vec r) = \sum \frac{1}{4 \pi \epsilon_0} \frac{q}{r^2} \hat r 
    \rightarrow 
    \frac{1}{4 \pi \epsilon_0} 
    \int 
    d^3 r' 
    \frac{\vec r - \vec r'}{|\vec r - \vec r'|^3} \rho(\vec r')
}

\section{Gauss's Law}
The flux of $\vec E$ throuhg a closed surface equations to the enclosed charge $C_0$

\equations{
    \frac{1}{C_0} \int_V d^3 r \rho(\vec r) = \int_{\del V} d a \rho(\vec r)
}

\section{Divergence Theorem}
\equations{
    \vec \nabla E = \del_x E_x + \del_y E_y + \del_z E_z
    \\
    \int_V d^3 r \vec \nabla \vec E(\vec r) = \int_{\del V} d \vec a \cdot \vec E(\vec r)
    \\
    \vec \nabla \vec E(\vec r) = \frac{\rho(\vec r)}{\epsilon_0}
}

\section{Faraday's Law}
The circulation of $\vec E$ around any closed path $N$ is equal to $(-1) \times$ 
the time derivative of the magnetic flux  through ANY surface bounded by 
the closed path.

\equations{
    \int_{\del S} d \vec l \cdot \vec E = - \frac{d}{dt} \int_S d \vec a \vec B 
}

\section{Stoke's Theorem}
\equations{
    \int_{\del S} d \vec l \cdot \vec E = 
    \int_S d \vec a \vec \nabla \times \vec E 
}

\subsection{Differential Laws}
\equations{
    \vec \nabla \times \vec E = - \frac{\del}{\del t} \vec B 
    \\
    \textrm{Gauss}
    \\
    \vec \nabla \cdot \vec E = - \frac{\rho(\vec r)}{\epsilon_0}
    \hp 
    \vec \nabla \cdot \vec B = 0
    \\
    \textrm{Ampere}
    \\
    \vec \nabla \times \vec B = \mu_0 J
}

\section{Electric Potential}
Start with Maxwell's Equations
\equations{
    \vec \nabla \cdot \vec E(\vec r) = \frac{ \rho(\vec r)}{\epsilon_0}
    \\
    \vec \nabla \times \vec E(\vec r) = - \frac{\del}{\del t} \vec B(\vec r, t)
    \\
    \vec \nabla \cdot \vec B(\vec r) = 0
    \\
    \vec \nabla \times \vec B = 
    \mu_0 \vec J + \mu_0 \epsilon_0 \frac{\del}{\del t} \vec E
}
remove the time dependent equations 
\equations{
    \vec \nabla \cdot \vec E(\vec r) = \frac{ \rho(\vec r)}{\epsilon_0}
    \\
    % \vec \nabla \cdot \vec B(\vec r) = 0
    % \\
    \vec \nabla \times \vec E(\vec r) = 0
}
the curl of $\vec E$ is 0 which means the electric field is conservative? 
A scalar potential function convenience.

consider the path integral 
\equations{
    \int\limits_P d \vec l \cdot \vec E 
}
We can show that the integral is path independent (because the curl is 0)
\equations{
    \int\limits_{P1} - \int\limits_{P2} = 
    \oint\limits_{\del S} d \vec l \cdot \vec E(\vec r)
    =
    \int\limits_S d \vec a \cdot \vec \nabla \times \vec E
    = 0
    \\
    \int\limits_{P1} d \vec l \cdot \vec E = \int\limits_{P2} d \vec l \cdot \vec E
}
That's actually a really smart proof damn 

Now we do actual potential stuff 
\equations{
    V(\vec r) = - \int^{\vec r}_{\vec 0_r} d \vec l \cdot \vec E(\vec r)
}
Where $\vec 0_r$ is the vector where the potential is 0
\equations{
    U(\vec a) - U(\vec b) = - \int^b_a d \vec l' \cdot \vec F(\vec r)
    \\
    \vec F_{Lorentz} = q \vec E(\vec r) + q \vec v(\vec r) \times \vec B(\vec r)
}
$q \vec E(\vec r)$ can do work, but $q \vec v(\vec r) \times \vec B(\vec r)$ 
cannot do any work (always in opposite direction of motion)

\equations{
    W 
    =
    q \int^{\vec r}_{\vec 0_{r}} d \vec l \cdot 
    \vec v \times \vec B 
    = 
    q \int^{\vec r}_{\vec 0} d \vec l \cdot 
    \frac{d \vec l}{dt} \times \vec B(\vec r)
    =
    \\
    q \int^{\vec r}_{\vec 0} dt \frac{d \vec l}{dt} \cdot 
    (\frac{d \vec l}{dt} \times \vec B(\vec r))
    = 0
}
That part cannot do any work 
\equations{
    W_{other} = U(\vec r) - U(\vec 0)
    \\
    \frac{U(\vec r) - U(\vec 0)}{q} = \Delta V
}
More Stuff 
\equations{
    V(\vec r) = - \int^{\vec r}_{\vec 0_r} d \vec l' \cdot \vec E(\vec r)
    \\
    - \vec \nabla V(\vec r) 
    =
    -
    \left[
        \hat x \frac{\del}{\del x} V(\vec r) + 
        \hat y \frac{\del}{\del y} V(\vec r) +
        \hat z \frac{\del}{\del z} V(\vec r)
    \right]
    \\
    d \vec r = \hat x dx + \hat y dy + \hat z dz 
    =
    r + dx 
}
consider the slightest motion $dx$ in the $\hat x$ direction so that 
$\vec r \to \vec r + \hat x dx$
\equations{
    V(\vec r + \hat x dx) = V(\vec r) + dx \hat x \cdot \vec E(\vec r)
    \hp 
    E_x(\vec r) = \hat x \cdot \vec E(\vec r)
    \\
    \frac{V(\vec r + \hat x dx) - V(\vec r)}{dx} = E_x(\vec r)
}
That gives us 
\[
\vec E(\vec r) = - \nabla V(\vec r)
\]
real important equation 

\subsection{Potential Equation}
consider a point mass 
\equations{
    \vec E(\vec r) = 
    \frac{1}{4 \pi \epsilon_0} q \frac{\vec r}{r^3} = 
    \frac{1}{4 \pi \epsilon_0} \frac{q}{r^2} \hat r
    \\
    V(\vec r) = - \int^{\vec r}_{\vec 0_r} d \vec l \cdot \vec E 
    \\
    E_y = \frac{1}{4 \pi \epsilon_0} \frac{q}{y^2}
    \hp 
    - \int^y_\infty dy' \frac{1}{4 \pi \epsilon_0} \frac{q}{y^2}
    \\
    V(\vec r) = \frac{1}{4 \pi \epsilon_0} \frac{q}{r}
}
Use the principle of superposition to get the general answer 
\equations{
    V(\vec r) = \frac{1}{ 4 \pi \epsilon_0} 
    \int d^3 r' \frac{\rho(\vec {r'})}{|\vec r - \vec {r'}|}
}

\subsection{Infinite Line Charge}
Consider a straight line of infinite length and constant charge density. 

Where should $\vec 0_{r}$ be?

I think we just pick an arbitrary point 

\equations{
    \vec E(s) = \frac{\lambda}{2 \pi \epsilon_0} \frac{1}{s}
    \\
    V(\vec r) = - \int d \vec l \vec E(s) = V(s) = 
    - \int^s_{O_r} ds \frac{1}{2 \pi \epsilon_0} \frac{\lambda}{s'}
    \\
    =
    - \frac{\lambda}{2 \pi \epsilon_0} \ln(s') \Big|^s_{\vec O_r}
    =
    \frac{\lambda}{2 \pi \epsilon_0} \ln(\frac{O_r}{s})
}

What PDE governs $V(\vec r)$

\equations{
    \vec \nabla \cdot \vec E = \frac{\rho}{\epsilon_0}
    \hp 
    \vec E = - \vec \nabla V 
    \\
    \vec \nabla \cdot \vec \nabla V(\vec r) = -\frac{\rho}{\epsilon_0}
    \\
    (\del_x^2 + \del_y^2 + \del_z^2) V = -\frac{\rho}{\epsilon_0}
    \\
    \nabla^2 V(\vec r) = -\frac{\rho(\vec r)}{\epsilon_0}
}


\section{Work}
If you move 1 charge, there is no work done because there are no other 
fields. 

If you bring in a 2nd charge, you get a total work of 
$
\frac{1}{4 \pi \epsilon_0} \frac{q_1 q_2}{|\vec r_1 - \vec r_2|}
$

If you bring in a third charge you just sum the things together 

\equations{
    U_{1 \to N} = \frac{1}{2} \sum^N_i \sum^N_{j > i} 
    \frac{1}{4 \pi \epsilon_0} \frac{q_i q_j}{|\vec r_i - \vec r_j|}
    =
    \frac{1}{2} \int d^3 r d^3 r' \frac{1}{4 \pi \epsilon_0} 
    \frac{\rho(\vec r) \rho(\vec{r'})}{|\vec r - \vec{r'}|}
    \\
    =
    \frac{1}{2} \frac{1}{4 \pi \epsilon_0} 
    \int d^3 r' \rho(\vec{r'}) V(\vec{r'})
    \hp 
    \rho(\vec{r'}) = - \epsilon_0 \nabla^2 V(\vec{r'})
    \\
    U = - \frac{\epsilon_0}{2} \int d^3 r V(\vec r) \nabla^2 V(\vec r)
}

\subsection{X-component}
Let's consider just the $x$-component for a little bit 

\equations{
    - \frac{\epsilon_0}{2} 
    \int d^3 r V(\vec r) \del_x \left[ \del_x V(\vec r) \right]
    \\
    \del_x \left[ V(\vec r) \del x V(\vec r) \right]
    =
    \del_x V \cdot \del_x V + V \del_x^2 V 
    \\
    \del_x \left[ V(\vec r) \del x V(\vec r) \right]
    - \del_x V \cdot \del_x V 
    =
    V \del_x^2 V 
}
So with that you get 
\equations{
    - \frac{\epsilon_0}{2} 
    \int d^3 r V(\vec r) \del_x \left[ \del_x V(\vec r)\right]
    =
    - \frac{\epsilon_0}{2} 
    \int d^3 r \del_x \left( V(\vec r) \del_x V (\vec r)\right)
    - E_x^2
}
generalize 
\equations{
    \frac{\epsilon_0}{2} 
    \int d^3 r \, \vec \nabla \cdot V(\vec r) \vec E(\vec r) + 
    \vec E \cdot \vec E 
}

The main point of all of this is that it goes to $0$ for large $r$ 
\equations{
    \int d^3 r \vec \nabla \cdot V \vec E 
    \rightarrow 
    \int d^3 r \vec \nabla \cdot \frac{C}{r} \frac{1}{r^2} \hat r 
    \rightarrow 
    \\
    \int da \frac{C}{r^3}
    \rightarrow 
    \frac{1}{r}
    \rightarrow \lim_{r \to \infty} = 0
}

So our potential somehow gets to 
\equations{
    U  =
    \int d^3 r \frac{\epsilon_0}{2} E^2 
}










































\end{document}