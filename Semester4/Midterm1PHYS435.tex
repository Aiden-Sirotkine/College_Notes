
\documentclass[fleqn]{report}
\usepackage{geometry}
\usepackage{amssymb}
\usepackage{fancyhdr}
\usepackage{multicol}
\usepackage{blindtext}
\usepackage{color}
\usepackage[fontsize=16pt]{fontsize}
\usepackage{lipsum}
\usepackage{pgfplots}
\usepackage{physics}
\usepackage{mathtools}
\usepackage[makeroom]{cancel}
\usepackage{ulem}
\usepackage{esint}

\graphicspath{ {../Images/} }
\setlength{\columnsep}{1cm}
\addtolength{\jot}{0.1cm}
\def\columnseprulecolor{\color{blue}}
\date{Spring 2025}

\newcommand{\textoverline}[1]{$\overline{\mbox{#1}}$}

\newcommand{\hp}{\hspace{1cm}}

\newcommand{\const}{\textrm{const}}

\newcommand{\del}{\partial}

\newcommand{\pdif}[2]{ \frac{\partial #1}{ \partial #2} }

\newcommand{\pderiv}[1]{ \frac{\partial}{ \partial #1} }

\newcommand{\comment}[1]{}

\newcommand{\equations} [1] {
\begin{gather*}
#1
\end{gather*}
}

\newcommand{\numequations} [1] {
\begin{gather}
#1
\end{gather}
}

\newcommand{\twovec}[2]{ 
\begin{pmatrix}
#1 \\ 
#2
\end{pmatrix}
}

\title{PHYS 435 Midterm 1}
\author{Aiden Sirotkine}

\begin{document}

\pagestyle{fancy}
\maketitle
\tableofcontents
\clearpage

\chapter{PHYS435 Midterm 1}
This midterm is open book open note so if I just write down 
absolutely every topic ever then I can look back at this notebook 
and be absolutely set 

\section{Calculus Junk}
Know my Divergence and curls and junk 

\subsection{Divergence Theorem}
Connects a volume to a surface 
\equations{
    \iiint\limits_V \, d^3 V \, 
    \nabla \cdot F 
    =
    \iint\limits_{S} \, d^2 S \, 
    \hat n \cdot F
}

\subsection{Poisson's Equation}
\equations{
    \vec \nabla \vec E(\vec r) = \frac{\rho(\vec r)}{\epsilon_0}
    \rightarrow
    -\nabla^2 V(\vec r) = \frac{\rho(\vec r)}{\epsilon_0}
}

\subsection{Green's Theorem}
Connects a surface to a loop
\equations{
    \oint\limits_{\del S} \, dS \,
    F
    =
    \iint\limits_V \, d^3 V \, 
    \nabla \times 
    F
}

There's probability some derivative identities I can use

\section{Electric Fields}
Everything can be solve with either Gauss's Law, superposition, or both. 

\equations{
    \vec E(r)
    =
    \frac{Q}{r^2} \hat r 
}

\section{Electric Potential}
Everything can be solved with either superposition or integrating the electric field 

\equations{
    V = -\int^r_{O} \, dr' \, 
    E(r')
    \hp 
    E = - \nabla V
}

\section{Conductors}
conductors are pretty neat. 
\subsection{Method of Images}
Given a conducting object and a point charge on the outside, the potential 
on the outside of the conductor is equivalent to the superposition of the 
real charge and an "image" point charge. 

We've found the image charges for both a plane and a sphere. 

\section{Capacitance} 
\equations{
    C = \frac{Q}{\Delta V}
}

\section{Boundary Value Problems}
This was a big part of the class we had, and luckily because of the law of 
superposition, 

\subsection{Potential Everywhere Given Boundary Potential (Don't Know 
Green's Function)}
This works because inside a conductor, there is no charge, so we can use 
Poisson's equation.
\equations{
    \nabla^2 V = 0
    \rightarrow 
    V(x, y, z) = X(x)Y(y)Z(z)
}
Solve the diff EQ from there to get some fourier series than 
can be solved for any surface potential.

This works if there is no charge on either the inside or outside of 
the conductor.

\subsection{Potential Given Charge Density}
It's just the superposition of the green's function and the charge density. 
This one is fine 

\section{Green's Function}
It' s a little wacky but its essentially the response function given a single 
point charge.

$G = 0$ on the boundary of a conductor 

The green's function changes for conductors, but is 
\equations{
    G(r, r') = \frac{1}{|\vec{r} - \vec{r'}|}
}
for insulators 

So our boundary value problem becomes 
\equations{
    V(\vec r)
    =
    \frac{1}{4 \pi \epsilon_0}
    \int\limits_V \, d^3 r' \, 
    G(\vec{r}, \vec{r'})
    \rho(\vec{r'})
    +
    \frac{1}{4 \pi}
    \int\limits_{\del V} \, da' \, 
    \hat{m'} \cdot \vec{\nabla}_r 
    G(\vec{r}, \vec{r'})
    V(\vec{r'})
}
The first part makes sense because it's just the superposition of the inside 
charges, but the second part is harder to derive 

THIS ONLY WORKS FOR CONDUCTORS

This derivation is obnoxious 
\equations{
    \nabla^2 V(\vec r) = -\frac{\rho}{\epsilon_0}
    \hp 
    \nabla^2 G(\vec r) = - 4 \pi \delta(\vec r - \vec{r'})
    \\
    \vec \nabla \cdot 
    \left( G \vec \nabla V - V \vec \nabla G \right)
    =
    \vec \nabla G \vec \nabla V + G \vec \nabla^2 V
    -
    \vec \nabla G \vec \nabla V - V \vec \nabla^2 G
    \\
    =
    G \vec \nabla^2 V
    - V \vec \nabla^2 G
    \rightarrow 
    \\
    \int\limits_V \, d^3 r' \, 
    G \vec \nabla^2 V
    - V \vec \nabla^2 G
    =
    \int\limits_V \, d^3 r' \, 
    \vec \nabla \cdot 
    \left( G \vec \nabla V - V \vec \nabla G \right)
    \rightarrow 
    \\
    \int\limits_V \, d^3 r' \, 
    G \vec \nabla^2 V
    - V \vec \nabla^2 G
    =
    \int\limits_{\del V} \, da \, 
    G \hat n \cdot \vec \nabla V - V \hat n \cdot \vec \nabla G
    \rightarrow 
    \\
    -\int\limits_V \, d^3 r' \, 
    G \frac{\rho}{\epsilon_0}
    +
    \int\limits_V \, d^3 r' \, 
    V \vec 4 \pi \delta(\vec r - \vec{r'})
    =
    \int\limits_{\del V} \, da \, 
    G \hat n \cdot \vec \nabla V - V \hat n \cdot \vec \nabla G
    \rightarrow 
    \\
    -\int\limits_V \, d^3 r' \, 
    G \frac{\rho}{\epsilon_0}
    +
    V(\vec{r'}) 4 \pi
    =
    \int\limits_{\del V} \, da \, 
    G \hat n \cdot \vec \nabla V - V \hat n \cdot \vec \nabla G
    \rightarrow 
    \\
    -\int\limits_V \, d^3 r' \, 
    G \frac{\rho}{\epsilon_0}
    +
    V(\vec{r'}) 4 \pi
    =
    \int\limits_{\del V} \, da \, 
    G * - \vec E \cdot \hat m - V \hat n \cdot \vec \nabla G
    \rightarrow 
    \\
    V(\vec{r'})
    =
    \frac{1}{4 \pi \epsilon_0}
    \int\limits_V \, d^3 r' \, 
    G \rho
    - 
    \frac{1}{4 \pi}
    \int\limits_{\del V} \, da \, 
    V \hat n \cdot \vec \nabla G
    \\
    +
    \frac{1}{4 \pi}
    \int\limits_{\del V} \, da \, 
    G * - \vec E \cdot \hat m 
}
And the last term is $0$ because the potential caused by Green's function 
on the surface of a conductor should be 0

That's a kind of need derivation that gives you a pretty comfy equation. 

THIS ONLY WORKS FOR CONDUCTORS

Because the Green's Function is specifically for a conducting sphere.

For a non-conducting surface you use the separation of variables. 


\section{Legendre Polynomials (FIGURE OUT WHAT'S HAPPENING)}
I think its just the separation of variables that happens with a sphere 

It's a consequence of the divergence of spherical coordinates 
\equations{
    \nabla^2 V
    =
    \frac{1}{r^2}
    \frac{\del}{\del r} 
    r^2 
    \frac{\del}{\del r}
    V
    +
    \frac{1}{r^2 \sin(\theta)}
    \frac{\del}{\del \theta} 
    \sin(\theta) 
    \frac{\del}{\del \theta} 
    V
    +
    \frac{1}{r^2 \sin^2(\theta)}
    \frac{\del^2}{\del \phi^2} V 
}

so now we say that the laplacian is 0 and the potential is separable 

\equations{
    V = R(r) Y(\theta, \phi)
    \hp
    \nabla^2 V = 0
    \\
    \frac{1}{r^2}
    \frac{\del}{\del r} 
    r^2 
    \frac{\del}{\del r}
    V
    +
    \frac{1}{r^2 \sin(\theta)}
    \frac{\del}{\del \theta} 
    \sin(\theta) 
    \frac{\del}{\del \theta} 
    V
    +
    \frac{1}{r^2 \sin^2(\theta)}
    \frac{\del^2}{\del \phi^2} V 
    =0
    \\
    \frac{\del}{\del r} 
    r^2 
    \frac{\del}{\del r}
    V
    +
    \frac{1}{\sin(\theta)}
    \frac{\del}{\del \theta} 
    \sin(\theta) 
    \frac{\del}{\del \theta} 
    V
    +
    \frac{1}{\sin^2(\theta)}
    \frac{\del^2}{\del \phi^2} V 
    =0
    \\
    Y
    \frac{\del}{\del r} 
    r^2 
    \frac{\del}{\del r}
    R(r) 
    +
    R(r)
    \frac{1}{\sin(\theta)}
    \frac{\del}{\del \theta} 
    \sin(\theta) 
    \frac{\del}{\del \theta} 
    Y
    +
    R(r)
    \frac{1}{\sin^2(\theta)}
    \frac{\del^2}{\del \phi^2}
    Y
    =0
    \\
    \frac{1}{R}
    \frac{\del}{\del r} 
    r^2 
    \frac{\del}{\del r}
    R(r) 
    +
    \frac{1}{Y}
    \frac{1}{\sin(\theta)}
    \frac{\del}{\del \theta} 
    \sin(\theta) 
    \frac{\del}{\del \theta} 
    Y
    +
    \frac{1}{Y}
    \frac{1}{\sin^2(\theta)}
    \frac{\del^2}{\del \phi^2}
    Y
    =0
}

This is our main diff eq that we now solve by taking various partial derivatives.

\equations{
    \frac{\del}{\del r}
    \left(
        \frac{1}{R}
        \frac{\del}{\del r} 
        r^2 
        \frac{\del}{\del r}
        R(r) 
    \right)
    =
    0
    \rightarrow 
    \\
    \frac{1}{r}
    \frac{\del}{\del r} 
    r^2 
    \frac{\del}{\del r}
    R(r) 
    =
    l(l+1)
    \\
    \frac{1}{Y}
    \frac{1}{\sin(\theta)}
    \frac{\del}{\del \theta} 
    \sin(\theta) 
    \frac{\del}{\del \theta} 
    Y
    +
    \frac{1}{Y}
    \frac{1}{\sin^2(\theta)}
    \frac{\del^2}{\del \phi^2}
    Y
    = -l(l+1)
}

With these, we get the equation 
\equations{
    \frac{1}{r(r)}
    \frac{\del}{\del r} 
    r^2 
    \frac{\del}{\del r}
    R(r) 
    -
    l(l+1)
    =
    0
    \rightarrow
    \frac{\del}{\del r} 
    r^2 
    \frac{\del}{\del r}
    R(r) 
    =
    l(l+1) R(r)
    \\
    \frac{1}{\sin(\theta)}
    \frac{\del}{\del \theta} 
    \sin(\theta) 
    \frac{\del}{\del \theta} 
    Y
    +
    \frac{1}{\sin^2(\theta)}
    \frac{\del^2}{\del \phi^2}
    Y
    +
    l(l+1) Y
    =
    0
    \rightarrow 
    \\
    \sin(\theta)
    \frac{\del}{\del \theta} 
    \sin(\theta) 
    \frac{\del}{\del \theta} 
    Y
    +
    \frac{\del^2}{\del \phi^2}
    Y
    +
    l(l+1) \sin^2(\theta) Y
    =
    0
}

You can separate $Y$ into $\Theta(\theta) \Phi(\phi)$

\equations{
    \sin(\theta)
    \frac{\del}{\del \theta} 
    \sin(\theta) 
    \frac{\del}{\del \theta} 
    \Theta \Phi
    +
    \frac{\del^2}{\del \phi^2}
    \Theta \Phi
    +
    l(l+1) \sin^2(\theta)
    \Theta \Phi
    \rightarrow 
    \\
    \frac{1}{\Theta(\theta)}
    \sin(\theta)
    \frac{\del}{\del \theta} 
    \sin(\theta) 
    \frac{\del}{\del \theta} 
    \Theta 
    +
    \frac{1}{\Phi(\phi)}
    \frac{\del^2}{\del \phi^2}
    \Phi
    +
    l(l+1) \sin^2(\theta)
    =
    0
}

Take our partial derivatives

\equations{
    \frac{\del}{\del \theta}
    \left(
    \frac{1}{\Theta(\theta)}
    \sin(\theta)
    \frac{\del}{\del \theta} 
    \sin(\theta) 
    \frac{\del}{\del \theta} 
    \Theta 
    +
    l(l+1) \sin^2(\theta)
    \right)
    =
    0
    \rightarrow
    \\
    \frac{1}{\Theta(\theta)}
    \sin(\theta)
    \frac{\del}{\del \theta} 
    \sin(\theta) 
    \frac{\del}{\del \theta} 
    \Theta 
    +
    l(l+1) \sin^2(\theta)
    =
    +m^2
}

and on the other side 

\equations{
    \frac{\del}{\del \phi}
    \left(
        \frac{1}{\Phi(\phi)}
        \frac{\del^2}{\del \phi^2}
        \Phi
    \right)
    = 
    0
    \rightarrow 
    \\
    \frac{1}{\Phi(\phi)}
    \frac{\del^2}{\del \phi^2}
    \Phi
    = 
    -m^2
    \rightarrow 
    \Phi(\phi)
    =
    A \sin(m \phi) + B \cos(m \phi)
}

The solution for $\Theta$ is where we get the obnoxious legendre function 

\equations{
    \frac{1}{\Theta(\theta)}
    \sin(\theta)
    \frac{\del}{\del \theta} 
    \sin(\theta) 
    \frac{\del}{\del \theta} 
    \Theta 
    +
    l(l+1) \sin^2(\theta)
    =
    +m^2
    \rightarrow 
    \\
    \sin(\theta)
    \frac{\del}{\del \theta} 
    \sin(\theta) 
    \frac{\del}{\del \theta} 
    \Theta 
    =
    \left(
    m^2
    -
    l(l+1) \sin^2(\theta)
    \right)
    \Theta 
}

The solution to this is the legendre function 

\equations{
    P^m_l(\theta)
}

So then you multiply the things together, and in spherical harmonics you get 
\equations{
    Y(\theta, \phi)
    =
    \sqrt{\frac{2l + 1}{4 \pi} \frac{(l - m)!}{(l + m)!}} 
    P^m_l(\theta) e^{i m \phi}
    \hp 
    -l \leq m \leq l
}

Let's assume that we have azimuthal symmetry, so no $\phi$ dependence. 

It basically just means that $m=0$ so our equation becomes 

\equations{
    \sin(\theta)
    \frac{\del}{\del \theta} 
    \sin(\theta) 
    \frac{\del}{\del \theta} 
    \Theta 
    =
    -
    l(l+1) \sin^2(\theta)
    \Theta 
    \rightarrow 
    \\
    \frac{\del}{\del \theta} 
    \sin(\theta) 
    \frac{\del}{\del \theta} 
    \Theta 
    =
    -
    l(l+1) \sin(\theta)
    \Theta 
    \rightarrow 
    \\
    \frac{1}{\sin(\theta)}
    \frac{\del}{\del \theta} 
    \sin^2(\theta) 
    \frac{1}{\sin(\theta)}
    \frac{\del}{\del \theta} 
    \Theta 
    =
    -
    l(l+1)
    \Theta 
}

You do a substitution to make solving easier 
\equations{
    \frac{dx}{d \theta} = - \sin(\theta)
    \rightarrow 
    dx = - \sin(\theta) d \theta
    \\
    \frac{\del}{\del x} 
    (1 - x^2)
    \frac{\del}{\del x} 
    \Theta 
    =
    -
    l(l+1)
    \Theta 
}

The legendre polynomials come back in the solution 

Legendre polynomials are 
\equations{
    P_l(x) = c_0 + c_2 x^2 + c_4 x^4 + \ldots + c_l x^l
    \hp 
    l \textrm{ even}
    \\
    P_l(x) = c_1 x + c_3 x^3 + c_5 x^5 + \ldots + c_l x^l
    \hp 
    l \textrm{ odd}
}

Now we put back in $x = \cos(\theta)$ and we get 

\equations{
    \Theta_l(\theta) = P_l(\theta)
    \\
    P_0 = 1
    \hp 
    P_1 = x
    \hp 
    P_2 = \frac{1}{2} (3x^2 - 1)
    \hp 
    P_3 = \frac{1}{2} (5x^3 - 3x)
    \hp 
    \ldots 
    \\
    P_0 = 1
    \hp 
    P_1 = \cos(\theta)
    \hp 
    P_2 = \frac{1}{2} (3 \cos^2(\theta) - 1)
    \\
    P_3 = \frac{1}{2} (5 \cos^3(\theta) - 3 \cos(\theta))
    \hp 
    \ldots 
}

You just take a superposition of a bajillion legendre polynomials to get 
the $\theta$ dependence of the potential 

Now we need to consider the $R$ potential 

\equations{
    \frac{1}{R(r)}
    \frac{\del}{\del r} 
    r^2 
    \frac{\del}{\del r}
    R_l(r) 
    -
    l(l+1)
    =
    0
    \rightarrow 
    \frac{\del}{\del r} 
    r^2 
    \frac{\del}{\del r}
    R_l(r) 
    =
    l(l+1)
    R(r)
    \\
    \rightarrow 
    R_l(r)
    =
    A_l r^l + B_l \frac{1}{r^{l + 1}}
}

We put everything together and get the potential for an azimuthally 
symmetric sphere 

\equations{
    V(r, \theta)
    =
    \sum_l
    \left(
        A_l r^l 
        +
        \frac{B_l}{r^{l + 1}}
    \right)
    P_l(\cos(\theta))
}

\subsection{Conducting Sphere in a Capacitor}

\section{Multipoles}
They're like a just a heuristic thing to figure out roughly what you 
distance dependence is.

It's based off of a taylor series that I'm sure you can derive if you really 
want to. 

monopole is 1/r, dipole is $1/r^2$, quadrupole is $1/r^3$

\section{Dielectrics}
I genuinely have no idea but there's some indentities and some polarizability 
stuff. 

Dielectrics are not conductors, but are made of atoms that can conduct 
electricity. 

Each atom has a symmetric charge distribution 

Now apply an electric field so that the dielectric has a dipole moment. 
\equations{
    \vec p = \alpha \vec E
}
Where $\alpha$ is the polarizability coefficient.

\subsection{Spring Model of Atoms}

The dipole moment is proportional to the displacement of the positive 
and negative charges relative to each other. 

\subsection{Polarization Density}
Because dielectrics are like continuous 

\equations{
    \vec P = \vec p \cdot \textrm{atomic density} = pn
    \\
    \vec P = n \alpha \vec E 
}

Consider an electronic suscebtilibility factor $X$ 

\equations{
    \vec P = \epsilon_0 X \vec E 
    \hp 
    X = \frac{n q^2}{\epsilon_0 \kappa}
}

We find the polarizability constants of single atoms but I doubt that's super 
necessary 

\subsection{Bound Charge Density}
I think this is finding the new charge density from a polarized dielectric 

\equations{
    V(\vec r) = 
    \frac{1}{4 \pi \epsilon_0}
    \frac{\vec p \cdot \vec r }{|\vec r - \vec{r'}|^3}
}
consider polarization density and $\vec p = \int \, d^3 r \, \vec P$

\equations{
    V(\vec r) = 
    \frac{1}{4 \pi \epsilon_0}
    \int\limits_{V} \, d^3 r' \, 
    \frac{\vec P \cdot \vec r }{|\vec r - \vec{r'}|^3}
}

we know that the derivative of $1/r$ is $-1/r^2 = -r/r^3$, so 

\equations{
    V(\vec r) = 
    \frac{1}{4 \pi \epsilon_0}
    \int\limits_{V} \, d^3 r' \, 
    \vec P
    \cdot 
    -\vec \nabla_r
    \frac{1 }{|\vec r - \vec{r'}|}
    =
    \frac{1}{4 \pi \epsilon_0} 
    \int\limits_{V} \, d^3 r' \, 
    \vec P
    \cdot 
    \vec \nabla_{r'}
    \frac{1 }{|\vec r - \vec{r'}|}
}

You do some math 
\equations{
    \vec \nabla_{r'}
    \cdot 
    \frac{\vec P }{|\vec r - \vec{r'}|}
    =
    \frac{\vec \nabla \cdot \vec P }{|\vec r - \vec{r'}|}
    +
    \vec P
    \cdot 
    \vec \nabla_{r'}
    \cdot 
    \frac{1 }{|\vec r - \vec{r'}|}
}

and then get 
\equations{
    V(\vec r)
    =
    \frac{1}{4 \pi \epsilon_0} 
    \int\limits_{V} \, d^3 r' \, 
    \vec P
    \cdot 
    \vec \nabla_{r'}
    \frac{1 }{|\vec r - \vec{r'}|}
    =
    \\
    \frac{1}{4 \pi \epsilon_0} 
    \int\limits_{V} \, d^3 r' \, 
    \vec \nabla_{r'}
    \cdot 
    \frac{\vec P }{|\vec r - \vec{r'}|}
    -
    \frac{1}{4 \pi \epsilon_0} 
    \int\limits_{V} \, d^3 r' \, 
    \frac{\vec \nabla \cdot \vec P }{|\vec r - \vec{r'}|}
}

Then do divergence theorem 

\equations{
    V(\vec r)
    =
    \frac{1}{4 \pi \epsilon_0} 
    \int\limits_{\del V} \, d a \, 
    \frac{\hat n \cdot \vec P }{|\vec r - \vec{r'}|}
    -
    \frac{1}{4 \pi \epsilon_0} 
    \int\limits_{V} \, d^3 r' \, 
    \frac{\vec \nabla \cdot \vec P }{|\vec r - \vec{r'}|}
}

With all of this shenanigans, we can say that 
\equations{
    V 
    =
    \frac{1}{4 \pi \epsilon_0}
    \int\limits_V \, d^3 r' \, 
    \frac{p_b(\vec{r'})}{|\vec r - \vec{r'}|}
}

Where the bound charge density is given by 
\equations{
    p_b(r) = - \vec \nabla \cdot \vec P(\vec r)
}
% Where $\vec P(\vec r)$ is the bound volume charge density????

and the surface bound charge density is given by 
\equations{
    \sigma_b(\vec r)
    =
    \hat n \cdot \vec p(\vec r)
}

So the total potential is given by integrating over the bulk bound charge 
and the surface bound charge. 

\subsection{$\vec E$ inside Dielectric}
Consider a linear dielectric 

\equations{
    \vec p = \alpha \vec E 
    \hp 
    \vec P = n \alpha \vec E 
    =
    \frac{\alpha}{a^3} \vec E 
}
Define the susceptibility to polarization $X$ 
\equations{
    \vec P = \epsilon_0 X_E \vec E 
    \hp 
    X_E = \frac{n \alpha}{\epsilon_0}
    =
    \frac{q^2 n}{\epsilon_0 \kappa}
}

A capacitor with a dielectric has both an external charge and a bound 
charge and a bound surface charge, the latter two are induced by both itself 
and the external charge. 

\equations{
    E = \frac{\sigma_{total}}{\epsilon_0}
    \hp 
    \sigma_{tot} = \sigma_{Free} - p 
    \\
    E = \frac{1}{\epsilon_0}
    \left(
        \sigma_{free} - p
    \right)
    =
    \frac{1}{\epsilon_0}
    \left(
        \sigma_{free} - \epsilon_0 X \vec E 
    \right)
    \\
    \vec E 
    = 
    \frac{\sigma_{Free}}{\epsilon_0 (1 + X)}
    =
    \frac{\sigma_{Free}}{\epsilon_0 \kappa }
}

There's some capacitance shenanigans but I don't really care at all 

Consider a 3d bulk charge density 
\equations{
    p_b(\vec r) = - \vec \nabla \cdot \vec p 
}
This happens if $X$, the polarize susceptibility function, is position dependent. 

The bound charges come from the dipole moment $\vec p = \epsilon_0 X \vec E$, 
which comes from the external electric field. 

consider a total charge density 
\equations{
    p_{tot} = p_{free} + p_{bound}
}

use Gauss's Law 
\equations{
    \vec \nabla \cdot \vec E(\vec r)
    =
    \frac{1}{\epsilon_0}
    \left(
        p_{free}(\vec r) + p_{bound}(\vec r)
    \right)
    =
    \frac{1}{\epsilon_0}
    \left(
        \vec p_{free}(\vec r) + \vec \nabla \cdot \vec P(\vec r)
    \right)
    \\
    \nabla \cdot 
    \left(
        \vec E(\vec r) + \frac{1}{\epsilon_0} \vec P(\vec r)
    \right)
    =
    \frac{p_f(\vec r)}{\epsilon_0}
}

This equation is why we make the "Displacement field"

\equations{
    \vec D(\vec r)
    =
    \epsilon_0 \vec E + \vec P(\vec r)
}
I think that's the polarization density and not the dipole moment 

\equations{
    \vec \nabla \cdot \vec D(\vec r) = \rho_{free}
}
In general nothing here should equal 0

\subsection{Helmholtz Theorem}
If you know the curl and divergence of a function, and the function does 
not diverge anywhere, then you know the function.

Consider a linear dielectric 
\equations{
    \vec p = \epsilon_0 X \vec E 
    \hp 
    \kappa = 1 + X 
    \\
    \vec D 
    = 
    \epsilon_0 \vec E + \epsilon_0 X \vec E 
    =
    \epsilon_0 \vec E (1 + X)
    =
    \epsilon_0 \vec E \kappa 
}

$\vec D$ doesn't actually do anything, it just lets us do math easier. 

If we have a parallel plate capacitor, and we consider the boundary between 
the bound charge density and the free charge density, we get 
\equations{
    \hat n \cdot \vec D_1 - \hat n \cdot \vec D_2 = \sigma_{free}
}

If we have no external potential and thus $\sigma_{free} = 0$, then 
\equations{
    \hat n \cdot \vec D_1 = \hat n \cdot \vec D_2 
    \rightarrow 
    \kappa_1 \hat n \cdot \vec E_1  = \kappa_2 \hat n \cdot \vec E_2
}

This lets you determine the electric fields of dielectrics given 
both of their spring constants and some other stuff. 

That is the end of lecture 16. 


\end{document}
