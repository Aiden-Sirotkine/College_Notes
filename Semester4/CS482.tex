\documentclass[fleqn]{report}
\usepackage{geometry}
\usepackage{amssymb}
\usepackage{fancyhdr}
\usepackage{multicol}
\usepackage{blindtext}
\usepackage{color}
\usepackage[fontsize=16pt]{fontsize}
\usepackage{lipsum}
\usepackage{pgfplots}
\usepackage{physics}
\usepackage{mathtools}
\usepackage[makeroom]{cancel}
\usepackage{ulem}

\setlength{\columnsep}{1cm}
\addtolength{\jot}{0.1cm}
\def\columnseprulecolor{\color{blue}}
\date{Spring 2025}

\newcommand{\textoverline}[1]{$\overline{\mbox{#1}}$}

\newcommand{\hp}{\hspace{1cm}}

\newcommand{\const}{\textrm{const}}

\newcommand{\del}{\partial}

\newcommand{\pdif}[2]{ \frac{\partial #1}{ \partial #2} }

\newcommand{\pderiv}[1]{ \frac{\partial}{ \partial #1} }

\newcommand{\comment}[1]{}

\newcommand{\equations} [1] {
\begin{gather*}
#1
\end{gather*}
}

\newcommand{\numequations} [1] {
\begin{gather}
#1
\end{gather}
}

\newcommand{\twovec}[2]{ 
\begin{pmatrix}
#1 \\ 
#2
\end{pmatrix}
}

\title{CS 482}
\author{Aiden Sirotkine}

\begin{document}

\pagestyle{fancy}
\maketitle
\tableofcontents
\clearpage

\chapter{CS 482}
This'll be an interesting class lets hope I can figure out stats.

\begin{itemize}
    \item 
    building complex models 
    \item 
    algorithmic randomness 
    \item 
    statistically analyzing data
\end{itemize}

Chuck data into a black box of modelling code and get result data out. 

Simulations give the most realistic answers for complex systems that cannot be 
linearly solved. 

Simulation is only good for basically unsolvable problems. 

\chapter{Terminology}

\section{System}
A collection of \textit{entities} that interact with a common 
purpose according to sets of \textit{laws} and \textit{policies}

example: a store, airport terminal, etc. \hp most things 

\section{Entities}
The components/objects that define a system

Physical or Logical objects 

Temporary or Permanent 

\section{Attributes}
The traits that define an entity 

static or dynamic

qualitative or quantitative 

A cashier in a store :
\begin{itemize}
    \item 
    has a high school diploma (static, qualitative)
    \item 
    has an IQ of 104 (static, quantitative)
    \item 
    can be busy or idle (dynamic, qualitative)
    \item 
    can process customers at a rate from 6/hour to 20/hour (dynamic, quantitative)
\end{itemize}

\section{Entity-Attribute Hierarchy}
determines the level of detail in the simulation 
\begin{itemize}
    \item 
    Regional Plants 
    \item 
    Production Lines 
    \item 
    Work Areas 
    \item 
    Machines, Tools, Operators 
\end{itemize}

The \textit{level of detail} in the simulation model 
is set by the entity/attribute hierarchy boundary, 
which is determined by the \textit{objectives of the study}. 

Attributes become entities if you want more detail in a system 

The boundary between attribute/entity determines the level of detail 
in the simulation. 

\section{Laws and Policies}
Both Laws and Policies govern the behavior of the system, but Laws cannot 
be changed while Policies can be changed. 

Laws are followed, Policies are set. 

\section{Model}
The thing that we're trying to do in this class 

A simplification of a system 

there are many ways to model a system 
\begin{enumerate}
    \item 
    Events List 
    \item 
    Difference Equations 
    \item 
    Markov Chains 
\end{enumerate}

\subsection{State Space}
A Collection of variables that represent and measure the condition of the system 
(busy, idle, broken, etc.)

The state space is the \textit{film} for a photo snapshot of a system.

\section{Event}
An instant of time when one of the following occurs 

\begin{itemize}
    \item
    the state(s) change(s)
    \item 
    other events are caused (scheduled) or prevented (cancelled)
    (i.e. other states change)
    \item 
    data is collecte dand statistics may be compiled 
    (based on or uses states)
\end{itemize}

examples:
\begin{itemize}
    \item
    a part arrives 
    \item
    A machine starts or stops or breaks down 
    \item
    The end of day of operation 
\end{itemize}

\section{Process}
An indexed set of states of events 

Let $N_t$ be hte number of customers in the system at time $t$ 

Let $B_j$ be a $0-1$ indicator of whether or not that $j$th customer does 
or doesn't get on hold. 

\section{Discrete-Event Simulation}
A model where the state $S$ changes at \textit{discrete} points in time 

what/when/how/what impact of changes 

\section{Single Server Queue}
$\lambda$ is the rate at which parts arrive 

$\mu$ is the rate at wich the server can process parts 

Number in system = number in queue + number in service. 

events schedule other events.

\section{Building Simulation Models}
\begin{itemize}
    \item 
    define states 
    \item 
    identify when and how the state change 
    \item 
    define events 
    \item 
    define initialization 
\end{itemize}

\begin{itemize}
    \item
    ALL simulation models have an initialization event. 
    \item
    assume a random number generator is available 
    \item
    a new random variable value must be generated each time a 
    random variable is used or called 
    \item
    arrival events schedule more arrival events (self-generating)
\end{itemize}

\subsection{When executing an event:}

\begin{itemize}
    \item
    State changes occur first 
    \item
    Events scheduled or cancelled occur second 
\end{itemize}

Once you have all of the states and changes and whatever, coding 
the simulation itself is not actually that hard. 
    
hop from most recent event to most recent event until done, 
changing the event queue as needed. 

\subsection{Dynamic Simulation}
Events are scheduled and executed in \underline{time sequence}

States are changed as each event is executed 

Data can be colelcted as events to do stats and math stuff 

Know where to collect data and what type of data to collect. 

    

 
\end{document}