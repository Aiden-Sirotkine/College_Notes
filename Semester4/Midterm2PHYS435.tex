\documentclass[fleqn]{report}
\usepackage{geometry}
\usepackage{amssymb}
\usepackage{fancyhdr}
\usepackage{multicol}
\usepackage{blindtext}
\usepackage{color}
\usepackage[fontsize=16pt]{fontsize}
\usepackage{lipsum}
\usepackage{pgfplots}
\usepackage{physics}
\usepackage{mathtools}
\usepackage[makeroom]{cancel}
\usepackage{ulem}
\usepackage{esint}

\graphicspath{ {../Images/} }
\setlength{\columnsep}{1cm}
\addtolength{\jot}{0.1cm}
\def\columnseprulecolor{\color{blue}}
\date{Spring 2025}

\newcommand{\textoverline}[1]{$\overline{\mbox{#1}}$}

\newcommand{\hp}{\hspace{1cm}}

\newcommand{\const}{\textrm{const}}

\newcommand{\del}{\partial}

\newcommand{\pdif}[2]{ \frac{\partial #1}{ \partial #2} }

\newcommand{\pderiv}[1]{ \frac{\partial}{ \partial #1} }

\newcommand{\comment}[1]{}

\newcommand{\equations} [1] {
\begin{gather*}
#1
\end{gather*}
}

\newcommand{\numequations} [1] {
\begin{gather}
#1
\end{gather}
}

\newcommand{\twovec}[2]{ 
\begin{pmatrix}
#1 \\ 
#2
\end{pmatrix}
}

\title{PHYS 435 Midterm 1}
\author{Aiden Sirotkine}

\begin{document}

\pagestyle{fancy}
\maketitle
\tableofcontents
\clearpage

\chapter{PHYS435 Midterm 2}
All magneto and electrostatics and maybe a little bit of electrodynamics 
with Ampere's Law. 

REMEMBER YOU MAGNETIC DIPOLE EQUATIONS

and you magnetic field from a current equations 

If you have both a current and a magnetization density, 
you can just put them together and try to solve. 

All you do is solve both parts independently. 

Solve the perma-magnetic with the magnetic pseudopotential and solve 
the moving current with the other math. 

This midterm is open book open note so if I just write down 
absolutely every topic ever then I can look back at this notebook 
and be absolutely set 

This midterm goes over basically all of electrostatics and magnetostatics
\section{Calculus Junk}
Know my Divergence and curls and junk 

\subsection{Divergence Theorem}
Connects a volume to a surface 
\equations{
    \iiint\limits_V \, d^3 V \, 
    \nabla \cdot F 
    =
    \iint\limits_{S} \, d^2 S \, 
    \hat n \cdot F
}

\subsection{Poisson's Equation and Green's Functions (Magnetic Pseudopotential)}
We also do have a kind of Green's function approach 
with the magnetic pseudo-potential 
\equations{
    \div \vec H 
    =
    - 
    \div \vec M 
    \hp 
    \vec H 
    =
    - \vec \nabla \phi_M(\vec r)
    \\
    \nabla^2 \phi_m(\vec r)
    =
    - \rho_m(\vec r)
    =
    + \div \vec M 
}

\section{Multipoles}
They're like a just a heuristic thing to figure out roughly what you 
distance dependence is.

It's based off of a taylor series that I'm sure you can derive if you really 
want to. 

monopole is 1/r, dipole is $1/r^2$, quadrupole is $1/r^3$

\section{Dielectrics}
I genuinely have no idea but there's some indentities and some polarizability 
stuff. 

Dielectrics are not conductors, but are made of atoms that can conduct 
electricity. 

Each atom has a symmetric charge distribution 

Now apply an electric field so that the dielectric has a dipole moment. 
\equations{
    \vec p = \alpha \vec E
}
Where $\alpha$ is the polarizability coefficient.

\subsection{Spring Model of Atoms}

The dipole moment is proportional to the displacement of the positive 
and negative charges relative to each other. 

\subsection{Polarization Density}
Because dielectrics are like continuous 

\equations{
    \vec P = \vec p \cdot \textrm{atomic density} = pn
    \\
    \vec P = n \alpha \vec E 
}

Consider an electronic suscebtilibility factor $X$ 

\equations{
    \vec P = \epsilon_0 X \vec E 
    \hp 
    X = \frac{n q^2}{\epsilon_0 \kappa}
}

We find the polarizability constants of single atoms but I doubt that's super 
necessary 

\subsection{Bound Charge Density}
I think this is finding the new charge density from a polarized dielectric 

\equations{
    V(\vec r) = 
    \frac{1}{4 \pi \epsilon_0}
    \frac{\vec p \cdot \vec r }{|\vec r - \vec{r'}|^3}
}
consider polarization density and $\vec p = \int \, d^3 r \, \vec P$

\equations{
    V(\vec r) = 
    \frac{1}{4 \pi \epsilon_0}
    \int\limits_{V} \, d^3 r' \, 
    \frac{\vec P \cdot \vec r }{|\vec r - \vec{r'}|^3}
}

we know that the derivative of $1/r$ is $-1/r^2 = -r/r^3$, so 

\equations{
    V(\vec r) = 
    \frac{1}{4 \pi \epsilon_0}
    \int\limits_{V} \, d^3 r' \, 
    \vec P
    \cdot 
    -\vec \nabla_r
    \frac{1 }{|\vec r - \vec{r'}|}
    =
    \frac{1}{4 \pi \epsilon_0} 
    \int\limits_{V} \, d^3 r' \, 
    \vec P
    \cdot 
    \vec \nabla_{r'}
    \frac{1 }{|\vec r - \vec{r'}|}
}

You do some math 
\equations{
    \vec \nabla_{r'}
    \cdot 
    \frac{\vec P }{|\vec r - \vec{r'}|}
    =
    \frac{\vec \nabla \cdot \vec P }{|\vec r - \vec{r'}|}
    +
    \vec P
    \cdot 
    \vec \nabla_{r'}
    \cdot 
    \frac{1 }{|\vec r - \vec{r'}|}
}

and then get 
\equations{
    V(\vec r)
    =
    \frac{1}{4 \pi \epsilon_0} 
    \int\limits_{V} \, d^3 r' \, 
    \vec P
    \cdot 
    \vec \nabla_{r'}
    \frac{1 }{|\vec r - \vec{r'}|}
    =
    \\
    \frac{1}{4 \pi \epsilon_0} 
    \int\limits_{V} \, d^3 r' \, 
    \vec \nabla_{r'}
    \cdot 
    \frac{\vec P }{|\vec r - \vec{r'}|}
    -
    \frac{1}{4 \pi \epsilon_0} 
    \int\limits_{V} \, d^3 r' \, 
    \frac{\vec \nabla \cdot \vec P }{|\vec r - \vec{r'}|}
}

Then do divergence theorem 

\equations{
    V(\vec r)
    =
    \frac{1}{4 \pi \epsilon_0} 
    \int\limits_{\del V} \, d a \, 
    \frac{\hat n \cdot \vec P }{|\vec r - \vec{r'}|}
    -
    \frac{1}{4 \pi \epsilon_0} 
    \int\limits_{V} \, d^3 r' \, 
    \frac{\vec \nabla \cdot \vec P }{|\vec r - \vec{r'}|}
}

With all of this shenanigans, we can say that 
\equations{
    V 
    =
    \frac{1}{4 \pi \epsilon_0}
    \int\limits_V \, d^3 r' \, 
    \frac{p_b(\vec{r'})}{|\vec r - \vec{r'}|}
}

Where the bound charge density is given by 
\equations{
    p_b(r) = - \vec \nabla \cdot \vec P(\vec r)
}
% Where $\vec P(\vec r)$ is the bound volume charge density????

and the surface bound charge density is given by 
\equations{
    \sigma_b(\vec r)
    =
    \hat n \cdot \vec p(\vec r)
}

So the total potential is given by integrating over the bulk bound charge 
and the surface bound charge. 

\subsection{$\vec E$ inside Dielectric}
Consider a linear dielectric 

\equations{
    \vec p = \alpha \vec E 
    \hp 
    \vec P = n \alpha \vec E 
    =
    \frac{\alpha}{a^3} \vec E 
}
Define the susceptibility to polarization $X$ 
\equations{
    \vec P = \epsilon_0 X_E \vec E 
    \hp 
    X_E = \frac{n \alpha}{\epsilon_0}
    =
    \frac{q^2 n}{\epsilon_0 \kappa}
}

A capacitor with a dielectric has both an external charge and a bound 
charge and a bound surface charge, the latter two are induced by both itself 
and the external charge. 

\equations{
    E = \frac{\sigma_{total}}{\epsilon_0}
    \hp 
    \sigma_{tot} = \sigma_{Free} - p 
    \\
    E = \frac{1}{\epsilon_0}
    \left(
        \sigma_{free} - p
    \right)
    =
    \frac{1}{\epsilon_0}
    \left(
        \sigma_{free} - \epsilon_0 X \vec E 
    \right)
    \\
    \vec E 
    = 
    \frac{\sigma_{Free}}{\epsilon_0 (1 + X)}
    =
    \frac{\sigma_{Free}}{\epsilon_0 \kappa }
}

There's some capacitance shenanigans but I don't really care at all 

Consider a 3d bulk charge density 
\equations{
    p_b(\vec r) = - \vec \nabla \cdot \vec p 
}
This happens if $X$, the polarize susceptibility function, is position dependent. 

The bound charges come from the dipole moment $\vec p = \epsilon_0 X \vec E$, 
which comes from the external electric field. 

consider a total charge density 
\equations{
    p_{tot} = p_{free} + p_{bound}
}

use Gauss's Law 
\equations{
    \vec \nabla \cdot \vec E(\vec r)
    =
    \frac{1}{\epsilon_0}
    \left(
        p_{free}(\vec r) + p_{bound}(\vec r)
    \right)
    =
    \frac{1}{\epsilon_0}
    \left(
        \vec p_{free}(\vec r) + \vec \nabla \cdot \vec P(\vec r)
    \right)
    \\
    \nabla \cdot 
    \left(
        \vec E(\vec r) + \frac{1}{\epsilon_0} \vec P(\vec r)
    \right)
    =
    \frac{p_f(\vec r)}{\epsilon_0}
}

This equation is why we make the "Displacement field"

\equations{
    \vec D(\vec r)
    =
    \epsilon_0 \vec E + \vec P(\vec r)
}
I think that's the polarization density and not the dipole moment 

\equations{
    \vec \nabla \cdot \vec D(\vec r) = \rho_{free}
}
In general nothing here should equal 0

\subsection{Helmholtz Theorem}
If you know the curl and divergence of a function, and the function does 
not diverge anywhere, then you know the function.

Consider a linear dielectric 
\equations{
    \vec p = \epsilon_0 X \vec E 
    \hp 
    \kappa = 1 + X 
    \\
    \vec D 
    = 
    \epsilon_0 \vec E + \epsilon_0 X \vec E 
    =
    \epsilon_0 \vec E (1 + X)
    =
    \epsilon_0 \vec E \kappa 
}

$\vec D$ doesn't actually do anything, it just lets us do math easier. 

If we have a parallel plate capacitor, and we consider the boundary between 
the bound charge density and the free charge density, we get 
\equations{
    \hat n \cdot \vec D_1 - \hat n \cdot \vec D_2 = \sigma_{free}
}

If we have no external potential and thus $\sigma_{free} = 0$, then 
\equations{
    \hat n \cdot \vec D_1 = \hat n \cdot \vec D_2 
    \rightarrow 
    \kappa_1 \hat n \cdot \vec E_1  = \kappa_2 \hat n \cdot \vec E_2
}

This lets you determine the electric fields of dielectrics given 
both of their spring constants and some other stuff. 

That is the end of lecture 16. 

\chapter{NEW STUFF}


\section{Various Magnetic Fields}
Everything can be solve with either Gauss's Law, superposition, or both. 

\begin{itemize}
    \item 
    So you have your magnetic field $\vec B$ 
    \item 
    You have your volume current density $\vec J$
    \item 
    you have your surface current density $\vec J_b$ 
    \item 
    you have your magnetization density $\vec M$ 
    \item 
    You have your magnetic field density $\vec A$ 
    \item 
    You have you helper field $\vec H$ 
    \item 
    You have your magnetic pseudopotential $\phi_m$
    \item 
    You have a magnetic dipole susceptibility $X_m$
\end{itemize}

I think that covers all of it. 

Then you use all the random equations that we learned to 
put them together 

\subsection{Coulomb Gauge}
That just means that 
\equations{
    \nabla^2 \vec A = - \mu \vec J
    \hp 
    \div \vec A = 0
}

know that 
\equations{
    \vec B 
    =
    \curl \vec A 
}

\section{Magnetic Multipoles}
I'm just gonna worry about the dipole moment for now 

\equations{
    \vec A 
    =
    \frac{\mu_0}{4 \pi}
    \frac{\vec m \times \vec r }{r^3}
}

dipole moment is calculated by 
\equations{
    \vec m 
    =
    \int \, d^3 r \, 
    \frac{\vec r \times \vec J}{2}
}

and if we consider a current we get 
\equations{
    \vec m 
    =
    I * Area * \hat n
}

Consider a magnetization density vector field 
\equations{
    \vec M 
    =
    n(\vec r) 
    \vec m(\vec r)
}
Where $n$ is the local density 

\equations{
    \vec A 
    =
    \int \, d^3 \, 
    \frac{\vec M \times \vec r}{r^3}
}

and we get the important 
\equations{
    J_{bound}
    =
    \curl \vec M 
    \hp 
    K_{bound surface}
    =
    \vec M \times \hat n 
}

\section{Boundary Value Problems}
This one is connected to the homework 8 that I did and that I have the answers for 

\equations{
    \curl \vec B 
    =
    \mu_0
    \left(
        \vec J_{free}
        +
        \vec J_{bound}
    \right)
}
That gets us the helper field 
\equations{
    \vec H 
    = 
    \frac{1}{\mu_0}
    \vec B 
    -
    \vec M
    \hp 
    \curl \vec H 
    =
    \mu_0 \vec J_{free}
}

That curl equation gives us some important stuff 
\equations{
    \vec H_{1, //}
    +
    \vec H_{2, //}
    =
    \vec K_{free}
    \times \hat n 
}
Where $\vec K_f$ is the free current density I think? 

And, since $\div \vec B = 0$, we get the other important 
\equations{
    \hat n(\vec B_{1, \perp} + \vec B_{2, \perp}) = 0 
    \rightarrow 
    \vec B_{1}
    \cdot \hat n
    =
    \vec B_{2}
    \cdot \hat n
}

For an ideal ferromagnet, there are no free currents and the bound 
currents are wholly determined by $\vec M$ 

There's a thingy 
\equations{
    \vec M 
    =
    X_m \hat H_{in}
    \hp 
    X_m 
    =
    \frac{M}{H_{in}}
}
So for a linear paramagnet we get 
\equations{
    \vec B_{in}
    =
    \mu_0 (\vec H_{in} + \vec M )
    =
    \mu_0 (1 + X_m) \vec H_{in}
}

okay I get it now 
so $\vec J$ is the volume current density which is amperes/m$^2$ and 
$\vec K$ is the surface current density which is just teslas / m. 

\section{Magnetic Pseudopotential}
It's just a thing that connects to $\vec H$ that allows you to get 
something that looks like a Green's function. 

\section{UNITS}
$\vec B$ is Volt-second / meters$^2$

\end{document}
