\documentclass[fleqn]{report}
\usepackage{geometry}
\usepackage{amssymb}
\usepackage{fancyhdr}
\usepackage{multicol}
\usepackage{blindtext}
\usepackage{color}
\usepackage[fontsize=16pt]{fontsize}
\usepackage{lipsum}
\usepackage{pgfplots}
\usepackage{physics}
\usepackage{mathtools}
\usepackage[makeroom]{cancel}
\usepackage{ulem}

\graphicspath{ {../Images/} }
\setlength{\columnsep}{1cm}
\addtolength{\jot}{0.1cm}
\def\columnseprulecolor{\color{blue}}
\date{Spring 2025}

\newcommand{\textoverline}[1]{$\overline{\mbox{#1}}$}

\newcommand{\hp}{\hspace{1cm}}

\newcommand{\const}{\textrm{const}}

\newcommand{\del}{\partial}

\newcommand{\pdif}[2]{ \frac{\partial #1}{ \partial #2} }

\newcommand{\pderiv}[1]{ \frac{\partial}{ \partial #1} }

\newcommand{\comment}[1]{}

\newcommand{\equations} [1] {
\begin{gather*}
#1
\end{gather*}
}

\newcommand{\numequations} [1] {
\begin{gather}
#1
\end{gather}
}

\newcommand{\twovec}[2]{ 
\begin{pmatrix}
#1 \\ 
#2
\end{pmatrix}
}

\title{Elastic Properties of 2D Origamis}
\author{Aiden Sirotkine}

\begin{document}

\pagestyle{fancy}
\maketitle
\tableofcontents
\clearpage

\chapter{Basis}
The current way to determine the elastic properties of 2D origamis is by 
calculating the free energy using Umbrella Sampling, and this data is not 
experimentally useful at all.

What I should do is figure out a framework for calculating the elastic properties 
of 2d materials using fluctuations. 

Hemani said that the most important quantities to determine are the 
axes dependent on the geometry of the sheet and the $E_1$ and $E_2$ values.



\end{document}